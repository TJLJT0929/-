\begin{document}


Optimal Dynamic Pricing of Inventories with Stochastic Demand over
Finite Horizons\\
Author(s): Guillermo Gallego and Garrett van Ryzin\\
Source: Management Science,Vol. 40, No.~8(Aug.,1994),pp.~999-1020\\
Published by: INFORMS\\
Stable URL: http://www.jstor.org/stable/2633090\\
Accessed: 03/03/2010 21:51

INFORMS is collaborating with JSTOR to digitize, preserve and extend
access to Management Science.

\section{Optimal Dynamic Pricing of Inventories with Stochastic Demand
over Finite
Horizons}\label{optimal-dynamic-pricing-of-inventories-with-stochastic-demand-over-finite-horizons}

Guillermo Gallego · Garrett van Ryzin Department of Industrial
Engineering and Operations Research,Columbia University,New York,New
York 10027 Graduate School of Business,Columbia University,New York,New
York 10027

n many industries, managers face the problem of selling a given stock of
items by a deadline. I We investigate the problem of dynamically pricing
such inventories when demand is price sensitive and stochastic and the
firm's objective is to maximize expected revenues. Examples that fit
this framework include retailers seling fashion and seasonal goods and
the travel and leisure industry, which markets space such as seats on
airline flights,cabins on vacation cruises, and rooms in hotels that
become worthless if not sold by a specific time.

We formulate this problem using intensity control and obtain structural
monotonicity results for the optimal intensity (resp., price)as a
function of the stock level and the length of the horizon. For a
particular exponential family of demand functions, we find the optimal
pricing policy in closed form.For general demand functions, we find an
upper bound on the expected revenue based on analyzing the deterministic
version of the problem and use this bound to prove that simple, fixed
price policies are asymptotically optimal as the volume of expected
sales tends to infinity. Finally,we extend our results to the case
where demand is compound Poisson; only a finite number of prices is
allowed; the demand rate is time varying; holding costs are incurred and
cash flows are discounted; the initial stock is a decision variable; and
reordering,overbooking,and random cancellations are allowed.

(Dynamic Pricing; Inventory; Yield Management; Intensity Control;
Stochastic Demand; Optimal Policies;Heuristics; Finite Horizon; Stopping
Times)

\section{1. Introduction and
Motivation}\label{introduction-and-motivation}

Given an initial inventory of items and a finite horizon overwhich sales
are allowed,we are concerned with the tactical problem of dynamically
pricing the items to maximize the total expected revenue. Two key
properties of this problem are the lack of short-term control over the
stock and the presence of a deadline after which selling must stop.
Demand is modeled as a pricesensitive stochastic point process with an
intensity that is a known decreasing function of the price; revenues are
collected as the stock is sold; no backlogging of demand is allowed;
unsold items have a given salvage value; and all costs related to the
purchase or production of items are considered sunk costs.

This generic problem arises in a variety of.industries. Retailers that
sell seasonal and style goods are an example(cf.Pashigan 1988 and
Pashigan and Bowen 1991).For instance, the authors recently have worked
with a major New York fashion producer-retailer that
designs,produces(via subcontractors),and sells fashion apparel through
its own line of retail outlets.(Other similar retailer-producers include
The GAP and The Limited.) The firm is known for the subtle and unique
colors of its garments,which are achieved using custommade fabrics.To
produce its garments,the firm must special order fabric directly from
mills.Theraw bolts of fabric are then shipped off-shore (usually by
surface freight) to a subcontractor that cuts and assembles the various
styles.The finished garments are then shipped back to the
U.S.(again,often by surface freight) where they are sorted,boxed,and
delivered to individual stores.This entire production process takes from
six to eight months to complete,yet the firm plans to ``sellthrough''
garments in as little as nine weeks!

The basic assumptions of the model fit this situation quite well. There
is a deadline for the sales period (nine weeks),and for all practical
purposes the company has no resupply option during the sales season.
Further,it is clear that the once the items are on the rack,the entire
production decision is sunk.Leftover garments are sold through an
affiliated outlet store yielding a given salvage value.(The salvage
value does impact the pricing decision,as we discuss below.) Demand for
garments is uncertain but is influenced by price.The merchandise
manager's job is to adjust the price(via markdowns or periodic sales)
throughout the selling season in response to the realized demand to
maximize revenues. Similar, though perhaps less extreme,instances of the
problem occur when selling seasonal appliances such as snow blowers,air
conditioners,etc.

The problem is also a fundamental one in the travel and leisure
industry.Managers in that industry face hard time constraints and have
almost no control in the short run over available space.For
example,airlines have a specified number of seats available on each
flight,and empty seats are worthless after the plane departs.To increase
their revenues,airlines give customers incentives to book in
advance.These incentives typically are adjusted in response to the
realized demand by opening and closing the various fare classes
available at any given point in time.This practice,known as yield
management,is now used by all majorairlines and is increasingly adopted
by major hotel chains and even by some car rental companies and cruise
ship lines (Kimes 1989).The benefits of yield management are often
staggering;American Airlines reportsa five-percent increase in
revenue,worth approximately \(\$ 1.4\) billion dollars over a
three-year period,attributable to effective yield management (Smith et
al.~1992).

Yet, despite its growing importance, there appears to be a certain
confusion about precisely what phenomenon yield management actually is
trying to exploit Indeed,in a recent survey,Weatherford and Bodily
(1992) conclude that''Several definitions of yield management have been
put forward,but to date no agreement exists on its meaning.'' They point
to market segmentation through time-of-purchase mechanisms (e.g.,
advance purchase requirements,cancellation penalties,

Saturday-night stays,etc.) as one possibility.Though it is certainly an
important factor,market segmentation provides only a partial---and
perhaps not the most central---explanation for the benefits of yield
management. This explanation appears somewhat biased by the
business-traveler/vacation-traveler division of the customer population
particular to the airline industry. In fact, for resort
hotels,cruiseship lines,and theaters, yield management mechanisms seem
to be beneficial even though the customer population is arguably much
more homogeneous.

Our results provide some important insights on this issue.In particular,
they suggest two alternative explanations for the benefits of yield
management: (1) Yield management is an attempt to adjust prices to
compensate for''normal''(to be made precise below) statistical
fluctuations in demand.For this first explanation,we have
anegativeresult.Namely,under some rathermild assumptions,we prove that
if demand as a function of price is known and prices are unconstrained,
then a single fixed-price policy is very nearly optimal. Thus, offering
multiple prices can at best capture only secondorder increases in
revenuedueto the statistical variability in demand.(Of course,even
second-order increases in revenue may be significant in practice,so this
explanation cannot be totally discounted.) Also, the relative
fluctuations of an optimal pricing policy appear to be small (on the
order of \(1 0 \%\) or less),while those found in the airline industry
in particular can differ by \(1 0 0 \%\) or more.

The second explanation revealed by our analysis is more compelling:(2)
Yield management is an attempt to ``synthesize'' a range of optimal
prices from a small,static set of pricesin response to a shiftingdemand
function.The above fixed-price results hold only when the firm knows the
demand function in advance and can price each instance of the
problem(e.g., day/flight/voyage) individually. In most applications,
these conditions do not hold.Airlines and hotels must, for a variety of
opera tional and customer-relations reasons,offer a limited number of
fares that remain relatively static,at least in the sense of spanning
several problem instances.Further,demand may shift significantly during
the week or over holidays,and also may not be easy to predict in
advance. In such a setting, we prove that a nearoptimal policy is to
allocate an appropriate fraction of time and capacity to each fare
class,much as is done in conventional yield management practice.In this
way, a static set of fare classes together with a dynamic allocation
scheme can be used to synthesize different prices for each instance.This
interpretation better explains both the magnitude of revenue increases
and the disparity in fare prices found in yield management practice.

Finally,we note that one important consideration which is ignored in
our formulation is the cost of price changes. Often, these costs are
small.For example, travel agents provide customers with current price
quotes based on information obtained from computer databases which can
easily be updated. In retailing,items may be bar-coded, and thus the
cost of a change involves only a computer entry and a change in the
displayed price. In such a case,assuming no cost for price changes is a
reasonable approximation. In many businesses,however,substantial
advertising or ticketing costs are associated with a price change.In
these cases,more stable pricing strategiesare needed.We show,however,
that policies that have no price changes are asymptotically (as the
expected volume of sales increases) optimal over the class of policies
that allow an unlimited number of price changes at no cost. This, of
course, implies asymptotic optimality for the problem with price change
costs as well.Further,we bound the additional expected revenue one can
obtain from a dynamic pricing policy over a fixed-price policy. This
bound can then be used in conjunction with cost information on price
changes to help determine if dynamic pricing is cost effective.

\section{1.1. Literature Review}\label{literature-review}

Research on pricing policies has been pursued by economists,marketing
scientists,and operationsresearchers from a range of perspectives.A
considerable body of work has evolved on joint ordering/production and
pricing models.A recent and comprehensive survey of this area is given
by Eliashberg and Steinberg(1991). In contrast, the main applications
and models we study fundamentally have few or no options for reordering.
However,in §5 we do analyze extensions to our model that consider
initial inventory decisions,reordering, holding costs,and discounting
under specialized (unit) cost structures.These extensions relate more
closely to the production-pricing literature.

Production-pricing problems are broadly categorized in Eliashberg and
Steinberg(1991)into convex and concave ordering cost cases.We shall
adopt this classification as well. In the convex case,several
discretetime stochastic models have been investigated in which ordering
and pricing decisions are allowed in each period. Single-period models
are analyzed by Hempenius (1970),Karlin and Carr
(1962),Mills(1959),and Whitin(1955)(his style goods model).These
singleperiod models are essentially price-sensitive versions of the
classic ``news-boy'' problem and are similar to our
initial-order-quantity extension discussed in \(\ S 5 . 4\) .The
difference is that these models assume static prices and demand,while
our model involves a continuous,dynamic demand process and allows
dynamic pricing decisions throughout the period.Lazear(1986) considers
a model of retail pricing with a single ordering decision and one
recourse option to change the price.He formulates a simple, two-stage
dynamic program to solve the problem.Pashigan(1988) and Pashigan and
Bowen (1991) investigate this model empirically.

Multi-period, finite-horizon models with convex costs are considered by
Hempenius (1970),Thowsen(1975) and Zabel(1972).Veinott(1980) uses the
theory of lattice programming to investigate monotonicity properties of
a class of deterministic,multi-period problems. Our reorder option
extension in \(\ S 5 . 5\) fits broadly in this class,though it is a
continuous time model and only unit ordering costs are considered.Karlin
and Carr also analyze a stationary,infinite-horizon discounted cost
problem,a problem which is also briefly discussed by Mills (1959).

To our knowledge,Li (1988) is the only other paper that considers a
continuous time model where demand is a controlled Poisson
processes.(Our reorder process in \(\ S 5 . 5\) is.deterministic while
Li's is Poisson.) The objective in his paper is to maximize expected
discounted profit over an infinite horizon. There is a cost for
production capacity, production and holding costs are linear,and both
production and pricing decisions are considered.Li's main result is that
a barrier policy is optimal for the production decision. He also gives
an implicit characterization of the optimal pricing policy when dynamic
pricing is allowed.

Concave order costs,usually due to the presence of fixed order costs,are
more difficult to analyze and most work has been confined to
deterministic models.EOQ models with price sensitive demand are
investigated in Keunreuther and Richard (1971) and Whitin(1955).
Cohen(1977) and Rajan et al.(1992) consider problems with decaying
inventories. Thomas(1970),Wagner (1960),and Wagner and Whitin
(1958)analyze discrete-time,multi-period models with concave costs. To
our knowledge,Thomas(1974) is the only paper that studies a
stochastic,multi-period model with fixed order costs.

In marketing science, dynamic models of pricing date back to Robinson
and Lakhani(1975)and the subsequent work of Bass (1980),Dolan and
Jeuland (1981), Jeuland and Dolan(1982),and Kalish(1983).(See Rao 1984
for an overview.) This research,however, focuses on strategic issues of
life cycle pricing based on deterministic models of how firm economics
and consumer behavior change with time.Several marketing scientists have
looked at tactical,dynamic pricing problems. Chakravarty and Martin
(1989) examine setting optimal quantity discounts in the face of
deterministic, dynamically changing demand. Kinberg and Rao(1975) model
consumer purchase behavior as a Markov chain and examine the problem of
selecting the optimal duration for a price promotion.(See also Nagle
1987 and Oren 1984.)

We have already mentioned that the area of yield management is quite
related to our problem. The study of yield management problems in the
airlines dates back to the work of Littlewood(1972) for a stochastic
twofare,single-leg problem and to Glover et al.(1982) for a
deterministic network model. Belobaba(1987,1989) proposed and tested a
multiple-fare-class extension of Littlewood's rule,which he termed the
expected marginal seat revenue(EMSR)heuristic.Extensions and
refinements of the multiple-fare-class problem include recent papers by
Brumelle et al.~(1990),Curry (1989), Robinson(1991),and
Wollmer(1992).Kimes(1989) gives a general overview of yield management
practice in the hotel industry.(See Bitran and Gilbert 1992,Liberman and
Yechiali 1978,and Rothstein 1974 for analytical models of hotel
problems.)A recent review of research on yield management is given by
Weatherford and Bodily (1992),where they adopt the term perishable
asset revenue management (PARM) to describe thisclass of problems.Our
problem can certainly be considered a continuous-time PARM problem.

Lastly,we mention three papers that address sufficiency conditions for
problems similar to our basic model.1 Miller(1968) studies a finite
horizon,continuous-time Markov decision process where only finitely many
actions (prices) are allowed.He obtains sufficient conditions for
optimality and shows that optimal policies are piecewise constant.
Kincaid and Darling(1963) analyze a problem that is functionally
equivalent to the basic single-commodity version of our problem. By
studying the problem from first principles, they again obtain sufficient
conditions; recently Stadje (1990) independently re-derived a similar
set of results.Unfortunately, the sufficient conditions derived in these
papers rarely lead to a solution; indeed,even for the basic version of
the problem few practical results have been obtained using these exact
approaches.

\section{1.2. Overview and Outline}\label{overview-and-outline}

In \(\ S 2\) we discuss our assumptions, formulate our basic model,
provide structural results,and find an exact solution for an exponential
demand function. We show that the stochastic optimal policy changes
prices continuously and thus may be undesirable in practice. This leads
us to try approximate methods. In §3 we find up per bounds on the
optimal revenue by considering a deterministic version of the problem.We
solve the deterministic problem and show that the optimal policy is to
set a fixed price throughout the horizon. Further, this deterministic
fixed-price policy is asymptotically optimal for the stochastic problem
as the volume of expected sales increases or as the time horizon tends
to zero.Numerical examples are given that indicate the performance of
fixed-price policies is quite good even when the expected volume of
sales is moderate. In \(\ S 4\) we analyze the case where only a finite
set of prices is allowed,a variant of the problem that is most closely
related to the yield management problem.

Finally, in \(\ S 5\) we examine several extensions to the basic
problem.First,we generalize our results to the case where demand is a
compound Poisson process.We then consider the case where the demand
function varies in time through a multiplicative seasonality factor;
holding costs are incurred and cash flows are discounted; the initial
inventory is a decision variable; additional items can be obtained at a
unit cost after the initial inventory is depleted.The last extension
allows for overbooking and cancellations.For all these cases, we find
asymptotically optimal heuristics. Our conclusions and thoughts for
future research in this area are given in \(\ S 6\)

\section{2. Assumptions, Formulation and Preliminary
Results}\label{assumptions-formulation-and-preliminary-results}

\section{2.1. Economic and Modeling
Assumptions}\label{economic-and-modeling-assumptions}

We assume our firm operates ina market with imperfect competition.For
example, the firm may bea monopolist, the product may be new and
innovative, in which case the firm holds a temporary monopoly,or the
market mayallow for product differentiation.Under imperfect
competition,a firm can influence demand by varying its price, \(p\) . We
express the demand as a rate (\# items / time) that depends only on the
current price \(p\) through a function \(\lambda ( p )\) . In the
monopoly or new product case, \(\lambda ( p )\) is the market demand and
is assumed to be nonincreasing in \(p\) due to substitution effects.For
example, if the arrival rate of customers is \(a\) ,and each customer
has an i.i.d. reservation price with tail probability
\(\bar { F } ( p )\) then the expected demand rate at price \(p\) is
\(a { \bar { F } } ( p )\) . In the case of product differentiation, the
demand function is unique to the firm and is assumed to be
non-increasing in \(p\) due to both lost sales to competitors and
substitution effects.In this case,for example,the demand rate seen by
the firm as a function of its price and those of its competitors may be
modeled using a multinomial logit (cf.Anderson et al.fora fairly
extensive treatment of discrete choice theory of product
differentiation).Here, we assume that \(\lambda ( p )\) is given and do
not explicitly model the competitive forces that give rise to this
demand function. (See Eliashberg and Steinberg 1991 and Dockner and
Jorgensen 1988 for examples of dynamic pricing models that represent
competition explicitly.)

The assumption that consumers respond only to the current price is,of
course,somewhat restrictive.In par ticular,it does not account for the
fact that consumers may act strategically,adjusting their buying
behavior in response to the firm's pricing strategy. To do so would
require a game theoretic formulation,which is beyond the scope of our
analysis.The current-price assumption is approximately true when
``impulse purchases'' are common (e.g.,fashion items).Further,the fact
that near-optimal strategies use very stable prices makes this
assumption reasonable in other applications as well. (See Lazear
1986,p.28 for further discussion of the importance of strategic
behavior.)

Realized demand is stochasticand modeled asaPoissonprocesswith intensity
\(\lambda ( p )\) .Thus,if the firmprices at \(p\) over an interval
\(\delta\) ,it sells one itemwith probability
\(\lambda ( p ) \delta + o ( \delta )\) ,no itemswith probability
\(1 - \lambda ( p ) \delta - o ( \delta )\) and more than one item with
probability \(o \left( \delta \right)\) . In \(\ S 5 . 2\) westudy the
casewhere the demand rate can also depend on the time.We initially
consider the case where no backlogging of demand is allowed, so once the
firm runs out of stock it collects no further revenues.

Several mild assumptions concerning the demand function are
imposed:First,we assume there is a oneto-one correspondence between
prices and demand rates so that \(\lambda ( p )\) has an inverse,denoted
\(p ( \lambda )\) .One can thereforealternativelyview the intensity
\(\lambda\) as the decision variable;the firms determines a target sales
intensity \(\lambda\) (i.e.,an output quantity)and the market
determines the price \(p ( \lambda )\) based on this quantity.From
ananalytical perspective,the intensity is more convenient towork with.

We assume the revenue rate,

\[
r ( \lambda ) \doteq \lambda p ( \lambda ) ,
\]

satisfies
\(\operatorname * { l i m } _ { \lambda  0 } r ( \lambda ) = 0 ,\) is
continuous,bounded and concave,and has a bounded least maximizer defined
by
\(\lambda ^ { * } \ = \ \operatorname* { m i n } \ \left\{ \lambda \colon \ r ( \lambda ) \ = \ \operatorname { m } a \mathsf { x } _ { \lambda \geq 0 } \ r ( \lambda ) \ \right\}\)
. Continuity, boundedness of the revenue rate and the maximizer
\(\lambda ^ { * }\) and the condition
\(\begin{array} { r } { \operatorname* { l i m } _ { \lambda \to 0 } r ( \lambda ) = 0 } \end{array}\)
are all reasonable requirements.Concavityof \(r ( \lambda )\) stems from
thestandard economic assumption that marginal revenue is decreasing in
output.

Cohen and Karlin and Carr consider demand functions with similar
conditions. Specifically, the condition
\(\begin{array} { r } { \operatorname* { l i m } _ { \lambda \to 0 } r ( \lambda ) = 0 } \end{array}\)
implies the existence of what Karlin and Carr {[}22{]} term a null price
\(p _ { \infty }\) (possibly \(+ \infty\) )for which
\(\begin{array} { r } { \operatorname* { l i m } _ { p \to p _ { \infty } } \lambda ( p ) = 0 } \end{array}\)
and
\(\begin{array} { r } { \operatorname* { l i m } _ { p \to p _ { \infty } } p \lambda ( p ) = 0 } \end{array}\)
.(Cohen requires the existence of a null price as well, though he does
not give it this term.) In our case, the null price allows us to model
the out-of-stock condition as an implicit constraint that forces the
firm to price at \$p = p \_ \{ \} \$ when inventory is zero.Note that
this modelingartifact partially blurs the distinction between demand and
sales,since in reality we can certainly have demand for items without a
corresponding sale when the firm is out of stock.However, in the context
of the model, no gen eralityis lost by making this assumption.

We call a function \(\lambda ( p )\) that satisfies all of the
assumptions above a regular demand function.An example of a regular
demand function is the exponential class
\(\lambda ( p ) = a e ^ { - p }\) . One can verify that this function is
decreasing in \(p\) , has a unique inverse
\(p ( \lambda ) = \log { ( a / \lambda ) }\) and results in a concave
revenue rate \(r ( \lambda ) = \lambda \log  ( a /\) \(\lambda\) ) with
unique maximizer \(\lambda ^ { * } = a e ^ { - 1 }\) . The null price in
this case is \(p _ { \infty } = + \infty\) . Linear demand functions are
also regular.

\section{2.2. Formulation}\label{formulation}

The pricing problem is formulated as follows: At time zero,the firm has
a stock \(n\) (a nonnegative integer) of items and a finite time
\(t > 0\) to sell them.The firm controls the intensity of the Poisson
demand \(\lambda _ { s } = \lambda ( p _ { s } )\) at time s using a
non-anticipating pricing policy \(p _ { s }\) . The intensity
\(\lambda ( \cdot )\) is assumed to be a regular demand function. Let
\(N _ { s }\) denote the number of items sold up to time s.A demand is
realized at time s if \(d N _ { s } = 1\) ,in which case the firm sells
one item and receives revenue of \(p _ { s }\)

The price \(p _ { s }\) must be chosen from the set of allowable price
\(\mathcal P = \mathfrak R ^ { + } \cup \{ p _ { \infty } \}\) . The set
of allowable rates is denoted
\(\Lambda = \{ \lambda ( p ) : p \in \mathcal { P } \}\) . Note that
since \(p _ { \infty } \in \mathcal { P }\) we always have
\(0 \in \Lambda\) .We consider other sets of allowable prices
\(\mathcal { P }\) in \(\ S 5\) . We denote by \(\mathcal { U }\) the
class of all non-anticipating pricing policies which satisfy

\[
\int _ { 0 } ^ { t } d N _ { s } \leq n \quad ( \mathsf { a . s }
\]

and

\[
p _ { s } \in \mathcal { P } \Leftrightarrow \lambda _ { s } \in \Lambda \quad \forall s .
\]

Constraint (2) is the modeling artifact mentioned above. It acts to
``turn off'' the demand process when the firm runs out of items to sell.
The existence of the null price \(p _ { \infty }\) in the set
\(\mathcal { P }\) guarantees that it can always be satisfied.

Without loss of generality,we assume the salvage value of any unsold
items at time \(t\) is zero,since for any positive salvage value \(q\)
we can always define a new regular demand function
\(\lambda ( p )  \lambda ( p - q )\) and a new price
\(p \gets p \gets q\) (the excess over salvage value) that transforms
the problem into the zero-salvage-value case.We also assume all costs
related to the purchase and production of the product are sunk.

Given a pricing policy \(u \in \mathcal { U }\) , an initial stock
\(n > 0\) and a sales horizon \(t > 0\) ,wedenote the expected revenue
by

\[
J _ { u } ( n , t ) \doteq E _ { u } \biggl [ \int _ { 0 } ^ { t } p _ { s } d N _ { s } \biggr ] ,
\]

where

\[
J _ { u } ( n , 0 ) \doteq 0 \quad \forall n
\]

and

\[
J _ { u } ( 0 , t ) \doteq 0 \quad \forall t .
\]

The firm's problem is to find a pricing policy \(u ^ { * }\) (if one
exists) that maximizes the total expected revenue generated over
\([ 0 , t ]\) ,denoted \(J ^ { * } ( n , t )\) . Equivalently,

\[
J ^ { * } ( n , t ) \doteq \operatorname* { s u p } _ { u \in \mathcal { U } } J _ { u } ( n , t ) .
\]

2.2.1. Optimality Conditions and Structural Results. One can informally
derive the Hamilton-Jacobi sufficient conditions for \(J ^ { * }\) by
considering what happens over a small interval of time \(\delta t\) .
Since by selecting the intensity \(\lambda\) (i.e., pricing at
\(p ( \lambda ) _ { . } ^ { \prime }\) )we sell one item over the next
\(\delta t\) with probability approximately \(\lambda \delta t\) and no
items with probability approximately \(1 - \lambda \delta t\) ,by the
Principle of Optimality,

\[
\begin{array} { r } { J ^ { * } ( n , t ) = \underset { \lambda } { \operatorname* { s u p } } \left[ \lambda \delta t ( p ( \lambda ) + J ^ { * } ( n - 1 , t - \delta t ) ) \right. } \\ { \left. + ( 1 - \lambda \delta t ) J ^ { * } ( n , t - \delta t ) + o ( \delta t ) \right] . } \end{array}
\]

Using \(r ( \lambda ) \doteq \lambda p ( \lambda )\) , rearranging and
taking the limit as \(\delta t  0\) ,weobtain

\[
\frac { \partial J ^ { * } ( n , t ) } { \partial t } = \operatorname* { s u p } _ { \lambda } \left[ r ( \lambda ) - \lambda ( J ^ { * } ( n , t ) - J ^ { * } ( n - 1 , t ) ) \right]
\]

with boundary conditions \(J ^ { * } ( n , 0 ) = 0 , \forall n\) and
\(J ^ { * } ( 0 , t )\) \(\mathbf { \varepsilon } = 0\) , Vt.The above
argument is not rigorous because we have not justified interchanging
\(\mathsf { s u p } _ { \lambda }\) and
\(\scriptstyle \operatorname* { l i m } _ { \delta t \to 0 } ;\) (204号
however, these conditions can be justified formally using Theorem II.1
in Bremaud,where general intensity control problems are studied. Thus,a
solution to equation (8) is indeed the optimal revenue
\(J ^ { * } ( n , t )\) and the intensities
\(\lambda ^ { * } ( n , t )\) that achieve the supremum form an optimal
intensity control. Equivalent conditions were derived in Kinkaid and
Darling(1963),Miller(1986), and Stadje(1990)without using the theory of
intensity control.

The existence of a unique solution to equation (8)is resolved by the
following proposition,which is proved in the appendix:

PROPOSITION 1.If \(\lambda ( \boldsymbol { p } )\) is a regular demand
function, then there exists a unique solution to equation(8).Further,
the optimal intensities satisfies
\(\lambda ^ { * } ( \mathfrak { n } , s ) \leq \lambda ^ { * }\) for all
n and forall \(0 \leq s \leq t\) ·

Although Proposition 1 guarantees the existence of a unique solution to
equation(8),obtaining it in closed formis quite difficult--if not
impossible---for arbitrary regular demand functions.However,we can make
a number of qualitative statements about the optimal expected
revenue,intensities and prices.We summarize these in the following
theorem.

THEOREM 1. \(J ^ { * } ( n , t )\) is strictly increasing and strictly
concave in both n and t.Furthermore,there exists an optimal intensity
\(\lambda ^ { * } ( n , t )\) (resp.,price \(p ^ { * } ( n , t )\)
)thatis strictly increasing(resp.,decreasing) in n and strictly
decreasing (resp.,increasing)in \(t\) ·

This theorem shows that more stock and/or time leads to higher expected
revenues.Further,at a given point in time, the optimal price drops as
the inventory increases; conversely, fora given level of inventory, the
optimal price rises if we have more time to sell. These properties are
not only intuitively satisfying,but they are also useful if one wants to
compute the optimal policy numerically because they significantly reduce
the set of policies over which one needs to optimize.A proof of a
slightly weaker version of Theorem 1 is implied by a sequence of results
in Kincaid and Darling (1963). A compact proof of Theorem 1 is presented
in the appendix.

\section{\texorpdfstring{2.3.An Optimal Solution for
\(\lambda ( p ) = a e ^ { - \alpha p }\)
(20}{2.3.An Optimal Solution for \textbackslash lambda ( p ) = a e \^{} \{ - \textbackslash alpha p \} (20}}\label{an-optimal-solution-for-lambda-p-a-e---alpha-p-20}

We can find an exact solution for the demand function
\(\lambda ( p ) = a e ^ { - \alpha p }\) ,where \(a > 0\) ,
\(\alpha > 0\) are arbitrary parameters.The solution is useful if one
can adequately fit demand to this particular function.More importantly,
however, the solution provides interesting insights into the behavior of
the optimal policy.

First note that without loss of generality we can take \(\alpha = 1\) by
simply changing units of price to \(p ^ { \prime } \gets \alpha p\) The
maximizer of \(r ( \lambda )\) in the case \(\alpha = 1\) is
\(\lambda ^ { * } = a / e\) and
\(p ^ { * } = p ( \lambda ^ { * } ) = 1\) . It is not hard to verify
(see also {[}25{]} and{[}44{]}) that the solution to equation (8) in
this case is

\[
J ^ { * } ( n , t ) = \log \biggr ( \sum _ { i = 0 } ^ { n } ( \lambda ^ { * } t ) ^ { i } \frac { 1 } { i ! } \biggr ) ,
\]

and the optimal price \(p ^ { * } ( n , t )\) is given by

\[
p ^ { * } ( n , t ) = J ^ { * } ( n , t ) - J ^ { * } ( n - 1 , t ) + 1 .
\]

Some sample paths of this optimal price \(p ^ { * } ( n , t )\) are
shown in Figure1 fora problem with 25 items and a unit time horizon.(The
line marked FP is explained below in \(\ S 3 . )\) The top graph shows a
sample price path when demand is low relative to the initial stock (
\(\mathit { i a } \ = \ 4 0\) ,
\(\lambda ^ { * } t = a e ^ { - 1 } \approx 9 . 5 ;\) ),while the
bottom graph showsaprice path for the same 25 itemswhen demand is high
relative to the initial stock ( \(\mathit { a } = 1 0 0\) 1
\(\lambda ^ { * } t = a e ^ { - 1 }\) \(\approx 3 6 . 8 \dot { }\)
).There are several interesting things to note about these
graphs.First,the upward jumps in price correspond to sales ( {\$d N \_
\{ s \} = 1 \textbackslash AA \_ \{ . \}\$} ).Aftera sale,theprice
decays until another sale is made,at which point the price takes another
jump.This behavior follows from Theorem 1. The upward jumps are due to
the fact that the firmpriceshigherif it has fewer itemstosell over a
given interval t. The decaying price between sales can be thought of as
a price promotion and follows from the fact \(p ^ { * } ( n , t )\) is
decreasing in \(t\) for fixed \(n\) . The firm gradually reduces the
price as time runs out in order to induce buying activity.

\pandocbounded{\includegraphics[keepaspectratio]{images/40db1ae5bf16ca70076470943deac4f35a13be9ba7f129bb4bcc79bb225ad478.jpg}}\\
Figure 1 Sample Paths of \(p ^ { \star } ( \eta - N _ { s } , 1 - s )\)
over \(s \in [ \mathbf { 0 } ,\) 1{]}: Top: n \(= 2 5\) \(a = 2 0\)
Bottom: \(n = 2 5\) \(a = 1 0 0\)

\section{3. Bounds and Heuristics}\label{bounds-and-heuristics}

For regular demand functions other than
\(\lambda ( p ) = a e ^ { - \alpha p }\) it is quite difficult,if not
impossible, to find closed form solutions to equation(8).Further,by
Theorem 1,the optimal price varies continuously over time. Yet in many
applications this degree of price flexibility is either impossible or
prohibitively expensive to implement. Therefore,one might often prefer
more stable policies that are close to optimal over a ``jittery''optimal
policy. In this section,we propose heuristics that meet these
criteria.They are easy to implement and also provably close to optimal
in many cases.Our approach is to first construct an upper bound based on
a deterministic version of the problem.The solution to this
deterministic problem then suggests a simple fixed-price heuristic that
we show is provably good when the volume of expected sales is large.

\section{3.1. An Upper Bound Based on a Deterministic
Problem}\label{an-upper-bound-based-on-a-deterministic-problem}

3.1.1.Formulation of Deterministic Problem. Consider the following
deterministic version of the problem:At time zero,the firm has a stock
\(x\) ,acontinuous quantity,of product and a finite time \(t > 0\) to
sell it.The instantaneous demand rate is deterministic and a function of
the price at time s, \(p ( s )\) , again denoted
\(\lambda ( p ( s ) )\) . (Our notation distinguishing a deterministic
policy follows that in Bremaud(1980).)We assume \(\lambda ( \cdot )\)
isaregulardemandfunction.Asbefore,without loss of generality,we assume
the salvage value of the product at time \(t\) is zero and that all
other costs are sunk.The price \(p ( s )\) must again be chosen from a
set \(\mathcal { P }\) of allowable prices.As before,we can equivalently
viewthe firmas setting the rate \(\lambda ( s ) \in \Lambda\) which
implies charging a price
\(p ( s ) = p ( \lambda ( s ) ) \in \mathcal { P }\) :

The firm's problem is to maximize the total revenue generated over
\([ 0 , t ]\) given \(x\) ,denoted \(J ^ { D } ( x , t )\) :

\[
J ^ { D } ( x , t ) = \operatorname* { m a x } _ { \{ \lambda ( s ) \} } \int _ { 0 } ^ { t } r ( \lambda ( s ) ) d s
\]

subject to

\[
\int _ { 0 } ^ { t } \lambda ( s ) d s \leq x
\]

\[
\lambda ( s ) \in \Lambda .
\]

\section{3.2. Optimal Solution of the Deterministic
Problem}\label{optimal-solution-of-the-deterministic-problem}

We begin with some definitions.Define the run-out rate, denoted
\(\lambda ^ { 0 }\) ,by \(\lambda ^ { 0 } \dot { = } x / t\) , the
run-out price,denoted \(p ^ { 0 }\) 号 by
\(p ^ { 0 } \doteq p ( \lambda _ { 0 } )\) , and the run-out-revenue
rate \(r ^ { 0 } \dot { = } p ^ { 0 } \lambda ^ { 0 }\) Notice that
\(\lambda ^ { 0 }\) (resp., \(p ^ { 0 }\) ) is the fixed intensity
(resp., price)at which the firm sells exactly its initial stock \(x\)
over the interval \([ 0 , t ]\) . Recall that \(\lambda ^ { * }\) is the
least maximizer of the revenue function
\(r ( \lambda ) = \lambda p ( \lambda )\) . We find it convenient to
define \(p ^ { * } \doteq p ( \lambda ^ { * } )\) ,and
\(r ^ { \ast } \doteq p ^ { \ast } \lambda ^ { \ast }\) . The quantity
\(r ^ { * }\) is the maximum instantaneous revenue rate. These
definitions allow us to state the following proposition,which is proved
in the appendix:

PROPOsiTiON2. The optimal solution to the deterministic problem(11)is
\(\lambda ( s ) = \lambda ^ { D } \doteq \operatorname* { m i n } \left\{ \lambda ^ { * } , \lambda ^ { 0 } \right\} , 0 \le\)
\(\leq t\) . In terms of price the optimal policy is
\(p ( s ) = p ^ { D }\)
\(\doteq \operatorname* { m a x } \left\{ p ^ { * } , p ^ { 0 } \right\}\)
1 \(0 \leq s \leq t\) . Finally,the optimal revenue is

\[
J ^ { D } ( x , t ) = t \operatorname* { m i n } \left\{ r ^ { * } , r ^ { 0 } \right\}
\]

The intuition for this solution is the following: If the firm has a
large number of items to sell \(( x \geq \lambda ^ { * } t\) ,it
ignores the problem of running out of stock and prices at the level that
maximizes the revenue rate.In this case the firm ends with
\(x - \lambda ^ { * } t\) unsold units.If the items are scarce
\(( x < \lambda ^ { * } t )\) , the firm can afford to price higher,and
it indeed prices at the highest level that still enables it to sell all
the items.Note thatin both cases the solution is to seta fixed price for
the entire interval.

3.2.1.The Deterministic Revenue as an Upper Bound. Intuitively, one
would expect that the uncertainty in sales in the stochastic problem
results in lower expected revenues.The following theorem formalizes this
idea:

THEOREM 2.If \(\lambda ( p )\) is a regular demand function, then for
all \(0 \leq n < + \infty\) and \(0 \leq t < + \infty\) 、

\[
J ^ { * } ( n , t ) \leq J ^ { D } ( n , t ) .
\]

PROOF OF THEOREM 2. As shown in Proposition 1,
\(\lambda _ { s } \le \lambda ^ { * } < \infty\) ,which implies
\(\int _ { 0 } ^ { t } \lambda _ { s } d s < \infty\) almost surely for
all \(t \geq 0\) . Recall \(\mathcal { U }\) denotes the class of
policies that satisfy
\(\begin{array} { r } { \int _ { 0 } ^ { t } d N _ { s } \le n } \end{array}\)
(a.s.);therefore by Bremaud 1980, Theorem II,

\[
E _ { u } \bigg [ \int _ { 0 } ^ { t } d N _ { s } \bigg ] = E _ { u } \bigg [ \int _ { 0 } ^ { t } \lambda _ { s } d s \bigg ] \leq n .
\]

Since the demand intensity in the control problem(7) is Markovian, it is
sufficient to consider only Markovian policies \(u\) (Bremaud
1980,Corollary II.2).That is, policies for which the price at time \(s\)
is a function \(p _ { s }\)
\(\mathbf { \Sigma } = p _ { u } ( n - N _ { s } , s )\)
only.(Equivalently, the intensity at time \(s\) is a function
\(\lambda _ { s } = \lambda ( n - N _ { s } , s )\) )By Bremaud 1980,
Theorem \({ \mathrm { I I } } ,\) we can write,

\[
\begin{array} { l } { \displaystyle { J _ { u } ( n , t ) = E _ { u } \bigg [ \int _ { 0 } ^ { t } p _ { u } ( n - N _ { s } , s ) d N _ { s } \bigg ] } } \\ { \displaystyle { \phantom { \sum _ { u } } } } \\ { \displaystyle { \phantom { \sum _ { u } } } } \end{array}
\]

and

\[
J ^ { * } ( n , t ) = \operatorname* { s u p } _ { u \in \mathcal { U } } J _ { u } ( n , t ) .
\]

Now for \(\mu \geq 0\) we define the augmented cost functional

\[
\begin{array} { c } { { J _ { u } ( n , t , \mu ) = E _ { u } \Bigg [ \int _ { 0 } ^ { t } \left( r ( \lambda _ { s } ) - \mu \lambda _ { s } \right) d s \Bigg ] } } \\ { { + n \mu \geq J _ { u } ( n , t ) , } } \end{array}
\]

and the augmented deterministic cost function

\[
J ^ { D } ( n , t , \mu ) = \operatorname* { m a x } _ { \lambda ( s ) \in \Lambda } \int _ { 0 } ^ { t } { ( r \lambda ( s ) - \mu \lambda ( s ) ) d s } + n \mu .
\]

We claim the following:

LEMMA 1.

\[
J _ { u } ( n , t , \mu ) \leq J ^ { D } ( n , t , \mu ) \quad \forall u \in \mathcal { U } , \mu \geq 0 .
\]

PROOF. This follows by viewing the integrand inside the expectation in
equation (15) as purely a function of \(\lambda\) and maximizing
pointwise:

\[
\begin{array} { l } { \displaystyle { J _ { u } ( n , t , \mu ) \leq \int _ { 0 } ^ { t } \operatorname* { m a x } _ { \lambda ( s ) \in \Lambda } \left\{ r ( \lambda ( s ) ) - \mu \lambda ( s ) \right\} d s + n \mu } } \\ { \displaystyle { = \operatorname* { m a x } _ { \left\{ \lambda ( s ) \in \Lambda \right\} } \int _ { 0 } ^ { t } ( r ( \lambda ( s ) ) - \mu \lambda ( s ) ) d s + n \mu } } \\ { \displaystyle { \ \leq J ^ { D } ( n , t , \mu ) . } } \end{array}
\]

MANAGEMENT SCIENCE/Vol.40,No.8,August 1994

Since Lemma 1 holds for all \(u \in \mathcal { U }\) and \(\mu \geq 0\)
,we have by equation (15)

\[
J ^ { * } ( n , t ) \leq \operatorname * { i n f } _ { \mu \geq 0 } J ^ { D } ( n , t , \mu ) .
\]

Theorem 2 then follows by noting that the quantity on the right above is
the optimal dual value of the infinite dimensional program

\[
J ^ { D } ( n , t ) = \operatorname* { m a x } _ { \lambda ( s ) \in \Lambda } \int _ { 0 } ^ { t } r ( \lambda ( s ) ) d s
\]

subject to

\[
\int _ { 0 } ^ { t } \lambda ( s ) d s \leq n .
\]

Since this is a convex program, and the null price together with the
fact that \(n > 0\) implies that \(\lambda ( s ) = 0 .\) 0
\(\le s \le t\) is a strictly interior solution,there exists a
multiplier \(\mu ^ { * }\) for which the duality gap is zero and
\(J ^ { D } ( n , t )\)
\(\ = \ \operatorname { i n f } _ { \mu \geq 0 } \ J ^ { D } ( n , \ t , \ \mu ) \ = \ J ^ { D } ( n , \ t , \ \mu ^ { * } )\)
. (See Luenberger 1969).□

Theorem 2 is useful for several reasons.It suggests that the solution of
the deterministic problem may provide insight into optimal or
near-optimal pricing strategies for the stochastic problem. It also
provides performance guarantees on the cost of such pricing
strategies.Together, these results can be used to establish a
strongrelationship between the stochastic and deterministic problems,as
we show next.

\section{3.3. Asymptotically Optimal Fixed-price
Heuristics}\label{asymptotically-optimal-fixed-price-heuristics}

The deterministic optimal solution suggests a simple fixed
price(FP)heuristic,namely,for the entire horizon set the price to
\(\boldsymbol { p } ^ { D } = \mathbf { m } \mathbf { a } \mathbf { x } \left\{ \boldsymbol { p } ^ { 0 } , \boldsymbol { p } ^ { * } \right\}\)
.Of course, one could improve on this heuristic bychoosing the best
fixed price; that is, the one maximizing
\(p E [ \operatorname* { m i n } \left\{ n , N _ { \lambda ( p ) t } \right\} ] ,\)
where \(N _ { \alpha }\) denotes a Poisson random variable with mean
\(\alpha\) .In general the best fixed price cannot be found
analytically; however,itis easy to find numerically.We let OFP denote
this optimal fixed-price heuristic and
\(J ^ { \mathrm { O F P } } ( n , \ t )\) (resp.,
\(J ^ { \mathrm { F P } } ( n , t ) )\) denote the revenue of the OFP
(resp., FP) heuristic. We will use the fact that
\(J ^ { \mathrm { O F P } } ( n , \ t )\)
\(\geq J ^ { \mathrm { F P } } ( n , t )\) :

Note that using a fixed price for the entire horizon is quite convenient
since there is no effort involved in monitoring time and inventory
levels and no cost incurred for changing prices.Further, the performance
of the heuristics turn out to be quite good in several cases, one of
which is shown by the following theorem:

THEOREM 3.

\[
\frac { J ^ { \mathrm { O F P } } ( n , t ) } { J ^ { * } ( n , t ) } \geq \frac { J ^ { \mathrm { F P } } ( n , t ) } { J ^ { * } ( n , t ) } \geq 1 - \frac { 1 } { 2 { \sqrt { \operatorname* { m i n } \left\{ n , \lambda ^ { * } t \right\} } } } .
\]

PROOF. The first inequality follows from the definition of the
heuristics.To show the second,note the expected revenue obtained when
the price is fixed at \(p\) is

\[
p E [ N _ { \lambda ( p ) t } - ( N _ { \lambda ( p ) t } - n ) ^ { + } ] .
\]

Gallego (1992) shows that for any random variable \(N\) with finite
mean \(\mu\) and finite standard deviation \(\sigma ,\) and for any real
number \(n\) 、

\[
E [ ( N - n ) ^ { + } ] \leq \frac { \sqrt { \sigma ^ { 2 } + ( n - \mu ) ^ { 2 } } - ( n - \mu ) } { 2 } ,
\]

where \({ { x } ^ { + } } \doteq \operatorname* { m a x } ( x , 0 )\)
.Consider first the case \(\lambda ^ { * } t > n\) That is,the case
where items are scarce and the FP heuristic uses the run-out price
\(p ^ { 0 }\) . Using the above inequality in equation (17) and noting
that when pricing at \(p ^ { 0 } , \mu = \sigma ^ { 2 } = n\) ,we
obtain

\[
J ^ { \mathrm { F P } } ( n , t ) \geq n p ^ { \circ } \Bigg ( 1 - \frac { 1 } { 2 \sqrt { n } } \Bigg ) = r ^ { 0 } t \Bigg ( 1 - \frac { 1 } { 2 \sqrt { n } } \Bigg ) .
\]

In the case where \(\lambda ^ { * } t \leq n\) we price at \(p ^ { * }\)
,by the same reasoning we obtain

\[
\begin{array} { l } { { \displaystyle { J ^ { \mathrm { F P } } ( n , t ) \geq p ^ { * } \bigg ( \lambda ^ { * } t - \frac { \sqrt { \lambda ^ { * } t + ( n - \lambda ^ { * } t ) ^ { 2 } } - ( n - \lambda ^ { * } t ) } { 2 } \bigg ) } } } \\ { { \displaystyle { \ \geq p ^ { * } \lambda ^ { * } t \bigg ( 1 - \frac { 1 } { 2 \sqrt { \lambda ^ { * } t } } \bigg ) = r ^ { * } t \bigg ( 1 - \frac { 1 } { 2 \sqrt { \lambda ^ { * } t } } \bigg ) . \qquad ( 1 9 ) } } } \end{array}
\]

Comparing these two cases to the deterministic revenue (12)and using
Theorem 2 completes the proof.□

REMARK.When \(\lambda ^ { * } t > n\) , one can determine the exact cost
of the FP heuristic by noting that
\(E ( N _ { n } - n ) ^ { + } = n ( 1\)
\(- \left. P \left\{ N _ { n } = n \right\} \right.\) ),which implies

\[
J ^ { \mathrm { F P } } ( n , t ) = n p ^ { 0 } \bigg ( 1 - \frac { n ^ { n } } { n ! } e ^ { - n } \bigg ) .
\]

This provides a slightly better guarantee for small \(n\) though it has
an identical rate of convergence since by Stirling's formula
\(( n ^ { n } / n ! ) e ^ { - n } \sim 1 / \sqrt { 2 \pi n }\) :

Theorem 3 shows that the FP heuristic,and consequently the OFP
heuristic,are asymptotically optimal in two limiting cases: (1) the
number of items is large (204号 \(( n \gg 1 )\) and there is plentyof
time to sell them \(( n < \lambda ^ { * } t )\) 0 or(2) there is the
potential for a large number of sales at the revenue maximizing price
\(( \lambda ^ { * } t \gg 1 )\) ,and there are enough items in stock to
satisfy this potential demand
\(( n \geq \lambda ^ { * } t ^ { \prime }\) ).Thus,we seethatif
thevolume of expected sales is large, the heuristics perform quite well.

One can gain an intuitive understanding of this result by examining
Figure 1,which shows the FP price and the optimal price for two sample
paths of the example in \(\ S 2 . 3\) . Note the that optimal price
paths in this figure appear roughly centered about the FP price shown by
the horizontal lines in Figure1.Also,onarelative basis the variations
about the FP price appear small. Thus, it seems the FP price is a
reasonable approximation to the optimal policy.

An example serves'to illustrate the utility of the bounds in Theorem 3:
Consider a firm that has 400 items and enough time to price at the
run-out price \(p ^ { 0 } ( \lambda ^ { * } t > n )\) .Theorem 3 then
guarantees that the expected revenue collected by simply offering a
fixed price of \(p ^ { 0 }\) is at least \(9 7 . 5 \%\) of what could be
obtained by using an optimal state-dependent strategy. For 100 items,
the guarantee drops to \(9 5 \%\) ,while for 25 items, it is only
\(9 0 \%\) . However,as we illustrate in the next subsection, these
guarantees are in fact quite pessimistic, and the actual performance of
fixed-price policies is good even for small ( \(_ { \approx 1 0 }\)
items)problems.

As a last example where fixed-price heuristics are asymptotically
optimal, we state without proof

THEOREM 4.

\[
\operatorname* { l i m } _ { t \to 0 } \frac { J ^ { \mathrm { O F P } } ( n , t ) } { J ^ { * } ( n , t ) } \geq \operatorname* { l i m } _ { t \to 0 } \frac { J ^ { \mathrm { F P } } ( n , t ) } { J ^ { * } ( n , t ) } = 1 \quad \forall n > 0 .
\]

\section{3.4. Numerical Example of the Performance of Fixed-Price
Heuristics}\label{numerical-example-of-the-performance-of-fixed-price-heuristics}

For the case where the demand function is \(a e ^ { - \alpha p }\) we
have a closed form expression for the optimal cost, which allows us to
examine the performance of the fixed-price heuristics for problems of
moderate size.Table 1 shows the prices and resulting revenues for a
series of problems with a unit horizon, \(\lambda ^ { * } t = 1 0\) and
starting inventories n ranging from 1 to 20.Note that the optimal fixed
price \(( \boldsymbol { p } ^ { \mathrm { o F P } } )\) is initially
lower than the deterministic price
\(( \boldsymbol { p } ^ { \mathtt { F P } } )\) when there are few items
to sell,but for \(n\) \(> 5\) it is higher.Thus, the OFP price seems to
smooth the transition between the low and high demand price extremes,
\(p ^ { * }\) and \(p ^ { 0 }\) . Note also that the worst relative
performance of the OFP heuristic is only \(5 . 5 \%\) ,and when
\(n > 1 2\) it is within \(1 \%\) of the optimal revenue. Indeed, in
numerical experiments on many different examples we never once observed
a value of \(J ^ { \mathrm { O F P } }\) that was more than seven
percent less than the optimal revenue.The relative performance of the FP
heuristic,on the other hand, is poorest at \(n = 1\) ( \(1 2 . 9 \%\)
below the optimal revenue), though for \(n > 1 5\) its revenue is
comparable to that of the OFP heuristic.

Table 1 Pricesand Revenues for the Case \(\lambda ^ { \star } t = 1 0\)

\begin{longtable}[]{@{}|l|l|l|l|l|l|@{}}
\toprule\noalign{}
\endhead
\bottomrule\noalign{}
\endlastfoot
\hline
n & DOFp & pFP & J* & JOFP/J* & JFp/J* \\
\hline
& & & & & \\
\hline
1 & 2.74 & 3.30 & 2.40 & 0.945 & 0.871 \\
\hline
2 & 2.36 & 2.61 & 4.11 & 0.947 & 0.926 \\
\hline
3 & 2.10 & 2.20 & 5.43 & 0.950 & 0.945 \\
\hline
4 & 1.90 & 1.92 & 6.47 & 0.954 & 0.954 \\
\hline
5 & 1.74 & 1.69 & 7.30 & 0.958 & 0.956 \\
\hline
6 & 1.61 & 1.51 & 7.96 & 0.962 & 0.956 \\
\hline
7 & 1.50 & 1.35 & 8.49 & 0.967 & 0.952 \\
\hline
8 & 1.41 & 1.22 & 8.89 & 0.971 & 0.946 \\
\hline
9 & 1.33 & 1.11 & 9.22 & 0.976 & 0.937 \\
\hline
10 & 1.26 & 1.00 & 9.46 & 0.980 & 0.925 \\
\hline
11 & 1.21 & 1.00 & 9.64 & 0.985 & 0.951 \\
\hline
12 & 1.16 & 1.00 & 9.77 & 0.989 & 0.970 \\
\hline
13 & 1.12 & 1.00 & 9.85 & 0.992 & 0.982 \\
\hline
14 & 1.08 & 1.00 & 9.91 & 0.995 & 0.990 \\
\hline
15 & 1.05 & 1.00 & 9.95 & 0.997 & 0.995 \\
\hline
16 & 1.04 & 1.00 & 9.97 & 0.998 & 0.997 \\
\hline
17 & 1.02 & 1.00 & 9.99 & 0.999 & 0.999 \\
\hline
18 & 1.01 & 1.00 & 9.99 & 0.999 & 0.999 \\
\hline
19 & 1.01 & 1.00 & 10.00 & 1.000 & 1.000 \\
\hline
20 & 1.00 & 1.00 & 10.00 & 1.000 & 1.000 \\
\hline
\end{longtable}

These results suggest that even for moderate sized problems the FP
heuristic,and especially the OFP heuristic,perform quite well. They also
suggest that dynamic pricing in response to the sort of statistical
variations in demand modeled here can at best provide only minimal
increases in revenue-on the order of one percent or less for moderate to
large problems.For this reason,we conclude that if demand functions are
well known and prices can be set freely,then one should not see great
benefits from the highly dynamic pricing practices,such as those found
in fashion retailing and yield management practice. Other explanations
of the benefits of these practices, one of which we propose in S4,are
needed. (An explanation based on the producer's imperfect knowledge of
customers' reservation prices is proposed by Lazear.)

\section{3.5. Some Structural Observations for the
Case}\label{some-structural-observations-for-the-case}

We next show that for \(\lambda ( p ) = a e ^ { - p }\) the optimal
intensity (resp.,price) for the stochastic problem is always smaller
(resp., larger) than the corresponding optimal intensity (resp.,price)
for the deterministic problem.

PROPOSITION3.If \(\lambda ( p ) = a e ^ { - p }\) ,then
\(\forall n \geq 0 , \forall t \geq 0 ,\) (2

\[
\lambda ^ { * } ( n , t ) \leq \lambda ^ { D } ( n , t )
\]

and

\[
p ^ { * } ( n , t ) \geq p ^ { D } ( n , t ) .
\]

PROOF. We can write the optimal intensity as

\[
\lambda ^ { * } ( n , t ) = \lambda ^ { * } \frac { P \left\{ N _ { \lambda ^ { * } t } \leq n - 1 \right\} } { P \left\{ N _ { \lambda ^ { * } t } \leq n \right\} } \leq \lambda ^ { * } .
\]

Since \(\lambda ^ { D } ( n , \textit { t } ) = n / t\) for
\(t \geq n / \lambda ^ { * }\) and
\({ \lambda } ^ { D } ( n ,  t ) = { \lambda } ^ { * }\) otherwise,and
by Proposition 1 \(, \lambda ^ { * } ( n , t ) \leq \lambda ^ { * }\)
always, we only need to show that
\(\lambda ^ { * } ( n , t ) \leq n / t .\) for \(t \geq n /\)
\(\lambda ^ { * }\) .Equivalently,

\[
\begin{array} { r } { \lambda ^ { * } t P \left\{ N _ { \lambda ^ { * } t } \leq n - 1 \right\} \leq n P \left\{ N _ { \lambda ^ { * } t } \leq n \right\} . } \end{array}
\]

But this holds since the left-hand side can be written
\(\textstyle \sum _ { i = 0 } ^ { n } i e ^ { - \lambda ^ { * } t } ( \lambda ^ { * } t ) ^ { i } / i !\)
,which is clearly less than \(n P \{ N _ { \lambda ^ { * } t }\)
\(\leq n \}\) . The corresponding properties for \(p ^ { * } ( n , t )\)
follow in a similar way.

This proposition helps address a question raised by Mills(1959)about
the relation between the optimal price for a stochastic model and its
deterministic counterpart. Karlin and Carr(1962) and Thowsen(1975)
analyzed this question for fixed price models underadditive or
multiplicative uncertainty. They showed the optimal stochastic price is
always higher (resp., lower) under multiplicative (resp.,additive)
uncertainty than the optimal deterministic price.Note that the demand
uncertainty in the exponential model is neitheradditive nor
multiplicative.Though Proposition 3 suggests that the optimal stochastic
price is always higher, this is not true for the revenue function
\(r ( \lambda ) = 1 - ( \lambda - 1 ) ^ { 2 }\) . It is
true,however,that for all regular demand functions over short time
horizons,i.e., \(t \leq n / \lambda ^ { * } .\) ,the optimal stochastic
price (resp., intensity) is always higher (resp., lower) than the
optimal deterministic price(resp.,intensity).

\section{4. Discrete Price Case}\label{discrete-price-case}

Consider the case where the set of allowable prices is
\(\mathcal { P } = \{ p _ { 1 } , . . . , p _ { K } , p _ { \infty } \}\)
, a discrete set. The restriction to a discrete set of prices may arise
if a firm decides,at a strategic level, to restrict itself to a given
set of prices in order to achieve market segmentation.Alternatively, the
discrete set of prices may be the result of an explicit or implicit
consensus at the industry level (e.g., ``price points'').We also
suggest below that a discrete price scheme together with dynamic
allocation of the units allows firms to synthesize a wide range of
effective prices to accommodate shifting demand functions while
retaining the appeal and practicality of having only a small,stable set
of prices for their product or service. We propose this as one plausible
explanation for the practice of yield management.

As before,we have \(n\) items and \(t\) units of time to sell them.
Corresponding to price \(p _ { k }\) , we have a known demand rate
\({ \lambda } _ { k }\) (24号 {\$\{ \textbackslash bf
\textbackslash nabla \} , k = 1 , \textbackslash ldots , K .\$} .Without
loss of generality, we assume that
\(p _ { 1 } < p _ { 2 } < \cdots , < p _ { K } .\) and
\(\lambda _ { 1 } > \lambda _ { 2 } > \cdots\) \(> \lambda _ { K }\)
.Equality in the demand rates is ruled out since equation (8) selects
the largest price corresponding to any given demand rate.

Let \(r _ { k } \doteq p _ { k } \lambda _ { k } ,\) .denote the revenue
rate associated with price \(p _ { k } , \ k = \ 1 , \ . \ . \ . \ K\)
.We assume that the revenue rates are monotonically decreasing:

\[
r _ { 1 } > r _ { 2 } > \cdots > r _ { K } .
\]

This assumption is again without loss of generality since
\(0 \leq \lambda \leq \lambda ^ { * }\) by Proposition 1,and because
\(r ( \cdot )\) isincreasing over this region.

4.0.1.Optimal Solution of the Deterministic Problem. The proof of
Theorem 2 goes through unchanged for the discrete price case; thus,we
can again use the deterministic revenue as an upper bound. The
deterministic solution also gives an asymptotically optimal heuristic,
though the resulting heuristic is no longer a fixed -price heuristic,but
consists of pricing at some price, \(p _ { k ^ { * } } ,\) for a
specified period of time and at a neighboring price,
\(p _ { k ^ { * } + 1 } ,\) ,for the balance of the horizon.

To solve the deterministic pricing problem let \(t _ { k }\)
\(\begin{array} { r } { = \int _ { 0 } ^ { t } 1 \left( p _ { s } = p _ { k } \right) d s } \end{array}\)
denote the amount of time we price the items at
\(p _ { k } , k = 1 , \ldots , K ,\) over the horizon \([ 0 , t ]\) Here
1 \(( p _ { s } = p _ { k } ) = 1\) if the price at time s is
\(p _ { k }\) and zero otherwise. Then equation(11) reduces to a linear
program.For convenience,set \(\lambda _ { 0 } \doteq \infty\) 、
\(\lambda _ { K + 1 } \doteq 0\) and \(r _ { 0 }\)
\(= r _ { K + 1 } \dot { = } 0\) . The next proposition states,without
proof, that this linear program can be solved in closed form:

PROPOSITION 4.For any \(( n , \ t )\) ,let \(k ^ { * }\) be such that
\(\lambda _ { k } { * } t \geq n > \lambda _ { k } { * } _ { + 1 } t\) ,
then the solution to the linear program is given by \(t _ { j } = 0\)
for \(j \not \in \{ k ^ { * } , k ^ { * } + 1 \}\) ,and

\[
\begin{array} { c } { \displaystyle t _ { k ^ { * } } = \frac { n - \lambda _ { k ^ { * } + 1 } t } { \lambda _ { k ^ { * } } - \lambda _ { k ^ { * } + 1 } } } \\ { \displaystyle t _ { k ^ { * } + 1 } = \frac { \lambda _ { k ^ { * } } t - n } { \lambda _ { k ^ { * } } - \lambda _ { k ^ { * } + 1 } } , } \end{array}
\]

where \(t _ { k } \doteq 0\) 、 \(t _ { k + 1 } \doteq t\) when
\(k ^ { * } = 0\) and
\(t _ { k } \doteq n / \lambda _ { K } , t _ { k + 1 }\)
\(\dot { = } 0\) when \(k ^ { * } = K\) :

REMARK. If there exists a salvage value \(q > 0\) then the above
results continue to hold provided (1)we eliminate all prices \(p < q\)
,and all prices \(p _ { i }\) such that
\(q ( \lambda _ { i } - \lambda _ { j } ) > r _ { i } - r _ { j }\) for
some \(p _ { j } > p _ { i } > q\) ,and (2) we set
\(r _ { i } = \lambda _ { i } ( p _ { i } - q )\) in the linear program.

Notice that the solution prices at \(p _ { k ^ { * } }\) for αt units of
time and at \(p _ { k ^ { * } + 1 }\) for \(( 1 - \alpha ) t\) units of
time where \(\alpha\) \(\in [ 0 , 1 )\) satisfies
\(\alpha \lambda _ { k } . . t + ( 1 - \alpha ) \lambda _ { k } . . _ { + 1 } t = n\)
Thus, \(\alpha \lambda _ { k } { * } t\) and
\(( 1 - \alpha ) \lambda _ { k ^ { * } + 1 } t\) are approximately the
number of items allocated to prices \(p _ { k }\) *and
\(p _ { k ^ { * } + 1 }\) respectively, and ( \(( \alpha r _ { k }\)
\(+ ( 1 - \alpha ) r _ { k ^ { * } + 1 } ) t / n\) is the effective
price paid,averaged across the \(n\) items.By adjusting the allocations
in this way,one can synthesize effective prices for many different
demand functions. There are many practical advantages to such a scheme.
It allows a firm to offer only a small set of stable prices,which are
easy for consumers to interpret and for the firm to advertise and
manage. Yet it also enables the firm to respond to short-term variations
in demand, such as those caused by day-ofthe-week cycles, holidays,
seasonalities, etc. This may be one reason why industries with highly
variable demand patterns, such as airlines, hotels and cruise-ships,
have adopted the fixed-fare-classes,dynamic allocation policies of yield
management.

\section{4.1. An Asymptotically Optimal
Heuristic}\label{an-asymptotically-optimal-heuristic}

The deterministic solution suggests a stopping-time (ST) heuristic for
the stochastic problem. Let
\(m \doteq \lceil \lambda _ { k } { * } t _ { k } { * } \rceil ,\)
(204号 \(T _ { m }\) be the (random) time the mth item is demanded when
the price is fixed at \(p _ { k }\) *,and let
\(t _ { m } = m / \lambda _ { k } * ,\) be the time it takes to sell
\(m\) items at price \(p _ { k ^ { \prime } }\) when demand is
deterministic.The heuristic is defined as follows:

ST Heuristic: Start pricing at \(p _ { k ^ { * } }\) and switch to
\(p _ { k ^ { * } + 1 }\) at (random) time

\[
\tau = \operatorname* { m i n } ( T _ { m } , t _ { m } ) .
\]

Let \(J ^ { \mathsf { S T } } ( n , t )\) denote the expected revenue
for the ST heuristic. The following theorem is proved in the appendix:

THEOREM5. Suppose \(n  \infty\) and \(t \to \infty\) such that
\(\lambda _ { k } t \geq n > \lambda _ { k + 1 } t\) .Then,

\[
\operatorname* { l i m } _ { t \to \infty } \frac { J ^ { 5 \mathrm { T } } ( n , t ) } { J ^ { D } ( n , t ) } = 1 .
\]

As an example illustrating the rate of convergence, considera flightwith
\(n = 3 0 0\) seats that is open for sale \(t = 3 6 0\) days before the
departure of the flight.Assume that at the promotional fare
\(p _ { 1 } = \$ 198\) ,theaverage demand rate is
\(\lambda _ { 1 } = 1\) seats per day,and that at the regular fare
\(p _ { 2 } = \$ 358\) ,the average demand is
\(\lambda _ { 2 } = 0 . 5\) seats perday. Then
\(m = t _ { 1 } = 2 4 0 .\) ,andthe promotional fare is stopped when
240 seats are sold or when 240 days elapse,whichever occurs first.Using
the bounds in the appendix,we obtain

\[
\ S 6 6 , 0 8 0 \le J ^ { \mathrm { S T } } ( n , t ) \le J ^ { * } ( n , t ) \le \ S 6 9 , 0 0 0 .
\]

To assess the performance of the ST heuristic we simulated 30o flights
with the above data. The expected revenue of the ST heuristic was
estimated to be \(\$ 67,546\) or about \(9 8 \%\) of the deterministic
upper bound.

The analysis of the ST heuristic can be sharpened when
\(n \geq \lambda _ { 1 } t ,\) .andwhen \(n \leq \lambda _ { K } t\) in
the sense that the absolute, rather than the relative,error goes to zero
as \(n\) and/or \(t\) goes to infinity. The first case occurs when
demand is so low that we cannot expect to sell all the items even at the
lowest price \(\left( { { p } _ { 1 } } \right)\) ; the second occurs
when demand is so high that we can expect to sell all the items at the
highest price \(\left( { p } _ { K } \right)\) . In both cases the
heuristic reserves the entire stock to the lowest (resp., the highest)
price.

Finally,we point out that a heuristic with the same asymptotic
properties can be constructed whereby initially the items are priced at
\(p _ { k ^ { * } + 1 }\) and subsequently reduced to \(p _ { k } .\) .
Thus,both the low-to-high and the highto-low stopping-time heuristics
are asymptotically optimal.For example,in air travel the desirability of
a seat usually increases as the date of flight is approached, while in
fashion retailing the desirability of garments decreases as the season
draws to a close; thus airlines pricelow-high,while fashion retailers
price high-low. See Feng and Gallego(1992) for structural results and
algorithms to compute optimal stopping-time rules in situations that
allow,at most, one price change.

\section{5. Extensions to the Basic
Problem}\label{extensions-to-the-basic-problem}

We next examine several extensions of the basic problem. The first
extension allows demand to be compound Poisson.Next we consider a demand
function that varies with time according to a multiplicative seasonality
factor.Then,weextendourresultsto thecasewhere there are holding costs
and cash flows are discounted.We then allow the initial stock \(n\) to
be a decision variable along with price. Finally,we allow a resupply
option in the presence of overbooking and random cancellations. For all
these cases,we find asymptotically optimal heuristics and,in some
instances,a closed-form optimal policy for the exponential demand case.

\section{5.1. Demand is a Compound Poisson
Processes}\label{demand-is-a-compound-poisson-processes}

Let \(N _ { s }\) be a Poisson Process with random intensity
\(\left\{ \lambda _ { u } \colon \right.\) \(0 \leq u \leq s \}\) and
let \(T _ { k }\) be the epoch of the \(k\) th arrival of \(N _ { s }\)
. That is, \(N _ { s } = k\) for \(T _ { k } \le \ s < T _ { k + 1 }\) .
At time \(T _ { k }\) we see a demand of size \(X _ { k }\) where the
\({ X } _ { k } ^ { \prime }\) s arei.i.d.random variableswith
\(E X > 0\) ,and \(E X ^ { 2 } < \infty\) . Let \(\mathcal { U }\) be
the set of nonanticipatory policies such that
\(\begin{array} { r } { \int _ { 0 } ^ { t } X _ { N _ { s } } d N _ { s } \leq n } \end{array}\)
almost surely.The expected revenue can be written as

\[
J _ { u } ( n , t ) = E _ { u } \int _ { 0 } ^ { t } p ( \lambda _ { s } ) X _ { N _ { s } } d N _ { s } .
\]

Let
\(\begin{array} { r } { J ^ { D } ( n , t ) = E X \operatorname* { m a x } _ { \lambda ( s ) \in \Lambda } \int _ { 0 } ^ { t } \big ( r ( \lambda ( s ) ) d s } \end{array}\)
subject to \(\int _ { 0 } ^ { t } X _ { N _ { s } } d N _ { s } \leq n\)
be the optimal revenue for the deterministic problem.We next show

THEOREM 6.

\[
J _ { u } ( n , t ) \le J ^ { D } ( n , t ) .
\]

PROOF.For \(\mu \geq 0\) ,we define

\[
J _ { u } ( n , t , \mu ) \doteq J _ { u } ( n , t ) + \mu E _ { u } \bigg ( n - \int _ { 0 } ^ { t } { X _ { N _ { s } } d N _ { s } } \bigg ) \ge J _ { u } ( n , t ) .
\]

Because \(X _ { k } = \ X _ { N _ { T _ { k } } }\) is independent of
\(N _ { s } s \le T _ { k }\) ,we can write

\[
\begin{array} { r l } { \mathbf { k } \{ \ln ( t , \phi ) - R \} = } & { \frac { 1 } { \phi } \sum _ { i = 1 } ^ { \infty } \{ \phi ( x _ { i } ) - R \} \mathbf { k } \{ \mathrm { H } \} \{ \mathrm { Z } , \phi , \varepsilon \} + \mathrm { i } \eta \mathbf { u } } \\ & { = \frac { 1 } { \phi } \sum _ { i = 1 } ^ { \infty } \mathbf { k } \{ x _ { i } , ( x _ { i } ) - \phi \} \{ \mathbf { k } \} \{ \mathrm { H } \} \{ \mathrm { S } , \varepsilon \} \{ t \} + \mathrm { i } \eta \mathbf { u } } \\ & { = \frac { 1 } { \phi } \sum _ { i = 1 } ^ { \infty } \{ x _ { i } , ( x _ { i } ) - \phi \} \{ \mathrm { H } \} \{ x _ { i } , ( x _ { i } ) - x _ { i } \} } \\ & { = \mathrm { i } \eta \sum _ { i = 1 } ^ { \infty } \{ x _ { i } , ( x _ { i } ) - \phi \} \ \mathrm { H } \{ S } ,  \\ & { = \mathrm { i } \eta \mathbf { k } \{ x _ { i } , ( x _ { i } ) - \phi \} \ \mathrm { H } \{ S } ,  \\ & { = \mathrm { i } \eta \mathbf { k } \{ x _ { i } , ( x _ { i } ) - \phi \} \ \mathrm { H } \{ S } ,  \\ & { = \mathrm { i } \eta \sum _ { i = 1 } ^ { \infty } \{ x _ { i } , ( x _ { i } ) - \phi \} \ \mathrm { H } \{ S } \ \mathrm { H } \  \\ & { = \mathrm { i } \eta \sum _ { i = 1 } ^ { \infty } \{ x _ { i } , ( x _ { i } ) - \phi \} \ \mathrm { H } \{ S } \ \mathrm { H } \  \\ & { = \mathrm { i } \eta \sum _ { i = 1 } ^ { \infty } \{ x _ { i } , ( x _ { i } ) - \phi \} \ \mathrm { H } \{ S } \ \mathrm { H } \  \\ & { = \mathrm { i } \eta \sum _ { i = 1 } ^ { \infty } \{ \phi ( x _ { i } ) - \phi \} \ \mathrm { H } \{ S } \ \mathrm { H } \{ S  \  \\ & { = \mathrm { i } \eta \sum _ { i = 1 } ^ { \infty } \{ \phi ( x _ { i } ) - \phi \} \ \mathrm { H } \{ S } \ \mathrm { H } \{ S  \  \\ & { = \mathrm { i } \eta ( \mathrm { H } \{ x _ { i } \} , \mathrm { H } \{ S } ) \ \mathrm { H } \  \\ & { = \mathrm { i } \eta ( \mathrm { H } \{ x _ { i } \} , \mathrm { H } \{ S } ) \ \mathrm { H } \{ S  \  \\ & { = \mathrm { i } \eta ( \mathrm { H } \{ x _ { i } \} , \mathrm { H } \{ S } ) \ \mathrm { H } \  \\ &  = \mathrm { i } \eta ( \mathrm { H } \end{array}
\]

Consequently,

\[
J _ { u } ( n , t ) \le J _ { u } ( n , t , \mu ) \le \operatorname * { i n f } _ { \mu \ge 0 } { J ^ { D } ( n , u , \mu ) } = J ^ { D } ( n , t )
\]

Thus,again the deterministic problem provides an upper bound.The
solution to the deterministic problem is easily seen to be
\(\lambda _ { s } = \lambda ^ { D }\) / \(0 \leq s \leq t\) ,where
\(\lambda ^ { D }\)
\(\doteq \operatorname* { m i n } \left\{ \lambda ^ { * } , \ \lambda ^ { 0 } \right\}\)
and \(\lambda ^ { 0 } \ \dot { = } \ n / ( t E X )\) . Consequently,
\(J ^ { D } ( n , \ t ) = \operatorname* { m i n } { \{ r ^ { * } , \ r ^ { 0 } \} } t E X\)
,where \(r ^ { 0 } \doteq \lambda ^ { 0 } p ( \lambda ^ { 0 } )\) . As
before,we can use the deterministic solution as a heuristic for the
stochastic problem.Let

\[
J ^ { \mathrm { F P } } ( n , t ) = p ^ { D } E \operatorname* { m i n } \left\{ n , \sum _ { k = 1 } ^ { N _ { t \lambda } \upsilon } X _ { k } \right\} .
\]

Following the arguments used in Theorem 3 to establish the asymptotic
optimality of the fixed price heuristic, we obtain

THEOREM 7.

\[
\frac { J ^ { \mathrm { O F P } } ( n , t ) } { J ^ { * } ( n , t ) } \geq \frac { J ^ { \mathrm { F P } } ( n , t ) } { J ^ { * } ( n , t ) } \geq 1 - \frac { \sqrt { \rho } } { 2 \sqrt { \operatorname* { m i n } \left\{ n , \lambda ^ { * } t \right\} } }
\]

where \(\rho \doteq E X ^ { 2 } / E X\)

\section{5.2. Time Varying Demand}\label{time-varying-demand}

Assume now that the demand rate \(\lambda ( p , s )\) depends both on
the price \(p\) and the time elapsed s since the start of

the selling season.Assume further that dependence in time is through a
positive multiplicative factor \(g ( s )\) 、so

\[
\lambda ( p , s ) = \lambda ( p ) g ( s ) \quad 0 \leq s \leq t .
\]

For example, \(g ( s )\) may be a concave function peaking near the
middle of the selling season.A simple method allows us to transform this
problem into one in which demand is time homogeneous. Let

\[
u = G ( s ) = \int _ { 0 } ^ { s } g ( z ) d z , \quad 0 \leq s \leq t ,
\]

and define

\[
\tilde { \lambda } ( p , u ) \doteq \lambda ( p ) , \quad 0 \le u \le G ( t ) .
\]

Then, for all \(\boldsymbol { s } < \boldsymbol { s } ^ { \prime }\)
,let \(u = G ( s )\) ,and \(u ^ { \prime } = G ( s ^ { \prime } )\)
,and note that

\[
\begin{array} { c } { { { \displaystyle { \int _ { s } ^ { s ^ { \prime } } \lambda ( p , z ) d z = \lambda ( p ) \int _ { s } ^ { s ^ { \prime } } g ( z ) d z = \lambda ( p ) [ G ( s ^ { \prime } ) - G ( s ) ] } } } } \\ { { { { } } } } \\ { { { = \lambda ( p ) [ u ^ { \prime } - u ] = \displaystyle { \int _ { u } ^ { u ^ { \prime } } { } ^ { } { \tilde { \lambda } ( p , v ) d v } } . } } } \end{array}
\]

Thus by using the clock \(u = G ( s ) , 0 \leq u \leq G ( t )\) ,instead
of the clock \(0 \leq s \leq t\) , we have transformed the problem into
one where demand is time homogeneous. Consequently,all of our results
apply to the transformed problem.In particular, the FPheuristic becomes:

\[
p ^ { \mathtt { F P } } = \operatorname* { m a x } \left\{ p ^ { * } , p ( n / G ( t ) ) \right\} .
\]

By Theorem 3,the performance guarantee of the FP heuristic is

THEOREM 8.

\[
\frac { J ^ { \mathrm { F P } } ( n , t ) } { J ^ { D } ( n , t ) } \geq 1 - \frac { 1 } { \sqrt { \operatorname* { m i n } ( n , \lambda ^ { * } G ( t ) ) } } .
\]

Theabove procedure can also be used in the discrete price case as
well.Indeed as in \(\ S 4 ,\) let \(k ^ { * }\) be such that
\(\lambda _ { k } . G ( t ) \geq n > \lambda _ { k ^ { * } + 1 } G ( t )\)
. Then the optimal solution to the transformed deterministic problem is
to price at \(p _ { k }\) for

\[
u _ { k ^ { * } } = \frac { n - \lambda _ { k } ^ { * } G ( t ) } { \lambda _ { k ^ { * } } - \lambda _ { k ^ { * } + 1 } }
\]

units of time, and to price at \(p _ { k ^ { * } + 1 }\) for

\[
u _ { k ^ { * } + 1 } = { \frac { \lambda _ { k ^ { * } } G ( t ) - n } { \lambda _ { k ^ { * } } - \lambda _ { k ^ { * } + 1 } } }
\]

units of time. The stopping-time heuristic for the original problem can
be constructed by pricing at \({ p } _ { k } ^ { * }\) for \(s _ { k }\)
\(= G ^ { - 1 } ( u _ { k } \ast )\) units of time in the original
clock,and by pricing at \(p _ { k ^ { * } + 1 }\) for
\(t \ : - \ : s _ { k } ,\) units of time,again in the original clock.

Now, let \(\tilde { \lambda } ^ { * } ( n , u )\) denote the optimal
intensity for the transformed problem when \(u\) units of time have
elapsed (with respect to the new clock) and there are \(n\) units in
inventory.We know from Theorem 1 that
\(\tilde { \lambda } ^ { * } ( n , u )\) is increasing in \(u\) . The
optimal intensity \(\lambda ^ { * } ( n , s )\) when s units of time
have elapsed(with respect to the original clock)and there are \(n\)
units in inventory is related to
\(\tilde { \lambda } ^ { * } ( n , u )\) by

\[
\lambda ^ { * } ( n , s ) = \tilde { \lambda } ^ { * } ( n , G ( s ) ) g ( s ) .
\]

Now let \(p ^ { * } ( n , s )\) denote the optimal price when s units of
time have elapsed with respect to the original clock and there are \(n\)
units in inventory. Then,by definition,

\[
\lambda ^ { * } ( n , s ) = \lambda ( p ^ { * } ( n , s ) , s ) = \lambda ( p ^ { * } ( n , s ) ) g ( s ) .
\]

Consequently,we have

\[
\lambda ( p ^ { * } ( n , s ) ) = \tilde { \lambda } ^ { * } ( n , G ( s ) ) .
\]

Now since \(\lambda ( p )\) is a decreasing function of \(p\) , it
follows that \(p ^ { * } ( n , s )\) is decreasing in s.This is
consistent with Theorem 1 viewing \(s\) as elapsed time.We note,however,
that the analogous result does not hold for
\(\lambda ^ { * } ( n , s )\) since its behavior also depends on
\(g ( s )\)

\section{5.3. Holding Cost and Discount
Rate}\label{holding-cost-and-discount-rate}

Now suppose cash flows are discounted at rate \(\beta ,\) and a linear
holding cost \(h\) is charged on existing inventories. Let \(Z ( s )\)
be the inventory level at time s.Then

\[
Z ( s ) = n - N _ { s }
\]

where \(N _ { s }\) is a Poisson process with random intensity
\(\left\{ \lambda _ { u } , 0 \le u \le s \right\}\) . The intensity
\(\lambda _ { s }\) is set to zero whenever \(Z ( s ) = 0\)

Recall \(\mathcal { U }\) denotes the class of nonanticipatory policies
that satisfy
\(\begin{array} { r } { \int _ { 0 } ^ { t } d N _ { s } \le n } \end{array}\)
almost surely. For any \(u \in \mathcal { U }\) the expected discounted
revenue is given by

\[
E _ { u } \int _ { 0 } ^ { t } e ^ { - \beta s } p ( \lambda _ { s } ) d N _ { s } = E _ { u } \int _ { 0 } ^ { t } e ^ { - \beta s } r ( \lambda _ { s } ) d s .
\]

The expected discounted holding cost is given by

\[
h E _ { u } \int _ { 0 } ^ { t } e ^ { - \beta s } Z ( s ) d s = h E _ { u } \int _ { 0 } ^ { t } e ^ { - \beta s } \bigg ( n - \int _ { 0 } ^ { s } \lambda _ { u } d u \bigg ) d s .
\]

Integrating the last expression by parts,we obtain

\[
h E _ { u } \int _ { 0 } ^ { t } e ^ { - \beta s } \biggr ( n - \frac { 1 } { \beta } ( 1 - e ^ { - \beta ( t - s ) } ) \lambda _ { s } \biggr ) d s .
\]

Consequently,the net expected discounted revenue
\(J _ { u } ( n , t )\) is given by

\[
J _ { u } ( n , t ) = E _ { u } \int _ { 0 } ^ { t } e ^ { - \beta s } \Bigg [ r ( \lambda _ { s } ) + \frac { h } { \beta } ( 1 - e ^ { - \beta ( t - s ) } ) \lambda _ { s } - h n \Bigg ] d s .
\]

Let {\$J \^{} \{ * \} ( n , \textbackslash{} t ) = \{
\textbackslash mathrm \{ m \} \} a \{ \{ \textbackslash bf x \} \} \_ \{
u \textbackslash in \textbackslash mathcal \{ U \} \} \textbackslash{} J
\_ \{ u \} ( n , \textbackslash{} t )\$} denote the maximal expected net
revenue among policies in \(\mathcal { U }\) . Let
\(\hat { r } ( \lambda _ { s } )\)
\(= e ^ { - \beta s } [ r ( \lambda _ { s } ) + h / \beta ( 1 - e ^ { - \beta ( t - s ) } ) \lambda _ { s } - h n ] ,\)
and let {\$J \^{} \{ \{ \textbackslash cal D \} \} ( n ,\$}
{\$\textbackslash begin\{array\} \{ r \} \{ t ) = \textbackslash mathrm
\{ m \} \{ \textbackslash bf a x \} \_ \{ \textbackslash lambda \_ \{ s
\} \} \textbackslash int \_ \{ 0 \} \^{} \{ t \} \textbackslash hat \{ r
\} ( \textbackslash lambda \_ \{ s \} ) d s \}
\textbackslash end\{array\}\$} subject to
\(\textstyle \int _ { 0 } ^ { t } \lambda _ { s } d s \leq n\) denote
the maximal net discounted revenue when demand is deterministic.Note
that \(\hat { r } ( \lambda _ { s } )\) inherits the concavity of
\(r ( \lambda _ { s } )\) , so by Theorem 2 we have

THEOREM 9.

\[
J ^ { * } ( n , t ) \leq J ^ { D } ( n , t ) .
\]

Again,one can show that the deterministic solution is an asymptotically
optimal heuristic for the stochastic problem, though in the presence of
holding cost and / or discount rates,itis no longer time
invariant.Indeed, let
\(J ^ { D } ( n , \textit { t , } \mu ) = \mathrm { m } a x _ { \lambda _ { s } }\)
\(\begin{array} { r } { \int _ { 0 } ^ { t } \big [ \bar { \hat { r } } ( \lambda _ { s } ) - \mu \lambda _ { s } \big ] d s + n \mu , } \end{array}\)
then
\(J ^ { D } ( n , t ) = \mathrm { i n f } _ { \mu \geq 0 } J ^ { D } ( n , t , \mu )\)
. Let \(\mu ^ { * }\) denote the optimal dual variable. Then for each
\(s \in ( 0 , \ t )\) we have
\(\hat { r } ^ { \prime } ( \lambda _ { s } )\)
\(\mathbf { \mu } = \mu ^ { * }\) . Or equivalently,

\[
\lambda _ { s } = g \bigg ( e ^ { \beta s } \bigg ( \mu ^ { * } + \frac { h } { \beta } e ^ { - \beta t } \bigg ) - \frac { h } { \beta } \bigg )
\]

where \(g ( \cdot ) \doteq \hat { r } ^ { \prime - 1 } ( \cdot )\) .
Now,since \(g ( \cdot )\) is a decreasing function,and the argument
\(e ^ { \beta s } ( \mu ^ { * } + ( h / \beta ) e ^ { - \beta t } )\)
\(\_ h / \beta\) is increasing in \(s\) , it follows that the optimal
intensity \(\lambda _ { s }\) (resp., price \(p _ { s }\) ) is
monotonically decreasing (resp.,increasing) in \(s \in ( 0 , t )\) :

At this point it is useful to isolate the holding and discounting
effects.If there were no discounting, then

\[
\lambda _ { s } = g ( \mu ^ { * } - h ( t - s ) ) ,
\]

and the argument is strictly increasing in \(s \in ( 0 , t )\)
regardless of the value of \(\mu ^ { * }\) . If there were no holding
costs,then

\[
\lambda _ { s } = g ( e ^ { \beta s } \mu ^ { \ast } )
\]

and the argument is strictly increasing in \(s \in ( 0 , t )\) only if
\(\mu ^ { * } > 0\) . From the holding cost point of view, the intuition
is that we want to sell initially at a faster rate in order to reduce
the cost of holding inventories.From the discounting point of view,we
are interested in the rate at which revenue is flowing in. If \(n\) is
large enough so that \(\mu ^ { * } = 0\) , we want to sell at
\(\lambda ^ { * } \doteq g ( 0 )\) to maximize therevenuerate
\(r ( \lambda ^ { * } )\) .If on the other hand, \(n\) is small enough
so that \(\mu ^ { * } > 0\) , thenwe want to sell ata lower rate
\(\lambda _ { s } = g ( e ^ { \beta s } \mu ^ { * } ) < \lambda ^ { * }\)
to avoid running out of stock before time \(t\) . However, since we are
discounting we start with higher revenue rates.

\section{5.4. Initial Stock as a Decision
Variable}\label{initial-stock-as-a-decision-variable}

Suppose we are allowed to determine the initial stock \(n\) ,the order
quantity,and also decide the subsequent pricing policy.If the initial
stock can be purchased at a unit cost \(c > 0\) , we want to find the
order quantity \(n ^ { * }\) that maximizes the expected profit

\[
\Pi ( n , t ) = J ^ { * } ( n , t ) - c n .
\]

This problem reduces to the classical newsboy problem if we replace
\(J ^ { * }\) above by the expected revenue for a given fixed price. If
we control this fixed price, then we obtain the problem studied by
Karlin and Carr (1962) and Whitin (1955).

By Theorem 2, \(J ^ { * } ( n , t ) \leq J ^ { D } ( n , t )\)
;consequently an upper bound on \(\Pi ( n , t )\) (cf.equation(12)) is
given by

\[
\Pi _ { B } ( n , t ) = { \left\{ \begin{array} { l l } { t r ( \lambda ^ { * } ) - c n } & { { \mathrm { i f ~ } } n > \lambda ^ { * } t } \\ { t r ( n / t ) - c n } & { { \mathrm { o t h e r w i s e . } } } \end{array} \right. }
\]

Treating \(n\) asa continuous variable, let \(n ^ { c }\) denote the
maximizer of \(\Pi ^ { D } ( n , \ t )\) . We see that for
\(n > \lambda ^ { * } t .\) {\$\textbackslash Pi \^{} \{
\textbackslash cal D \} ( n , t )\$} is strictly decreasing in \(n\)
,so \(n ^ { c } \leq \lambda ^ { * } t\) .For \(n\) (204号
\(\leq \lambda ^ { * } t\) 、 {\$\textbackslash Pi \^{} \{
\textbackslash cal D \} ( n , t )\$} is concave in \(n\) ,so

\[
n ^ { c } = \lambda ^ { c } t ,
\]

where
\(\lambda ^ { c } \doteq { r ^ { \prime } } ^ { - 1 } ( c ) \le \lambda ^ { * }\)
. Thus

\[
\Pi ( n ^ { * } , t ) \leq \Pi ^ { D } ( n ^ { c } , t ) = t [ r ( \lambda ^ { c } ) - c \lambda ^ { c } ] .
\]

THEOREM 10. The deterministic solution
\(( n ^ { c } , \lambda ^ { c } )\) isasymptotically optimal as
\(t \to \infty\) :

PROOF OF THEOREM 10.By Theorem 3,

\[
\begin{array} { l } { \displaystyle \Pi ( n ^ { c } , t ) = J ^ { * } ( n ^ { c } , t ) - c n ^ { c } } \\ { \displaystyle \ge \Pi ^ { D } ( n ^ { c } , t ) - \frac { 1 } { 2 } r ( \lambda ^ { c } ) \sqrt { t / \lambda ^ { c } } . } \end{array}
\]

Consequently,

\[
\frac { \Pi ( n ^ { c } , t ) } { \Pi ( n ^ { * } , t ) } \ge \frac { \Pi ( n ^ { c } , t ) } { \Pi ^ { D } ( n ^ { c } , t ) } \ge 1 - \frac { r ( \lambda ^ { c } ) } { 2 ( r ( \lambda ^ { c } ) - c \lambda ^ { c } ) \sqrt { \lambda ^ { c } t } } .
\]

Thus,we have

\[
\operatorname* { l i m } _ { t \to \infty } \frac { \Pi ( n ^ { c } , t ) } { \Pi ( n ^ { * } , t ) } = 1 . \sqsupset
\]

REMARk. Using Theorem 1,one can show that \(n ^ { * }\) and
\(\lambda ^ { c }\) are related in a rather interesting way,namely,

\[
{ \lambda } ( n ^ { * } , t ) \le { \lambda } ^ { c } \le { \lambda } ( n ^ { * } + 1 , t ) .
\]

\section{5.5. Resupply, Cancellations,and
Overbooking}\label{resupply-cancellationsand-overbooking}

Suppose additional units can be secured at a unit cost \(b > 0\) , so
the firm now has the option of selling beyond its initial
inventory(overbooking).We view this option in one of two ways: (1)
demand is satisfied by placing a special order every time a sale is made
while out of stock,or(2) demand is backlogged and at time \(t\) the
firm orders as many additional units as needed to satisfy the
backlog.The first case is most common when items are hard goods
(clothes,appliances,etc.),in which case \(b\) may represent unit
transshipment costs or special handling charges.The second case applies
to a model of overbooking in the airline and hotel industry,where \(b\)
may correspond to the cost of a seat on an alternate flight or a room at
an alternate hotel site (i.e.,a secondary supply) or may also be a
loss-of-goodwill penalty for not providing on time service.

Overbooking is often practiced to compensate for cancellations.Here we
assume that each reservation is canceled independently,at time \(t\)
,with probability 1 \(- \rho\) .In addition,we assume that customers
who cancel are refunded the purchase price less a penalty,which
consists of a fixed plus variable component. Specifically, let \(c\)
represent the fixed fee and \(\beta\) represent the fraction of the
price paid that composes the variable fee. Thus, a customer who pays
price \(p\) and cancels gets a refund of \(p ( 1 - \beta ) - c\) .(See
Bitran and Gilbert 1992 and Liberman and Yechiali 1978 for alternative
models that consider cancellations and overbooking.)

Given a non-anticipating intensity control policy \(\lambda _ { s }\)
based on the initial inventory \(n\) and the current history
ofreservations, thenumberof reservations \(N _ { s }\) is Poisson with
random intensity \(\int _ { 0 } ^ { s } \lambda _ { v } d v\) . Let
\(\left\{ T _ { k } \colon k \geq 1 \right\}\) denote the jump points of
the counting process \(N _ { s }\) 、 \(0 \leq s \leq t\) 1 and let
\(\{ Z _ { k } \colon k \geq 1 \}\) be a sequence of independent
Bernoulli random variables taking value 1 with probability \(\rho\) and
taking value O with probability \(1 \mathrm { ~ - ~ } \rho\) . By our
above assumptions about the cancellation process, these random variables
are also independent of the counting process \(N _ { s }\)

We assume that revenues are collected as reservations are made and
refunds for canceled reservations are paid at the end of the horizon.If
we disregard the time value of money, the net expected revenue can be
written as

\[
E \sum _ { k \geq 1 } p ( \lambda _ { T _ { k } } ) 1 ( T _ { k } \leq t ) 1 ( Z _ { k } = 1 ) = E \int _ { 0 } ^ { t } \rho r ( \lambda _ { s } ) d s .
\]

If the firm imposes a \(1 0 0 \beta \%\) penalty of the price paid for
each canceled reservation, then the expected net revenue is obtained by
replacing \(\rho\) by
\(\rho \mathrm { ~ + ~ } \beta ( 1 \mathrm { ~ - ~ } \rho )\) above.In
addition,if each canceled reservation is subject to a fixed penalty
\(c\) , then we add to the expected net revenue the quantity

\[
\begin{array} { r } { E c \sum _ { k \geq 1 } 1 ( T _ { k } \leq t ) 1 ( Z _ { k } = 0 ) = E \displaystyle \int _ { 0 } ^ { t } \sum c ( 1 - \rho ) \lambda _ { s } d s . } \end{array}
\]

The number of uncanceled reservations is

\[
\sum _ { k \geq 1 } 1 ( T _ { k } \leq t ) 1 ( Z _ { k } = 1 ) = \int _ { 0 } ^ { t } d \bar { N } _ { s }
\]

where \(\bar { N _ { s } }\) is Poisson with random intensity
\(\rho \ \int _ { 0 } ^ { s } \lambda _ { v } d v\) 旺
\(\int _ { 0 } ^ { t } d \bar { N _ { s } } > n\) , we must purchase (
{\$\textbackslash smash \{ \textbackslash int \_ \{ 0 \} \^{} \{ t \} d
\textbackslash bar \{ N \_ \{ s \} \} \textgreater{} n
\textbackslash bar \{ \textbackslash rangle \} \^{} \{ + \} \}\$}
additional units at \(b\) dollars each. Therefore, the expected net
revenue under a nonanticipating policy \(u\) is

\[
\begin{array} { l } { { \displaystyle V _ { u } ( n , t ) = E _ { u } \int _ { 0 } ^ { t } \big ( \rho + \beta ( 1 - \rho ) \big ) r ( \lambda _ { s } ) d s } } \\ { { \displaystyle ~ + ~ E _ { u } \int _ { 0 } ^ { t } c ( 1 - \rho ) \lambda _ { s } d s - b E _ { u } \Bigg ( \int _ { 0 } ^ { t } d \bar { N _ { s } } - n \Bigg ) ^ { + } , } } \end{array}
\]

where

\[
\begin{array} { r } { V _ { u } ( m , 0 ) \doteq \left\{ \begin{array} { l l } { 0 } & { m \ge 0 } \\ { b E ( X _ { m } - n ) ^ { + } } & { m < 0 , } \end{array} \right. } \end{array}
\]

\(n\) denotes the initial inventory (capacity), \(m\) denotes the
possibly negative unsold capacity at time \(t\) before learning about
cancellations,and \(X _ { m }\) is a binomial random variable with
parameters \(n - m\) and \(\rho\) Thus, \(X _ { m }\) \(- n ,\) if
positive,is the number of uncanceled reservations in excess of the
initial capacity.

As before, the firm's problem is to find a pricing policy \(u ^ { * }\)
(if one exists) that achieves an expected revenue

\[
V ^ { * } ( n , t ) = \operatorname* { s u p } _ { u \in \mathcal { U } } V _ { u } ( n , t ) ,
\]

MANAGEMENT SCIENCE/Vol. 40,No.8,August 1994 where we let
\(\mathcal { U }\) denote the class of all Markovian policies satisfying
\(p _ { s } \in \mathcal { P }\) \(\forall s\) :

In the next subsection we find a closed-form solution to the stochastic
problems when demand is exponen tially decaying and no cancellations
occur ( \(\mathbf { \rho } _ { \rho } = 1\) ).We then solve the
deterministic counterpart for the general case and present an
asymptotically optimal heuristic.

5.5.1.An Optimal Policy for the Exponential Demand Function with no
Cancellations. Let \(\lambda ( p )\) \(= a e ^ { - \alpha p }\) and
\(\rho = 1\) ,and \(b > 0\) . This case corresponds to having no
cancellations and a unit reorder cost. It perhaps most appropriate for
applications where items are hard goods and the cost \(b\) is a per-unit
special-order cost or per-unit transshipment cost for obtaining
additional units. One can verify that in this case
\(V ^ { * } ( n , t )\) is the solution to equation(8) with boundary
conditions \(V ^ { * } ( n , t ) = 0\) if \(n \geq 0\) and
\(V ^ { * } ( n , t ) = n b\) if \(n < 0\) .As before,without loss of
generality we take \(\alpha = 1\) . Let \(\lambda ^ { \textit { b } }\)
\(\dot { = }\) argmax
\(( r ( \lambda ) - \lambda b ) = \lambda ^ { * } e ^ { - b }\) . Then
\(V ^ { * } ( n , t )\) is given by

\[
V ^ { * } ( n , t ) = \left\{ \begin{array} { l l } { \log \left( \displaystyle \sum _ { i = 0 } ^ { n } \frac { ( \lambda ^ { * } t ) ^ { i } } { i ! } + e ^ { n b } \displaystyle \sum _ { i = n + 1 } ^ { \infty } \frac { ( \lambda ^ { t } t ) ^ { i } } { i ! } \right) , } & { n > 0 } \\ { \lambda ^ { b } t + n b } & { n \leq 0 } \end{array} \right.
\]

and the optimal price is given by
\(p ^ { * } ( n , \ t ) = V ^ { * } ( n , \ t )\)
\(- V ^ { * } ( n - 1 , t ) + 1\) :

Note,for \(n > 0\) we can write,

\[
\exp ( V ^ { * } ( n , t ) ) = \exp ( J ^ { * } ( n , t ) ) + e ^ { n b } \sum _ { t = n + 1 } ^ { \infty } \frac { ( \lambda ^ { b } t ) ^ { i } } { i ! } ,
\]

where \(J ^ { * } ( n , t )\) is the optimal revenue with no reorder
option (the basic problem),and the second term above is always
nonnegative. Thus, the expected revenue is strictly greater than without
the reorder option as expected. The price trajectory itself has
characteristics similar to the basic problem(
\(\mathbf { \nabla } \cdot \boldsymbol { b } = \infty .\)
),takingupward jumps as items are sold and decaying as time elapses
without a sale. The exception is when the inventory drops to zero,at
which point the policy switches to a fixed price of \(b + 1\) :

5.5.2.An Asymptotically Optimal Heuristic for the General Case. For the
general case with both cancellations and reordering,the deterministic
problem corresponding to equation(21) can be written as

\[
\begin{array} { l } { \displaystyle { V ^ { D } ( x , t ) = \operatorname* { m a x } _ { \lambda ( s ) } \int _ { 0 } ^ { t } ( \rho + \beta ( 1 - \rho ) ) r ( \lambda ( s ) ) d s } } \\ { \displaystyle { + \int _ { 0 } ^ { t } c ( 1 - \rho ) \lambda ( s ) d s - b \Bigg ( \rho \int _ { 0 } ^ { t } \lambda ( s ) d s - x \Bigg ) ^ { + } . } } \end{array}
\]

Now \(V ^ { * } ( x , t ) \leq V ^ { D } ( x , t )\) follows by applying
Jensen's inequality to the third term of \(V ^ { u } ( x , t )\) and by
viewing the integrand inside the expectation as purely a.function of
\(\lambda\) and maximizing pointwise.

To solve the deterministic problem we need to introduce notation that is
pertinent only to this section. Let

\[
\hat { r } ( \lambda ) \doteq ( \rho + \beta ( 1 - \rho ) ) r ( \lambda ) + c ( 1 - \rho ) \lambda ,
\]

denote the modified revenue rate. Let
\(\lambda ^ { 0 } \doteq x / ( \rho t )\) be the expected-run-out rate.
At rate \(\lambda ^ { 0 }\) , we book \(\lambda ^ { 0 } t = x /\)
(204号 \(\rho \geq x\) units over the horizon, of which
\(\rho \lambda ^ { 0 } t = x\) show at time \(t\) Let
\(p ^ { 0 } \doteq p ( \lambda _ { 0 } )\) be the expected-run-out
price, and \(\hat { r } ^ { 0 } \doteq \hat { r } ( \lambda _ { 0 } )\)
. Let \(\lambda ^ { * }\) denote the least maximizer of
\(\hat { r } ( \lambda )\)
\(p ^ { * } \dot { = } p ( \lambda ^ { * } )\) its corresponding
price,and \(\hat { r } ^ { * } \doteq \hat { r } ( \lambda ^ { * } )\)
Finally, let \(\lambda ^ { \textit { b } }\) be the least maximizer of
\(\hat { r } ( \lambda ) - b \rho \lambda , p ^ { b }\)
\(\dot { = } p ( { \lambda } ^ { b } )\) its corresponding price,and
\(\hat { r } ^ { b } \doteq \hat { r } ( \lambda ^ { b } )\) . The
following proposition is given without proof:

PROPOsiTiON5. The optimal solution to the deterministic problem(22)is

\[
\begin{array} { r l } & { p _ { D } ( s ) = \left\{ \begin{array} { l l } { p ^ { * } } & { \rho \lambda ^ { * } t \leq x } \\ { p ^ { 0 } } & { \rho \lambda ^ { * } t \leq x < \rho \lambda ^ { * } t \quad 0 \leq s \leq t , } \\ { p ^ { b } } & { x < \rho \lambda ^ { b } t } \end{array} \right. } \\ & { \lambda _ { D } ( s ) = \left\{ \begin{array} { l l } { \lambda ^ { * } } & { \rho \lambda ^ { * } t \leq x } \\ { \lambda ^ { 0 } } & { \rho \lambda ^ { b } t \leq x < \rho \lambda ^ { * } t \quad 0 \leq s \leq t , } \\ { \lambda ^ { b } } & { x < \rho \lambda ^ { b } t } \end{array} \right. } \end{array}
\]

and

\[
\begin{array} { r } { V ^ { D } ( x , t ) = \left\{ \begin{array} { l l } { \hat { r } ^ { * } t } & { \rho \lambda ^ { * } t \leq x } \\ { \hat { r } ^ { 0 } t } & { \rho \lambda ^ { b } t \leq x < \rho \lambda ^ { * } t } \\ { \big ( \hat { r } ^ { b } - b \rho \lambda ^ { b } \big ) t + x b } & { x < \rho \lambda ^ { b } t . } \end{array} \right. } \end{array}
\]

Thus if capacity is high \(( x \ge \rho \lambda ^ { * } t )\) , we price
to maximize the modified revenue rate (equation(22)).If capacity is low
\(( x < \rho \lambda ^ { b } )\) ,we price at \(p ^ { \flat }\) ,since
in this case \(\lambda ^ { \textit { b } }\) (20 maximizes the modified
profit rate \({ \hat { r } } ( \lambda ) - b \lambda\) For intermediate
capacity \(( \rho \lambda ^ { b } t \le x < \rho \lambda ^ { * } t )\)
)we price at the expected-run-out price.

REMARK.If \(\rho b > \hat { r } ^ { \prime } ( 0 )\) ,then
\(\lambda ^ { b } = 0 .\) ,and the solution reduces to the case with no
reorder option provided \(x\) \(\geq 0\)

Notice that the deterministic solution consists of a fixed price over
the entire horizon. Consider the fixedprice(FP)heuristic that prices
according to the deterministic intensities \(\lambda _ { D } ( s )\) 1
\(0 \leq s \leq t\) . The expected value of the FP heuristic is given by

\[
\begin{array} { l } { { \displaystyle V ^ { \mathrm { F P } } ( x , t ) = \int _ { 0 } \big ( \rho + \beta ( 1 - \rho ) \big ) r \big ( \lambda _ { D } ( s ) \big ) d s } } \\ { { \displaystyle \qquad + \int _ { 0 } ^ { t } c ( 1 - \rho ) \lambda _ { D } ( s ) d s - E b ( \hat { N _ { t } } - x ) ^ { + } , } } \end{array}
\]

where \(\hat { N _ { t } }\) is Poisson with intensity
\(\int _ { 0 } ^ { t } \rho \lambda _ { D } ( v ) d v\) .Note that the
first two terms of \({ V } ^ { \mathrm { F P } } ( x , t )\) are equal
to those of \({ V ^ { D } } ( x , t )\) . To establish the asymptotic
optimality of \({ V } ^ { \mathrm { F P } } ( x , t )\) we need a slight
variant of (18). Let \(N\) be a random variable with mean \(\mu\) and
variance \(\sigma ^ { 2 }\) , writing
\(( N - x ) ^ { + } = { \textstyle \frac { 1 } { 2 } } ( | N - x | + ( N - x ) )\)
, taking expectations and using the Cauchy-Schwartz inequality,we obtain

\[
\begin{array} { l } { { \displaystyle E ( N - x ) ^ { + } \le \frac { 1 } { 2 } \sqrt { \sigma ^ { 2 } + ( \mu - x ) ^ { 2 } } + \frac { 1 } { 2 } ( \mu - x ) } } \\ { { \displaystyle \qquad \le \frac { 1 } { 2 } \sigma + \frac { 1 } { 2 } ( | \mu - x | + ( \mu - x ) ) = \frac { 1 } { 2 } \sigma + \frac { 1 } { 2 } ( \mu - x ) ^ { - } } } \end{array}
\]

We can now state

THEOREM 11.

\[
\begin{array} { r l } & { \frac { V ^ { \mathrm { F P } } \left( n , t \right) } { V ^ { * } \left( n , t \right) } } \\ & { \qquad \quad \ge \left\{ \begin{array} { l l } { 1 - \frac { b \sqrt { \rho \lambda ^ { * } t } } { 2 \hat { r } ^ { * } t } } & { \rho \lambda ^ { * } t \le n } \\ { 1 - \frac { b \sqrt { \rho \lambda ^ { 0 } t } } { 2 \hat { r } ^ { 0 } t } } & { \rho \lambda ^ { b } t \le n < \rho \lambda ^ { * } t } \\ { 1 - \frac { b \sqrt { \rho \lambda ^ { b } t } } { 2 ( \hat { r } ^ { b } - b \rho \lambda ^ { b } t ) } } & { n < \rho \lambda ^ { b } t . } \end{array} \right. } \end{array}
\]

PROOF. Applying the above bound to
\(E ( \hat { N _ { t } } - x ) ^ { + }\) in
\({ V } ^ { \mathtt { F P } } ( n , t )\) we obtain

\[
E b ( \hat { N _ { t } } - n ) ^ { + } \leq \frac { 1 } { 2 } b \sqrt { \rho \lambda _ { D } t } + \frac { 1 } { 2 } b ( \rho \lambda _ { D } t - n ) ^ { + } .
\]

Consequently,
\(\begin{array} { r } { V ^ { \operatorname { F P } } ( n , \ t ) \ge V ^ { D } ( n , \ t ) - \frac { 1 } { 2 } b \sqrt { \rho \lambda _ { D } t } } \end{array}\)
The result follows after dividing by \({ V ^ { D } } ( n , t )\) .□

By observing that when \(n \leq 0\)
\(E [ ( N _ { t } - n ) ^ { + } ] = \lambda ^ { b } t\) \(- \ : n .\)
,we obtain the following corollary to Theorem 11:

COROLLARY 1.If \(n \leq 0\) ,then

\[
V ^ { \mathrm { F P } } ( n , t ) = V ^ { * } ( n , t ) .
\]

That is,when there are no items in inventory, the optimal policy is to
fix the price at \(p ^ { b }\) . The reason for this is that backlogged
items represent a sunk cost that cannot be influenced by our pricing
policy. Thus, we ignore \(n\) and simply try to maximize the net revenue
rate \(\hat { r } ( \lambda )\)
\(\mathbf { \Pi } - \mathbf { \Pi } _ { \rho } \lambda b\) over the
remaining time,which implies pricing at \(p ^ { b }\)

\section{6. Conclusions}\label{conclusions}

We have shown how a range of inventory pricing problems can be analyzed
using intensity control theory, bounds,and heuristics.By analyzing the
deterministic version of different versions of the basic problem,we were
able to obtain both upper bounds on the expected revenue and insights
into the form of near-optimal policies.Exact optimal policies were found
in certain cases for a family of exponential demand functions.Perhaps
the strongest conclusion from our results is that using simple
fixed-price policies appears to work surprisingly well in many
instances. This is encouraging since the optimal dynamic policies are
quite jittery and require constant
priceadjustments,anundesirablecharacteristic in practical
applications.In the discrete-price case,we showed that a policy that
varies the allocation of units and time to two neighboring prices is
nearly optimal. The policy provides a good explanation of the structure
of current yield management practice.

We believe that this class of inventory pricing models represents a
fertile area for future research.From a practical standpoint,revenue
maximization holds the potential for dramatic improvements in
profitability and thus is likely to be a topic of intense interest to
managers in a wide range of industries. On a methodological level, we
think that formulating problems in the framework of intensity control is
a promising approach. Though exact solutions appear limited to special
cases,one can easilyobtain bounds similar to those in Theorem2that
relate the stochastic and deterministic variants of the problem. No
doubt other variants of the problem can be attacked using precisely this
approach. Similar bounds could potentially be useful for a wide range of
intensity control problems in other application contexts as well.2

² We thank three anonymous referees and the associate editor for
providing several references and many helpful comments.The research of
G.Gallego was supported in part by the National Science Foundation grant
DDM 9109636.The research of both authors was supported by the National
Science Foundation grant SES93-09394.

\section{Appendix}\label{appendix}

PROOF OF PROPOSITiON 1. We first show that the supremum in equation (8)
can be replaced by
\(\mathbf { m a x } _ { \lambda \in [ 0 , \lambda ^ { \bullet } ] }\)
.To do so,let \(\lambda _ { i }\) be an arbitrary intensity satisfying
\({ \lambda } _ { i } > { \lambda } ^ { * }\) .By concavity of
\(r ( \lambda )\) and the definition of \(\lambda ^ { * }\) ,we have
\(r ( \lambda ^ { * } ) \geq r ( \lambda _ { i } )\) ,and since
\(J ( n , \ t )\) is nondecreasing in \(n\) ,we have

\[
r ( \lambda ^ { * } ) - \lambda ^ { * } [ J ( n , t ) - J ( n - 1 , t ) ] \geq r ( \lambda _ { i } ) - \lambda _ { i } [ J ( n , t ) - J ( n - 1 , t ) ] .
\]

Hence the optimal choice of \(\lambda\) is always within the set
\([ 0 , \lambda ^ { * } ] ,\) a compact set.Combining compactness with
the fact that \(r ( \lambda )\) is continuousand bounded establishes the
conditions required by Bremaud Theorem I.3 for the existence of a unique
solution to equation (8).

PROOF OF THEOREM 1.The fact that \(J ^ { * } ( n , t )\) is strictly
increasing in \(n\) and \(t\) is straightforward to show,so we omit the
details.We next show by induction that \(\lambda ^ { * } ( n , \ t )\)
is strictly decreasing in \(t\) and in so doing establish that
\(\lambda ^ { * } ( n , \ t )\) is strictly increasing in \(n\) and that
\(J ^ { * } ( n , t )\) is strictly concave in both \(n\) and t.The
results for \(p ^ { * } ( n , t )\) follow from the fact that
\(\lambda ( p )\) is a regular demand function.

We begin with the case \(n = 1\) .Note that from equation (8),

\[
J ^ { * } ( n , t ) = J ^ { * } ( n - 1 , t ) + r ^ { \prime } ( \lambda ^ { * } ( n , t ) ) \geq J ^ { * } ( n - 1 , t )
\]

with strict inequality holding when \(t > 0\) .For \(n = 1\) ,observe
that
\(J ^ { * } ( 1 , t ) = r ^ { \prime } ( \lambda ^ { * } ( 1 , t ) )\)
.Thus,since \(J ^ { * } ( 1 , t )\) is strictly increasing in \(t\) we
have

\[
0 < \frac { \partial J ^ { * } ( 1 , t ) } { \partial t } = r ^ { \prime \prime } ( \lambda ^ { * } ( 1 , t ) ) \lambda ^ { * \prime } ( 1 , t ) ,
\]

which along with the concavity of \(r ( \cdot )\) implies that
\(\lambda ^ { * } ( 1 , t )\) is strictly decreasing in t.It also
follows from equation(8) that

\[
\frac { \partial J ^ { * } ( 1 , t ) } { \partial t } = r ( \lambda ^ { * } ( 1 , t ) ) - \lambda ^ { * } ( 1 , t ) J ^ { * } ( 1 , t ) .
\]

Again,combining thiswiththe fact that
\(J ^ { * } ( 1 , t ) = r ^ { \prime } ( \lambda ^ { * } ( 1 , t ) )\)
,we find

\[
\frac { \partial ^ { 2 } J ^ { * } ( 1 , t ) } { \partial t ^ { 2 } } = - \lambda ^ { * } ( 1 , t ) \frac { \partial J ^ { * } ( 1 , t ) } { \partial t } < 0 ,
\]

which shows \(J ^ { * } ( 1 , t )\) is strictly concave in t.Thus,all of
the claimed propertieshold for \(n = 1\)

Next assume that \(\lambda ^ { * } ( n - 1 , t )\) is strictly
decreasing int.From equation (26)we see that
\(r ^ { \prime } ( \lambda ^ { * } ( n , \textit { t } ) ) > 0\) for
\(t > 0\) ,implying \(\lambda ^ { * } ( n , \ t )\) (204号
\(< \lambda ^ { * }\) .Note that as t approaches zero from the right in
problem(26),
\(\begin{array} { r } { \operatorname* { l i m } _ { t  0 } r ^ { \prime } ( \lambda ^ { * } ( n , t ) ) = 0 . } \end{array}\)
so \(\lambda ^ { * } ( n , 0 ^ { + } ) = \lambda ^ { * }\) Hence,
\(\lambda ^ { * } ( n , t )\) is initially strictly decreasing int.Now
assume for the sake of contradiction that \(\lambda ^ { * } ( n , t )\)
is strictly decreasing over \([ 0 , t _ { 0 } )\) butis nondecreasing
over a nonempty interval \(\left[ t _ { 0 } , \ t _ { 1 } \right]\)
.Taking derivatives with respect to \(t\) in equation (8),we find that
over \([ 0 , t _ { 0 } )\) (20

\[
\frac { \partial J ^ { * } ( n , t ) } { \partial t } > \frac { \partial J ^ { * } ( n - 1 , t ) } { \partial t }
\]

with the opposite inequality holding over
\(\left[ t _ { 0 } , t _ { 1 } \right]\) .From this and equation(8),it
then follows that over
\([ 0 , t _ { 0 } ) , J ^ { * } ( n , t ) - J ^ { * } ( n - 1 , t ) < J ^ { * } ( n\)
\(- 1 , t ) - J ^ { * } ( n - 2 , t )\) ,and consequently
\(\lambda ^ { * } ( n , t ) > \lambda ^ { * } ( n - 1 , t )\) again with
the opposite inequalities holding over
\(\left[ t _ { 0 } , \ t _ { 1 } \right]\) .But this implies that
\(\lambda ^ { * } ( n - 1 , t )\) must be nondecreasing in the
neighborhood of \(t _ { 0 }\) which contradicts the inductive
hypothesis.Therefore,we conclude that \(\lambda ^ { * } ( n , t )\) must
be strictly decreasing in \(t\) and that
\(\lambda ^ { * } ( n , t ) > \lambda ^ { * } ( n\)
\(\mathbf { \Sigma } - \mathbf { \Sigma } _ { 1 , \textit { t } } )\) :

We now use these facts to show concavity of \(J ^ { * } ( n , t )\)
.Indeed,the fact that \(\lambda ^ { * } ( n , t )\) is strictly
increasing in \(n\) , the concavity of \(r ( \cdot )\) and (26) imply

\[
J ^ { * } ( n , t ) - J ^ { * } ( n - 1 , t ) < J ^ { * } ( n - 1 , t ) - J ^ { * } ( n - 2 , t ) ,
\]

so \(J ^ { * } ( n , t )\) is strictly concave in \(n\)
.Also,equations(8) and(27) imply

\[
\frac { \partial ^ { 2 } J ^ { * } ( n , t ) } { \partial t ^ { 2 } } = \lambda ^ { * } ( n , t ) \bigg [ \frac { \partial J ^ { * } ( n , t ) } { \partial t } - \frac { J ^ { * } ( n - 1 , t ) } { \partial t } \bigg ] < 0 ,
\]

so \(J ^ { * } ( n , \ t )\) is strictly concave in \(t\) .The claims
for \(p ^ { * } ( n , \ t )\) follow directly from the results for
\(\lambda ^ { * } ( n , t )\) :

PROOF OF PROPOsITiON 2. Consider the deterministic problem(11). Note
that the integrand in problem(11) is simply the revenue function,
\(r ( \lambda )\) ,which is concave by assumption.There are two
cases.First, suppose the maximizer of
\(r ( \lambda ) , \lambda ^ { * } ,\) satisfies
\(\lambda ^ { * } t \leq x\) ,thenclearly \(\lambda _ { s }\)
\(= \lambda ^ { * }\) \(0 \leq s \leq t\) is the optimal solution since
this choice maximizes the integrand pointwise.In the second case,
\(\lambda ^ { * } t > x\) ,it follows from the fact that
\(r ( \lambda )\) is concave that for a given value
\(\begin{array} { r } { y = \int _ { 0 } ^ { t } \lambda _ { s } d s , \lambda _ { s } = y / t . } \end{array}\)
\(0 \leq s \leq t\) maximizes the integral. The maximum revenue given
\(y\) is therefore \(t ( y / t ) p ( y / t ) = t r ( y / t )\) Now since
\(y / t < \lambda ^ { * }\) and \(r ( \lambda )\) is increasing for
\(\lambda < \lambda ^ { * } ,\) it follows that \(y = x\) in any optimal
solution, and thus \(\lambda _ { s } = ( x / t ) = \lambda ^ { 0 }\)
\(0 \leq s \leq t\) maximizes the integral. Converting these rates to
their corresponding prices and computing the corresponding total revenue
associated with this solution establishes the proposition.

PROOF OF THEOREM 5.Notice that \(\tau\) is a stopping time, since
\(\tau\) is finite with probability one and the event \(\tau < s\) ,can
be determined by the history of the arrivalsup to times.To obtaina lower
bound on \(J ^ { \mathsf { s r } } ( n , \ t )\) consider a wasteful
heuristic that reserves \(m\) (resp., \(n\) (204号 \(- m )\) units to
be priced at \(p _ { k }\) (resp., \(p _ { k + 1 } .\) over
\(t _ { m }\) (resp., \(t - t _ { m } )\) units of time. Let
\(J ^ { w } ( n , \ t )\) denote the expected revenue of the wasteful
heuristic.Evidently,the wasteful heuristic is a lower bound on the
stopping-time heuristic since if \(\tau = T _ { m } < t _ { m }\) the
wasteful heuristic delays the selling of the remaining \(n - m\) units
until time \(t _ { m }\) . On the other hand,if
\(\tau = t _ { m } < T _ { m }\) more than \(n - m\) units are left at
time \(t _ { m }\) 1 and the wasteful heuristic only makes \(n - m\) of
them available for sale at price \(p _ { k + 1 }\) .Inspite of these
limitations,we will show that the wasteful heuristic,and consequently
the ST heuristic,is asymptotically optimal.

To do this, let \(t _ { n - m } \doteq n - m / \lambda _ { k + 1 }\)
.Note that \(t _ { n - m }\) is the time it takes to sell \(n - m\)
items at price \(p _ { k + 1 }\) when the demand is deterministic.
Consequently, \(t ^ { \prime } = t _ { m } + t _ { n - m }\) is the
total time it takes to dispose of the \(n\) items when the demand rates
are deterministic.Observe that by our choice of \(m\) we have

\[
t - \frac { \lambda _ { k } - \lambda _ { k + 1 } } { \lambda _ { k } \lambda _ { k + 1 } } < t ^ { \prime } \leq t .
\]

Consequently,a lower bound on the wasteful heuristic can be obtained by
delaying the start of the sales by
\(\mathbf { \Delta } t \gets \mathbf { \Delta } t ^ { \prime }\) so that
effectively the horizon is shrunk to \(t ^ { \prime }\) .Recall that
thedeterministic revenueis

\[
J ^ { D } ( n , t ) = \frac { r _ { k } - r _ { k + 1 } } { \lambda _ { k } - \lambda _ { k + 1 } } n + \frac { \lambda _ { k } r _ { k + 1 } - \lambda _ { k + 1 } r _ { k } } { \lambda _ { k } - \lambda _ { k + 1 } } t ,
\]

so by equation (28)
\(J ^ { D } ( n , t ) < J ^ { D } ( n , t ^ { \prime } ) + ( p _ { k + 1 } - p _ { k } )\)
.We thus have

\[
\frac { J ^ { ^ { 5 \top } } ( n , t ) } { J ^ { ^ { * } ( n , t ) } } \geq \frac { J ^ { ^ { w } } ( n , t ) } { J ^ { ^ { D } } ( n , t ) } \geq \frac { J ^ { ^ { w } } ( n , t ^ { \prime } ) } { J ^ { ^ { D } } ( n , t ^ { \prime } ) + ( p _ { k + 1 } - p _ { k } ) } .
\]

Now if \(t \to \infty\) with
\(\lambda _ { k } t \geq n > \lambda _ { k + 1 } t\) .Then,by
construction, \(t ^ { \prime } \to \infty\) with
\(\lambda _ { k } t ^ { \prime } \geq n > \lambda _ { k + 1 } t ^ { \prime }\)
.Evidently \(J ^ { w } ( n , t ^ { \prime } ) \to \infty\) and
\(J ^ { D } ( n , t ^ { \prime } )  \infty\) as \(t \to \infty\)
,hence,if we can show that

\[
\operatorname* { l i m } _ { t \to \infty } \frac { J ^ { w } ( n , t ^ { \prime } ) } { J ^ { D } ( n , t ^ { \prime } ) } = 1 ,
\]

we can conclude that

\[
\operatorname* { l i m } _ { t \to \infty } \frac { J ^ { w } ( n , t ^ { \prime } ) } { J ^ { D } ( n , t ^ { \prime } ) + ( p _ { k + 1 } - p _ { k } ) } = 1
\]

and by(29) that the ST heuristic is asymptotically optimal.

Notice that showing this first limit is equivalent to showing it holds
for a subsequence
\(\{ t _ { m } ^ { \prime } \doteq m / ( \alpha \lambda _ { k } ) , m = 1 , . . . , \}\)
where

\[
m = [ \alpha \lambda _ { k } t _ { m } ^ { \prime } ] = \alpha \lambda _ { k } t _ { m } ^ { \prime } .
\]

Thus,we drop the prime notation and assume that
\(m = \alpha \lambda _ { k } t\) and that
\(n - m = \bar { \alpha } \lambda _ { k + 1 } t\) are integers.

Let \(N _ { \lambda _ { k } s }\) bea Poisson random variable with rate
\(\lambda _ { k } s\) .Clearly the expected revenue for the wasteful
heuristic is

\[
J ^ { w } ( n , t ) = p _ { k } E \operatorname* { m i n } \left\{ N _ { \lambda _ { k } \alpha t } , \alpha \lambda _ { k } t \right\} + p _ { k + 1 } E \operatorname* { m i n } \left\{ N _ { \lambda _ { k + 1 } \bar { \alpha } t } , \bar { \alpha } \lambda _ { k + 1 } t \right\} .
\]

Notingthat
\(\Xi \operatorname * { m i n } \ ( N _ { \lambda s } , \ \lambda s ) = E N _ { \lambda s } - E ( N _ { \lambda s } - \lambda s ) ^ { + }\)
,and that \(E N _ { \lambda s }\)
\(= V a r ( N _ { \lambda s } ) = \lambda s\) ,and using equation(18)
we obtain the following lower bound on the performance of the wasteful
heuristic.

\[
J ^ { w } ( n , t ) \geq p _ { k } [ \alpha \lambda _ { k } t - 1 / 2 \sqrt { \alpha \lambda _ { k } t } ] + p _ { k + 1 } [ \bar { \alpha } \lambda _ { k + 1 } t - 1 / 2 \sqrt { \bar { \alpha } \lambda _ { k + 1 } t } ] .
\]

From the deterministic solution,we know that
\(J ^ { D } ( n , \ t ) \ = \ p _ { k } \alpha \lambda _ { k } t\)
\(+ \ p _ { k + 1 } \bar { \alpha } \lambda _ { k + 1 } t\) . Taking
ratios,we observe that

\[
\frac { J ^ { w } ( n , t ) } { J ^ { D } ( n , t ) } \geq 1 - 1 / 2 \biggl [ \frac { 1 } { \sqrt { \alpha \lambda _ { k } t } } + \frac { 1 } { \sqrt { \bar { \alpha } \lambda _ { k + 1 } t } } \biggr ] ,
\]

which establishes the result.

\end{document}
