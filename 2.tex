\begin{document}


\section{4.2. Asymptotic Analysis of Bid Prices Derived from the Upper
Bound
Problem}\label{asymptotic-analysis-of-bid-prices-derived-from-the-upper-bound-problem}

Below, \(\mu^{*}\) denotes an optimal solution to (9). Again, we note
that \(\mu^{*}\) depends on the initial capacity \(x\) and the initial
time- to- go \(k\) . Consider the following fixed- bid- price heuristic:

\section{Fixed-Bid-Price Heuristic
(H)}\label{fixed-bid-price-heuristic-h}

At time \(k\) with remaining capacity \(x\) compute \(\mu^{*}\) once by
solving (9). Then, for all times \(t \leq k\) , accept a request for
itinerary \(j\) with revenue \(r\) if and only if there is sufficient
capacity to satisfy it and \(r > \mu^{*}A^{j}\) .

Let \(J_{k}^{H}(x)\) denote the expected revenue of this heuristic given
initial capacity \(x\) and time- to- go \(k\) . Let \(\theta\) be a
positive integer, and consider a sequence of problems, indexed by
\(\theta\) , with initial capacity vectors \(\theta x\) , time- to- go
\(\theta k\) and revenues, denoted \(R_{t}(\theta)\) , where

\[
R_{t}(\theta) = _{D}R_{t / \theta} \tag{12}
\]

and \(\mathbf{\Sigma} = _{D}\) denotes equality in distribution. This
construction corresponds to splitting each period \(t\) in the original
problem into \(\theta\) statistically independent and identical periods
in the scaled problem and at the same time increasing the initial
capacity by a factor of \(\theta\) . As a result, the relative values of
demand, capacity and time are preserved.

For the scaled problem, let \(J_{\theta k}^{H}(\theta x)\) denote the
optimal expected revenue and \(J_{\theta k}^{H}(\theta x)\) denote the
expected

revenue of the fixed- bid- price heuristic. (We show below that the
vector \(\mu^{*}\) solves (9) for all \(\theta\) , so the same vector of
bid prices is used for each problem in the sequence.) The following
result shows that the fixed- bid- price heuristic is asymptotically
optimal as the scale of the problem, as measured by \(\theta\) ,
increases:

THEOREM 1. If \(R_{i}^{j} \equiv C\) (a.s.), then

\[
\frac{J_{\theta k}^{H}(\theta x)}{J_{\theta k}(\theta x)} \geq 1 - O(\theta^{-1 / 2}).
\]

In particular,

\[
\lim_{\theta \to \infty} \frac{J_{\theta k}^{H}(\theta x)}{J_{\theta k}(\theta x)} = 1.
\]

PROOF. By considering (8) and (12), we have

\[
\begin{array}{l}{{\sum_{t=1}^{\theta k}\sum_{j=1}^{n}E(R_{t}^{j}(\theta)-\mu A^{j})^{+}+\theta\mu x}}\\ {{=\theta\bigg[\sum_{t=1}^{k}\sum_{j=1}^{n}E(R_{t}^{j}-\mu A^{j})^{+}+\mu x\bigg]=\theta\overline{{J}}_{k}(x,\mu).}}\end{array} \tag{14}
\]

Therefore, the vector \(\mu^{*}\) that solves (9) is the same for all
value of \(\theta\) . We therefore have by Lemma 1

\[
J_{\theta k}(\theta x) \leq \theta v_{k}(x). \tag{13}
\]

Now consider the fixed- bid- price heuristic with \(\mu^{*}\) as the
fixed vector of bid prices. We will construct a sample path bound on the
revenue with these bid prices using a coupling argument. To do so, we
consider an alternate system which follows the bid- price policy for
accepting sales, but has no capacity constraint; rather, in the
alternate system we subtract a revenue of \(C\) for each set sold in
excess of \(\theta x_{i}\) on each leg \(i\) , where \(C\) is the
uniform upper bound on the itinerary revenues. Consider the net revenues
collected in each system for a given sample path of arrivals. Note in
the two systems, the same revenues are collected up until the time one
or more of the leg capacities is exhausted.

Now, suppose a request arrives for an itinerary \(j\) that needs a leg
\(i\) whose capacity is exhausted. If
\(R_{i}^{j}(\theta) \leq \mu^{*} A^{j}\) , it will be rejected in both
systems and no revenues will be collected in either system. If
\(R_{i}^{j}(\theta) > \mu^{*} A^{j}\) , the request will be rejected in
the original system because of the capacity constraint. In the alternate
system, it will be accepted but at least one penalty of \(C\) will be
charged because one or more leg capacities are exhausted. No revenues
will be collected in the original system, and the alternate system will
realize a loss since \(R_{i}^{j}(\theta) - C \leq 0\) . Moreover, the
alternate system will have even less capacity remaining because it
accepted the request. Hence, it follows that the net revenues in the
alternate system are, pathwise, a lower bound on the revenues obtained
under the bid- price heuristic. Therefore,

\[
\begin{array}{l}{{J_{\theta k}^{H}(\theta x)\geq\sum_{t=1}^{\theta k}\sum_{j=1}^{n}E(R_{t}^{j}(\theta)-\mu^{*}A^{j})^{+}}}\\ {{+\sum_{t=1}^{\theta k}\sum_{j=1}^{n}P(R_{t}^{j}(\theta)>\mu^{*}A^{j})\mu^{*}A^{j}}}\\ {{-C\sum_{i=1}^{m}E(N^{i}-\theta x^{i})^{+},}}\end{array} \tag{14}
\]

where \(N^{i}\) is the number of leg \(i\) seats sold under the bid-
price heuristic, which is given by

\[
N^{i} = \sum_{t = 1}^{\theta k} \sum_{j = 1}^{n} \mathbf{1}\{R_{t}^{j}(\theta) > \mu^{*} A^{j}\} a_{ij}.
\]

The first two terms in (14) are the actual revenues collected; the last
term is the total penalties charged.

Let \(Y_{ijt} = \mathbf{1}\{R_{t}^{j} > \mu^{*} A^{j}\} a_{ij}\) and
note that by (12) and the independence of the vectors \(R_{t}(\theta)\)
, that \(EN^{i} = \theta \sum_{t = 1}^{k} \sum_{j = 1}^{n} EY_{ijt}\)
and that
\(\operatorname {Var}(N^{i}) = \theta \sum_{t = 1}^{k} \sum_{j = 1}^{n} \operatorname {Var}(Y_{ijt})\)
. We now use a bound due to Gallego (1992), which states that for any
random variable \(Z\) with mean \(\mu\) and finite variance
\(\sigma^{2}\) ,

\[
E(Z - z)^{+} \leq \frac{\sqrt{\sigma^{2} + (z - \mu)^{2}} - (z - \mu)}{2}.
\]

Applying this bound to the terms in the last sum in (14) and using the
fact that

\[
EN^{i} = \sum_{t = 1}^{\theta k} \sum_{j = 1}^{n} P(R_{t}^{j} > \mu^{*} A^{j}) a_{ij} \leq \theta x^{i}
\]

by the Kuhn Tucker conditions (10) for \(\mu^{*}\) , implies that

\[
E(N^{i} - \theta x^{i})^{+}\leq \frac{\sqrt{\operatorname{Var}(N^{i}) + (\theta x^{i} - E N^{i})^{2}} - (\theta x^{i} - E N^{i})}{2}
\]

\[
\leq \frac{\sqrt{\operatorname{Var}(N^{i})} + |\theta x^{i} - E N^{i}| - (\theta x^{i} - E N^{i})}{2}
\]

\[
= \frac{\sqrt{\operatorname{Var}(N^{i})}}{2}. \tag{15}
\]

Also note that by the Kuhn Tucker conditions (10),
\(\mu^{*}(\Sigma_{t = 1}^{\theta k}\Sigma_{j = 1}^{n}P(R_{t}^{j}(\theta) > \mu^{*}A^{j})A^{j} - \theta x) = 0\)
, so that

\[
\sum_{t = 1}^{\theta k}\sum_{j = 1}^{n}P(R_{t}^{j}(\theta) > \mu^{*}A^{j})\mu^{*}A^{j} = \theta \mu^{*}x. \tag{16}
\]

Using (15) and (16) in the second and third sums in (14) and using (13),
we obtain

\[
J_{\theta k}^{H}(\theta x)\geq \sum_{t = 1}^{\theta k}\sum_{j = 1}^{n}E(R_{t}^{j}(\theta) - \mu^{*}A^{j})^{+}
\]

\[
+\theta \mu^{*}x - \frac{C}{2}\sqrt{\theta}\sum_{i = 1}^{m}\sqrt{\sum_{t = 1}^{k}\sum_{j = 1}^{n}\operatorname{Var}(Y_{ijt})}
\]

\[
= \theta v_{k}(x) - O(\sqrt{\theta}),
\]

which completes the proof.

\section{4.3. Uniqueness of the Asymptotic Bid
Prices}\label{uniqueness-of-the-asymptotic-bid-prices}

We next address the uniqueness of the asymptotically optimal bid prices.
Although the upper bound problem (9) is convex in \(\mu\) , in general
it is not strictly convex and hence it may not have a unique solution.

To see this, we can write the function \(\overline{J}_{k}(x,\mu)\) as

\[
\overline{J}_{k}(x,\mu) = g(\mu A) + \mu x. \tag{17}
\]

where \(g:R^{n}\to R\) is defined as

\[
g(r) = \sum_{t = 1}^{k}\sum_{j = 1}^{n}E(R_{t}^{j} - r^{j})^{+}, \tag{18}
\]

and \(r = (r^{1},\ldots ,r^{n})\) . Now if there exists a \(t\) such
that \(P(R_{t}^{j} > r^{j}) > 0\) , \(\forall r^{j}\geq 0\) , then \(g\)
is strictly convex on \(R^{n}\) .

However, even if \(g\) is strictly convex, in general \(g(\mu A)\) is
only (weakly) convex in \(\mu\) . It is not hard to see that the
function \(g(\mu A)\) will be strictly convex in \(\mu\) if and

only if \(xA = yA\) implies \(x = y\) , which is equivalent to the
condition: \(xA = 0\) implies \(x = 0\) . But this is true if and only
if \(A\) has rank \(m\) . Therefore, we have the following sufficient
condition for uniqueness:

PROPOSITION 3. Suppose for all \(j\) there exists a \(t\) such that
\(P(R_{t}^{j} > r) > 0\) on \([0, + \infty)\) . Further, suppose rank
\((A) = m\) . Then the solution to (9) (and hence the vector of
asymptotically optimal bid prices) is unique.

In most practical settings, we would expect \(n\geqslant m\) and hence
it is highly likely that \(A\) will have rank \(m\) . In this case,
sufficiently large tails on the fare distributions will result in unique
asymptotically optimal bid prices.

However, multiple asymptotically optimal bid- price can occur if fare
distributions are highly concentrated. For example, one can easily
construct situations in which there is a range of bid prices that are
high enough to block a low fare class while still being low enough to
allow higher fare classes to book; each value produces the same
acceptance decision (with probability one) and hence all are
asymptotically optimal bid prices. Alternatively, it is possible that
because rank \((A)< m\) multiple solutions to (9) exist. As a simple
example, consider the case of two legs in series \((m = 2)\) with one
itinerary \((n = 1)\) that traverses both legs. In this case
\(\operatorname {rank}(A) = 1< m\) , and multiple solutions exist.
Specifically, it is easy to see in this case \(\overline{J}_{k}(x,\mu)\)
depends on \(\mu\) only through the sum \(\mu^{1} + \mu^{2}\) .

\section{4.4. Bid Prices and Opportunity
Cost}\label{bid-prices-and-opportunity-cost}

The above observations illustrate an important point; namely, there is
not a one- to- one correspondence between optimal bid prices and the
opportunity cost of leg capacity. That is, one can generate examples of
bid prices that give near optimal accept/deny decisions but at the same
time are very poor approximations to the marginal value of leg capacity.

As a simple example of this difference, consider a single- leg problem
in which high revenue fare classes arrive strictly before low revenue
fare classes. In this case, it is clear that it is optimal to accept
arrivals in first- come- first- serve (FCFS) order. Therefore, a
constant bid price of zero is optimal. On the other hand, the
opportunity cost \(J_{k}(x) - J_{k}(x - 1)\) at each point in time \(k\)
is certainly not zero. In other words, while it is sufficient to compare
the revenue to the true opportunity cost

\(J_{k}(x) - J_{k}(x - 1)\) at each point in time to make optimal
accept/deny decisions, it is not necessary to do so; other threshold
values may produce the same accept/deny decision and same optimal
revenues, as the value of zero does in this case.

One might argue that the real goal is to make the right accept/deny
decision and therefore it is not worth worrying about the difference
between optimal bid- price values and opportunity costs; however, in
practice a good estimate of opportunity cost is often essential. In
particular, for special event requests- especially ad- hoc group
bookings- which are typically not part of the forecast, one needs an
accurate assessment of opportunity cost to make a good decision.

The point, simply, is that one has to be careful about the
interpretation of the bid prices produced by any optimization algorithm.
Ideally, we would like the sum of leg bid prices along an itinerary to
represent the itinerary's opportunity cost. On the other hand, due to
``degeneracy'' of the value function, this may not always be achievable.
Yet despite this difficulty, Theorem 1 shows that a properly constructed
bid- price control rule is still asymptotically optimal against
forecasted demand. The algorithmic challenge, therefore, is to construct
bid prices which produce near- optimal acceptance decisions against
forecasted demand, while simultaneously providing good estimates of the
opportunity cost (whenever possible), so that special event (group)
requests can be properly evaluated.

\section{5. A Unified View of Bid Price Approximation
Schemes}\label{a-unified-view-of-bid-price-approximation-schemes}

It is quite helpful, to view bid price methods as corresponding to
various approximations of the optimal value function. That is, a given
approximation method \(A\) yields a function \(J_{k}^{A}(x)\) that
approximates \(J_{k}(x)\) . The bid prices are then the gradients of
\(J_{k}^{A}(x)\) , i.e.

\[
J_{k}(z) - J_{k}(x - A^{j})\approx \nabla_{x}J_{k}^{A}(x)A^{j},
\]

If the gradient does not exist, then \(\nabla_{x}J_{k}^{A}(x)\) above is
typically replaced (at least implicitly) by a subgradient of
\(J_{k}^{A}(x)\) . This interpretation of approximations schemes raises
two important questions: Is \(J_{k}^{A}(x)\) a good approximation of the
value function? And more importantly, is \(\nabla_{x}J_{k}^{A}(x)A^{j}\)
a good approximation of the opportunity cost? In this section, we
examine these questions for two popular approximation schemes and also
the asymptotic bid prices developed in Theorem 1.

cost? In this section, we examine these questions for two popular
approximation schemes and also the asymptotic bid prices developed in
Theorem 1.

Before proceeding, we note that in actual applications, the
approximation \(J_{k}^{A}(x)\) is usually resolved frequently to allow
the vector of bid prices to adjust to changes in remaining capacity
\(x\) and remaining time \(k\) . In this section, we assume the
approximations are solved for each \(x\) and \(k\) and compare the
resulting bid prices to optimal bid prices.

\section{5.1. Deterministic Linear Program
(DLP)}\label{deterministic-linear-program-dlp}

The deterministic linear programming method corresponds to the
approximation

\[
\begin{array}{c}{{J_{k}^{\mathrm{LP}}(x)=\min \sum_{j=1}^{n}E R_{j}y_{j},}}\\ {{A y\leq x,}}\\ {{0\leq y\leq E D,}}\end{array}
\]

where \(D = (D_{1},\ldots ,D_{n})\) and \(D_{i}\) denotes the demand to
come for itinerary \(j\) ( \(ED\) is the expected value of \(D\) ) and
\(R_{j}\) is the (possibly random) revenue associated with itinerary
\(j\) . In our earlier notation,
\(D_{j} = \Sigma_{t = 1}^{n} \mathbf{1}\{R_{t}^{\prime} > 0\}\) . The
decision variables \(y_{j}\) represent a discrete (nonnested), static
allocation of capacity to each itinerary \(j\) .

If the constraints \(A y\leq x\) are not degenerate (linearly dependent)
at the optimal solution, then \(\nabla J_{k}^{\mathrm{LP}}(x)\) exists
and is given by the unique vector of optimal dual prices associated with
these constraints; if these constraints are degenerate, then there are
multiple optimal dual price vectors, each of which is only a subgradient
of the function \(J_{k}^{\mathrm{LP}}(x)\) .

The most serious weakness of the DLP formulation is that it considers
only the mean demand and ignores all other distributional information.
As a consequence, the dual values are zero on any leg that has a mean
demand less than capacity. Despite this deficiency, Williamson's (1992,
Chapter 6) extensive simulation studies showed that with frequent
reoptimization, the performance of DLP bid prices is quite good,
producing higher revenue than both probabilistic math programming models
(see below) and a variety of leg- based EMSR heuristics.

5.2. Probabilistic Nonlinear Program (PNLP) The probabilistic nonlinear
programming method corresponds to the approximation

\[
J_{k}^{\mathrm{PNLP}}(x) = \min \sum_{j = 1}^{n}E R_{j}E\min \{D_{j},y_{j}\} ,
\]

\[
A y\leq x,
\]

\[
y = 0,
\]

where again \(D_{j}\) and \(R_{j}\) are defined as in the DLP case. As
in the DLP, the decision variables \(y_{j}\) represent a discrete,
static allocation of capacity to each itinerary \(j\) . If the
constraints \(A y \leq x\) are not degenerate at the optimal solution,
then \(\nabla J_{k}^{\mathrm{LP}}(x)\) exists and is given by the unique
vector of optimal dual prices associated with these constraints; if
these constraints are degenerate, the multiple optimal dual vectors are
subgradients of the function \(J_{k}^{\mathrm{LP}}(x)\) .

This formulation appears somewhat better than the DLP, in that the term
\(E \min \{D_{j}, y_{j}\}\) in the objective function captures the
randomness in demand. However, the assumption of a discrete, static
allocations of capacity to each fare class can lead to poor behavior.
This behavior was demonstrated empirically in Williamson's (1992,
Chapter 6) simulation studies, in which she observed that the PNLP bid
prices consistently produced lower revenues than the DLP bid prices. In
a further computational comparison between the DLP and PNLP, Talluri
(1996) found similar behavior.

To understand the weakness of the PNLP approximation, consider a problem
with \(m = 1\) leg and \(n\) itineraries on the leg, each of which is
identical. Assume each itinerary \(j\) has demand, \(D_{j} \sim D\) ,
where \(D\) is normally distributed with mean \(\mu\) and standard
deviation \(\sigma\) and that each itinerary has the same deterministic
revenue \(r\) . The PNLP formulation is then

\[
J_{k}^{\mathrm{PNLP}}(x) = \min \sum_{j = 1}^{n}r E\min \{D,y_{j}\} ,
\]

\[
\sum_{j = 1}^{n}y_{j} = x,
\]

\[
y\geq 0,
\]

where \(D \sim N(\mu , \sigma)\) . By symmetry, the optimal solution is
\(y_{j} = x / n\) , \(j = 1, \ldots , n\) and hence the Kuhn- Tucker
conditions imply the optimal dual price \(\lambda\) satisfies

\[
\lambda = r P(n D > x) = r\left(1 - \Phi \left(\frac{x - n\mu}{n\sigma}\right)\right), \tag{19}
\]

where \(\Phi (x)\) is the CDF of the standard normal distribution. The
value \(\lambda\) above then forms our estimate of the marginal value of
the \(x\) th seat.

But since all itineraries are identical, the marginal value in this
problem should be the unchanged if we aggregate all \(n\) fare classes
into one fare class with mean \(n\mu\) and variance \(n\sigma^{2}\) .
Aggregating and applying the PNLP we find the optimal dual multiplier in
this case satisfies

\[
\lambda = r P\left(\sum_{j = 1}^{n}D_{j} > x\right) = r\left(1 - \Phi \left(\frac{x - n\mu}{\sqrt{n}\sigma}\right)\right). \tag{20}
\]

If \(x \neq n\mu\) and \(n\) is large, (19) and (20) give very different
estimates of the marginal value. Of course, from first principles we
know the true opportunity cost is completely independent of how we
aggregate (or disaggregate) these identical itineraries.

While on the surface this seems like a contrived example, it is not
unreasonable to expect a similar type of behavior in large hub- and-
spoke networks. For example, if many uncongested in- bound legs have
connecting passengers traveling on a single congested outbound leg and
passengers pay comparable revenues, then the situation is quite similar
to the example above. That is, one would like to treat all passengers as
a single fare class (i.e.~``nest'' the fare classes) but the PNLP
allocates space to each separately, resulting in a distorted estimate of
the marginal value of the leg. In contrast, the DLP method does indeed
posses this ``nesting'' property, since, in the above example,
aggregating all \(n\) fare classes does not change the resulting DLP bid
price.

\section{5.3. Prorated EMSR}\label{prorated-emsr}

Another method for computing estimates of bid prices is to use a
prorated expected marginal seat revenue (PEMSR) scheme. Originally
proposed in Williamson (1992), PEMSR schemes involve allocating a
portion of the revenue of each itinerary to the legs of the itinerary.
One then solves \(m\) leg- level problems using the expected marginal
revenue (EMSR) heuristic proposed by Belobaba (1989). The resulting EMSR
values from each leg are then used as bid prices.

Specifically, let \(\alpha = (\alpha_{1}, \ldots , \alpha_{m})\) be a
non- negative real vector. For each itinerary \(j\) , define new
revenues, one for each leg in the itinerary, by

Table 2 Problem Data for Iterative Allocation Example

\begin{longtable}[]{@{}llll@{}}
\toprule\noalign{}
\endhead
\bottomrule\noalign{}
\endlastfoot
Time (k) & Itin. (A) & Fare & Prob. \\
\multirow{2}{*}{2} & AB & \$100 & 0.5 \\
& CD & \$100 & 0.5 \\
\multirow{2}{*}{1} & ABC & \$1000 & 0.5 \\
& BCD & \$1000 & 0.5 \\
\end{longtable}

\[
R_{ij} = \frac{\alpha_i}{\Sigma_{i\in A_j}\alpha_i} R_{j}, i\in A_j.
\]

Next, treat each leg \(i\) independently as if it received demand
\(D_{j},\) but with reduced revenue \(R_{ij}\) and solve the
corresponding leg- level EMSR.1 The approximation to the value function
is then

\[
J_{k}^{\mathrm{PEMSR}}(x) = \sum_{i = 1}^{m}J_{i}(x_{i},\alpha),
\]

where \(J_{i}(x_{i},\alpha)\) denotes the expected revenue of leg \(i\)
under the allocation \(\alpha\)

Williamson (1992) investigated several methods for determining the
allocation \(\alpha_{j}\) including prorating based on mileage, number
of legs and the relative revenue value of local demand on each leg. Her
conclusion is that none of these fixed allocations is very robust in
general. Indeed, it is not hard to see that if one leg of an itinerary
is highly congested and all others have abundant capacity, then the
revenue of the itinerary should be entirely allocated to the congested
leg. Depending on the realization of demand, however, the congested leg
could be any of the legs on the itinerary; hence, no fixed allocation
scheme can be expected to work well in all cases.

An intriguing idea along these lines, again appearing in Williamson
(1992, p.~107) but not pursued fully there, is to prorate revenues using
an iterative loop. That is, first obtain the marginal values based on
some initial proration scheme and EMSR calculations. Then, use these
marginal values to do the next round of proration and EMSR calculations.
Repeat these iterations until the marginal values (hopefully) converge.
The hope here is that the bid- prices will converge to a near- optimal
set of bid- prices. Unfortunately there is little theoretical
justification behind this idea.

Even if the bid prices converge to some value (it is not entirely clear
if convergence is guaranteed), they do not necessarily converge to a
good set of bid- prices. For example, consider a three- leg line
network, with nodes A, B, C and D. Each of the three legs, AB, BC and
CD, has one remaining seat. Suppose \(t = 2\) and we have data for
itinerary arrivals as shown in Table 2. If we start with an allocation
of fares for \(t = 1\) using equal weights, prorate the fares in period
\(t = 1\) by these weights, and then compute the expected marginal value
of each leg we get \(\mu_{\mathrm{AB}} = 250\) ,
\(\mu_{\mathrm{BC}} = 500\) and \(\mu_{\mathrm{CD}} = 250\) .

The results of repeated applications of this procedure are shown in
Table 3. Note that the bid prices converge to \(\mu_{\mathrm{AB}} = 0\)
, \(\mu_{\mathrm{BC}} = 1000\) and \(\mu_{\mathrm{CD}} = 0\) . However,
by inspection of the data in Table 2, it is clear that we want to reject
both of the itineraries arriving in period \(t = 2\) , so we need
\(\mu_{\mathrm{AB}} > 100\) and \(\mu_{\mathrm{CD}} > 100\) . Such a
policy yields an expected revenue of
{\$\textbackslash)1,000\textbackslash(. Because the iterative proration
scheme produces zero bid prices for legs AB and CD, it accepts both of
the itineraries in period\$} t=2 {\$\textbackslash), generating an
expected revenue of only \textbackslash(\$} 600\$.

One problem with the iterative proration scheme is that once the problem
is broken up into leg- level problems, all network information is lost.
Additional problems can arise due to the fact that many EMSR methods
assume fare classes have ordered arrivals, usually with lower fare
classes arriving before high fare classes, see Belobaba (1987, 1989),
Brumelle et al.~(1990), and Curry

Table 3 Example of Convergence of Iterative Proration Scheme \((t = 1)\)

\begin{longtable}[]{@{}|l|l|l|l|@{}}
\toprule\noalign{}
\endhead
\bottomrule\noalign{}
\endlastfoot
\hline
Iter. & μAB & μBC & μCD \\
\hline
0 & 250.0 & 500.0 & 250.0 \\
\hline
1 & 166.7 & 666.7 & 166.7 \\
\hline
2 & 100 & 800 & 100 \\
\hline
. & . & . & . \\
\hline
. & . & . & . \\
\hline
. & . & . & . \\
\hline
∞ & 0 & 1000 & 0 \\
\hline
\end{longtable}

(1989). In a proration scheme, the regular ordering of arrivals is often
destroyed, and certainly any assumption of low prorated revenues
arriving strictly before high prorated revenues becomes untenable.

\section{5.4. Asymptotic Bid Prices}\label{asymptotic-bid-prices}

The asymptotic analysis provides an alternative approximation approach.
Indeed, note from (9) that \(\nabla_{x}v_{k}(x) = \mu^{*}\) , so we can
view the upper bound \(v_{k}(x)\) as an approximation of \(J_{k}(x)\)
with \(\mu^{*}\) its (sub)gradient. The approximation (9) is somewhat
unique in that it is solved directly in the space of the bid prices
\(\mu ,\) whereas in the DLP and NLP methods, \(\mu\) is a dual value of
a problem whose primal variables are inventory allocations.

The approximation (9) has the ``nesting'' property because the objective
function in (9) sums all arriving itineraries \(j\) whose revenue
exceeds the fixed thresholds \(\mu A^{j}\) . As a result, it does not
suffer from the discrete allocation problem of the PNLP method. For
example, suppose two itineraries \(j_{1}\) and \(j_{2}\) are entered as
separate columns of \(A_{j}\) but in reality \(A^{j_{1}} = A^{j_{2}}\)
and each has the same fare distribution. If these two itineraries were
combined into a new itinerary (by adding the probabilities of arrival in
each period together) the asymptotic bid prices would not change, which
is the correct behavior.

At the same time, the approximation (9) suffers from the same weakness
as the DLP in that its value only depends on the first moment of demand.
Indeed, if, as above, we let \(D_{j}\) denote the demand to come for
itinerary \(j\) and \(R_{j}\) denote the random revenue associated with
itinerary \(j,\) then (9) becomes

\[
v_{k}(x) = \min_{\mu \geq 0}\sum_{j = 1}^{n}ED_{j}E(R_{j} - \mu A_{j})^{+} + \mu^{T}x,
\]

which only depends on \(ED_{j}\)

The problem here is that the asymptotic analysis is too ``coarse'' to
capture some of the second- order stochastic effects which the PNLP
captures. As a result, as in the DLP case, the asymptotic bid- prices
will turn out to be zero if the mean demand on a leg is strictly less
than its capacity. This is indeed correct behavior asymptotically, but
for demands with high variance near capacity, the actual bid price could
be significantly larger than zero, possibly even more than some of the
fares of the lower fare classes. Indeed, under appropriate scaling, one
can show that, as fare variances tend to zero, the asymptotic bid prices
from (9) approach the DLP bid prices. One can then combine this result
with Theorem 1 to show that as fare variances tend to zero, the DLP bid
prices are also asymptotically optimal.

However, the asymptotic approximation, unlike the DLP and PNLP methods,
accounts for variability in itinerary revenues, which provides a
significant advantage in approximating optimal bid prices when fares
vary significantly within itinerary / fare- classes.

As a simple example of this effect, consider a singleleg problem with
only one fare class. Suppose the actual fares \(R\) vary and have
distribution \(F(r)\) with mean \(ER\) If the mean demand, \(ED_{i}\) is
significantly higher than capacity, \(x_{i}\) the bid price produced by
DLP and PNLP methods both approach \(ER\) . That is, reducing the leg
capacity by one almost certainly results in a lost sale, which each of
these models values at \(ER\) . Under an optimal bid- price policy,
however, the bid price rises above \(ER\) as the demand / capacity ratio
increases. This occurs because, with many requests to choose from, it is
optimal to be selective and accept only the higher fares within the fare
class (i.e.~fares in the ``right tail'' of the distribution) rather than
accepting all fares. From (9), one can show that the optimal bid price
in the above example tends to a value \(r^{*}\) satisfying

\[
ED(1 - F(r^{*})) = x.
\]

Depending on the distribution of \(F_{\prime}\) the value of \(r^{*}\)
can be significantly higher than \(ER\) ; hence the DLP and PNLP methods
will under- estimates the optimal bid price. A complimentary effect can
occur when the demand / capacity ratio is small; in this case, the DLP
and NLP models may tend to over- estimate the optimal bid price.



\end{document}
