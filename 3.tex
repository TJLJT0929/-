\begin{document}

\section{5 Numerical Studies}\label{numerical-studies}

In this section, we validate the methodology and examine the theoretical
results introduced in Sections 3 and 4. We conducted comprehensive
simulation studies, but due to the page limit, we relegate the results
to the Appendix A and only briefly summarize our key findings here. We
first validate consistency and asymptotic distribution of the spectral
estimator, check the efficacy of the Gaussian bootstrap. All simulations
match with our theoretical results perfectly. We also provide examples
for constructing one- sample and two- sample confidence intervals for
ranks and carrying out hypothesis testing Examples 3.2- 3.5. Finally, we
empirically investigate the connections between the spectral method and
MLE. In particular, we found the two- step or oracle weight spectral
method behaves almost identically to MLE.

\section{6 Real Data Analysis}\label{real-data-analysis}

We present the spectral ranking inferences for two real datasets in this
section. The first one is about the ranking of statistics journals,
which is based on pairwise comparisons (Journal B citing Journal A means
A is preferred over B, which is consistent with the fact that good
papers usually

have higher citations). In particular, we can test for two periods of
time, whether the journal ranking has changed significantly. The second
application is about movie ranking, which is based on multiway
comparisons of different sizes (3 or 4 movies are given to people to
rank). The movie comparison graph shows strong heterogeneity in node
degrees. Therefore, this comparison graph should be better modeled as a
fixed graph without homogeneous sampling. In both cases, we will report
the results from the Two- Step spectral estimator and the vanilla
spectral estimator in Appendix B.

\section{6.1 Ranking of Statistics
Journals}\label{ranking-of-statistics-journals}

In this section, we study the Multi- Attribute Dataset on Statisticians
(MADStat) which contains citation information from 83,331 papers
published in 36 journals between 1975- 2015. The data set was collected
and studied by Ji et al.~(2022, 2023+).

We follow Ji et al.~(2023+)'s convention to establish our pairwise
comparison data. We will use journals' abbreviations given in the data.
We refer interested readers to the complete journal names on the data
website
https://dataverse.harvard.edu/dataset.xhtml?persistentId=doi:10.7910/DVN/V7VUF0.
Firstly, we excluded three probability journals---AOP, PTRF, and AIHPP,
due to their fewer citation exchanges with other statistics
publications. Hence, our study comprises of a total of 33 journals.
Secondly, we only examine papers published between 2006- 2015. For
instance, if we treat 2010 as our reference year, we only count
comparison results indicating `Journal A is superior to Journal B' if,
and only if, a paper published in Journal B in 2010 has cited another
paper that was published in Journal A between the years 2001 and 2010.
This approach favors more recent citations, thus allowing the journal
rankings to better reflect the most current trends and information.
Finally, we chose to divide our study into two periods, 2006- 2010 and
2011- 2015, to detect the possible rank changes of journals.

Utilizing the data, we showcase the ranking inference results,
summarized in Table 1. These results include two- sided, one- sided, and
uniform one- sided confidence intervals for ranks within each of the two
time periods (2006- 2010 and 2011- 2015). We calculate these intervals
using the Two- Step Spectral method over the fixed comparison graph (the
results based on the one- step Vanilla Spectral method are presented in
Appendix B.1) and using the bootstrap method detailed in Sections 3.3 -
3.5.

From Table 1, we can easily get answers to the following questions on
ranking inference. For example, is each journal's rank maintained
unchanged across the two time periods? At a significance level of
\(\alpha = 10\%\) , we find that the ranks of the following journals in
alphabetical order demonstrate significant differences between the two
time frames (Example 3.4):

AISM, AoAS, Biost, CSTM, EJS, JMLR, JoAS, JSPI.

This aligns with real- world observations. For instance, JMLR, EJS, and
AOAS are newer journals that emerged after 2000. As a result, these
journals received fewer citations in the earlier period and got
recognized better in the more recent period.

We then turn our attention to the stability of the highest- ranked
journals. Referring to Table 1, we observe that the top four journals
(AoS, Bka, JASA, and JRSSB, known as the Big- Four

Table 1: Ranking inference results for 33 journals in 2006-2010 and
2011-2015 based on the TwoStep Spectral estimator. For each time period,
there are 6 columns. The first column \(\tilde{\theta}\) denotes the
estimated underlying scores. The second through the fifth columns denote
their relative ranks, two-sided, one-sided, and uniform one-sided
confidence intervals for ranks with coverage level \(95\%\)
respectively. The sixth column denotes the number of comparisons in
which each journal is involved.

\begin{longtable}[]{@{}lllllllllllll@{}}
\toprule\noalign{}
\endhead
\bottomrule\noalign{}
\endlastfoot
\multirow{2}{*}{Journal} & \multicolumn{6}{l}{%
2006 -- 2010} & \multicolumn{6}{l@{}}{%
2011 -- 2015} \\
& θ & r & TCI & OCI & UOCI & Count & θ & r & TCI & OCI & UOCI & Count \\
JRSSB & 1.654 & 1 & {[}1,1{]} & {[}1,n{]} & {[}1,n{]} & 5282 & 1.553 & 1
& {[}1,2{]} & {[}1,n{]} & {[}1,n{]} & 5513 \\
AoS & 1.206 & 3 & {[}2,4{]} & {[}2,n{]} & {[}2,n{]} & 7674 & 1.522 & 2 &
{[}1,2{]} & {[}1,n{]} & {[}1,n{]} & 11316 \\
Bka & 1.316 & 2 & {[}2,3{]} & {[}2,n{]} & {[}2,n{]} & 5579 & 1.202 & 3 &
{[}3,3{]} & {[}3,n{]} & {[}3,n{]} & 6399 \\
JASA & 1.165 & 4 & {[}3,4{]} & {[}3,n{]} & {[}3,n{]} & 9652 & 1.064 & 4
& {[}4,4{]} & {[}4,n{]} & {[}4,n{]} & 10862 \\
JMLR & -0.053 & 20 & {[}14,25{]} & {[}15,n{]} & {[}13,n{]} & 1100 &
0.721 & 5 & {[}5,7{]} & {[}5,n{]} & {[}5,n{]} & 2551 \\
Biost & 0.288 & 13 & {[}10,18{]} & {[}10,n{]} & {[}9,n{]} & 2175 & 0.591
& 6 & {[}5,9{]} & {[}5,n{]} & {[}5,n{]} & 2727 \\
Bcs & 0.820 & 5 & {[}5,7{]} & {[}5,n{]} & {[}5,n{]} & 6614 & 0.571 & 7 &
{[}5,9{]} & {[}6,n{]} & {[}5,n{]} & 6450 \\
StSci & 0.668 & 7 & {[}5,9{]} & {[}5,n{]} & {[}5,n{]} & 1796 & 0.437 & 8
& {[}6,13{]} & {[}6,n{]} & {[}6,n{]} & 2461 \\
Sini & 0.416 & 10 & {[}9,14{]} & {[}9,n{]} & {[}8,n{]} & 3701 & 0.374 &
9 & {[}8,13{]} & {[}8,n{]} & {[}8,n{]} & 4915 \\
JRSSA & 0.239 & 14 & {[}10,20{]} & {[}10,n{]} & {[}9,n{]} & 893 & 0.370
& 10 & {[}6,13{]} & {[}8,n{]} & {[}6,n{]} & 865 \\
JCGS & 0.605 & 8 & {[}6,9{]} & {[}6,n{]} & {[}6,n{]} & 2493 & 0.338 & 11
& {[}8,13{]} & {[}8,n{]} & {[}8,n{]} & 3105 \\
Bern & 0.793 & 6 & {[}5,8{]} & {[}5,n{]} & {[}5,n{]} & 1575 & 0.336 & 12
& {[}8,13{]} & {[}8,n{]} & {[}8,n{]} & 2613 \\
ScaJS & 0.528 & 9 & {[}7,12{]} & {[}7,n{]} & {[}6,n{]} & 2442 & 0.258 &
13 & {[}8,13{]} & {[}9,n{]} & {[}8,n{]} & 2573 \\
JRSSC & 0.113 & 15 & {[}11,22{]} & {[}11,n{]} & {[}11,n{]} & 1401 &
0.020 & 14 & {[}14,19{]} & {[}14,n{]} & {[}12,n{]} & 1492 \\
AoAS & -1.463 & 30 & {[}30,33{]} & {[}30,n{]} & {[}30,n{]} & 1258 &
-0.017 & 15 & {[}14,20{]} & {[}14,n{]} & {[}14,n{]} & 3768 \\
CanJS & 0.101 & 17 & {[}11,22{]} & {[}11,n{]} & {[}11,n{]} & 1694 &
-0.033 & 16 & {[}14,20{]} & {[}14,n{]} & {[}14,n{]} & 1702 \\
JSPI & -0.327 & 26 & {[}24,26{]} & {[}24,n{]} & {[}22,n{]} & 6565 &
-0.046 & 17 & {[}14,20{]} & {[}14,n{]} & {[}14,n{]} & 6732 \\
JTSA & 0.289 & 12 & {[}9,18{]} & {[}10,n{]} & {[}8,n{]} & 751 & -0.101 &
18 & {[}14,22{]} & {[}14,n{]} & {[}14,n{]} & 1026 \\
JMVA & -0.126 & 22 & {[}17,25{]} & {[}17,n{]} & {[}15,n{]} & 5833 &
-0.148 & 19 & {[}14,22{]} & {[}15,n{]} & {[}14,n{]} & 6454 \\
SMed & -0.131 & 23 & {[}17,25{]} & {[}18,n{]} & {[}17,n{]} & 6626 &
-0.242 & 20 & {[}18,25{]} & {[}18,n{]} & {[}17,n{]} & 6857 \\
Extrem & -2.099 & 33 & {[}30,33{]} & {[}31,n{]} & {[}30,n{]} & 173 &
-0.312 & 21 & {[}16,30{]} & {[}18,n{]} & {[}14,n{]} & 487 \\
AISM & 0.317 & 11 & {[}9,18{]} & {[}10,n{]} & {[}9,n{]} & 1313 & -0.359
& 22 & {[}19,30{]} & {[}20,n{]} & {[}18,n{]} & 1605 \\
EJS & -1.717 & 32 & {[}30,33{]} & {[}30,n{]} & {[}30,n{]} & 1366 &
-0.367 & 23 & {[}20,29{]} & {[}20,n{]} & {[}19,n{]} & 4112 \\
SPLet & -0.033 & 19 & {[}15,25{]} & {[}15,n{]} & {[}13,n{]} & 3651 &
-0.384 & 24 & {[}21,29{]} & {[}21,n{]} & {[}19,n{]} & 4439 \\
CSDA & -0.975 & 29 & {[}27,30{]} & {[}27,n{]} & {[}27,n{]} & 6732 &
-0.467 & 25 & {[}21,30{]} & {[}21,n{]} & {[}21,n{]} & 8717 \\
JNS & -0.255 & 25 & {[}19,26{]} & {[}20,n{]} & {[}17,n{]} & 1286 &
-0.484 & 26 & {[}21,30{]} & {[}21,n{]} & {[}21,n{]} & 1895 \\
ISRe & 0.082 & 18 & {[}11,25{]} & {[}11,n{]} & {[}10,n{]} & 511 & -0.491
& 27 & {[}21,30{]} & {[}21,n{]} & {[}20,n{]} & 905 \\
AuNZ & 0.108 & 16 & {[}11,23{]} & {[}11,n{]} & {[}10,n{]} & 862 & -0.504
& 28 & {[}21,30{]} & {[}21,n{]} & {[}20,n{]} & 816 \\
JClas & -0.185 & 24 & {[}15,26{]} & {[}15,n{]} & {[}11,n{]} & 260 &
-0.535 & 29 & {[}18,30{]} & {[}20,n{]} & {[}14,n{]} & 224 \\
SCmp & -0.096 & 21 & {[}15,25{]} & {[}15,n{]} & {[}14,n{]} & 1309 &
-0.561 & 30 & {[}23,30{]} & {[}24,n{]} & {[}21,n{]} & 2650 \\
Bay & -1.494 & 31 & {[}30,33{]} & {[}30,n{]} & {[}27,n{]} & 279 & -1.102
& 31 & {[}31,32{]} & {[}31,n{]} & {[}30,n{]} & 842 \\
CSTM & -0.843 & 27 & {[}27,29{]} & {[}27,n{]} & {[}27,n{]} & 2975 &
-1.296 & 32 & {[}31,32{]} & {[}31,n{]} & {[}31,n{]} & 4057 \\
JoAS & -0.912 & 28 & {[}27,30{]} & {[}27,n{]} & {[}27,n{]} & 1055 &
-1.904 & 33 & {[}33,33{]} & {[}33,n{]} & {[}33,n{]} & 2780 \\
\end{longtable}

in statistics) maintain their positions strongly across different time
periods. Furthermore, with a significance level of \(\alpha = 10\%\) ,
we reject the hypothesis that the top seven ranked items remain constant
across the two time periods (Example 3.5). Specifically, for 2006- 2010,
the \(95\%\) confidence set for the top- 7 items includes:

AoS, Bern, Bcs, Bka, JASA, JUGS JRSSB, ScaJS, StSci.

And for 2011- 2015, the \(95\%\) confidence set for the top- 7 items
includes:

AoS, Bcs, Biost, Bka, JASA, JMLR, JRSSA, JRSSB, StSci.

Clearly, these sets intersect only at 6 items, smaller than 7,
reflecting a shift in the rankings over the two periods.

\section{6.2 Ranking of Movies}\label{ranking-of-movies}

In this section, we construct confidence intervals for the ranks of
movies or television series featured within the The Netflix Prize
competition (Bennett et al., 2007), which aims to enhance the precision
of the Netflix recommendation algorithm. The dataset we examine
corresponds to 100 random 3 and 4 candidate elections drawn from Data
Set 1 of Mattei et al.~(2012), which was extracted from the original The
Netflix Prize dataset, devoid of any ties. The dataset contains 196
movies in total and 163759 3- way or 4- way comparisons. For simplicity,
we only use the top ranked movie, although it is straightforward to
apply the multi- level breaking to use the complete ranking data. This
dataset can be accessed at the website:
https://www.preflib.org/dataset/00004.

We compute two- sided, one- sided, and uniform one- sided confidence
intervals employing the bootstrap method as described in Sections 3.3 -
3.5, based on the Two- Step Spectral method. The results are shown in
Table 2. Additionally, the results from the one- step Vanilla Spectral
method are detailed in Table 11 in Appendix B.2.

First note that the heterogeneity of the number of comparisons
(``Count'' column in Table 2) is more severe relative to the journal
ranking data, which leads to the adaptive length of rank confidence
intervals. From ``OCI'' column of Table 2, we can test whether each
individual movie belong to the top- 10 rated movies ( \(K = 10\) in
Example 3.2). We end up failing to reject the hypothesis for the first
12 films listed in Table 2. Furthermore, we can use the uniform one-
sided confidence interval (``UOCI'') column to build candidate
confidence set for the true top- 10 movies ( \(K = 10\) in Example 3.3).
The result suggests that except for ``Harry Potter and the Sorcerer's
Stone'', we should include the other top 15 ranked films in our top- 10
confidence set. The reason we cannot exclude the films ``High Noon'' and
``Sex and the City: Season 6: Part 2'', despite these movies rank lower
than ``Harry Potter and the Sorcerer's Stone'', is due to their fewer
number of comparisons and thus wider confidence intervals. Similarly,
the top- 5 confidence set is the first 9 movies in Table 2.

\begin{longtable}[]{@{}|l|l|l|l|l|l|l|l|@{}}
\toprule\noalign{}
\endhead
\bottomrule\noalign{}
\endlastfoot
\hline
Movie & θ & τ & TCI & OCI & UOCI & Count & \\
\hline
The Silence of the Lambs & 3.002 & 1 & {[}1, 1{]} & {[}1, n{]} & {[}1,
n{]} & 19589 & \\
\hline
The Green Mile & 2.649 & 2 & {[}2, 4{]} & {[}2, n{]} & {[}2, n{]} & 5391
& \\
\hline
Shrek (Full-screen) & 2.626 & 3 & {[}2, 4{]} & {[}2, n{]} & {[}2, n{]} &
19447 & \\
\hline
The X-Files: Season 2 & 2.524 & 4 & {[}2, 7{]} & {[}2, n{]} & {[}2, n{]}
& 1114 & \\
\hline
Ray & 2.426 & 5 & {[}4, 7{]} & {[}4, n{]} & {[}4, n{]} & 7905 & \\
\hline
The X-Files: Season 3 & 2.357 & 6 & {[}4, 10{]} & {[}4, n{]} & {[}2,
n{]} & 1442 & \\
\hline
The West Wing: Season 1 & 2.278 & 7 & {[}4, 10{]} & {[}4, n{]} & {[}4,
n{]} & 3263 & \\
\hline
National Lampoon\textquotesingle s Animal House & 2.196 & 8 & {[}6,
10{]} & {[}6, n{]} & {[}5, n{]} & 10074 & \\
\hline
Aladdin: Platinum Edition & 2.154 & 9 & {[}6, 13{]} & {[}6, n{]} & {[}5,
n{]} & 3355 & \\
\hline
Seven & 2.143 & 10 & {[}6, 11{]} & {[}7, n{]} & {[}6, n{]} & 16305 & \\
\hline
Back to the Future & 2.030 & 11 & {[}9, 15{]} & {[}9, n{]} & {[}8, n{]}
& 6428 & \\
\hline
Blade Runner & 1.968 & 12 & {[}10, 16{]} & {[}10, n{]} & {[}9, n{]} &
5597 & \\
\hline
Harry Potter and the Sorcerer\textquotesingle s Stone & 1.842 & 13 &
{[}12, 22{]} & {[}12, n{]} & {[}11, n{]} & 7976 & \\
\hline
High Noon & 1.821 & 14 & {[}11, 25{]} & {[}11, n{]} & {[}10, n{]} & 1902
& \\
\hline
Sex and the City: Season 6: Part 2 & 1.770 & 15 & {[}11, 30{]} & {[}11,
n{]} & {[}8, n{]} & 532 & \\
\hline
Jaws & 1.749 & 16 & {[}13, 25{]} & {[}13, n{]} & {[}13, n{]} & 8383 & \\
\hline
The Ten Commandments & 1.735 & 17 & {[}13, 28{]} & {[}13, n{]} & {[}12,
n{]} & 2186 & \\
\hline
Willy Wonka \&amp; the Chocolate Factory & 1.714 & 18 & {[}13, 26{]} &
{[}13, n{]} & {[}13, n{]} & 9188 & \\
\hline
Stalag & 17 & 697 & 19 & {[}12, 34{]} & {[}12, n{]} & 11, n{]} & 806 \\
\hline
Unforgiven & 1.633 & 20 & {[}14, 29{]} & {[}14, n{]} & {[}14, n{]} &
9422 & \\
\hline
\end{longtable}

Table 2: Ranking inference results for top- 20 Netflix movies or tv
series based on the Two- Step Spectral estimator. The first column
\(\widetilde{\theta}\) denotes the estimated underlying scores. The
second through the fifth columns denote their relative ranks, two-
sided, one- sided, and uniform one- sided confidence intervals for ranks
with coverage level \(95\%\) , respectively. The sixth column denotes
the number of comparisons in which each movie is involved.

\section{7 Conclusion and Discussion}\label{conclusion-and-discussion}

In this work, we studied the performance of the spectral method in
preference score estimation, quantified the asymptotic distribution of
the estimated scores, and explored one- sample and two- sample inference
on ranks. In particular, we worked with general multiway comparisons
with fixed comparison graphs, where the size of each comparison can vary
and can be as low as only one. This is much closer to real applications
than the homogeneous random sampling assumption imposed in the BTL or PL
models. The applications of journal ranking and movie ranking have
demonstrated the clear usefulness of our proposed methodologies.
Furthermore, we studied the relationship between the spectral method and
the MLE in terms of estimation efficiency and revealed that with a
carefully chosen weighting scheme, the spectral method can approximately
achieve the same efficiency as the MLE, which is also verified using
numerical simulations. Finally, to the best of our knowledge, it is the
first time that effective two- sample rank testing methods have been
proposed in the literature.

Although we have made significant improvements in relaxing conditions,
the role of general comparison graphs is still not fully understood,
especially in the setting of multiway comparisons. Questions like how to
design a better sampling regime, either online or offline, remain open.
In addition, the spectral method essentially encodes multiway
comparisons into pairwise comparisons, where the encoding will break
data independence. The best encoding or breaking method should be
further investigated. Finally, a set of recent works on ranking
inferences opens the door to many possibilities of theoretical studies
on ranking inferences and related problems such as assortment
optimization, under the setting of, say, rank time series, rank change
point detection, rank panel data, recommendation based on rank
inferences, uncertainty quantification and inference for properties of
the optimal assortment. These may find potential application in numerous
management settings.

\section{References}\label{references}

Aouad, A., Farias, V., Levi, R. and Segev, D. (2018). The
approximability of assortment optimization under ranking preferences.
\emph{Operations Research}, \textbf{66} 1661- 1669. Avery, C. N.,
Glickman, M. E., Hoxby, C. M. and Metrick, A. (2013). A revealed
preference ranking of us colleges and universities. \emph{The Quarterly
Journal of Economics}, \textbf{128} 425- 467. Azari Soufiani, H., Chen,
W., Parkes, D. C. and Xia, L. (2013). Generalized method- of- moments
for rank aggregation. \emph{Advances in Neural Information Processing
Systems}, \textbf{26. Baltrunas, L., Makcinskas, T. and Ricci, F.
(2010). Group recommendations with rank aggregation and collaborative
filtering. In \emph{Proceedings of the fourth ACM conference on
Recommender systems}.Bennett, J., Lanning, S. et al.~(2007). The Netflix
Prize. In \emph{Proceedings of KDD cup and workshop}, vol.~2007. New
York.Caron, F., Teh, Y. W. and Murphy, T. B. (2014). Bayesian
nonparametric Plackett- Luce models for the analysis of preferences for
college degree programmes. \emph{The Annals of Applied Statistics},} 8**
1145- 1181. Chen, P., Gao, C. and Zhang, A. Y. (2022). Partial recovery
for top- \(K\) ranking: Optimality of mle and suboptimality of the
spectral method. \emph{The Annals of Statistics}, \textbf{50} 1618-
1652. Chen, X., Krishnamurthy, A. and Wang, Y. (2023). Robust dynamic
assortment optimization in the presence of outlier customers.
\emph{Operations Research}.Chen, X., Wang, Y. and Zhou, Y. (2020).
Dynamic assortment optimization with changing contextual information.
\emph{The Journal of Machine Learning Research}, \textbf{21} 8918- 8961.
Chen, Y., Fan, J., Ma, C. and Wang, K. (2019). Spectral method and
regularized mle are both optimal for top- \(K\) ranking. \emph{Annals of
statistics}, \textbf{47} 2204. Chen, Y. and Suh, C. (2015). Spectral
mle: Top- \(K\) rank aggregation from pairwise comparisons. In
\emph{International Conference on Machine Learning}. PMLR.Cheng, W.,
Dembzynski, K. and Hüllermeier, E. (2010). Label ranking methods based
on the Plackett- Luce model. In \emph{ICML}.Chernozhukov, V.,
Chetverikov, D. and Kato, K. (2017). Central limit theorems and
bootstrap in high dimensions. \emph{The Annals of Probability},
\textbf{45} 2309- 2352. Chernozhukov, V., Chetverikov, D., Kato, K. and
Koike, Y. (2019). Improved central limit theorem and bootstrap
approximations in high dimensions. \emph{arXiv preprint
arXiv:1912.10529}.Davis, J. M., Gallego, G. and Topaloglu, H. (2014).
Assortment optimization under variants of the nested logit model.
\emph{Operations Research}, \textbf{62} 250- 273.

Dwork, C., Kumar, R., Naor, M. and Sivakumar, D. (2001). Rank
aggregation methods for the web. In Proceedings of the 10th
international conference on World Wide Web.Fan, J., Hou, J. and Yu, M.
(2022a). Uncertainty quantification of mle for entity ranking with
covariates. arXiv preprint arXiv:2212.09961. Fan, J., Lou, Z., Wang, W.
and Yu, M. (2022b). Ranking inferences based on the top choice of
multiway comparisons. arXiv preprint arXiv:2211.11957. Gallego, G. and
Topaloglu, H. (2014). Constrained assortment optimization for the nested
logit model. Management Science, 60 2583- 2601. Gao, C., Shen, Y. and
Zhang, A. Y. (2021). Uncertainty quantification in the Bradley- Terry-
Luce model. arXiv preprint arXiv:2110.03874. Guiver, J. and Snelson, E.
(2009). Bayesian inference for Plackett- Luce ranking models. In
proceedings of the 26th annual international conference on machine
learning.Hajek, B., Oh, S. and Xu, J. (2014). Minimax- optimal inference
from partial rankings. Advances in Neural Information Processing
Systems, 27. Han, R. and Xu, Y. (2023). A unified analysis of
likelihood- based estimators in the Plackett- Luce model. arXiv preprint
arXiv:2306.02821. Han, R., Ye, R., Tan, C. and Chen, K. (2020).
Asymptotic theory of sparse Bradley- Terry model. The Annals of Applied
Probability, 30 2491- 2515. Hunter, D. R. (2004). MM algorithms for
generalized Bradley- Terry models. The annals of statistics, 32 384-
406. Jang, M., Kim, S. and Suh, C. (2018). Top- \(K\) rank aggregation
from \(m\) - wise comparisons. IEEE Journal of Selected Topics in Signal
Processing, 12 989- 1004. Jang, M., Kim, S., Suh, C. and Oh, S. (2016).
Top- \(K\) ranking from pairwise comparisons: When spectral ranking is
optimal. arXiv preprint arXiv:1603.04153. Ji, P., Jin, J., Ke, Z. T. and
Li, W. (2022). Co- citation and co- authorship networks of
statisticians. Journal of Business \& Economic Statistics, 40 469- 485.
Ji, P., Jin, J., Ke, Z. T. and Li, W. (2023+). Meta- analysis on
citations for statisticians. To Appear.Johnson, V. E., Deaner, R. O. and
Van Schaik, C. P. (2002). Bayesian analysis of rank data with
application to primate intelligence experiments. Journal of the American
Statistical Association, 97 8- 17. Li, H., Simchi- Levi, D., Wu, M. X.
and Zhu, W. (2019). Estimating and exploiting the impact of photo
layout: A structural approach. Available at SSRN 3470877.

Li, W., Shrotriya, S. and Rinaldo, A. (2022). \(\ell_{\infty}\) - bounds
of the mle in the btl model under general comparison graphs. In
Uncertainty in Artificial Intelligence. PMLR.Liu, Y., Fang, E. X. and
Lu, J. (2022). Lagrangian inference for ranking problems. Operations
Research.Luce, R. D. (1959). Individual choice behavior: A theoretical
analysis. John Wiley \& Sons, Inc., New York; Chapman \& Hall, Ltd.,
London.Massey, K. (1997). Statistical models applied to the rating of
sports teams. Bluefield College, 1077. Mattei, N., Forshee, J. and
Goldsmith, J. (2012). An empirical study of voting rules and
manipulation with large datasets. Proceedings of COMSOC, 59. Mattei, N.
and Walsh, T. (2013). Preflib: A library for preferences
http://www.preflib.org. In International conference on algorithmic
decision theory. Springer.Maystre, L. and Grossglauser, M. (2015). Fast
and accurate inference of Plackett- Luce models. Advances in Neural
Information Processing Systems, 28. Negahban, S., Oh, S. and Shah, D.
(2012). Iterative ranking from pair- wise comparisons. Advances in
Neural Information Processing Systems, 25. Ouyang, L., Wu, J., Jiang,
X., Almeida, D., Wainwright, C., Mishkin, P., Zhang, C., Agarwal, S.,
Slama, K., Ray, A. et al.~(2022). Training language models to follow
instructions with human feedback. Advances in Neural Information
Processing Systems, 35 27730- 27744. Plackett, R. L. (1975). The
analysis of permutations. Journal of the Royal Statistical Society:
Series C (Applied Statistics), 24 193- 202. Portnoy, S. (1986). On the
central limit theorem in \(\mathbf{R}^p\) when \(p \to \infty\) .
Probab. Theory Related Fields, 73 571- 583. Rusmevichientong, P., Shen,
Z.- J. M. and Shmoys, D. B. (2010). Dynamic assortment optimization with
a multinomial logit choice model and capacity constraint. Operations
research, 58 1666- 1680. Rusmevichientong, P. and Topaloglu, H. (2012).
Robust assortment optimization in revenue management under the
multinomial logit choice model. Operations research, 60 865- 882. Shah,
N., Balakrishnan, S., Bradley, J., Parekh, A., Ramchandran, K. and
Wainwright, M. (2015). Estimation from pairwise comparisons: Sharp
minimax bounds with topology dependence. In Artificial intelligence and
statistics. PMLR.Shen, S., Chen, X., Fang, E. and Lu, J. (2023).
Combinatorial inference on the optimal assortment in multinomial logit
models. Available at SSRN 4371919.

Simons, G. and Yao, Y.- C. (1999). Asymptotics when the number of
parameters tends to infinity in the Bradley- Terry model for paired
comparisons. \emph{The Annals of Statistics}, \textbf{27} 1041--1060.
Sumida, M., Gallego, G., Rusmevichientong, P., Topaloglu, H. and Davis,
J. (2021). Revenue- utility tradeoff in assortment optimization under
the multinomial logit model with totally unimodular constraints.
\emph{Management Science}, \textbf{67} 2845--2869. Szörényi, B., Busa-
Fekete, R., Paul, A. and Hüllermeier, E. (2015). Online rank elicitation
for Plackett- Luce: A dueling bandits approach. \emph{Advances in Neural
Information Processing Systems}, \textbf{28}.Talluri, K. and Van Ryzin,
G. (2004). Revenue management under a general discrete choice model of
consumer behavior. \emph{Management Science}, \textbf{50} 15--33. Tropp,
J. A. (2012). User- friendly tail bounds for sums of random matrices.
\emph{Foundations of computational mathematics}, \textbf{12} 389--434.
Turner, H. and Firth, D. (2012). Bradley- Terry models in R: the
BradleyTerry2 package. \emph{Journal of Statistical Software},
\textbf{48} 1--21. Vulcano, G., Van Ryzin, G. and Ratliff, R. (2012).
Estimating primary demand for substitutable products from sales
transaction data. \emph{Operations Research}, \textbf{60} 313--334.
Wang, X., Bendersky, M., Metzler, D. and Najork, M. (2016). Learning to
rank with selection bias in personal search. In \emph{Proceedings of the
39th International ACM SIGIR conference on Research and Development in
Information Retrieval}.Zhang, H., Rusmevichientong, P. and Topaloglu, H.
(2020). Assortment optimization under the paired combinatorial logit
model. \emph{Operations Research}, \textbf{68} 741--761.



\end{document}
