\begin{document}


An Analysis of Bid- Price Controls for Network Revenue Management
Author(s): Kalyan Talluri and Garrett van Ryzin
Source: Management Science, Vol. 44, No.~11, Part 1 of 2 (Nov., 1998)

\section{An Analysis of Bid-Price Controls for Network Revenue Management}\label{an-analysis-of-bid-price-controls-for-network-revenue-management}


Bid- prices are becoming an increasingly popular method for controlling
the sale of inventory in revenue management applications. In this form
of control, threshold---or ``bid''---prices are set for the resources or
units of inventory (seats on flight legs, hotel rooms on specific dates,
etc.) and a product (a seat in a fare class on an itinerary or room for
a sequence of dates) is sold only if the offered fare exceeds the sum of
the threshold prices of all the resources needed to supply the product.
This approach is appealing on intuitive and practical grounds, but the
theory underlying it is not well developed. Moreover, the extent to
which bid- price controls represent optimal or near optimal policies is
not well understood. Using a general model of the demand process, we
show that bid- price control is not optimal in general and analyze why
bid- price schemes can fail to produce correct accept/deny decisions.
However, we prove that when leg capacities and sales volumes are large,
bid- price controls are asymptotically optimal, provided the right bid
prices are used. We also provide analytical upper bounds on the optimal
revenue. In addition, we analyze properties of the asymptotically
optimal bid prices. For example, we show they are constant over time,
even when demand is nonstationary, and that they may not be unique.

\section{Introduction and Overview}\label{introduction-and-overview}

Bid- price control is a revenue management method in which threshold
values (called bid prices) are set for each leg of a network and an
itinerary (path on the network) is sold only if its fare exceeds the sum
of the bid prices along the path. (Throughout the paper, the term bid-
price control refers to such an additive, leg- based bid- price scheme.
See Definition 1. ) The technique, which originated with the work of
Simpson (1989) at MIT and was later studied by Williamson (1992) in her
Ph.D.~thesis, is being adopted by a number of airlines and hotels.
Indeed, it is fast becoming the method of choice for origin- destination
(OD) revenue management.

Yet the theory underlying bid- price controls is scant. The early
development by Simpson (1989) and Williamson (1992), while quite
innovative and intuitively ap

Yet the theory underlying bid- price controls is scant. The early
development by Simpson (1989) and Williamson (1992), while quite
innovative and intuitively appealing, is based on different mathematical
programming formulations of the OD control problem. However, the
deterministic mathematical programming models used in these analyses are
clearly oversimplified models of the true OD revenue management problem.
Even the more sophisticated probabilistic mathematical programming
formulations, such as those proposed by Glover et al.~(1982), Williamson
(1992) and Wollmer (1986) assume one- time, static allocations of
capacity.

In this paper, we propose a general model of the OD control problem and
analyze it via dynamic programming. The model incorporates demand
uncertainty and makes no assumptions about the network structure or the
sequence of arrivals (timing of high and low fare class arrivals). It
allows for random fares within a fare class, which is of significant
practical importance in some applications (see §1). Using this general
model,

we formulate a dynamic program to analyze both the structure of the
optimal control and the performance of bid price controls.

\section{Related Literature}\label{related-literature}

The study of revenue management problems (or yield management) in the
airlines dates back to the work of Rothstein (1971) on an overbooking
model and to Littlewood (1972) on a model of space allocation for a
stochastic two- fare, single- leg (a network with one leg) problem.
Belobaba (1987a, 1987b) proposed and tested a multiple- fare- class
extension of Littlewood's rule which he termed the expected marginal
seat revenue (EMSR) heuristic. Extensions and refinements of the
multiple- fare- class problem include recent papers by Brumelle and
McGill (1993), Curry (1989), Robinson (1991) and Wollmer (1992) (all of
which with minor differences in models give the optimal nested seat
allocations when fare classes book sequentially). A recent review of
research on revenue management as well as a taxonomy of perishable asset
revenue management (PARM) problems is given by Weatherford and Bodily
(1992). See also Barnhart and Talluri (1996) for a recent survey on
airline operations that covers the practice of revenue management.

Dynamic programming has been applied to analyze single- leg problems in
prior work. For example, Lee and Hersh (1993) use discrete time dynamic
programming to develop optimal rules for the single- leg problem when
demand in each fare class is modeled as a stochastic process. In terms
of modeling approaches, their work is closest to ours. Diamond and Stone
(1991) and Feng and Gallego (1995) develop optimal threshold rules when
demands are modeled as continuous time stochastic process. Other dynamic
programming models for the single- leg problem are analyzed by Chatwin
(1992) and Janakiram et al.~(1994).

In a network setting, various mathematical programming approaches have
been proposed. Glover et al.~(1982) address a deterministic network flow
model for the allocation of seats between passenger itineraries and fare
classes. Wang (1983) provides an algorithm for the sequential
allocations of seats on a plane to different origin- destination city
pairs and fare classes within a flight segment when demands are random.
Dror et al.~(1988) present a rolling horizon network flow formu lation
for the seat inventory control problem assuming deterministic demands.
In two internal McDonell Douglas reports, Wollmer (1986a) and (1986b)
proposes a mathematical programming formulation incorporating random
demands, where the objective is to maximize the total expected network
revenue.

Yet to date, few dynamic programming models of network revenue
management have been analyzed. An exception is the work of Gallego and
van Ryzin (1994), who address a network revenue management problem using
a continuous time, dynamic pricing model. Their bounding techniques and
asymptotic analysis are quite similar to ours. The fundamental
difference between this work and ours is that Gallego and van Ryzin
assume prices are set for each itinerary (so the number of control
variables equals the number of itineraries), and customers either accept
or reject the offered prices. In contrast, we focus on bid- price
controls, in which values are set for the legs (so the number of
controls equals the number of legs) and a booking has a random revenue
that can only be accepted or rejected. Our situation, therefore, models
the case where fares are set exogenously, and the main goal of our work
is to analyze the effectiveness of bid prices as a mechanism for making
these accept/deny decisions.

As mentioned, Simpson (1989) and Williamson (1992) introduced the idea
of bid- price controls and they proposed many of the main approximation
approaches in the area. Williamson (1992) in particular used extensive
simulation studies to analyze a variety of approaches to network revenue
management. Simpson and Williamson's work had a significant impact on
the practice of revenue management. However, they do not provide a
rigorous analysis of the structure of the optimal network policy nor do
they provide a theoretical foundation for the bid- price approach. Our
work puts this important practical development on a sound theoretical
footing.

\section{Organization of the Paper}\label{organization-of-the-paper}

In §1 we formally define our network model and define bid- price
control. In §2, we analyze the structure of the optimal policy and show
that, in general, it is not a bid- price control. Two counter- examples
are given in §3 to illustrate why a bid- price structure can be
suboptimal. In §4, we develop an upper bound on the optimal rev

enue. This bound is used to show that a bid- price policy is
asymptotically optimal as the sales volumes and capacities in the
network grow. The analysis also provides a constructive method for
computing a set of asymptotically optimal bid prices. In §5, we briefly
analyze some of the more common methods for computing bid prices and
discuss their strengths and weaknesses. Conclusions and a discussion of
future research is presented in §6.

\section{1. Formulation}\label{formulation}

We use superscripts to denote components of a vector and subscripts to
denote time. We generally try to follow the convention that \(k\)
denotes current time (timetoo- go) while \(t\) denotes an arbitrary
time. Time is counted backwards, so time \(t\) represents a point \(t\)
periods from the end of the horizon. We do not distinguish between row
and column vectors, since the proper interpretation is usually clear
from the context. Finally, if \(A\) is a matrix, then the \(j\) - th
column of \(A\) is denoted \(A^{j}\) and the ith row is denoted
\(A_{i}\) . Finally, we let \(x^{+} = \max \{x,0\} ,1\{E\}\) denote the
indicator function of the event \(E\) and \((a,s)\) denotes almost
surely.

Our model is formulated as follows: An airline network has m arcs or
legs which can be used to provide \(n\) origin- destination itineraries.
We let \(a_{ij}\) be the number of seats on leg \(i\) used by itinerary
\(j\) \((a_{ij} = 0\) if leg \(i\) is not part of itinerary \(j\) ).
Define the matrix \(A = [a_{ij}]\) . Thus, the \(j\) - th column of
\(A,A^{j},\) is a multiple of the incidence vector for itinerary \(j\)
(or the incidence vector itself if itinerary \(j\) requires only one
seat). We use the notation \(i\in A^{j}\) to indicate that leg \(i\) is
used by itinerary \(j\) and \(j\in A_{i}\) to mean that itinerary \(j\)
uses leg \(i\)

While it is simplest to imagine that each origindestination itinerary
requires only one unit of capacity from each of the legs it traverses,
we do not impose this restriction. Indeed, the model can accommodate
group requests (e.g.~a family booking four seats together). To model
this situation, we simply introduce one column in the matrix \(A\) for
each possible group size, with the nonzero elements of the column equal
to the group size and the itinerary revenue set equal to the total
revenue of the group (e.g.~for a group of size four, add a column \(j\)
with \(a_{ij} = 4\) for \(i\) on the path and with revenue \(R_{t}^{j}\)
that is four times as large). The interpretation of the proba bility
model in this case is that it reflects the likelihood of having requests
for particular group sizes. (See Young and van Slyke (1994) for an exact
analysis of monotonicity properties for a single- leg problem with group
requests.)

The state of the network is described by a vector \(x\)
\(= (x^{1},\ldots ,x^{m})\) of leg capacities. If itinerary \(j\) is
sold, the state of the network changes to \(x - A^{j}\) . To keep our
analysis simple, we will assume there are no cancellations or no- shows
and, consequently, overbooking is not needed. Alternately, capacity may
include so- called overbooking pads. (An overbooking pad is the number
of seats an airline makes available for sale beyond the actual cabin
capacity. These pads are sometimes set prior to performing fare class
allocations.) If overbooking pads are computed independently, one can
consider leg capacities in our model to be the sum of actual physical
capacity and the overbooking pad.

In our formulation time is discrete, and \(k\) represents the number of
periods left before departure. Within each time period, \(t,\) we assume
that at most one request for an itinerary can arrive; that is, the
discretization of time is sufficiently fine so that the probability of
more than one request is negligible.

The way we model a request at time \(t\) is somewhat nonstandard but
useful for analytical purposes. All booking events in time \(t\) are
modeled as the realization of a single random vector
\(R_{t} = (R_{t}^{1},\ldots ,R_{t}^{n})\) . If \(R_{t}^{j} > 0,\) this
indicates a request for itinerary \(j\) occurred and that its associated
revenue is \(R_{t}^{j};\) if \(R_{t}^{j} = 0\) this indicates no request
for \(j\) occurred. A realization \(R_{t} = 0\) (all components equal to
zero) indicates that no request from any itinerary occurred at time
\(t\) . For example, if we have \(n\) \(= 3\) itineraries, then a value
\(R_{t} = (0,0,0)\) indicates no requests arrived, a value
\(R_{t} = (120,0,0)\) indicates a request for itinerary 1 with revenue
of {\$\textbackslash) 120,\textbackslash(a value of\$} R\_\{t\} \(=\)
(140,0,0) indicates a request for itinerary 1 with revenue of
{\$\textbackslash) 140\textbackslash(a value\$} R\_\{t\}=(0,70,0)
\(indicates a request for itinerary 2 with revenue of\) ~\(70,\) etc.

Note by our assumption that at most one arrival occurs in each time
period, at most one component of \(R_{t}\) can be positive (as indicated
in the example above). More formally, if we let
\(E_{n} = \{e_{0},e_{1},\ldots ,e_{n}\}\) where \(e_{j}\) is the \(j\) -
th unit \(n\) - vector and \(e_{0}\) is the zero \(n\) - vector and
define the set
\(\mathcal{S} = \{R:R = \alpha e,e\in E_{n},\alpha \geq 0\}\) then
\(R_{t}\) \(\in \mathcal{S}\) . The sequence \(\{R_{t};t\geq 1\}\) is
assumed to be

independent with known joint distributions \(F_{t}(r)\) whose support is
on S. We further require that the marginal distributions,
\(F_{t}^{i}(r) = P(R_{t}^{i}\leq r)\) be continuous on \((0,\)
\(+\infty)\) and that all \(R_{k}^{j}\) have finite means. For the
asymptotic results, we require that the revenues have bounded support
(i.e.~\(P(R_{t}^{j}\leq C) = 1\) for some constant C). Note that
\(F_{t}^{i}(r)\) is not the revenue distribution directly. Rather,
\(P(R_{t}^{i} > 0) = 1 - F_{t}^{i}(0)\) is the probability of getting a
request from itinerary \(j\) in period \(t,\) and
\(P(R_{t}^{j}\leq r\mid R_{t}^{j}\)
\(>0) = F_{t}^{i}(r) / (1 - F_{t}^{i}(0))\) is the distribution of the
actual revenue from itinerary \(j\) at time \(t\)

The time dependence of the revenue distribution models a variety of
nonstationarities in the arrival process. For example, due to purchase
timing restrictions, the mix of available fare products changes with
time. Purchase patterns in various customer segments (e.g.~the ratio of
leisure/business purchases) change as the flight departure date
approaches as well. No assumption is made on the particular order of
arrival in our model.

Allowing uncertainty in revenues is important for several reasons.
First, airlines often offer a variety of fares in each fare class for
each itinerary and also pay varying commissions on these fares. Under
such conditions, there is a potential to generate more revenue by
discriminating among the various net revenues within a particular fare
class, and the value function should reflect this potential. Second, the
practice of negotiating fares in some industries (advertising,
broadcasting, hotels) contributes to uncertainty in fares. Including
fare variance provides more modeling flexibility in these emerging
revenue management applications. Finally, modeling fare variance
provides flexibility in constructing forecasts. Specifically, one can
decrease the relative forecast error by aggregating fare classes, at the
expense of increasing the variance in the fares within a fare class.
This ability to vary the level of aggregation in the forecast data in
this way has the potential of leading to a better overall forecasting-
optimization scheme.

Given the time- to- go, \(k\) , the current seat inventory \(x\) and the
current request \(R_{k}\) , we are faced with a decision: Do we or do we
not accept the current request?

Let an \(n\) - vector \(u_{k}\) denote this decision, where
\(u_{k}^{j} = 1\) if we accept a request for itinerary \(j\) at time
\(k,\) and \(u_{k}^{j}\) \(= 0\) otherwise. In general, the decision to
accept, \(u_{k}^{j},\) is a function of the capacity vector \(x\) and
the fare \(r^{j}\) offered for itinerary \(j,\)
i.e.~\(u_{k}^{j} = u_{k}^{j}(x,r^{j})\) and hence
\(u_{k} = u_{k}(x,r),\) where
\(r = (r^{1},\ldots ,r^{n})\in \mathcal{S}.\) Since we can accept at
most one request in any period, \(u_{k}\in E_{n},\) where \(E_{n}\) is
the collection of unit \(n\) - vectors as defined above. Since we assume
cancellations and no- shows do not occur and that legs cannot be
oversold, if the current seat inventory is \(x,\) then \(u_{k}\) is
restricted to the set \(\mathcal{U}(x) = \{e\in E_{n}:A e\) \(\leq x\}\)

We can now define precisely what a bid- price control scheme is in the
context of this model.

DEFINITION 1. A control \(u_{k}(x,r)\) is said to be a bid- price
control if there exist real- valued functions
\(\mu_{k}(x) = (\mu_{k}^{1}(x), \ldots , \mu_{k}^{n}(x)), k = 1,2, \dots\)
(called bid prices) such that

\[
u_{k}^{j}(x,r^{j}) = \left\{ \begin{array}{l l}{1} & {r^{j}\geq \sum_{i\in A^{j}}\sum_{h = 0}^{a_{i j} - 1}\mu_{k}^{i}(x - h),A^{j}\leq x,}\\ {0} & {o t h e r w i s e.} \end{array} \right. \tag{1}
\]

That is, a bid- price control specifies a set of bid prices for each leg
at each point in time and for each capacity, such that we accept a
request for a particular itinerary if and only if there is available
capacity and the fare exceeds the sum of the bid prices for all the
units of capacity used by the itinerary. We next examine whether this
bid- price structure is optimal.

\section{2. Structure of the Optimal
Control}\label{structure-of-the-optimal-control}

In order to formulate a dynamic program to determine optimal decisions
\(u_{k}^{*}(x,r)\) , let \(J_{k}(x)\) denote the maximum expected
revenue (cost- to- go) for a given seat inventory \(x\) at time \(k\) .
Then \(J_{k}(x)\) must satisfy the Bellman equations (see Bertsekas
(1995), p.~18)

\[
\begin{array}{r l} & {J_{k}(x) = \underset {u_{k}(\cdot)\in \mathcal{U}(x)}{\max}E[R_{k}u_{k}(x,R_{k})}\\ & {\qquad +J_{k - 1}(x - A u_{k}(x,R_{k}))]} \end{array} \tag{2}
\]

with the boundary condition

\[
J_{0}(x) = 0, \quad \forall x. \tag{3}
\]

This leads to our first proposition, which establishes the existence of
an optimal policy and characterizes the form of the optimal control.

PROPOSITION 1. If \(R_{k}^{j}\) has finite first moments for all \(k\)
and \(j\) , then \(J_{k}(x)\) is finite for all finite \(x\) , and an
optimal control \(u_{k}^{*}\) exists of the form

\[
u_{k}^{*}(x,r^{j}) = \left\{ \begin{array}{ll}1 & r^{j}\geq J_{k - 1}(x) - J_{k - 1}(x - A^{j})andA^{j}\leq x,\\ 0 & otherwise. \end{array} \right.
\]

PROOF. Note that \(u_{k}^{*}(x,r)\) defined by (4) maximizes

\[
r u + J_{k - 1}(x - A u)
\]

subject to the constraint \(u\in \mathcal{U}(x)\) . Therefore,

\[
\begin{array}{r l} & {E[R_{k}u_{k}^{*}(x,R_{k}) + J_{k - 1}(x - A u_{k}^{*}(x,R_{k}))]}\\ & {\quad \geq \underset {u_{k}(\cdot)\in \mathcal{U}(x)}{\max}E[R_{k}u_{k}(x,R_{k}) + J_{k - 1}(x - A u_{k}(x,R_{k}))].} \end{array}
\]

Thus, \(u_{k}^{*}\) satisfies the Bellman equation provided we can show
that the expectation on the left- hand side exists. To do so, we use
induction. First, assume that \(J_{k - 1}(x)\) is finite for all finite
\(x\) . Then applying \(u_{k}^{*}\) we have that

\[
\begin{array}{l}{{ E[R_{k}u_{k}^{*}(x,R_{k})+J_{k-1}(x-A u_{k}^{*}(x,R_{k}))]}}\\ {{\quad=J_{k-1}(x)+\sum_{j:A^{j}=x}E(R_{k}^{j}+J_{k-1}(x-A^{j})-J_{k-1}(x))^{+}.}}\end{array}
\]

By the induction assumption, \(J_{k - 1}(x - A^{j}) + J_{k - 1}(x)\) is
finite. Therefore, since
\(E(R - \epsilon)^{*}\equiv ER + \left\vert \epsilon \right\vert\) the
righthand- side above is finite if the revenues \(R_{k}^{j}\) have
finite first moments. The finiteness of \(J_{k - 1}\) then follows using
induction on \(k\) and the fact that \(J_{0}(x) = 0\) for all \(x\) ,
and that under \(u_{k}^{*}\)

\[
\begin{array}{l}{{J_{k}(x)=J_{k-1}(x)+\sum_{j:A^{j}=x}E(R_{k}^{j}+J_{k-1}(x-A^{j})}}\\ {{\mathrm{~}-J_{k-1}(x))^{+}.\quad\sqcup}}\end{array} \tag{5}
\]

\section{3. Nonoptimality of Bid-Price
Controls}\label{nonoptimality-of-bid-price-controls}

Proposition 1 says that an optimal policy for accepting requests is of
the form: accept fare \(r^{j}\) for itinerary \(j\) if and only if we
have sufficient remaining capacity and

\[
r^{j}\geq J_{k - 1}(x) - J_{k - 1}(x - A^{j}).
\]

This reflects the rather intuitive notion that we accept a fare of \(r\)
for a given itinerary only when it exceeds the opportunity cost of the
reduction in leg capacities. It is precisely this intuition and its
analogy to the role of dual prices in deterministic optimization that
motivated the early development of bid- price control schemes (Simp

son 1989, Williamson 1992). However, in general this form of control is
not a bid price control, a fact which we illustrate next via two counter
examples.

3.1. A Counter Example to Bid- Price Optimality In this first example,
we have a simple network with two legs. There is one unit of capacity on
each leg and two time periods remaining in the horizon. The itinerary
data are shown in Table 1. In period 2, there are two local itineraries.
(A local itinerary on a leg is a nonstop itinerary consisting of that
leg, while a through itinerary on a leg is a multi- leg itinerary
involving that leg.) each with a fare of
\(250 and probability of arrival 0.3, and one through itinerary with a fare of\)
500 and probability of arrival 0.4; in the last period, there is only a
through fare with the same \$500 revenue and a probability of arrival of
0.8. Recall, that arrivals in each period are mutually exclusive (i.e.,
only one itinerary per period arrives).

In this example, we report the data in the form of an arrival
probability and a fare for each itinerary. This can easily be translated
into a single distribution of arriving revenues
\(R_{t} = (R_{t}^{1}, \ldots , R_{t}^{n})\) where \(R_{t}^{j}\) is the
revenue associated with a request for itinerary \(j\) at time \(k\) and
\(R_{t}^{j} = 0\) indicates no request for itinerary \(j\) occurred.
Note also that the assumption of a deterministic fare violates the
continuity assumption on the distribution of \(R_{t}\) . It is not hard
to show, however, that we can get essentially the same counter example
by replacing the deterministic fare with a random fare that has a
continuous distribution arbitrarily close to the (degenerate)
deterministic distribution.

It is not hard to see by inspection what an optimal policy is for this
example. Accepting either of the local itineraries in period 2 yields
\$250 in revenue and prevents us from accepting a through itinerary in
period 1.

Table 1 Problem Data for Bid-Price Counter Example

\begin{longtable}[]{@{}llll@{}}
\toprule\noalign{}
\endhead
\bottomrule\noalign{}
\endlastfoot
Time (t) & Itin. (A\textquotesingle) & Fare & Prob. \\
\multirow{3}{*}{2} & (1 1) & \$500 & 0.4 \\
& (1 0) & \$250 & 0.3 \\
& (0 1) & \$250 & 0.3 \\
\multirow{2}{*}{1} & (1 1) & \$500 & 0.8 \\
& No arrival & & 0.2 \\
\end{longtable}

However, if we do not accept a local itinerary in period 2 and leave
both legs available for the through demand in period 1, the expected
revenue is () 400\$ . So it is optimal to reject both local itineraries
in period 2.

On the other hand, we clearly want to accept the through itinerary in
period 2. Together, this implies that the bid prices, \(\mu_{1}\) and
\(\mu_{2},\) in period 2 must satisfy \(\mu_{1}\) \(>250\)
\(\mu_{2} > 250\) and \(\mu_{1} + \mu_{2}\leq 500,\) which is, of
course, impossible. Therefore, no bid- price policy can produce an
optimal decision in period 2. Indeed, it is not hard to show that the
best a bid- price policy can do in this example is to reject all demand
in period 2 and accept only the through fare (if it arrives) in period
1, yielding a {\$\textbackslash) 400\textbackslash(expected revenue. The
optimal policy, in contrast, generates an expected revenue of\$}
~\(440\) - fully \(10\%\) more expected revenue than the best possible
bid- price policy.

Finally, it is possible to construct other counter examples in which
bid- price sub- optimality occurs with arbitrarily large remaining leg
capacities. Thus, the problem of bid price sub- optimality is not
confined to the end of the horizon, but can in fact occur at any point
in the booking process.

\section{3.2. Bid-Price Optimality and the Structure of the Value
Function}\label{bid-price-optimality-and-the-structure-of-the-value-function}

A structural insight into why bid prices are not optimal in general is
obtained by considering the implication of bid- price optimality for the
value function \(J_{k}(x)\) . In some cases, it implies a certain
linearity of the value function, as the next proposition demonstrates.

Proposirion 2. Suppose the elements of A are only zero or one (i.e.~no
multiple requests) and that A has the identity matrix as a sub matrix.
Further, suppose the marginal distributions \(F_{i}(x)\) are strictly
increasing on \((0, + \infty)\) for all t and \(j\) . Then a bid- price
control scheme is optimal only if \(J_{k}\) satisfies

\[
J_{k}(x) - J_{k}(x - A^{j}) = \sum_{i\in A^{j}}(J_{k}(x) - J_{k}(x - e_{i}))
\]

for all \(k,j\) and \(x\geq A^{j}\)

Proor. Without loss of generality, let the first m columns of \(A\) be
the identity matrix. If a bid- price control scheme is optimal, then by
considering the first m itineraries we must have that

\[
\mu_{k}^{i}(x) = J_{k}(x) - J_{k}(x - e_{i}),\quad i = 1,\ldots ,m,
\]

by Proposition 1 and the definition of bid- price control. Now suppose
for some \(k,j\) and \(x\geq A^{j}\) , that \(J_{k}(x - A^{j})\)
\(>J_{k}(x) - \Sigma_{i\in A^{j}}(J_{k}(x) - J_{k}(x - e_{i}))\) . Then
by Proposition 1 the threshold for accepting fares for itinerary \(j\)
in state \(x\) in period \(k + 1\) is

\[
\begin{array}{l}{{J_{k}(x)-J_{k}(x-A^{j})}}\\ {{< \sum_{i\in A^{j}}(J_{k}(x)-J_{k}(x-e_{i}))=\sum_{i\in A^{j}}\mu_{k}^{i}(x).}}\end{array}
\]

Using this inequality together with the fact that the distributions
\(F_{k}^{i}(x)\) are strictly increasing, one can show that the
threshold as determined by the bid prices,
\(\Sigma_{i\in A^{j}}\mu_{k}^{i}(x),\) violates the optimality
conditions (2); hence, bid prices cannot be optimal. A similar
contradiction is obtained when
\(J_{k}(x - A^{j})< J_{k}(x) - \Sigma_{i\in A^{j}}(J_{k}(x) - J_{k}(x\)
\(- e_{i})\) .

Proposition 2 shows that bid prices are only optimal in this case when
the opportunity cost of the itinerary, \(J_{k}(x) - J_{k}(x - A^{j}),\)
is equal to the sum of the opportunity costs of selling each leg \(i\)
separately, \(\Sigma_{i\in A^{j}}(J_{k}(x) - J_{k}(x\) \(- e_{i})\) .
This sort of linearity in the value function cannot be expected to hold
in general.

Indeed, the counter example above illustrates the two main reasons why
it does not hold. Bid prices in this example fail in part because
selling a seat is a ``large'' change in the capacity of a leg. Large
relative changes in capacity on several legs simultaneously cannot, in
general, be expected to have the same revenue effect as the sum of the
individual changes. This is one reason why gradient- based reasoning
falls short in explaining bid- price optimality. Indeed, this issue was
first raised by Curry (1992), who used an analogy to the Taylor series
expansion of the value function to argue that second- order
``interaction'' terms may be significant in determining optimal revenue
thresholds.

The second reason bid prices may fail to capture the opportunity cost is
that future revenues may depended in a highly nonlinear way on the
remaining capacity. Specifically, in the counter example note that it is
the minimum capacity on the two legs that determines future expected
revenues. Hence, the opportunity cost of using a single leg exactly
equals the opportunity cost of using both legs simultaneously, so the
linearity required in Proposition 2 is destroyed. This phenomenon is
very

similar to degeneracy in mathematical programming, and it can occur in
the optimal value function or in various approximation to the optimal
value function, as shown in §5.5.

\section{4. An Asymptotic Analysis of Bid-Price
Controls}\label{an-asymptotic-analysis-of-bid-price-controls}

We next analyze the degree of suboptimality of a bid- price control
scheme. Our first step is to consider an upper bound based on a
relaxation of the original problem. The upper bound provides dual prices
that are then used in our second step to construct a bid- price policy.
In contrast to the negative results of the previous section, we show
that the control generated by this particular set of bid prices has good
asymptotic properties if the number of seats sold on each leg is large.

\section{4.1. An Upper Bound Problem}\label{an-upper-bound-problem}

Let \(u_{t}\) represent a given control policy. We consider \(u_{t}\)
somewhat more abstractly as simply a process which is adapted to history
of requests from \(k\) to \(t\) . That is, if
\(\mathcal{F}_{t} = \sigma (\{R_{k}; T \geq k \geq t\})\) , then
\(u_{t}\) is a process that is \(\mathcal{F}_{t}\) - measurable. The
problem can then be stated as finding such a process \(u_{t}\) that
solves the problem

\[
J_{k}(x) = \max E\left[\sum_{t = 1}^{k}R_{t}u_{t}\right],
\]

\[
\sum_{t = 1}^{k}A u_{t}\leq x\quad (\mathrm{a.s.}),
\]

\[
u_{t}\in E_{n}. \tag{6}
\]

Note that the process \(u_{t}\) is defined for a given starting state
\(x\) and \(k\) , and therefore optimizing over \(u_{t}\) only provides
a control policy for those states reachable from \((x, k)\) .
Nevertheless, this formulation is sufficient for determining
\(J_{k}(x)\) .

For any \(m\) - vector \(\mu \geq 0\) , consider a relaxed version of
this problem

\[
\begin{array}{r l} & {\overline{{J}}_{k}(x,\mu) = \underset {\{u_{t}\in E_{n}\}}{\max}E\Bigg[\underset {t = 1}{\overset{k}{\sum}}R_{t}u_{t}\Bigg]}\\ & {\qquad +E\Bigg[\mu \Bigg(x - \underset {t = 1}{\overset{k}{\sum}}A u_{t}\Bigg)\Bigg]}\\ & {\qquad = \underset {\{u_{t}\in E_{n}\}}{\max}E\Bigg[\underset {t = 1}{\overset{k}{\sum}}(R_{t} - \mu A)u_{t}\Bigg] + \mu x.} \end{array} \tag{7}
\]

Here, \(u_{t}\) is a process which is adapted to \(\mathcal{F}_{t}\) and
satisfies \(u_{t} \in E_{n}\) , but need not satisfy
\(\Sigma_{t = 1}^{k}A u_{t} \leq x\) (a.s.). That is, the policy might
oversell a leg \(i\) , but at a cost of \(\mu_{i}\) for each oversold
seat.

The values of these two problems are related as follows:

LEMMA 1. For any \(\mu \geq 0\) ,
\(J_{k}(x) \leq \overline{J}_{k}(x, \mu)\) .

Proor. For any control process \(\{u_{t}:1\leq t\leq k\}\) which is an
optimal policy for (6), we have that \(\mu (x - \Sigma_{t = 1}^{k}\)
\(A u_{t})\geq 0\) (a.s.), and hence because \(u_{t}\) is bounded,
\(E[\mu (x\) \(- \Sigma_{t = 1}^{k}A u_{t})]\geq 0\) as well. In
addition, by definition \(E[\Sigma_{t = 1}^{k}R_{t}u_{t}] = J_{k}(x)\) .
Since such a policy is also feasible for (7), the above inequality
follows.

To create the best upper bound possible from this relation, we will
minimize \(\overline{J}_{k}(x, \mu)\) over \(\mu \geq 0\) . Fortunately,
evaluating \(\overline{J}_{k}(x, \mu)\) is not difficult, since it is
easy to see that in (7) the problem decomposes by periods. Hence an
optimal policy for (7) is simply

\[
u_{t}^{j} = \left\{ \begin{array}{ll}1 & r^{j} > \mu A^{j}, \\ 0 & \text{otherwise} \end{array} \right.
\]

Note this is in fact a bid- price control with bid prices given by the
vector \(\mu\) for all times \(t \leq k\) and all states \(x\) .
Evaluating the cost under this optimal control yields

\[
\overline{J}_{k}(x, \mu) = \sum_{t = 1}^{k} \sum_{j = 1}^{n} E(R_{t}^{j} - \mu A^{j})^{+} + \mu x. \tag{8}
\]

The partial expectation \(E(Z - z)^{+}\) is a convex function in \(z\)
for any random variable \(Z\) for which the expectation exists.
Therefore, \(\overline{J}_{k}(x, \mu)\) is convex in \(\mu\) . As a
result the problem,

\[
v_{k}(x) = \min_{\mu = 0} \overline{J}_{k}(x, \mu) \tag{9}
\]

is a convex program. This minimization problem generates the least upper
bound from our relaxation. Let \(\mu^{*}\) denote an optimal solution to
(9). Note that \(\mu^{*}\) clearly depends on \(x\) and \(k\) ; that is,
\(\mu^{*} = \mu^{*}(x, k)\) . However, to economize on notation we do
not include the arguments \(x\) and \(k\) .

Note that \(\frac{d}{dz} E(Z - z)^{+} = - P(Z > z)\) . Therefore, if the
distributions of \(R_{t}^{j}\) are continuous on \((0, +\infty)\) , then
an optimal solution \(\mu^{*}\) to (9) satisfies the Kuhn- Tucker
necessary conditions

\[
\begin{array}{c}{{\sum_{t=1}^{k}\sum_{j=1}^{n}P(R_{t}^{j}> \mu^{*}A^{j})A^{j}-\lambda=x,}}\\ {{\lambda\mu^{*}=0,}}\\ {{\lambda\equiv0.}}\end{array} \tag{10}
\]

Since (9) is a convex program, these conditions are also sufficient.

The Kuhn- Tucker conditions (10) are quite intuitive. Note that since
\(P(R_{t}^{j} > \mu^{*}A^{j}) = E u_{t}^{j},\) the term
\(\Sigma_{t = 1}^{k}\Sigma_{j = 1}^{n}\)
\(P(R_{t}^{j} > \mu^{*}A^{j})A^{j}\) is the vector of expected number of
requests for each leg over the remaining horizon \(k\) from itineraries
whose revenue exceeds the bid prices defined by \(\mu^{*}\) . Since
\(\lambda \geq 0\) , the first and second condition in (10) imply that
if \(\mu^{*i} > 0,\) then \(\lambda^{i} = 0\) and the expected number of
such request for leg \(i,\) across all itineraries, is precisely the
capacity \(x^{i};\) if \(\mu^{*i} = 0\) \(\lambda^{i}\geq 0\) and the
expected number of such requests for leg \(i\) is no more than the
capacity \(x^{i}\) . Indeed, (9) is equivalent to the problem of
maximizing the expected revenue subject to the constraint that the
expected number of requests is no more than \(x\) (i.e.~the constraint
\(E[\Sigma_{t = 1}^{k}A u_{t}]\leq x)\)

The optimal value obtained by solving (9) also provides an analytical
alternative to the ``perfect hindsight'' upper bound, which is used
frequently in many practical simulation studies. The perfect hindsight
bound is obtained by solving the linear program

\[
\begin{array}{r l} & {V_{k}(x,\omega) = \max \sum_{t = 1}^{k}R_{t}(\omega)u_{t}(\omega),}\\ & {\qquad \sum_{t = 1}^{k}A u_{t}(\omega)\leq x,}\\ & {u_{t}(\omega)\in [0,1]^{n},} \end{array} \tag{11}
\]

where we use \(\omega\) to indicate that the optimization is performed
using perfect information about the actual realization of demand. From
strong duality, we then have

\[
\begin{array}{r l} & {V_{k}(x,\omega) = \underset {\mu (\omega) = 0}{\min}\underset {u_{t}(\omega)\in [0,1]^{n}}{\max}\underset {t = 1}{\sum}^{k}R_{t}(\omega)u_{t}(\omega)}\\ & {\qquad +\mu (\omega)(x - A u_{t}(\omega))}\\ & {= \underset {\mu (\omega) = 0}{\min}\underset {t = 1}{\sum}^{k}\underset {j = 1}{\sum}(R_{t}^{j}(\omega) - \mu (\omega)A^{j})^{+} + \mu (\omega)x}\\ & {\qquad \leq \underset {t = 1}{\sum}^{k}\underset {j = 1}{\sum}(R_{t}^{j}(\omega) - \mu^{*}A^{j})^{+} + \mu^{*}x} \end{array} \tag{12}
\]

where again \(\mu^{*}\) denotes an optimal solution to (9). Taking
expectations on both sides above and using (8) and (9) yields

\[
E V_{k}(x,\omega)\leq v_{k}(x).
\]

Hence, as a bound on optimal expected revenues, (9) is weaker than what
one obtains by simulating and averaging (11); however, because it is
analytical, requiring only one optimization and no simulation, it is
much more computationally efficient.

In summary, (9) provides an upper bound on optimal revenues, and its
optimal solutions, \(\mu^{*}\) , satisfy the constraints of the original
problem in expectation. We next show that if one fixes the set of bid
prices at \(\mu^{*}\) for all times \(t \leq k\) , the resulting revenue
is asymptotically optimal in a certain scaling of the problem. That is,
a fixed bid price policy with bid prices equal to \(\mu^{*}\) is in fact
asymptotically optimal.

\section{4.2. Asymptotic Analysis of Bid Prices Derived from the Upper
Bound
Problem}\label{asymptotic-analysis-of-bid-prices-derived-from-the-upper-bound-problem}

Below, \(\mu^{*}\) denotes an optimal solution to (9). Again, we note
that \(\mu^{*}\) depends on the initial capacity \(x\) and the initial
time- to- go \(k\) . Consider the following fixed- bid- price heuristic:

\section{Fixed-Bid-Price Heuristic
(H)}\label{fixed-bid-price-heuristic-h}

At time \(k\) with remaining capacity \(x\) compute \(\mu^{*}\) once by
solving (9). Then, for all times \(t \leq k\) , accept a request for
itinerary \(j\) with revenue \(r\) if and only if there is sufficient
capacity to satisfy it and \(r > \mu^{*}A^{j}\) .

Let \(J_{k}^{H}(x)\) denote the expected revenue of this heuristic given
initial capacity \(x\) and time- to- go \(k\) . Let \(\theta\) be a
positive integer, and consider a sequence of problems, indexed by
\(\theta\) , with initial capacity vectors \(\theta x\) , time- to- go
\(\theta k\) and revenues, denoted \(R_{t}(\theta)\) , where

\[
R_{t}(\theta) = _{D}R_{t / \theta} \tag{12}
\]

and \(\mathbf{\Sigma} = _{D}\) denotes equality in distribution. This
construction corresponds to splitting each period \(t\) in the original
problem into \(\theta\) statistically independent and identical periods
in the scaled problem and at the same time increasing the initial
capacity by a factor of \(\theta\) . As a result, the relative values of
demand, capacity and time are preserved.

For the scaled problem, let \(J_{\theta k}^{H}(\theta x)\) denote the
optimal expected revenue and \(J_{\theta k}^{H}(\theta x)\) denote the
expected

revenue of the fixed- bid- price heuristic. (We show below that the
vector \(\mu^{*}\) solves (9) for all \(\theta\) , so the same vector of
bid prices is used for each problem in the sequence.) The following
result shows that the fixed- bid- price heuristic is asymptotically
optimal as the scale of the problem, as measured by \(\theta\) ,
increases:

THEOREM 1. If \(R_{i}^{j} \equiv C\) (a.s.), then

\[
\frac{J_{\theta k}^{H}(\theta x)}{J_{\theta k}(\theta x)} \geq 1 - O(\theta^{-1 / 2}).
\]

In particular,

\[
\lim_{\theta \to \infty} \frac{J_{\theta k}^{H}(\theta x)}{J_{\theta k}(\theta x)} = 1.
\]

PROOF. By considering (8) and (12), we have

\[
\begin{array}{l}{{\sum_{t=1}^{\theta k}\sum_{j=1}^{n}E(R_{t}^{j}(\theta)-\mu A^{j})^{+}+\theta\mu x}}\\ {{=\theta\bigg[\sum_{t=1}^{k}\sum_{j=1}^{n}E(R_{t}^{j}-\mu A^{j})^{+}+\mu x\bigg]=\theta\overline{{J}}_{k}(x,\mu).}}\end{array} \tag{14}
\]

Therefore, the vector \(\mu^{*}\) that solves (9) is the same for all
value of \(\theta\) . We therefore have by Lemma 1

\[
J_{\theta k}(\theta x) \leq \theta v_{k}(x). \tag{13}
\]

Now consider the fixed- bid- price heuristic with \(\mu^{*}\) as the
fixed vector of bid prices. We will construct a sample path bound on the
revenue with these bid prices using a coupling argument. To do so, we
consider an alternate system which follows the bid- price policy for
accepting sales, but has no capacity constraint; rather, in the
alternate system we subtract a revenue of \(C\) for each set sold in
excess of \(\theta x_{i}\) on each leg \(i\) , where \(C\) is the
uniform upper bound on the itinerary revenues. Consider the net revenues
collected in each system for a given sample path of arrivals. Note in
the two systems, the same revenues are collected up until the time one
or more of the leg capacities is exhausted.

Now, suppose a request arrives for an itinerary \(j\) that needs a leg
\(i\) whose capacity is exhausted. If
\(R_{i}^{j}(\theta) \leq \mu^{*} A^{j}\) , it will be rejected in both
systems and no revenues will be collected in either system. If
\(R_{i}^{j}(\theta) > \mu^{*} A^{j}\) , the request will be rejected in
the original system because of the capacity constraint. In the alternate
system, it will be accepted but at least one penalty of \(C\) will be
charged because one or more leg capacities are exhausted. No revenues
will be collected in the original system, and the alternate system will
realize a loss since \(R_{i}^{j}(\theta) - C \leq 0\) . Moreover, the
alternate system will have even less capacity remaining because it
accepted the request. Hence, it follows that the net revenues in the
alternate system are, pathwise, a lower bound on the revenues obtained
under the bid- price heuristic. Therefore,

\[
\begin{array}{l}{{J_{\theta k}^{H}(\theta x)\geq\sum_{t=1}^{\theta k}\sum_{j=1}^{n}E(R_{t}^{j}(\theta)-\mu^{*}A^{j})^{+}}}\\ {{+\sum_{t=1}^{\theta k}\sum_{j=1}^{n}P(R_{t}^{j}(\theta)>\mu^{*}A^{j})\mu^{*}A^{j}}}\\ {{-C\sum_{i=1}^{m}E(N^{i}-\theta x^{i})^{+},}}\end{array} \tag{14}
\]

where \(N^{i}\) is the number of leg \(i\) seats sold under the bid-
price heuristic, which is given by

\[
N^{i} = \sum_{t = 1}^{\theta k} \sum_{j = 1}^{n} \mathbf{1}\{R_{t}^{j}(\theta) > \mu^{*} A^{j}\} a_{ij}.
\]

The first two terms in (14) are the actual revenues collected; the last
term is the total penalties charged.

Let \(Y_{ijt} = \mathbf{1}\{R_{t}^{j} > \mu^{*} A^{j}\} a_{ij}\) and
note that by (12) and the independence of the vectors \(R_{t}(\theta)\)
, that \(EN^{i} = \theta \sum_{t = 1}^{k} \sum_{j = 1}^{n} EY_{ijt}\)
and that
\(\operatorname {Var}(N^{i}) = \theta \sum_{t = 1}^{k} \sum_{j = 1}^{n} \operatorname {Var}(Y_{ijt})\)
. We now use a bound due to Gallego (1992), which states that for any
random variable \(Z\) with mean \(\mu\) and finite variance
\(\sigma^{2}\) ,

\[
E(Z - z)^{+} \leq \frac{\sqrt{\sigma^{2} + (z - \mu)^{2}} - (z - \mu)}{2}.
\]

Applying this bound to the terms in the last sum in (14) and using the
fact that

\[
EN^{i} = \sum_{t = 1}^{\theta k} \sum_{j = 1}^{n} P(R_{t}^{j} > \mu^{*} A^{j}) a_{ij} \leq \theta x^{i}
\]

by the Kuhn Tucker conditions (10) for \(\mu^{*}\) , implies that

\[
E(N^{i} - \theta x^{i})^{+}\leq \frac{\sqrt{\operatorname{Var}(N^{i}) + (\theta x^{i} - E N^{i})^{2}} - (\theta x^{i} - E N^{i})}{2}
\]

\[
\leq \frac{\sqrt{\operatorname{Var}(N^{i})} + |\theta x^{i} - E N^{i}| - (\theta x^{i} - E N^{i})}{2}
\]

\[
= \frac{\sqrt{\operatorname{Var}(N^{i})}}{2}. \tag{15}
\]

Also note that by the Kuhn Tucker conditions (10),
\(\mu^{*}(\Sigma_{t = 1}^{\theta k}\Sigma_{j = 1}^{n}P(R_{t}^{j}(\theta) > \mu^{*}A^{j})A^{j} - \theta x) = 0\)
, so that

\[
\sum_{t = 1}^{\theta k}\sum_{j = 1}^{n}P(R_{t}^{j}(\theta) > \mu^{*}A^{j})\mu^{*}A^{j} = \theta \mu^{*}x. \tag{16}
\]

Using (15) and (16) in the second and third sums in (14) and using (13),
we obtain

\[
J_{\theta k}^{H}(\theta x)\geq \sum_{t = 1}^{\theta k}\sum_{j = 1}^{n}E(R_{t}^{j}(\theta) - \mu^{*}A^{j})^{+}
\]

\[
+\theta \mu^{*}x - \frac{C}{2}\sqrt{\theta}\sum_{i = 1}^{m}\sqrt{\sum_{t = 1}^{k}\sum_{j = 1}^{n}\operatorname{Var}(Y_{ijt})}
\]

\[
= \theta v_{k}(x) - O(\sqrt{\theta}),
\]

which completes the proof.

\section{4.3. Uniqueness of the Asymptotic Bid
Prices}\label{uniqueness-of-the-asymptotic-bid-prices}

We next address the uniqueness of the asymptotically optimal bid prices.
Although the upper bound problem (9) is convex in \(\mu\) , in general
it is not strictly convex and hence it may not have a unique solution.

To see this, we can write the function \(\overline{J}_{k}(x,\mu)\) as

\[
\overline{J}_{k}(x,\mu) = g(\mu A) + \mu x. \tag{17}
\]

where \(g:R^{n}\to R\) is defined as

\[
g(r) = \sum_{t = 1}^{k}\sum_{j = 1}^{n}E(R_{t}^{j} - r^{j})^{+}, \tag{18}
\]

and \(r = (r^{1},\ldots ,r^{n})\) . Now if there exists a \(t\) such
that \(P(R_{t}^{j} > r^{j}) > 0\) , \(\forall r^{j}\geq 0\) , then \(g\)
is strictly convex on \(R^{n}\) .

However, even if \(g\) is strictly convex, in general \(g(\mu A)\) is
only (weakly) convex in \(\mu\) . It is not hard to see that the
function \(g(\mu A)\) will be strictly convex in \(\mu\) if and

only if \(xA = yA\) implies \(x = y\) , which is equivalent to the
condition: \(xA = 0\) implies \(x = 0\) . But this is true if and only
if \(A\) has rank \(m\) . Therefore, we have the following sufficient
condition for uniqueness:

PROPOSITION 3. Suppose for all \(j\) there exists a \(t\) such that
\(P(R_{t}^{j} > r) > 0\) on \([0, + \infty)\) . Further, suppose rank
\((A) = m\) . Then the solution to (9) (and hence the vector of
asymptotically optimal bid prices) is unique.

In most practical settings, we would expect \(n\geqslant m\) and hence
it is highly likely that \(A\) will have rank \(m\) . In this case,
sufficiently large tails on the fare distributions will result in unique
asymptotically optimal bid prices.

However, multiple asymptotically optimal bid- price can occur if fare
distributions are highly concentrated. For example, one can easily
construct situations in which there is a range of bid prices that are
high enough to block a low fare class while still being low enough to
allow higher fare classes to book; each value produces the same
acceptance decision (with probability one) and hence all are
asymptotically optimal bid prices. Alternatively, it is possible that
because rank \((A)< m\) multiple solutions to (9) exist. As a simple
example, consider the case of two legs in series \((m = 2)\) with one
itinerary \((n = 1)\) that traverses both legs. In this case
\(\operatorname {rank}(A) = 1< m\) , and multiple solutions exist.
Specifically, it is easy to see in this case \(\overline{J}_{k}(x,\mu)\)
depends on \(\mu\) only through the sum \(\mu^{1} + \mu^{2}\) .

\section{4.4. Bid Prices and Opportunity
Cost}\label{bid-prices-and-opportunity-cost}

The above observations illustrate an important point; namely, there is
not a one- to- one correspondence between optimal bid prices and the
opportunity cost of leg capacity. That is, one can generate examples of
bid prices that give near optimal accept/deny decisions but at the same
time are very poor approximations to the marginal value of leg capacity.

As a simple example of this difference, consider a single- leg problem
in which high revenue fare classes arrive strictly before low revenue
fare classes. In this case, it is clear that it is optimal to accept
arrivals in first- come- first- serve (FCFS) order. Therefore, a
constant bid price of zero is optimal. On the other hand, the
opportunity cost \(J_{k}(x) - J_{k}(x - 1)\) at each point in time \(k\)
is certainly not zero. In other words, while it is sufficient to compare
the revenue to the true opportunity cost

\(J_{k}(x) - J_{k}(x - 1)\) at each point in time to make optimal
accept/deny decisions, it is not necessary to do so; other threshold
values may produce the same accept/deny decision and same optimal
revenues, as the value of zero does in this case.

One might argue that the real goal is to make the right accept/deny
decision and therefore it is not worth worrying about the difference
between optimal bid- price values and opportunity costs; however, in
practice a good estimate of opportunity cost is often essential. In
particular, for special event requests- especially ad- hoc group
bookings- which are typically not part of the forecast, one needs an
accurate assessment of opportunity cost to make a good decision.

The point, simply, is that one has to be careful about the
interpretation of the bid prices produced by any optimization algorithm.
Ideally, we would like the sum of leg bid prices along an itinerary to
represent the itinerary's opportunity cost. On the other hand, due to
``degeneracy'' of the value function, this may not always be achievable.
Yet despite this difficulty, Theorem 1 shows that a properly constructed
bid- price control rule is still asymptotically optimal against
forecasted demand. The algorithmic challenge, therefore, is to construct
bid prices which produce near- optimal acceptance decisions against
forecasted demand, while simultaneously providing good estimates of the
opportunity cost (whenever possible), so that special event (group)
requests can be properly evaluated.

\section{5. A Unified View of Bid Price Approximation
Schemes}\label{a-unified-view-of-bid-price-approximation-schemes}

It is quite helpful, to view bid price methods as corresponding to
various approximations of the optimal value function. That is, a given
approximation method \(A\) yields a function \(J_{k}^{A}(x)\) that
approximates \(J_{k}(x)\) . The bid prices are then the gradients of
\(J_{k}^{A}(x)\) , i.e.

\[
J_{k}(z) - J_{k}(x - A^{j})\approx \nabla_{x}J_{k}^{A}(x)A^{j},
\]

If the gradient does not exist, then \(\nabla_{x}J_{k}^{A}(x)\) above is
typically replaced (at least implicitly) by a subgradient of
\(J_{k}^{A}(x)\) . This interpretation of approximations schemes raises
two important questions: Is \(J_{k}^{A}(x)\) a good approximation of the
value function? And more importantly, is \(\nabla_{x}J_{k}^{A}(x)A^{j}\)
a good approximation of the opportunity cost? In this section, we
examine these questions for two popular approximation schemes and also
the asymptotic bid prices developed in Theorem 1.

cost? In this section, we examine these questions for two popular
approximation schemes and also the asymptotic bid prices developed in
Theorem 1.

Before proceeding, we note that in actual applications, the
approximation \(J_{k}^{A}(x)\) is usually resolved frequently to allow
the vector of bid prices to adjust to changes in remaining capacity
\(x\) and remaining time \(k\) . In this section, we assume the
approximations are solved for each \(x\) and \(k\) and compare the
resulting bid prices to optimal bid prices.

\section{5.1. Deterministic Linear Program
(DLP)}\label{deterministic-linear-program-dlp}

The deterministic linear programming method corresponds to the
approximation

\[
\begin{array}{c}{{J_{k}^{\mathrm{LP}}(x)=\min \sum_{j=1}^{n}E R_{j}y_{j},}}\\ {{A y\leq x,}}\\ {{0\leq y\leq E D,}}\end{array}
\]

where \(D = (D_{1},\ldots ,D_{n})\) and \(D_{i}\) denotes the demand to
come for itinerary \(j\) ( \(ED\) is the expected value of \(D\) ) and
\(R_{j}\) is the (possibly random) revenue associated with itinerary
\(j\) . In our earlier notation,
\(D_{j} = \Sigma_{t = 1}^{n} \mathbf{1}\{R_{t}^{\prime} > 0\}\) . The
decision variables \(y_{j}\) represent a discrete (nonnested), static
allocation of capacity to each itinerary \(j\) .

If the constraints \(A y\leq x\) are not degenerate (linearly dependent)
at the optimal solution, then \(\nabla J_{k}^{\mathrm{LP}}(x)\) exists
and is given by the unique vector of optimal dual prices associated with
these constraints; if these constraints are degenerate, then there are
multiple optimal dual price vectors, each of which is only a subgradient
of the function \(J_{k}^{\mathrm{LP}}(x)\) .

The most serious weakness of the DLP formulation is that it considers
only the mean demand and ignores all other distributional information.
As a consequence, the dual values are zero on any leg that has a mean
demand less than capacity. Despite this deficiency, Williamson's (1992,
Chapter 6) extensive simulation studies showed that with frequent
reoptimization, the performance of DLP bid prices is quite good,
producing higher revenue than both probabilistic math programming models
(see below) and a variety of leg- based EMSR heuristics.

5.2. Probabilistic Nonlinear Program (PNLP) The probabilistic nonlinear
programming method corresponds to the approximation

\[
J_{k}^{\mathrm{PNLP}}(x) = \min \sum_{j = 1}^{n}E R_{j}E\min \{D_{j},y_{j}\} ,
\]

\[
A y\leq x,
\]

\[
y = 0,
\]

where again \(D_{j}\) and \(R_{j}\) are defined as in the DLP case. As
in the DLP, the decision variables \(y_{j}\) represent a discrete,
static allocation of capacity to each itinerary \(j\) . If the
constraints \(A y \leq x\) are not degenerate at the optimal solution,
then \(\nabla J_{k}^{\mathrm{LP}}(x)\) exists and is given by the unique
vector of optimal dual prices associated with these constraints; if
these constraints are degenerate, the multiple optimal dual vectors are
subgradients of the function \(J_{k}^{\mathrm{LP}}(x)\) .

This formulation appears somewhat better than the DLP, in that the term
\(E \min \{D_{j}, y_{j}\}\) in the objective function captures the
randomness in demand. However, the assumption of a discrete, static
allocations of capacity to each fare class can lead to poor behavior.
This behavior was demonstrated empirically in Williamson's (1992,
Chapter 6) simulation studies, in which she observed that the PNLP bid
prices consistently produced lower revenues than the DLP bid prices. In
a further computational comparison between the DLP and PNLP, Talluri
(1996) found similar behavior.

To understand the weakness of the PNLP approximation, consider a problem
with \(m = 1\) leg and \(n\) itineraries on the leg, each of which is
identical. Assume each itinerary \(j\) has demand, \(D_{j} \sim D\) ,
where \(D\) is normally distributed with mean \(\mu\) and standard
deviation \(\sigma\) and that each itinerary has the same deterministic
revenue \(r\) . The PNLP formulation is then

\[
J_{k}^{\mathrm{PNLP}}(x) = \min \sum_{j = 1}^{n}r E\min \{D,y_{j}\} ,
\]

\[
\sum_{j = 1}^{n}y_{j} = x,
\]

\[
y\geq 0,
\]

where \(D \sim N(\mu , \sigma)\) . By symmetry, the optimal solution is
\(y_{j} = x / n\) , \(j = 1, \ldots , n\) and hence the Kuhn- Tucker
conditions imply the optimal dual price \(\lambda\) satisfies

\[
\lambda = r P(n D > x) = r\left(1 - \Phi \left(\frac{x - n\mu}{n\sigma}\right)\right), \tag{19}
\]

where \(\Phi (x)\) is the CDF of the standard normal distribution. The
value \(\lambda\) above then forms our estimate of the marginal value of
the \(x\) th seat.

But since all itineraries are identical, the marginal value in this
problem should be the unchanged if we aggregate all \(n\) fare classes
into one fare class with mean \(n\mu\) and variance \(n\sigma^{2}\) .
Aggregating and applying the PNLP we find the optimal dual multiplier in
this case satisfies

\[
\lambda = r P\left(\sum_{j = 1}^{n}D_{j} > x\right) = r\left(1 - \Phi \left(\frac{x - n\mu}{\sqrt{n}\sigma}\right)\right). \tag{20}
\]

If \(x \neq n\mu\) and \(n\) is large, (19) and (20) give very different
estimates of the marginal value. Of course, from first principles we
know the true opportunity cost is completely independent of how we
aggregate (or disaggregate) these identical itineraries.

While on the surface this seems like a contrived example, it is not
unreasonable to expect a similar type of behavior in large hub- and-
spoke networks. For example, if many uncongested in- bound legs have
connecting passengers traveling on a single congested outbound leg and
passengers pay comparable revenues, then the situation is quite similar
to the example above. That is, one would like to treat all passengers as
a single fare class (i.e.~``nest'' the fare classes) but the PNLP
allocates space to each separately, resulting in a distorted estimate of
the marginal value of the leg. In contrast, the DLP method does indeed
posses this ``nesting'' property, since, in the above example,
aggregating all \(n\) fare classes does not change the resulting DLP bid
price.

\section{5.3. Prorated EMSR}\label{prorated-emsr}

Another method for computing estimates of bid prices is to use a
prorated expected marginal seat revenue (PEMSR) scheme. Originally
proposed in Williamson (1992), PEMSR schemes involve allocating a
portion of the revenue of each itinerary to the legs of the itinerary.
One then solves \(m\) leg- level problems using the expected marginal
revenue (EMSR) heuristic proposed by Belobaba (1989). The resulting EMSR
values from each leg are then used as bid prices.

Specifically, let \(\alpha = (\alpha_{1}, \ldots , \alpha_{m})\) be a
non- negative real vector. For each itinerary \(j\) , define new
revenues, one for each leg in the itinerary, by

Table 2 Problem Data for Iterative Allocation Example

\begin{longtable}[]{@{}llll@{}}
\toprule\noalign{}
\endhead
\bottomrule\noalign{}
\endlastfoot
Time (k) & Itin. (A) & Fare & Prob. \\
\multirow{2}{*}{2} & AB & \$100 & 0.5 \\
& CD & \$100 & 0.5 \\
\multirow{2}{*}{1} & ABC & \$1000 & 0.5 \\
& BCD & \$1000 & 0.5 \\
\end{longtable}

\[
R_{ij} = \frac{\alpha_i}{\Sigma_{i\in A_j}\alpha_i} R_{j}, i\in A_j.
\]

Next, treat each leg \(i\) independently as if it received demand
\(D_{j},\) but with reduced revenue \(R_{ij}\) and solve the
corresponding leg- level EMSR.1 The approximation to the value function
is then

\[
J_{k}^{\mathrm{PEMSR}}(x) = \sum_{i = 1}^{m}J_{i}(x_{i},\alpha),
\]

where \(J_{i}(x_{i},\alpha)\) denotes the expected revenue of leg \(i\)
under the allocation \(\alpha\)

Williamson (1992) investigated several methods for determining the
allocation \(\alpha_{j}\) including prorating based on mileage, number
of legs and the relative revenue value of local demand on each leg. Her
conclusion is that none of these fixed allocations is very robust in
general. Indeed, it is not hard to see that if one leg of an itinerary
is highly congested and all others have abundant capacity, then the
revenue of the itinerary should be entirely allocated to the congested
leg. Depending on the realization of demand, however, the congested leg
could be any of the legs on the itinerary; hence, no fixed allocation
scheme can be expected to work well in all cases.

An intriguing idea along these lines, again appearing in Williamson
(1992, p.~107) but not pursued fully there, is to prorate revenues using
an iterative loop. That is, first obtain the marginal values based on
some initial proration scheme and EMSR calculations. Then, use these
marginal values to do the next round of proration and EMSR calculations.
Repeat these iterations until the marginal values (hopefully) converge.
The hope here is that the bid- prices will converge to a near- optimal
set of bid- prices. Unfortunately there is little theoretical
justification behind this idea.

Even if the bid prices converge to some value (it is not entirely clear
if convergence is guaranteed), they do not necessarily converge to a
good set of bid- prices. For example, consider a three- leg line
network, with nodes A, B, C and D. Each of the three legs, AB, BC and
CD, has one remaining seat. Suppose \(t = 2\) and we have data for
itinerary arrivals as shown in Table 2. If we start with an allocation
of fares for \(t = 1\) using equal weights, prorate the fares in period
\(t = 1\) by these weights, and then compute the expected marginal value
of each leg we get \(\mu_{\mathrm{AB}} = 250\) ,
\(\mu_{\mathrm{BC}} = 500\) and \(\mu_{\mathrm{CD}} = 250\) .

The results of repeated applications of this procedure are shown in
Table 3. Note that the bid prices converge to \(\mu_{\mathrm{AB}} = 0\)
, \(\mu_{\mathrm{BC}} = 1000\) and \(\mu_{\mathrm{CD}} = 0\) . However,
by inspection of the data in Table 2, it is clear that we want to reject
both of the itineraries arriving in period \(t = 2\) , so we need
\(\mu_{\mathrm{AB}} > 100\) and \(\mu_{\mathrm{CD}} > 100\) . Such a
policy yields an expected revenue of
{\$\textbackslash)1,000\textbackslash(. Because the iterative proration
scheme produces zero bid prices for legs AB and CD, it accepts both of
the itineraries in period\$} t=2 {\$\textbackslash), generating an
expected revenue of only \textbackslash(\$} 600\$.

One problem with the iterative proration scheme is that once the problem
is broken up into leg- level problems, all network information is lost.
Additional problems can arise due to the fact that many EMSR methods
assume fare classes have ordered arrivals, usually with lower fare
classes arriving before high fare classes, see Belobaba (1987, 1989),
Brumelle et al.~(1990), and Curry

Table 3 Example of Convergence of Iterative Proration Scheme \((t = 1)\)

\begin{longtable}[]{@{}|l|l|l|l|@{}}
\toprule\noalign{}
\endhead
\bottomrule\noalign{}
\endlastfoot
\hline
Iter. & μAB & μBC & μCD \\
\hline
0 & 250.0 & 500.0 & 250.0 \\
\hline
1 & 166.7 & 666.7 & 166.7 \\
\hline
2 & 100 & 800 & 100 \\
\hline
. & . & . & . \\
\hline
. & . & . & . \\
\hline
. & . & . & . \\
\hline
∞ & 0 & 1000 & 0 \\
\hline
\end{longtable}

(1989). In a proration scheme, the regular ordering of arrivals is often
destroyed, and certainly any assumption of low prorated revenues
arriving strictly before high prorated revenues becomes untenable.

\section{5.4. Asymptotic Bid Prices}\label{asymptotic-bid-prices}

The asymptotic analysis provides an alternative approximation approach.
Indeed, note from (9) that \(\nabla_{x}v_{k}(x) = \mu^{*}\) , so we can
view the upper bound \(v_{k}(x)\) as an approximation of \(J_{k}(x)\)
with \(\mu^{*}\) its (sub)gradient. The approximation (9) is somewhat
unique in that it is solved directly in the space of the bid prices
\(\mu ,\) whereas in the DLP and NLP methods, \(\mu\) is a dual value of
a problem whose primal variables are inventory allocations.

The approximation (9) has the ``nesting'' property because the objective
function in (9) sums all arriving itineraries \(j\) whose revenue
exceeds the fixed thresholds \(\mu A^{j}\) . As a result, it does not
suffer from the discrete allocation problem of the PNLP method. For
example, suppose two itineraries \(j_{1}\) and \(j_{2}\) are entered as
separate columns of \(A_{j}\) but in reality \(A^{j_{1}} = A^{j_{2}}\)
and each has the same fare distribution. If these two itineraries were
combined into a new itinerary (by adding the probabilities of arrival in
each period together) the asymptotic bid prices would not change, which
is the correct behavior.

At the same time, the approximation (9) suffers from the same weakness
as the DLP in that its value only depends on the first moment of demand.
Indeed, if, as above, we let \(D_{j}\) denote the demand to come for
itinerary \(j\) and \(R_{j}\) denote the random revenue associated with
itinerary \(j,\) then (9) becomes

\[
v_{k}(x) = \min_{\mu \geq 0}\sum_{j = 1}^{n}ED_{j}E(R_{j} - \mu A_{j})^{+} + \mu^{T}x,
\]

which only depends on \(ED_{j}\)

The problem here is that the asymptotic analysis is too ``coarse'' to
capture some of the second- order stochastic effects which the PNLP
captures. As a result, as in the DLP case, the asymptotic bid- prices
will turn out to be zero if the mean demand on a leg is strictly less
than its capacity. This is indeed correct behavior asymptotically, but
for demands with high variance near capacity, the actual bid price could
be significantly larger than zero, possibly even more than some of the
fares of the lower fare classes. Indeed, under appropriate scaling, one
can show that, as fare variances tend to zero, the asymptotic bid prices
from (9) approach the DLP bid prices. One can then combine this result
with Theorem 1 to show that as fare variances tend to zero, the DLP bid
prices are also asymptotically optimal.

However, the asymptotic approximation, unlike the DLP and PNLP methods,
accounts for variability in itinerary revenues, which provides a
significant advantage in approximating optimal bid prices when fares
vary significantly within itinerary / fare- classes.

As a simple example of this effect, consider a singleleg problem with
only one fare class. Suppose the actual fares \(R\) vary and have
distribution \(F(r)\) with mean \(ER\) If the mean demand, \(ED_{i}\) is
significantly higher than capacity, \(x_{i}\) the bid price produced by
DLP and PNLP methods both approach \(ER\) . That is, reducing the leg
capacity by one almost certainly results in a lost sale, which each of
these models values at \(ER\) . Under an optimal bid- price policy,
however, the bid price rises above \(ER\) as the demand / capacity ratio
increases. This occurs because, with many requests to choose from, it is
optimal to be selective and accept only the higher fares within the fare
class (i.e.~fares in the ``right tail'' of the distribution) rather than
accepting all fares. From (9), one can show that the optimal bid price
in the above example tends to a value \(r^{*}\) satisfying

\[
ED(1 - F(r^{*})) = x.
\]

Depending on the distribution of \(F_{\prime}\) the value of \(r^{*}\)
can be significantly higher than \(ER\) ; hence the DLP and PNLP methods
will under- estimates the optimal bid price. A complimentary effect can
occur when the demand / capacity ratio is small; in this case, the DLP
and NLP models may tend to over- estimate the optimal bid price.



\end{document}
