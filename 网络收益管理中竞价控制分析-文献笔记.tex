\documentclass[12pt,a4paper]{article}
\usepackage[UTF8]{ctex}
\usepackage{amsmath}
\usepackage{amsfonts}
\usepackage{amssymb}
\usepackage{xcolor}
\usepackage{geometry}
\usepackage{hyperref}

\geometry{left=2.5cm,right=2.5cm,top=2.5cm,bottom=2.5cm}

\title{网络收益管理中竞价控制分析——文献笔记}
\author{研究者}
\date{\today}

\begin{document}

\maketitle

\section{引言}

网络收益管理中的竞价控制是一个重要的研究领域,它在航空公司、酒店等行业中有着广泛的应用。本文献笔记总结了该领域的主要理论成果和实践方法。

\textcolor{blue}{\textbf{定义1:竞价控制}}

一个控制策略,其中为每个网络边设置阈值价格(称为竞价),只有当产品的票价超过其所需资源的竞价之和时,才会销售该产品。

\section{理论基础}

\textcolor{blue}{\textbf{定理1:最优性定理}}

在腿容量和销售量较大的情况下,竞价控制策略是渐近最优的,前提是使用正确的竞价。

网络收益管理的动态规划模型可以表示为:
\[
J_k(x) = \max_{u_k(\cdot) \in \mathcal{U}(x)} E[R_k u_k(x, R_k) + J_{k-1}(x - A u_k(x, R_k))]
\]

其中$J_k(x)$表示在时间$k$、容量状态$x$下的最大期望收益。

\textcolor{blue}{\textbf{命题1:控制结构命题}}

如果$R_k^j$具有有限的一阶矩,则存在最优控制$u_k^*$,其形式为:
\[
u_k^*(x,r^j) = \begin{cases}
1 & \text{如果 } r^j \geq J_{k-1}(x) - J_{k-1}(x - A^j) \text{ 且 } A^j \leq x \\
0 & \text{其他情况}
\end{cases}
\]

\section{竞价控制的非最优性}

\textbf{例子1:竞价控制非最优性反例}

考虑一个包含两条腿的简单网络,每条腿有一个单位容量,剩余时间为两个周期。在第2周期,有两个本地行程,每个票价为\$250,到达概率为0.3;一个通程行程票价为\$500,到达概率为0.4。在最后一个周期,只有一个通程票价,收益为\$500,到达概率为0.8。

在这个例子中,无法找到满足以下条件的竞价$\mu_1$和$\mu_2$:
\begin{itemize}
\item $\mu_1 > 250$
\item $\mu_2 > 250$  
\item $\mu_1 + \mu_2 \leq 500$
\end{itemize}
这表明竞价控制不是普遍最优的。

\textcolor{red}{\textbf{重要观察:}}

竞价控制失效的根本原因在于价值函数的非线性性质。具体来说,机会成本$J_k(x) - J_k(x - A^j)$不等于各腿机会成本的和$\sum_{i \in A^j}(J_k(x) - J_k(x - e_i))$。

\section{渐近分析}

\textcolor{blue}{\textbf{定义2:上界问题}}

考虑放松版本的问题:
\[
\bar{J}_k(x,\mu) = \max_{\{u_t \in E_n\}} E\left[\sum_{t=1}^k (R_t - \mu A) u_t\right] + \mu x
\]
其中$\mu \geq 0$是拉格朗日乘子向量。

\textcolor{blue}{\textbf{定理2:渐近最优性定理}}

如果$R_t^j \leq C$(几乎处处),则
\[
\frac{J_{\theta k}^H(\theta x)}{J_{\theta k}(\theta x)} \geq 1 - O(\theta^{-1/2})
\]
特别地,
\[
\lim_{\theta \to \infty} \frac{J_{\theta k}^H(\theta x)}{J_{\theta k}(\theta x)} = 1
\]

\section{竞价近似方法}

\subsection{确定性线性规划(DLP)}

\textcolor{blue}{\textbf{定义3:DLP方法}}

对应于近似值函数:
\[
J_k^{LP}(x) = \max \sum_{j=1}^n E[R_j] y_j
\]
约束条件为:
\begin{align}
A y &\leq x \\
0 &\leq y \leq E[D]
\end{align}

\subsection{概率非线性规划(PNLP)}

\textcolor{blue}{\textbf{定义4:PNLP方法}}

对应于近似值函数:
\[
J_k^{PNLP}(x) = \max \sum_{j=1}^n E[R_j] E[\min\{D_j, y_j\}]
\]
约束条件为:
\begin{align}
A y &\leq x \\
y &\geq 0
\end{align}

\textcolor{red}{\textbf{方法比较:}}

DLP方法只考虑需求的均值而忽略其他分布信息,但在频繁重新优化的情况下表现良好。PNLP方法虽然考虑了需求的随机性,但由于离散静态分配的假设,实际表现常常不如DLP方法。

\section{实际应用考虑}

\textbf{例子2:航空公司应用}

在实际的航空收益管理系统中,竞价控制已被多家航空公司采用。其优势包括:
\begin{itemize}
\item 计算复杂度低
\item 易于实现和维护
\item 适用于大规模网络
\item 可处理动态定价环境
\end{itemize}

\textcolor{red}{\textbf{挑战与限制:}}

\begin{itemize}
\item 在网络容量受限的情况下可能不是最优的
\item 需要准确的需求预测
\item 对价格弹性的假设可能不现实
\item 竞争环境下的表现需要进一步研究
\end{itemize}

\section{结论}

\textcolor{blue}{\textbf{命题2:主要结论}}

虽然竞价控制在一般情况下不是最优的,但在腿容量和销售量较大的渐近情况下,使用适当的竞价可以实现最优性。这为实际应用提供了理论基础。

\textcolor{red}{\textbf{未来研究方向:}}

\begin{itemize}
\item 非平稳需求环境下的竞价控制
\item 竞争市场中的策略行为
\item 机器学习方法在竞价优化中的应用
\item 实时动态调整机制的设计
\end{itemize}

\section{参考文献}

主要参考了Talluri和van Ryzin (1998)关于网络收益管理中竞价控制分析的经典论文,以及相关的后续研究成果。

\end{document}