% This LaTeX document needs to be compiled with XeLaTeX.
\documentclass[
  10
]{article}
\usepackage[margin=2cm,includehead=true,includefoot=true,centering,]{geometry}
\usepackage{xcolor}
\usepackage{ucharclasses}
\usepackage{hyperref}
\usepackage{amsmath,amssymb}
\usepackage{amsfonts}
\usepackage[version=4]{mhchem}
\usepackage{stmaryrd}
\usepackage{bbold}
\usepackage{polyglossia}
\usepackage{fontspec}
\usepackage{ctex}
\usepackage[export]{adjustbox}
\usepackage{tabularx}
\usepackage{booktabs}

\usepackage{setspace}
\setstretch{1.2}

\makeatletter
\@ifundefined{KOMAClassName}{% if non-KOMA class
  \IfFileExists{parskip.sty}{%
    \usepackage{parskip}
  }{% else
    \setlength{\parindent}{0pt}
    \setlength{\parskip}{6pt plus 2pt minus 1pt}}
}{% if KOMA class
  \KOMAoptions{parskip=half}}
\makeatother
\usepackage{longtable,booktabs,array}
\usepackage{calc} % for calculating minipage widths
% Correct order of tables after \paragraph or \subparagraph
\usepackage{etoolbox}
\makeatletter
\patchcmd\longtable{\par}{\if@noskipsec\mbox{}\fi\par}{}{}
\makeatother
% Allow footnotes in longtable head/foot
\IfFileExists{footnotehyper.sty}{\usepackage{footnotehyper}}{\usepackage{footnote}}
\makesavenoteenv{longtable}
\usepackage{graphicx}
\makeatletter
\newsavebox\pandoc@box
\newcommand*\pandocbounded[1]{% scales image to fit in text height/width
  \sbox\pandoc@box{#1}%
  \Gscale@div\@tempa{\textheight}{\dimexpr\ht\pandoc@box+\dp\pandoc@box\relax}%
  \Gscale@div\@tempb{\linewidth}{\wd\pandoc@box}%
  \ifdim\@tempb\p@<\@tempa\p@\let\@tempa\@tempb\fi% select the smaller of both
  \ifdim\@tempa\p@<\p@\scalebox{\@tempa}{\usebox\pandoc@box}%
  \else\usebox{\pandoc@box}%
  \fi%
}
% Set default figure placement to htbp
\def\fps@figure{htbp}
\makeatother
\setlength{\emergencystretch}{3em} % prevent overfull lines
\providecommand{\tightlist}{%
  \setlength{\itemsep}{0pt}\setlength{\parskip}{0pt}}


\hypersetup{colorlinks=true, linkcolor=blue, filecolor=magenta, urlcolor=cyan,}
\urlstyle{same}

\usepackage{colortbl}
\definecolor{table-row-color}{HTML}{999999}
\definecolor{table-rule-color}{HTML}{999999}
%\arrayrulecolor{black!40}
\arrayrulecolor{table-rule-color}     % color of \toprule, \midrule, \bottomrule
\setlength{\aboverulesep}{0pt}
\setlength{\belowrulesep}{0pt}


\setotherlanguages{english}
\IfFontExistsTF{Source Han Serif CN}
{\newfontfamily\chinesefont{Source Han Serif CN}}
{\IfFontExistsTF{Noto Serif CJK SC}
  {\newfontfamily\chinesefont{Noto Serif CJK SC}}
  {\IfFontExistsTF{SimSun}
    {\newfontfamily\chinesefont{SimSun}}
    {\IfFontExistsTF{FangSong}
      {\newfontfamily\chinesefont{FangSong}}
      {\newfontfamily\chinesefont{Arial Unicode MS}}
}}}
\IfFontExistsTF{Times New Roman}
{\newfontfamily\englishfont{Times New Roman}}
{\IfFontExistsTF{Liberation Serif}
  {\newfontfamily\englishfont{Liberation Serif}}
  {\IfFontExistsTF{DejaVu Serif}
    {\newfontfamily\englishfont{DejaVu Serif}}
    {\newfontfamily\englishfont{Arial}}
}}


\author{}
\date{}

\begin{document}


\section{Optimal Subsampling for Data Streams with Measurement
Constrained Categorical
Responses}\label{optimal-subsampling-for-data-streams-with-measurement-constrained-categorical-responses}

Jun Yu, Zhiqiang Ye, Mingyao Ai \& Ping Ma

To cite this article: Jun Yu, Zhiqiang Ye, Mingyao Ai \& Ping Ma (18 Dec
2024): Optimal Subsampling for Data Streams with Measurement Constrained
Categorical Responses, Journal of Computational and Graphical
Statistics, DOI: 10.1080/10618600.2024.2421990

To link to this article: https://doi.org/10.1080/10618600.2024.2421990

\section{Optimal Subsampling for Data Streams with Measurement
Constrained Categorical
Responses}\label{optimal-subsampling-for-data-streams-with-measurement-constrained-categorical-responses-1}

Jun \(\mathsf { Y u ^ { a } }\) , Zhiqiang
\(\mathsf { Y e } ^ { \mathsf { b } }\) , Mingyao
\(\mathsf { A i } ^ { \mathsf { b } } \oplus\) , and Ping Mac

aSchool of Mathematics and Statistics, Beijing Institute of Technology,
Beijing, China; bLMAM, School of Mathematical Sciences and Center for
Statistical Science, Peking University, Beijing, China; cDepartment of
Statistics, University of Georgia, Athens, GA

\section{ABSTRACT}\label{abstract}

High-velocity, large-scale data streams have become pervasive.
Frequently, the associated labels for such data prove costly to measure
and are not always available upfront. Consequently, the analysis of such
data poses a significant challenge. In this article, we develop a method
that addresses this challenge by employing an online subsampling
procedure and a multinomial logistic model for efficient analysis of
high-velocity, large-scale data streams. Our algorithm is designed to
sequentially update parameter estimation based on the A-optimality
criterion. Moreover, it significantly increases computational efficiency
while imposing minimal storage requirements. Theoretical properties are
rigorously established to quantify the asymptotic behavior of the
estimator. The method's efficacy is further demonstrated through
comprehensive numerical studies on both simulated and real-world
datasets. Supplementary materials for this article are available online.

ARTICLE HISTORY Received February 2024 Accepted October 2024

\section{KEYWORDS}\label{keywords}

Massive data; Multinomial logistic model; Online updating; Poisson
sampling; Semi-supervised learning

\section{1. Introduction}\label{introduction}

Data streams represent a sequence of data points, with each point
arriving sequentially---a phenomenon that has become ubiquitous in
contemporary settings (Li et al.~2019; Xie, Bai, and Ma 2023). Unlike
the analysis of static data, the examination of data streams presents
two distinct challenges. First, data volumes often surpass the storage
capacity of a single device, especially when dealing with high-velocity
data streams. This poses a significant hurdle in storing the data points
and providing timely analysis. Second, the process of labeling data can
be both expensive and time-consuming, leading to a scenario where labels
for many records may be unavailable, despite the presence of covariates.
This challenge is commonly referred to as measurement constraints or
semi-supervised learning in the statistical literature. For example, in
the analysis of forest cover types, environmental information can be
easily obtained by remote sensing, but relaying that information from
the remote sensor to the forest cover types is usually handled manually
and slower than the data collection velocity of the sensors. Addressing
these multifaceted challenges is imperative to unlock the full potential
of data-driven analytics.

To analyze high-velocity, unlabeled streaming data effectively, a
desired statistical approach needs to meet the following three
principles. First, online analytical processing should provide a timely
and accurate response. Second, to solve the measurement constraint
problem, the algorithm can sample a relatively small portion of the data
batch at a time when the expensive responses will be measured and the
resultant sampling estimator is statistically efficient. Finally, the
analysis should reduce data storage, and the analytical processing tries
to avoid accessing the historical data (Schifano et al.~2016).

To meet the first two principles, one natural solution is subsampling.
Ample evidence shows that the subsampling efficiently balances the
statistical accuracy and computational cost. Typical works in this line
of research include but are not limited to, Wang, Zhu, and Ma (2018),
Wang, Yang, and Stufken (2019), Han et al.~(2020), and Wang et
al.~(2021). Particularly, the leverage score subsampling (Ma, Mahoney,
and \(\mathrm { Y u } 2 0 1 5\) ; Ma et al.~2022), Lowcon (Meng et
al.~2020), and OSMUC subsampling (Zhang, Ning, and Ruppert 2021b) were
proposed for linear/generalized linear model raise an efficient solution
on analysis the massive data under measurement constraint. Recently, Wu
et al.~(2024) proposed a predictive-inference-based subsampling for
semi-supervised learning. Unfortunately, these approaches typically
assume access to the full sample before initiating the subsampling
procedure. Thus, they may fail to give a timely response to analyzing
high-velocity data streams. Moreover, most subsampling strategies do not
consider how the selected subsample at the current stage affects the
online data analysis in the future stages. Methods that hold promise in
addressing the challenges of analyzing data streams are distributed
subsampling methods. These methods divide the data into several subsets
and then apply the subsampling method or subdata selection to each
subset separately. Examples of such works include Yu et al.~(2022) and
Zhang and Wang (2021), along with mini-batch stochastic gradient descent
methods proposed by Robbins and Monro (1951), Dekel et al.~(2012), and
Zhang et al.~(2021a). Nevertheless, these methods largely adhere to the
three principles when applied directly to high-velocity data streams,
especially in cases where storing or accessing historical data is not
feasible. In an online learning framework, the full dataset cannot be
used by iterating over it, as unselected data points are discarded and
cannot be revisited. As a result, conventional mini-batch stochastic
gradient descent may become inefficient, and leveraging informative data
points for parameter updates can naturally enhance statistical
efficiency. However, the large sampling variance from simple random
sampling leads to inexact gradient information, affecting the
optimality, convergence, and overall implementation of algorithms.
Therefore, an optimal sequential subsampling method and an online
updating estimator, based on selected subsamples, are crucial to
addressing the challenges posed by data storage and measurement
constraints in large-scale, high-velocity data streams.

The primary goal of our work is to develop a sequential subsampling
strategy for online estimating procedures for measurement constrained
categorical responses when data arrives in a high-velocity stream. Here
we restrict ourselves to categorical responses since they are prevalent
in big data, particularly under measurement constraints. In pursuing
this goal, we opt for the multinomial logistic model, widely recognized
for its efficacy in modeling categorical responses across diverse
scientific disciplines. Note that multinomial logistic models encompass
logistic regression and softmax regression as special cases. In our
proposed method, we introduce an online updating estimation approach
based on the sequentially selected subsample with theoretical backups,
including consistency and asymptotic normality. We employ a first-order
gradient-based method to meet the needs of online updating without
revisiting the historical information. Moreover, we develop optimal
sequential subsampling procedures that minimize the asymptotic MSE of
onestep updating and online updating with the number of blocks going to
infinity, respectively. We show that consecutively applying the optimal
probabilities for a one-step estimator does not yield an optimal
solution for online data analysis. Furthermore, the first-order
gradient-based method can further accelerate computation compared to
Newton-type methods, making the method more responsive to high-velocity
data streams. More precisely, the proposed method eliminates the need to
update the Hessian matrix during online updating, making it particularly
well-suited for scenarios where the selected subsampling in the data
chunk is small, and the corresponding Hessian matrix does not exist.

The rest of this article is organized as follows. In Section 2, we
introduce the multinomial logistic models and illustrate the general
online subsampling and estimation procedure. The consistency of the
resulting estimator is also derived. In Section 3, the optimal
subsampling probabilities for one-step updating and online updating are
derived. Some practical implementation of the optimal subsampling
procedures is also considered. Simulation studies and real data analyses
are provided in Section 4. Section 5 concludes this article. All proofs
are postponed to the supplementary material.

\section{2. Multinomial Logistic Model and its Online
Updating}\label{multinomial-logistic-model-and-its-online-updating}

\section{2.1. Multinomial Logistic
Model}\label{multinomial-logistic-model}

Let \(\{ ( \pmb { x } _ { i } , y _ { i } ) \} _ { i = 1 } ^ { N }\) be
a sequence of independent and identically distributed random variables
with response \(y _ { i } \in \{ 1 , . . . , J \} ( J \geq\) 2). The
response \(y _ { i }\) can be regarded as a random variable coming from
a multinomial distribution with \(\pi _ { i j } ( \beta )\) being the
conditional probability that the response \(y _ { i }\) is
\(j ( j = 1 , \ldots , J )\) given the covariate \(x _ { i } , ~ \beta\)
is the unknown parameter vector. More precisely, the \(\pi _ { i j }\)
is characterized by the following parametric form:

\[
\begin{array} { l } { \pi _ { i j } ( \beta ) = \displaystyle \frac { \exp ( f ( x _ { i } , \beta _ { 0 } , \beta _ { j } ) ) } { 1 + \sum _ { k = 1 } ^ { J - 1 } \exp ( f ( x _ { i } , \beta _ { 0 } , \beta _ { k } ) ) } , \quad \mathrm { f o r ~ } j = 1 , \ldots , J - 1 , } \\ { \pi _ { i J } ( \beta ) = \displaystyle \frac { 1 } { 1 + \sum _ { k = 1 } ^ { J - 1 } \exp ( f ( x _ { i } , \beta _ { 0 } , \beta _ { k } ) ) } , } \end{array}
\]

where \(f ( \cdot )\) is a known function,
\(\pmb { \beta } _ { 0 } \in \mathbb { R } ^ { d _ { 0 } }\) stands for
the common parameters among all categories,
\(\beta _ { j } ~ \in ~ \mathbb { R } ^ { d _ { j } } ( j ~ = ~ 1 , \ldots , J ~ -\)
1) stands for the individual parameters belonging to the jth category
only, and
\(\pmb { \beta } = ( \pmb { \beta } _ { 0 } ^ { \mathrm { T } } , \dots , \pmb { \beta } _ { J - 1 } ^ { \mathrm { T } } ) ^ { \mathrm { T } } \in \mathbb { R } ^ { d }\)
being parameters of interest for the entire model. Here
\(d = d _ { 0 } + \dots + d _ { J - 1 }\) denotes the dimension of the
parameter space. Consequently, we can rewrite the model as

\[
\log \left( \frac { \pi _ { i j } ( \pmb { \beta } ) } { \pi _ { i J } ( \pmb { \beta } ) } \right) = f ( \pmb { x } _ { i } , \pmb { \beta } _ { 0 } , \pmb { \beta } _ { j } ) ,
\]

which is known as the multinomial logistic model with the baseline
category link function. The true parameter vector
\(\beta _ { \mathrm { { _ { t r u e } } } }\) is the maximizer of the
expected population likelihood

\[
\begin{array} { l } { \displaystyle \mathrm { E L } ( \pmb { \beta } ) = \mathrm { E } \left( \sum _ { j = 1 } ^ { J - 1 } \mathbb { I } ( y = j ) f ( \pmb { x } , \pmb { \beta } _ { 0 } , \pmb { \beta } _ { j } ) \right. } \\ { \displaystyle \left. - \log \left( 1 + \sum _ { j = 1 } ^ { J - 1 } \exp \left( f ( \pmb { x } , \pmb { \beta } _ { 0 } , \pmb { \beta } _ { j } ) \right) \right) \right) , } \end{array}
\]

where \(\mathbb { I } ( \cdot )\) is an indicator function. Clearly,
when
\(f ( \pmb { x } _ { i } , \beta _ { 0 } , \pmb { \beta } _ { j } ) =\)
\(\pmb { x } _ { i } ^ { \mathrm { T } } \pmb { \beta } _ { j }\) with
\(\beta _ { 0 } \quad = \quad 0\) , Model (1) reduces to be the softmax
regression whose subsampling strategies are considered in Yao and Wang
(2019). More specifically, for the case \(J = 2\) , Models (1) become
the well-known logistic regression, and the subsampling methods are
studied in Wang, Zhu, and Ma (2018). It is worth mentioning that the
baseline category function is not the only link function associated with
a multinomial logistic model. The cumulative link function,
adjacent-categories link function, and continuation-ratio link functions
are also widely adopted in practice. For clarity, we state the results
for Model (1) in the main text and relegate the results for the other
three links to the supplementary material.

\section{2.2. Online Subsampling and Estimation
Procedure}\label{online-subsampling-and-estimation-procedure}

In the online semi-supervised learning framework, the full dataset is no
longer provided all at once. Instead, it arrives in a data block
\(\{ \boldsymbol { z } _ { t i } \} _ { i = 1 } ^ { n }\) for
\(t = 0 , . . . , T\) from a large data stream, where \(n\) is the
number of observations in each block and \(T\) is the number of blocks.
We assume the first block contains all the labeled historical data, that
is,
\(\boldsymbol { z } _ { t i } ~ = ~ ( \boldsymbol { x } _ { t i } ^ { \mathrm { T } } , \boldsymbol { y } _ { t i } ) ^ { \mathrm { T } }\)
for \(t = 0\) and the unlabeled data arrives in the rest blocks with
\(z _ { t i } = x _ { t i }\) for \(t = 1 , . . . , T\) . An online
subsampling is referred to a method that selects \(r \leq n\) unlabeled
data from tth block
\(\{ \boldsymbol { z } _ { t i } \} _ { i = 1 } ^ { n }\) with \(t\)
ranging from 1 to \(T\) . Once the selected sample of each block is
obtained, we can get access to the labels of them and then update the
estimator via the labeled data in the current batch.

To develop an appealing online sampling strategy, we propose to use a
gradient-based method that only requires the information on the previous
estimator and the selected subsample points, and does not demand the
whole data storage. Compared with the divide-and-conquer based approach,
our method can handle the scenario where only a small portion of data in
each block can be measured, even when the selected data points are less
than the number of parameters. To be precise, the estimator of \(\beta\)
is updated via the following formula when the tth block arrives:

\[
\tilde { \pmb { \beta } } ^ { ( t ) } = \tilde { \pmb { \beta } } ^ { ( t - 1 ) } - \eta _ { t } \tilde { \pmb { g } } _ { t } ( \tilde { \pmb { \beta } } ^ { ( t - 1 ) } ) ,
\]

where \(\tilde { \pmb { \beta } } ^ { ( 0 ) }\) is defined as the
initial estimator obtained from the historical data, \(\eta _ { t }\) is
the pre-specified learning rate, and
\(\tilde { g } _ { t } ( \pmb { \beta } )\) is the subsample estimator
of the gradient, that is,

\[
g _ { t } ( \beta ) = - \frac { 1 } { n } \sum _ { i = 1 } ^ { n } \sum _ { j = 1 } ^ { J } \mathbb { I } ( y _ { t i } = j ) \frac { \partial \log \pi _ { t i , j } ( \beta ) } { \partial \beta } .
\]

The reasons we adopt (3) as the online updating equation are 2- fold.
First, only the gradient and latest estimator are needed to be held in
the memory. Thus, data storage is not necessary. Second, the
gradient-based method is more computationally tractable compared with
Newton-type methods, since it only requires the information on
\(\tilde { \pmb { \beta } } ^ { ( t - 1 ) }\) and the selected subsample
data points. It is worth mentioning that the proposed method can handle
the case that only an extremely small portion of the data in the tth
block can be measured, especially for the scenario that the selected
data points are less than the number of parameters.

To further relax memory constraints, it is natural to select a subsample
in the tth block through Poisson sampling (Särndal, Swensson, and
Wretman 2003) since it only needs scanning the data line by line via
independent Bernoulli trials. To ease the presentation, denote
\(\overset { - } { \pmb { u } } _ { t i } ( { \tilde { \pmb { \beta } } } ^ { ( t - 1 ) } )\)
be the negative score function of ith observation in tth block, that is,
− jJ=1 I(yti = j)∂ log πti,j(β)/∂β. The gradient gt(β˜ (t−1)) is
naturally estimated by the Horvitz-Thompson method (Horvitz and Thompson
1952), that is,

\[
\begin{array} { r l r } {  { \tilde { \boldsymbol g } _ { t } ( \tilde { \boldsymbol \beta } ^ { ( t - 1 ) } ) = \sum _ { i = 1 } ^ { n } \frac { R _ { t i } } { n p _ { t i } } \boldsymbol u _ { t i } ( \tilde { \boldsymbol \beta } ^ { ( t - 1 ) } ) } } \\ & { } & { = - \sum _ { i = 1 } ^ { n } \frac { R _ { t i } } { n p _ { t i } } \sum _ { j = 1 } ^ { J } { \mathbb I } ( y _ { t i } = j ) \frac { \partial \log \pi _ { t i , j } ( { \boldsymbol \beta } ) } { \partial { \boldsymbol \beta } } , } \end{array}
\]

where \(R _ { t i }\) is an indicator function with \(R _ { t i } = 1\)
if and only if the ith sample in the tth block is selected and
\(p _ { t i }\) is the inclusion probability. We denote the expected
number of subsampled items as \(r\) , which implies that the subsampling
probabilities are selected such that
\(p _ { t 1 } + \cdot \cdot \cdot + p _ { t n } = r .\) . While the
subsample sample size in each data block is random, as shown in Ai et
al.~(2021), the number of subsampled items, say \(\hat { r }\) , will
naturally concentrate around \(r\) with
\(\mathrm { p r } ( | \hat { r } - r | / r \le t ) \le 1 - 2 \mathrm { e x p } ( - 2 r ^ { 2 } t ^ { 2 } / n )\)
.

Thus, the computational cost is still under control. Without loss of
generality, we assume the targeted subsample sizes are the same in all
blocks. As suggested in Polyak and Juditsky (1992), the step sizes can
be chosen by
\(\eta _ { t } ~ = ~ \eta _ { 0 } t ^ { - \alpha } ( t ~ \geq ~ 1 )\)
with \(\eta _ { 0 }\) and \(\alpha\) being predetermined
hyper-parameters. For clarity, the proposed method is summarized in
Algorithm 1.

\begin{longtable}[]{@{}
  >{\raggedright\arraybackslash}p{(\linewidth - 0\tabcolsep) * \real{1.0000}}@{}}
\toprule\noalign{}
\endhead
\bottomrule\noalign{}
\endlastfoot
Algorithm1: Online Poisson subsampling for multino- mial logistic
regression. \\
\begin{minipage}[t]{\linewidth}\raggedright
\begin{enumerate}
\def\labelenumi{\arabic{enumi}.}
\setcounter{enumi}{-1}
\tightlist
\item
  Input An initial estimator β and sampling probabilities
  \{ti=1,.,t=..T. for t=1,\ldots,Tdo \textsubscript{(t)}(t-1) Setβ←β
  fori=1,\ldots,n do Generate a Bernoulli variable Rti \textasciitilde{}
  Bernoulli(pti). if Rti=1 then Measure the response yti under covariate
  Xti
\end{enumerate}
\end{minipage} \\
\end{longtable}

Output: The estimator β˜ (T).

To establish our asymptotic results, we need the following regularity
assumptions.

Assumption 1. The regression parameters lie in a compact parameter space
\(\Lambda\) , with the true parameter vector
\(\beta _ { \mathrm { t r u e } }\) being the inner points of this
parameter space. Further assume that
\(\operatorname { E } g _ { t } ( \beta )\) is well defined for all
\(\pmb { \beta } \in \mathbb { R } ^ { d }\) and
\(\begin{array} { r } { \operatorname* { l i m } _ { \| \beta \|  \infty } \operatorname { E L } ( \beta ) = \infty } \end{array}\)
with EL \(( \beta )\) and \({ \pmb g } _ { t } ( { \pmb \beta } )\)
defining in (2) and (5), respectively.

Assumption 2. Suppose that
\(\mathrm { p r } ( f ( \pmb { x } , \pmb { \beta } ) \neq f ( \pmb { x } , \pmb { \beta } _ { \mathrm { t r u e } } ) ) > 0\)
with probability one for all
\(\beta \neq \beta _ { \mathrm { t r u e } }\) and \(\pmb { x }\) comes
from \(F _ { x }\) , where \(F _ { x }\) is the distribution of
\(\pmb { x }\) .

Assumption 3. Assume that for all \(\pmb { \beta } \in \Lambda\) , the
risk function (2) is twice continuously differentiable with respect to
\(\beta\) . Moreover, the matrix

\[
\Psi = - \mathrm { E } \left( \sum _ { j = 1 } ^ { J } \mathbb { I } ( y = j ) \frac { \partial ^ { 2 } \log \pi _ { j } ( \pmb { \beta } _ { \mathrm { t r u e } } ) } { \partial \pmb { \beta } \partial \pmb { \beta } ^ { \operatorname { T } } } \right)
\]

is positive definite.

Assumption 4. \(\mathrm { F o r } j = 1 , \dotsc , J - 1\) , the
quantity
\(\| \partial f ( \pmb { x } , \pmb { \beta } _ { 0 } , \pmb { \beta } _ { j } ) /\)
\(\begin{array} { r l r } { \partial \pmb { \beta } \| } & { { } \le } & { h ( \pmb { x } ) } \end{array}\)
, such that
\(\begin{array} { r l r } { \mathrm { ~ E ~ } ( h ^ { 3 } ( { \pmb x } ) ) } & { { } < } & { \infty } \end{array}\)
. Furthermore, E
\(\mathbf { \Gamma } ( \| \partial ^ { 2 } f ( \pmb { x } , \pmb { \beta } _ { 0 } , \pmb { \beta } _ { j } ) / \partial \pmb { \beta } ^ { \mathrm { T } } \partial \pmb { \beta } \| _ { s } ) \ < \ \infty\)
, where \(\| \cdot \| _ { s }\) denotes the matrix spectrum norm.

Assumption 5. The following conditions hold with probability one:

\[
\begin{array} { l } { \displaystyle \operatorname* { s u p } _ { t } \mathrm { E } \left( \sum _ { i = 1 } ^ { n } \frac { r h ^ { 2 } ( x _ { t i } ) } { n ^ { 2 } \dot { p } _ { t i } } \bigg | \mathcal { F } _ { t - 1 } \right) < \infty , } \\ { \displaystyle \operatorname* { s u p } _ { t } \mathrm { E } \left( \frac { r ^ { 3 / 2 } } { n ^ { 3 } } \sum _ { i = 1 } ^ { n } \frac { h ^ { 3 } ( x _ { t i } ) } { { P } _ { t i } ^ { 2 } } \bigg | \mathcal { F } _ { t - 1 } \right) < \infty , } \end{array}
\]

where \(\mathcal { F } _ { t - 1 }\) be the \(\sigma\) -field generated
by
\(\{ ( \boldsymbol { x } _ { k i } ^ { \mathrm { T } } , \boldsymbol { y } _ { k i } , \boldsymbol { R } _ { k i } ) ^ { \mathrm { T } } : 1 \leq k \leq\)
\(t - 1 , 1 \leq i \leq n \}\) .

Assumptions 1 are the regularity conditions of multinomial logistic
regressions. This assumption is to ensure the boundedness of the
sequence
\(\tilde { \pmb { \beta } } ^ { ( t ) } ( t \ = \ 0 , 1 , \ldots )\) .
We also require the boundedness of
\(\beta _ { \mathrm { { t r u e } } }\) as in McCullagh and Nelder
(1989). Assumption 2 is the identification condition to ensure that the
\(\beta _ { \mathrm { t r u e } }\) is unique, which is also required in
Han et al.~(2020). Assumption 3 can be satisfied when the risk function
is convex. Assumptions 4 and 5 are the moment assumptions imposed on the
full sample and subsample. Specifically, the envelop function
\(h ( \pmb { x } )\) in Assumption 4 can be simply specified as
\(\| \pmb { x } \|\) for the softmax regression problem. Similar
assumptions are used in Wang, Zhu, and Ma (2018), and
\(\mathrm { Y u }\) et al.~(2022). The reason for using the conditional
expectation is that the subsampling probabilities
\(\{ { p } _ { t i } \} _ { i = 1 } ^ { n }\) may depend on
\(\mathcal { F } _ { t - 1 }\) . A typical example is that the
subsampling probabilities rely on the pilot estimator derived in the
previous step.

At the end of this section, we establish the consistency of the
estimator achieved by Algorithm 1.

Theorem 2.1. If Assumptions 1--5 hold, and
\(\eta _ { t } = \eta _ { 0 } t ^ { - \alpha }\) for some constants
\(\eta _ { 0 } > 0\) , \(\alpha \in ( 1 / 2 , 1 )\) , and
\(r / ( \sum _ { t = 1 } t ^ { - \alpha } ) ^ { 2 } \to 0\) , then β(T)
c onverges to \(\beta _ { \mathrm { { t r u e } } }\) almost surely as
\(T \to \infty\) .

Compared with the divide-and-conquer-based online updating methods, such
as Schifano et al.~(2016), in which the number of blocks is usually less
than the root of the total sample size, our method focuses on the case
that the number of blocks \(( T )\) goes to infinity and is probably
faster than \(n\) . Moreover, Theorem 2.1 implies \(r\) can be either
fixed or go to infinity, which differs from most existing subsampling
works. We believe this is more suitable for the online system which
needs responsive service for data streams. It is worth mentioning that
when \(T\) is finite or much slower than \(r ^ { 1 / 3 }\) , the
distributed subsampling method proposed in Yu et al.~(2022) can be
applied directly. Thus, we focus our attention on the scenario that
\(T \to \infty\) . As for the case that \(T\) goes to infinity with a
fixed and small \(n\) , the subsampling is not essential since we can
handle each block timely.

\section{3. Subsampling Strategies for Online
Updating}\label{subsampling-strategies-for-online-updating}

\section{3.1. Optimal Poisson Subsampling for One-Step
Updating}\label{optimal-poisson-subsampling-for-one-step-updating}

For the one-step updating in (4), simple calculation yields that
\(\mathrm { E } ( \Vert \tilde { \pmb { \beta } } ^ { ( t ) } - \pmb { \beta } _ { \mathrm { t r u e } } \Vert ^ { 2 } \ \vert \ \tilde { \pmb { \beta } } ^ { ( t - 1 ) } , \boldsymbol { X } _ { t } )\)
can be decomposed into the following three terms:
\(\| \mathrm { E } ( \tilde { \pmb { \beta } } ^ { ( t ) } \mathrm { ~  ~ \cdot ~ } | \tilde { \pmb { \beta } } ^ { ( t - 1 ) } , X _ { t } ) - \pmb { \beta } _ { \mathrm { t r u e } } \| ^ { 2 } ,\)
\(\mathrm { t r } [ \mathrm { v a r } \{ \mathrm { E } ( \tilde { \pmb { \beta } } ^ { ( t ) } | \tilde { \pmb { \beta } } ^ { ( t - 1 ) } , X _ { t } , Y _ { t } ) | \tilde { \pmb { \beta } } ^ { ( t - 1 ) } , X _ { t } \} ] ,\)
and
\(\operatorname { t r } [ \operatorname { E } \{ \operatorname { v a r } ( { \tilde { \boldsymbol { \beta } } } ^ { ( t ) } |\)
\(\tilde { \boldsymbol { \beta } } ^ { ( t - 1 ) } , \boldsymbol { X } _ { t } , \boldsymbol { Y } _ { t } ) \quad | \quad \tilde { \boldsymbol { \beta } } ^ { ( t - 1 ) } , \boldsymbol { X } _ { t } \} ]\)
. From the detailed calculation given in Section S.2.2, one can observe
that the first two terms are not related to the subsampling
probabilities and the third term corresponds to the sampling variance of
g˜t(β˜ (t−1)). Thus, it is desirable to design subsampling probabilities
to well approximate the gradient information. Due to measurement
constraints, the responses are unknown. Thus, we propose to minimize its
expectation instead, that is, we resort to minimizing the matrix
\(\begin{array} { r } { \mathrm { E } \{ \mathrm { v a r } ( \tilde { \pmb g } _ { t } ( \tilde { \pmb \beta } ^ { ( t - 1 ) } ) \mid \tilde { \pmb \beta } ^ { ( t - 1 ) } , \boldsymbol X _ { t } , Y _ { t } ) \mid \tilde { \pmb \beta } ^ { ( t - 1 ) } , \boldsymbol X _ { t } \} , } \end{array}\)
, where
\(X _ { t } ~ = ~ \{ \pmb { x } _ { t i } \} _ { i = 1 } ^ { n }\) ,
\({ Y _ { t } } ~ = ~ \{ y _ { t i } \} _ { i = 1 } ^ { n }\) . In
general, one cannot expect to optimize the matrix
\(\begin{array} { r l } { \mathrm { E } \{ \mathrm { v a r } ( \tilde { \pmb g } _ { t } ( \tilde { \pmb \beta } ^ { ( t - 1 ) } ) } & { { } | \quad \tilde { \pmb \beta } ^ { ( t - 1 ) } , X _ { t } , Y _ { t } ) } \end{array}\)
\textbar{} β˜ (t−1), X in the sense of the Loewner ordering (Horn and
Johnson 2012). Therefore, we opt to minimize the trace of
\(\operatorname { E } \{ \operatorname { v a r } ( \tilde { \pmb { g } } _ { t } ( \tilde { \pmb { \beta } } ^ { ( t - 1 ) } ) \ | \ \tilde { \pmb { \beta } } ^ { ( t - 1 ) } , X _ { t } , Y _ { t } ) \ | \ \tilde { \pmb { \beta } } ^ { ( t - 1 ) } , X _ { t } \}\)
, which is known as the \(A\) -optimality criterion in the design of
experiments.

The explicit form of
\(\mathrm { E } \{ \mathrm { v a r } ( \tilde { \pmb g } _ { t } ( \tilde { \pmb \beta } ^ { ( t - 1 ) } ) \quad | \quad \bar { \pmb \beta } ^ { ( t - 1 ) } , \boldsymbol { X } _ { t } , \boldsymbol { Y } _ { t } )\)
\(\tilde { \pmb { \beta } } ^ { ( t - 1 ) } , X _ { t } \}\) can be
derived as follows.

Lemma 3.1. For each data block \(( X _ { t } , Y _ { t } )\) , we have

{\$\$ \textbackslash begin\{array\} \{ r l \} \& \{
\textbackslash mathrm \{ E \} \textbackslash\{ \textbackslash mathrm \{
v a r \} ( \textbackslash tilde \{ g \} \_ \{ t \} (
\textbackslash tilde \{ \textbackslash beta \} \^{} \{ ( t - 1 ) \} )
\textbackslash mid \textbackslash tilde \{ \textbackslash beta \} \^{}
\{ ( t - 1 ) \} , X \_ \{ t \} , Y \_ \{ t \} ) \textbackslash mid
\textbackslash tilde \{ \textbackslash beta \} \^{} \{ ( t - 1 ) \} , X
\_ \{ t \} \textbackslash\} \} \textbackslash\textbackslash{} \& \{
\textbackslash quad = \textbackslash cfrac \{ 1 \} \{ n \^{} \{ 2 \} \}
\textbackslash cfrac \{ n \} \{ i = 1 \} \textbackslash cfrac \{ 1 -
\textbackslash gamma \_ \{ t i \} \} \{ \textbackslash gamma \_ \{ t i
\} \} \textbackslash displaystyle \textbackslash sum \_ \{ j = 1 \} \^{}
\{ J \} \textbackslash pi \_ \{ t i , j \} ( \textbackslash beta \_ \{
\textbackslash mathrm \{ t r u e \} \} ) \}
\textbackslash\textbackslash{} \& \{ \textbackslash quad
\textbackslash cfrac \{ \textbackslash partial \textbackslash log
\textbackslash pi \_ \{ t i , j \} ( \textbackslash tilde \{
\textbackslash beta \} \^{} \{ ( t - 1 ) \} ) \} \{
\textbackslash partial \textbackslash beta \} \textbackslash left(
\textbackslash frac \{ \textbackslash partial \textbackslash log
\textbackslash pi \_ \{ t i , j \} ( \textbackslash tilde \{
\textbackslash beta \} \^{} \{ ( t - 1 ) \} ) \} \{
\textbackslash partial \textbackslash beta \} \textbackslash right) \^{}
\{ \textbackslash top \} . \} \textbackslash end\{array\} \$\$}

From Lemma 3.1, the optimal subsampling probabilities that minimize the
expected variance of the gradient
\(\tilde { g } _ { t } ( \tilde { \boldsymbol { \beta } } ^ { ( t - 1 ) } )\)
, that is the trace of (6), are derived as follows. The corresponding
subsampling probabilities are so-called
\(\mathrm { ^ { \ast } M V g ^ { \prime } }\) .

Theorem 3.2. Let
\(\begin{array} { r } { \pmb { h } _ { t i } ^ { \mathrm { M V g } } = ( \sum _ { j = 1 } ^ { J } \pi _ { t i , j } ( \pmb { \beta } _ { t r u e } ) \Vert \partial \log \pi _ { t i , j } ( \tilde { \pmb { \beta } } ^ { ( t - 1 ) } ) / } \end{array}\)
\(\partial \pmb { \beta } \| ^ { 2 } ) ^ { 1 / 2 }\) . Then the
asymptotic mean square error of the gradient, that is,
\(\mathrm { t r } ( \mathrm { E } [ \mathrm { v a r } \{ \tilde { \pmb { g } } _ { t } ( \tilde { \pmb { \beta } } ^ { ( t - 1 ) } ) \ | \ \tilde { \pmb { \beta } } ^ { ( t - 1 ) } , X _ { t } , Y _ { t } \} \ | \ \tilde { \ { \pmb { \beta } } } ^ { ( t - 1 ) } , X _ { t } ] )\)
attains its minimum if \(\mathbf { \nabla } P t i\) 's are chosen to be

\[
p _ { t i } ^ { \mathrm { M V g } } = r \frac { \hbar _ { t i } ^ { \mathrm { M V g } } \wedge M _ { t } ^ { \mathrm { M V g } } } { \sum _ { j = 1 } ^ { n } \hbar _ { t j } ^ { \mathrm { M V g } } \wedge M _ { t } ^ { \mathrm { M V g } } } , \quad i = 1 , \dots , n ,
\]

where \(M _ { t } ^ { \mathrm { M V g } }\) is the maximum number such
that \$r ( \_ \{ t i \} \^{} \{ \} \$
\(\begin{array} { r } { M _ { t } ^ { \mathrm { M V g } } ) \leq \sum _ { j = 1 } ^ { n } \hbar _ { t j } ^ { \mathrm { M V g } } \wedge M _ { t } ^ { \mathrm { M V g } } } \end{array}\)
and \(a \wedge b = \operatorname* { m i n } ( a , b )\) .

Remark 1. In Theorem 3.2, the thresholds
\(M _ { t } ^ { \mathrm { M V g } } s ^ { \prime }\) are used to ensure
the inclusion probabilities lie between zero and one. When \(r / n  0\)
, which is common in big data subsampling settings, one may expect that
\(r \hbar _ { t i } ^ { \mathrm { M V g } } / ( \sum _ { j = 1 } ^ { n } \hbar _ { t j } ^ { \mathrm { M V g } } ) ~ < ~ 1\)
. As a result, the optimal subsampling probabilities in each data block
reduce to pti
\(\begin{array} { r } { \boldsymbol { p } _ { t i } ^ { \mathrm { M V g } } ~ = ~ { r } \hbar _ { t i } ^ { \mathrm { M V g } } / ( \sum _ { j = 1 } ^ { n } \hbar _ { t j } ^ { \mathrm { M V g } } ) } \end{array}\)
. As for the cases that \(r / n > 0\) , it may exist some \(i\) such
that
\(\begin{array} { r } { r \hbar _ { t i } ^ { \mathrm { M V g } } / ( \sum _ { j = 1 } ^ { n } \hbar _ { t j } ^ { \mathrm { M V g } } ) > 1 } \end{array}\)
for the tth block. The explicit form solution is presented given in the
supplementary material. As pointed out in Yu et al.~(2022), finding the
exact solution of \(M _ { t } ^ { \mathrm { { \hat { M } V g } } }\)
needs \(O ( n + r \log r )\) times. To reduce the computational cost,
the \(M _ { t } ^ { \mathrm { M V g } }\) can be estimated by some
quantile of \(\hbar _ { t i } ^ { \mathrm { \tilde { M V g } } }\) under
this scenario.

\section{3.2. Optimal Poisson Subsampling for Online
Updating}\label{optimal-poisson-subsampling-for-online-updating}

The subsampling method proposed in Section 3.1 leads to an estimator
that achieves the smallest mean squared error of β˜ (t) in the tth step.
One of the naive ideas is to adopt the proposed subsampling
probabilities in each step recursively which is known as the greedy
method in numerical analysis. It is well known that the greedy method
will not always lead to an optimal solution for the final estimator
(Cormen et al.~2009). Thus, some calibration is needed on the
subsampling probabilities to achieve the minimal asymptotic mean squared
error (AMSE) of \(\beta\) as \(T \to \infty\) .

In our online learning setting, we not only allow \(p _ { i t }\) to
depend on \(\{ \pmb { x } _ { t i } \} _ { i = 1 } ^ { n }\) at the tth
batch but also on \(\mathcal { F } _ { t - 1 }\) which is defined in
Assumption 5. To calibrate the greedy subsampling probabilities obtained
in Theorem 3.2, we restrict our attention to the following scenario in
the rest of this article.

Assumption 6. The subsampling probabilities are designed such that
\(\mathrm { E } ( \Gamma _ { t } ( \pmb { \beta } _ { \mathrm { t r u e } } ) | \mathcal { F } _ { t - 1 } )\)
is a stable quantity against \(t ,\) , where
\(\begin{array} { r } { \Gamma _ { t } ( \pmb { \beta } ) = r n ^ { - 2 } \sum _ { i = 1 } ^ { n } { p _ { t i } } ^ { - 1 } \pmb { u } _ { t i } ( \pmb { \beta } ) \pmb { u } _ { t i } ^ { \mathrm { T } } ( \pmb { \beta } ) } \end{array}\)
. That is to say, as \$T \$ \(\infty\) ,
\(\mathrm { E } ( \Gamma _ { t } ( \pmb { \beta } _ { \mathrm { t r u e } } ) | \mathcal { F } _ { t - 1 } )  \Gamma\)
almost surely, where \(\Gamma\) is a positive definite matrix.

This assumption is to ensure that the sampling effect is stable against
\(t\) and can be satisfied when the proposed subsampling probability in
Theorems 3.2 or 3.4 (will be shown later) is adopted. Clearly, the
uniform subsampling also satisfied this assumption under Assumption 2.
The asymptotic normality of (T) as \(T \to \infty\) is presented as
follows.

Theorem 3.3. If Assumptions 1--6 hold, and
\(\eta _ { t } = \eta _ { 0 } t ^ { - \alpha }\) for some constants
\(\eta _ { 0 } > 0 , \alpha \in ( 1 / 2 , 1 )\) , then the following
results hold.

\begin{enumerate}
\def\labelenumi{\roman{enumi}.}
\tightlist
\item
  As \(T , n , r  \infty\) with
  \(r / ( \sum _ { t = 1 } t ^ { - \alpha } ) ^ { 2 } \to \mathsf { \bar { 0 } }\)
  , and further assume that \(r\) is functional independent of \(t , T\)
  , it follows that
  \(( r ^ { 1 / \alpha } T ) ^ { \alpha / 2 } ( \tilde { \beta } ^ { ( T ) } - \beta _ { \mathrm { t r u e } } )  N ( \stackrel { \sim } { 0 } , \eta _ { 0 } V ) .\)
  , where \(V\) is the unique solution of the following equation:
\end{enumerate}

\[
\Psi V + V \Psi = \Gamma ,
\]

with \(\Psi\) defined in Assumption 3.

\begin{enumerate}
\def\labelenumi{\roman{enumi}.}
\setcounter{enumi}{1}
\tightlist
\item
  As \(T , n \ \to \ \infty\) with fixed r,
  \(T ^ { \alpha / 2 } ( \tilde { \pmb \beta } ^ { ( T ) } - { \pmb \beta } _ { \mathrm { t r u e } } ) \ \)
  \(N ( 0 , \eta _ { 0 } V _ { \mathrm { f i x } } )\) , where
  \(V _ { \mathrm { f i x } }\) is the unique solution of the equation
  \(\Psi V _ { \mathrm { f i x } } + V _ { \mathrm { f i x } } \Psi = r ^ { - 1 } \Gamma\)
  .
\end{enumerate}

Theorem 3.3 characterizes the asymptotic behavior when both the number
of data blocks \(T\) and the size of each block \(n\) goes to infinity.
These results enable us to calibrate the subsampling probabilities in
Theorem 3.2 such that \(\tilde { \boldsymbol { \beta } } ^ { ( T ) }\)
achieves the minimum asymptotic variance (and the minimum AMSE due to
the asymptotic unbiasedness). Thus, we denote the sampling probabilities
by \(^ { \ast } \mathrm { { M V } ^ { \ast } }\) .

Theorem 3.4. Let
\(\begin{array} { r l r } { \hbar _ { t i } ^ { \mathrm { M V } } } & { { } } & { = } \end{array} \quad \begin{array} { r l r } { \{ \sum _ { j = 1 } ^ { J } \pi _ { t i , j } ( \beta _ { t r u e } ) \| \Psi ^ { - 1 / 2 } } \end{array}\)
\(( \partial \log \pi _ { t i , j } ( \pmb { \beta } _ { t r u e } ) / \partial \pmb { \beta } ) \| ^ { 2 } \} ^ { 1 / 2 }\)
. Then the AMSE of \(\tilde { \pmb { \beta } } ^ { ( T ) }\) attains its
minimum if \(\mathop { p _ { t i } { ' } s }\) are chosen to be

\[
p _ { t i } ^ { \mathrm { { M V } } } = r \frac { \hbar _ { t i } ^ { \mathrm { { M V } } } \wedge M _ { t } ^ { \mathrm { { M V } } } } { \sum _ { j = 1 } ^ { n } \hbar _ { t j } ^ { \mathrm { { M V } } } \wedge M _ { t } ^ { \mathrm { { M V } } } } , \quad t = 1 , \dots , T , i = 1 , \dots , n ,
\]

where
\(\begin{array} { r } { \sum _ { j = 1 } ^ { n } \hbar _ { t j } ^ { \mathrm { M V } } \wedge M _ { t } ^ { \mathrm { M V } } } \end{array}\)
\(M _ { t } ^ { \mathrm { M V } }\) is the maximum number such that .
\(r ( \hbar _ { t i } ^ { \mathrm { M V } } \wedge M _ { t } ^ { \mathrm { M V } } ) \leq\)

Clearly, the proposed optimal subsampling probabilities depend only on
the covariates, but not on the responses. Thus, it is applicable to
scenarios with measurement constraints. Moreover, one can clearly see
that the optimal subsampling probabilities for minimizing the AMSE of
\(\tilde { \boldsymbol { \beta } } ^ { ( T ) }\) are different from the
optimal subsampling probabilities for minimizing the expected variance
of the gradient in each step. In contrast to the subsampling
probabilities which focus on well estimating the gradient information in
each step, the Fisher information matrix also plays an important role in
the subsampling probabilities to achieve the AMSE of
\(\tilde { \boldsymbol { \beta } } ^ { ( T ) }\)

To take a close look at the proposed subsampling probabilities, we take
the logistic regression as an example. Clearly, it is a special case of
Model (1) with \(J = 2\) and
\(f ( \pmb { x } , \pmb { \beta } _ { 0 } , \pmb { \beta } _ { 1 } ) =\)
\(\pmb { x } ^ { \mathrm { T } } \pmb { \beta } _ { 1 }\) . As discussed
in Remark 1, we focus on the cases that
\(\begin{array} { r l r } { r \hbar _ { t i } ^ { \mathrm { M V } } / ( \sum _ { j = 1 } ^ { n } \hbar _ { t j } ^ { \mathrm { M V } } ) } & { { } < } & { 1 } \end{array}\)
for all \(i\) . Simple calculation yields that
\(\begin{array} { r l } { \dot { P } _ { t i } ^ { \mathrm { M V } } \propto } & { { } \{ \pi _ { t i , 1 } ( \beta _ { \mathrm { t r u e } } ) ( 1 - \pi _ { t i , 1 } ( \beta _ { \mathrm { t r u e } } ) ) { \boldsymbol x } _ { i } ^ { \mathrm { T } } \Psi ^ { - 1 } { \boldsymbol x } _ { i } \} ^ { 1 / 2 } } \end{array}\)
, which corresponds to the square root leverage score of logistic models
(Lee 1987). Similar to the results of leverage score sampling for the
linear model discussed in Ma et al.~(2022), the data point that has a
large impact on the prediction naturally contains more information.
Thus, it is reasonable to select the large leverage score points to
improve the estimation performance. For the one step updating considered
in Section 3.1, the optimal subsampling probabilities turn into ptiMV
\(\begin{array} { r l r } { \dot { P } _ { t i } ^ { M V } } & { { } \propto } & { \{ \hat { \pi } _ { t i , 1 } ( \pmb { \beta } _ { \mathrm { t r u e } } ) ( 1 - \pi _ { t i , 1 } ( \pmb { \beta } _ { \mathrm { t r u e } } ) ) \} ^ { 1 / 2 } \| \pmb { x } _ { i } \| _ { 2 } } \end{array}\)
which can be regarded as \(L\) -optimal sampling probabilities under
measurement constraints with
\(\begin{array} { c c c c c } { T } & { = } & { 1 , r , n } & { \to } & { \infty . } \end{array}\)
. It is also interesting to see that when the learning rate in (3) is
selected as the second-order derivative of the loss function
\(L _ { n T } ( \pmb { \beta } )\) at \(\pmb { \beta } ^ { ( t - 1 ) }\)
, the optimal subsampling probabilities reduce to ptMiV ∝
\(\{ \pi _ { t i , 1 } ( \widehat { \pmb { \beta } } _ { \mathrm { t r u e } } ) ( 1 - \pi _ { t i , 1 } ( \widehat { \pmb { \beta } } _ { \mathrm { t r u e } } ) ) \} ^ { \bar { 1 } / 2 } \| \Psi ^ { - 1 } \pmb { x } _ { i } \| _ { 2 }\)
, which is the \(A\) -optimal sampling probabilities under measurement
constraints with \(T = 1 , r , n  \infty\) (Zhang, Ning, and Ruppert
2021b). Compared with the optimal subsampling strategies for one-step
updating described in Theorem 3.2, one can find that the greedy methods
will not lead to the optimal solution as \(T \to \infty\) . Instead, the
squared root leverage score subsampling is much more suitable for online
learning.

\section{3.3. Practical Considerations}\label{practical-considerations}

Note that the optimal subsampling probabilities presented in both
Theorems 3.2 and 3.4 depend on the unknown true parameter vector
\(\beta _ { \mathrm { t r u e } }\) . Thus, the optimal subsampling
strategies cannot be applied directly. A practical way is to replace the
unknown true parameter in the tth iteration with the estimator obtained
in the (t − 1)th iteration, that is, β˜ (t−1), s ince it is a consistent
estimator of \(\beta _ { \mathrm { { t r u e } } }\) as \(t\) goes to
infinity. To better approximate the subsampling probabilities, we use a
consistent estimator estimated by the historical data or the data in the
first block as the initial values in Algorithm 1. In the streaming
setting, if the historical data is limited, we can initialize \(\Psi\)
based on the historical data at first, and periodically update it using
new stream subsamples to ensure that \(\Psi\) is consistent. As
suggested in Xie et al.~(2019), an infrequent update of \(\Psi\) with a
small probability is sufficient.

In the big data era, it is common to see that the subsample budget \(r\)
in each data block is relatively small compared with

Algorithm 2: Optimal online Poisson subsampling for multinomial logistic
regression.\\
Input Historical labeled data
\(\mathcal { F } _ { 0 } = \{ ( \boldsymbol { x } _ { 0 i } ^ { \mathrm { T } } , \boldsymbol { y } _ { 0 i } ) ^ { \mathrm { T } } \} _ { i = 1 } ^ { n }\)
and unlabeled data stream
\(\{ \pmb { x } _ { t i } \} _ { i = 1 } ^ { n }\) with
\(t = 1 , \dots , T\) . The (shrinkage)\\
parameter \(\rho\) .\\
Initial estimator Find the maximum likelihood estimator
\(\tilde { \boldsymbol { \beta } } ^ { ( 0 ) }\) for multinomial
logistic regression based on \(\mathcal { F } _ { 0 }\) . Calculate\\
\(\begin{array} { r } { \hat { \Psi } = - \frac { 1 } { n } \sum _ { i = 1 } ^ { n } \bigg ( \sum _ { j = 1 } ^ { J } \mathbb { I } ( y _ { 0 i } = j ) \frac { \partial ^ { 2 } \log \pi _ { 0 i , j } ( \tilde { \pmb { \beta } } ^ { ( 0 ) } ) } { \partial \beta \partial \beta ^ { \mathrm { T } } } \bigg ) . } \end{array}\)\\
for \(t = 1 , \dots , T\) do Set β˜ (t) ← β˜ (t−1). Calculate
\(\begin{array} { r } { \boldsymbol { \lambda } ^ { \mathrm { M V } }  \sum _ { i = 1 } ^ { n } \Big \{ \sum _ { j = 1 } ^ { J } \pi _ { t i , j } ( \tilde { \pmb { \beta } } ^ { ( t - 1 ) } ) \| \hat { \Psi } ^ { - 1 / 2 } ( \partial \log \pi _ { t i , j } ( \tilde { \pmb { \beta } } ^ { ( t - 1 ) } ) / \partial \pmb { \beta } ) \| ^ { 2 } \Big \} ^ { 1 / 2 } . } \end{array}\)
for \(i = 1 , \ldots , n\) do Calculate
\(\begin{array} { r } { \cdot \tilde { p } _ { t i } ^ { \mathrm { M V } } \gets \left[ ( 1 - \rho ) \frac { r \left\{ \sum _ { j = 1 } ^ { J } \pi _ { t i , j } ( \tilde { \beta } ^ { ( t - 1 ) } ) \| \hat { \Psi } ^ { - 1 / 2 } ( \partial \log \pi _ { t i , j } ( \tilde { \beta } ^ { ( t - 1 ) } ) / \partial \beta ) \| ^ { 2 } \right\} ^ { 1 / 2 } } { \lambda ^ { \mathrm { M V } } } + \rho \frac { r } { n } \right] \wedge 1 . } \end{array}\)
Generate a Bernoulli variable \(R _ { t i } \sim\) Bernoulli
\(\dot { ( p _ { t i } ^ { \mathrm { M V } } ) }\) . if
\(R _ { t i } = 1\) then Measure the response \(y _ { t i }\) under
covariate \(\pmb { x } _ { t i }\) and update
\(\tilde { \pmb \beta } ^ { ( t ) } \gets \tilde { \pmb \beta } ^ { ( t - 1 ) } - \frac { \eta _ { t } } { n \dot { p } _ { t i } ^ { \mathrm { M V } } } \pmb u _ { t i } ( \tilde { \pmb \beta } ^ { ( t - 1 ) } ) .\)

tsheledonummtboermeofe otbhse revaetinotntsh \(n\) oArs
\(\hbar _ { t i } ^ { \mathrm { M V g } } > M _ { t } ^ { \mathrm { M V g } }\)
\(\hbar _ { t i } ^ { \mathrm { M V } } ~ > ~ M _ { t } ^ { \mathrm { M V } }\)
happens. In practice, when r/n is relatively small, the MtMVg or
\(M _ { t } ^ { \mathrm { M V } }\) can be taken as infinity with no
significant effect on the optimal subsampling probabilities in Theorems
3.2 and 3.4.

Let \({ \lambda } ^ { \mathrm { M V g } }\) or
\({ \boldsymbol { \lambda } } ^ { \mathrm { M V } }\) denote the
denominator for optimal onestep subsampling and optimal online
subsampling, respectively. To make the estimators more stable and
robust, Ma, Mahoney, and Yu (2015) first proposed a shrinkage-based
subsampling method, which is a linear combination of the optimal
probabilities and the uniform subsampling probabilities. Thereby it is
natural to obtain the benefits of both two methods.

The computational cost is \(O ( n T )\) of Algorithm 2. The storage
demand is \(O ( 1 )\) since the analysis only need to keep
\(\tilde { \pmb { \beta } } ^ { ( t - \top ) }\) and
\({ \pmb u } _ { t i } ( { \tilde { \pmb \beta } } ^ { ( t - 1 ) } )\) .
When the pilot estimator or the initial guess is consistent, the
denominators in the subsampling probabilities are very similar across
different data blocks. Thus, we can use the first block to estimate the
denominator. In this scenario,
\(\dot { P } _ { t i } ^ { \mathrm { M V } }\) only depends on the ith
data points itself together with the estimated denominator. Thus, the
cost of the RAM is also \(O ( 1 )\) since the Poisson subsampling can be
implemented by loading the data line-by-line. Note that the responses of
the data stream are expensive. It should be emphasis that the main cost
in our problem is the data annotation.

The results of consistency and asymptotic normality for the practical
implementation are presented in the following theorems.

Assumption 7. Suppose that

\[
\mathrm { E } \left( \frac { r } { n ^ { 2 } } \sum _ { i = 1 } ^ { n } \frac { u _ { \mathrm { 1 } i } ( \beta _ { \mathrm { t r u e } } ) u _ { \mathrm { 1 } i } ^ { \mathrm { { T } } } ( \beta _ { \mathrm { t r u e } } ) } { \left\{ ( 1 - \rho ) \frac { r h _ { \mathrm { 1 } i } ^ { \mathrm { { M V } } } } { \sum _ { i = 1 } ^ { n } \hbar _ { \mathrm { 1 } i } ^ { \mathrm { { M V } } } } + \rho \frac { r } { n } \right\} \wedge 1 } \right) = \Gamma ^ { \mathrm { M V } }
\]

holds almost surely with \(\Gamma ^ { \mathrm { M V } }\) being a
positive definite matrix, where
\(\hbar _ { 1 i } ^ { \mathrm { M V } }\) is defined in Theorem 3.4.

Theorem 3.5. If Assumptions 1--4 and 7 hold, and
\(\eta _ { t } = \eta _ { 0 } t ^ { - \alpha }\) for some constants
\(\eta _ { 0 } ~ > ~ 0 , \alpha ~ \in ~ ( 1 / 2 , 1 )\) , then
\(\tilde { \boldsymbol { \beta } } ^ { ( t ) }\) achieved by Algorithm 1
with \(\{ \grave { p } _ { t i } ^ { \mathrm { M V } } \}\) converges to
\(\beta _ { \mathrm { { t r u e } } }\) almost surely as
\(t \to \infty\) . Moreover, the following results hold.

\begin{enumerate}
\def\labelenumi{\roman{enumi}.}
\item
  As \(T , n , r \ \to \ \infty\) with
  \(\bar { r } / ( \sum _ { t = 1 } t ^ { - \alpha } ) ^ { 2 } \ \to \ 0\)
  and \(r\) is functionally independent of \$t , T , ( r \^{} \{ 1 / \}
  T ) \^{} \{ / 2 \} ( \^{} \{ ( T ) \} - \_ \{ \} ) \$
  \(N ( 0 , \eta _ { 0 } V _ { \mathrm { p r a c } } )\) , where \(V\)
  is the unique solution of the equation
  \(\Psi V _ { \mathrm { p r a c } } + V _ { \mathrm { p r a c } } \Psi = \Gamma ^ { \mathrm { M V } } .\)
  .
\item
  As \(T , n  \infty\) with fixed
  \(r , \ T ^ { \alpha / 2 } ( \tilde { \pmb { \beta } } ^ { ( T ) } \ - \ \pmb { \beta } _ { \mathrm { t r u e } } ) \ \)
  \(N ( 0 , \eta _ { 0 } V _ { \mathrm { p r a c , f } } )\) , where
  \(V\) is the unique solution of the equation
  \(\Psi V _ { \mathrm { p r a c , f } } + V _ { \mathrm { p r a c , f } } \Psi = r ^ { - 1 } \Gamma ^ { \mathrm { M V } }\)
  .
\end{enumerate}

The Assumption 7 is to ensure that when the
\(\beta _ { \mathrm { t r u e } }\) is known, the proposed subsampling
method will not lead to a trivial solution, that is, the
\(V _ { \mathrm { p r a c } }\) goes to zero. Furthermore, if (8) also
holds when \(\hbar _ { 1 i } ^ { \mathrm { M V } }\) is replaced by
\(\hbar _ { 1 i } ^ { \mathrm { M V g } }\) and denote the expectation
as \(\Gamma ^ { \mathrm { M V g } }\) with
\(\Gamma ^ { \mathrm { M V g } }\) being a positive definite matrix. The
results are also hold when
\(\{ \grave { p } _ { t i } ^ { \mathrm { M V g } } \}\) are adopted by
replacing \(\Gamma ^ { \mathrm { M V } }\) with
\(\Gamma ^ { \mathrm { M V g } }\) .

\section{4. Numerical Studies}\label{numerical-studies}

This section is to evaluate the performances of the methods proposed in
Section 3 via some simulated and real datasets. To account for the
randomness in the subsampling phases, the implementation is repeated
1000 times for each case in this section. The empirical mean squared
error (MSE) of the resultant estimator is used to evaluate the accuracy
of the algorithms in estimating the model parameter. To be precise, for
the tth iteration \(( t = 1 , . . . , T )\) , we calculate
\(\begin{array} { r l } { } & { { } 5 0 0 ^ { - 1 } \sum _ { k = 1 } ^ { 5 0 0 } \| \tilde { \boldsymbol { \beta } } _ { p , k } ^ { ( t ) } - \boldsymbol { \beta } _ { \mathrm { t r u e } } \| ^ { 2 } ; } \end{array}\)
swcherme \(\tilde { \pmb { \beta } } _ { p , k } ^ { ( t ) }\)
fhroprmotbhaeb
\(\pmb { \dot { p } } = \{ p _ { t i } : t = 1 , . . . , T ; i = 1 , . . . , n \}\)
\(k { \mathrm { t h } }\) subsample. All the computations are performed
via \(\mathtt { R }\) (R Core Team 2021).

\pandocbounded{\includegraphics[keepaspectratio]{images/4122950f4ef661eb51a7bd31d1f9f86db1c4953ca1e270ca84742355269b3f6c.jpg}}\\
Figure 1. The log of MSEs for Model (9) with the number of updates based
on the MV (triangle), MVg (square), and uniform subsampling (circle)
method

\section{4.1. Simulations}\label{simulations}

In the following, we take Model (1) with \(J = 3\) as an example to
illustrate our methods. More precisely, we consider the following model:

\[
\begin{array} { r l } & { \log \left( \frac { \pi _ { i 1 } ( \pmb { \beta } ) } { \pi _ { i 3 } ( \pmb { \beta } ) } \right) = \pmb { x } _ { t i ( 0 ) } ^ { \mathrm { { T } } } \pmb { \beta } _ { 0 } + \pmb { x } _ { t i ( 1 ) } ^ { \mathrm { { T } } } \pmb { \beta } _ { 1 } , } \\ & { \log \left( \frac { \pi _ { i 2 } ( \pmb { \beta } ) } { \pi _ { i 3 } ( \pmb { \beta } ) } \right) = \pmb { x } _ { t i ( 0 ) } ^ { \mathrm { { T } } } \pmb { \beta } _ { 0 } + \pmb { x } _ { t i ( 2 ) } ^ { \mathrm { { T } } } \pmb { \beta } _ { 2 } . } \end{array}
\]

Here we set the true parameter vector
\(\pmb { \beta } _ { 0 } = \pmb { \beta } _ { 1 } = \pmb { \beta } _ { 2 } =\)
\(( 1 , 1 , 1 , 1 ) ^ { \mathrm { T } }\) . The number of data blocks is
set to \(T = 1 0 0 0\) with \(n = 1 0 0 0\) observations in each block.
The corresponding covariates
\(\pmb { x } _ { t i } = ( \pmb { x } _ { t i ( 0 ) } ^ { \mathrm { T } } , \pmb { x } _ { t i ( 1 ) } ^ { \mathrm { T } } , \pmb { x } _ { t i ( 2 ) } ^ { \mathrm { T } } ) ^ { \mathrm { T } }\)
, are independent and identically distributed and generated as follows.

Case 1 The covariate \(x\) follows a multivariate t distribution with
degree four, that is, \(t _ { 4 } ( \mathbf { 0 } , \Sigma _ { 1 } )\) ,
where \(\Sigma _ { 1 }\) is a matrix with all diagonal elements equal to
one and off-diagonal elements equal to 0.5.\\
Case 2 The covariate \(x\) follows a multivariate normal distribution
\(N ( \mathbf { 0 } , 2 \Sigma _ { 1 } )\) , where
\({ \boldsymbol { \Sigma } } _ { 1 }\) is the same as in Case 1.\\
Case 3 The covariate \(x\) is the same as Case 1 expect the correlation
structure is changed to \(\Sigma _ { 2 }\) , that is,
\(t _ { 4 } ( \mathbf { 0 } , \Sigma _ { 2 } )\) . Here
\(\Sigma _ { 2 }\) is a matrix with the \(( i , j )\) th entry being
\(0 . 5 ^ { | i - j | }\) .\\
Case 4 The covariate \(x\) follows a mixture distribution
\(0 . 5 N ( \mathbf { 0 } . 5 , 2 \Sigma _ { 1 } ) + 0 . 5 t _ { 4 } ( \mathbf { 0 } , \Sigma _ { 1 } )\)
.

We first test our method in a multivariate t distribution case since it
is common to see the heavy-tailed distribution in the big data era. Here
we select four as the degree of multivariate t distribution since we
need to ensure that the third order moment exists as we imposed in
Assumption 4. Case 2 considers a lighter-tailed distribution compared
with Case 1. The constant in front of the correlation matrix is to
ensure the variances of the four cases are the same. Case 3 considers a
different correlation structure. Case 4 further considers the case that
covariates come from mixture distribution and are asymmetric.

Now we evaluate the performances of our methods based on different
choices of subsampling probabilities. The uniform subsampling method is
also considered for comparison. We regard the first block as historical
data and set the hyperparameters
\(\rho ~ = ~ 0 . 2 , \eta _ { 0 } ~ = ~ 1\) , and
\(\alpha \ = \ 0 . 9 9\) throughout this section. To verify the
convergence of the subsample-based estimator, the empirical log of MSEs
with different numbers of updates are presented in Figure 1, where the
number of updates \(t\) ranges from 1 to 1000 and the target subsample
size is fixed at \(r = 6\) .

From Figure 1, one can see that the MSEs for all subsampling methods
decrease as \(t\) increases, which confirms the theoretical result in
Theorem 2.1. In addition, subsampling methods based on the MV and
\(\operatorname { M V g }\) always result in smaller empirical MSEs
compared with uniform subsampling. Clearly, when the number of updates
is small, the \(\operatorname { M V g }\) method may perform better than
the MV method. However, the MV method results in smaller MSEs than those
in the \(\operatorname { M V g }\) method as \(t\) grows up. This is
because the \(\operatorname { M V g }\) method is a greedy approach and
may not lead to the smallest MSE of the final estimator.

To evaluate the influence of the target subsample size \(r _ { : }\) ,
we implement Algorithm 1 by fixing \(T\) at 100, changing \(r\) among 6,
18, 30, 60, and 120. Case 1 is selected as an example and the results
are reported in Figures 2(a). The results for all the data in each block
are also reported by the dashed line. We also illustrate the case that
the total target subsample size, that is, \(r T _ { : }\) , is fixed at
6000, which can explore the effect of the allocations between \(r\) and
\(T\) . The results are reported in Figure 2(b).

Clearly, the MSEs for all subsampling methods decrease as \(r\)
increases since more data are used. One can find that there is a linear
relationship between the log(MSE) and \(\log ( r )\) with the slope
around \(- 0 . 5\) , which implies as \(r\) increasing the convergence
rate is around \(r ^ { 1 / 2 }\) . This echoes the convergence results
in the first part of Theorem 3.3. As \(r\) increased, the difference
between our method and uniform sampling became smaller, since the
uniform subsampling has more chance to select the informative points.
One can see that when \(r = 1 2 0\) , the MSE is very close to the MSE
that all the data in the block is used, although only \(12 \%\) data in
each block is used. This reveals that our methods are particularly
useful for the scenario where the responses are expensive to be measured
when the data is updated rapidly. In Figure 2(b), the \(T\) ranges among
10, 50, 100, 250, 500, and 1000. One can see that the
\(\operatorname { M V g }\) and MV have similar performance when the
number of blocks is small. This may be because the gradient-based method
may not lead to a convergence result when the iteration is small. The
phenomenon also reflects that the \(r\) can not be too big compared with
\(T\) , which echos the assumption that
\(r / ( \sum _ { t = 1 } \bar { t ^ { - \alpha } } ) ^ { 2 } \stackrel { - } {  } 0\)
in Theorem 2.1.

\pandocbounded{\includegraphics[keepaspectratio]{images/8231c0810cb807f5d47e93c4d4029d8a91c93398ed3725f81de3003b188866a2.jpg}}\\
Figure 2. The log of MSEs for Model (9) with different combinations of r
and T based on the MV (triangle), \(\mathsf { M V } \mathsf { g }\)
(square), and uniform subsampling (circle) methods, where \(n _ { 0 }\)
in all cases are fixed at 1000 for a fair comparison. The dashed line in
(a) is the log MSE when all data in each block have been used. The data
is generated as mentioned in Case 1.

\pandocbounded{\includegraphics[keepaspectratio]{images/becda7cfca06ce269e9f377df85db9b3ce7e557e3c11b20396bdb4f04eb60682.jpg}}\\
Figure 3. The relative efficiencies against the MV methods (fixed
\(r = 6 ,\) ) for Model (9) with the number of updates. (a) The relative
efficiencies of \(\mathsf { M V } \mathsf { g }\) (square), and uniform
subsampling (circle) with r fixed at 6. (b) The relative efficiencies
for different subsample sizes are based on the uniform subsampling
method. The log dash line and dot-dash line stand for the cases
\(r = 8\) and 10, respectively. The data is generated as mentioned in
Case 1.

Note that for the measurement constraints problem, the main cost is the
data labeling. In the language of experimental design, relative design
A-efficiency is a measure that compares the cost of any two design
strategies. For example, the efficiency of design \(\xi _ { 1 }\) is
\(\kappa\) compared with design \(\xi _ { 2 }\) , the design
\(\xi _ { 1 }\) would need \(1 0 0 ( \kappa - 1 ) \%\) more subjects
than the design \(\xi _ { 2 }\) to achieve the same accuracy of
estimators in the sense of asymptotic mean squared error. Thus, we also
report the relative design A-efficiency calculated by the ratio of MSE
for the two sampling strategies, namely uniform and
\(\operatorname { M V g }\) methods, over MSE for the MV method. The
results for Case 1 are reported in Figure 3(a). For reference, we also
drew a horizontal line at one. To numerically verify the effect of \(r\)
, we also compared MV methods with the uniform subsampling method with
\(r \ = \ 8\) and 10. The corresponding results are reported in Figure
3(b).

From Figure 3, we can see that the \(\operatorname { M V g }\) method is
sometimes superior to the MV method in the first few rounds. This echos
the results in Theorems 3.2 and 3.4. As time passed, the MV method is
better than \(\operatorname { M V g }\) , which implies the MV method is
more suitable for continuous online updating under measurement
constraints. Simple calculation yields that the relative efficiencies
are around \(1 1 5 \%\) and \(1 0 5 \%\) for the uniform sampling and
\(\operatorname { M V g }\) methods, respectively. Moreover, from Figure
3(b), one may expect to label additional \(6 0 \%\) samples in each
iteration for the uniform subsampling method to achieve a similar
performance.

To further evaluate the tradeoff between statistical efficiency and
computational cost, we report the computing time plots for the four
cases. All the computations are carried out on an iMac with a 3.6GHz
Intel Core i9 processor. To mimic the cost of data annotating, we use
Sys.sleep() function in \(\mathtt { R }\) and assume the time spent on
labeling each instance is only 0.01 sec.~To mimic the situation that
data arrives block by block, we use fread() function in data.table
package (Dowle and Srinivasan 2021) to load the data in each block. The
running time is recorded by the Sys.time() function and repeat each case
50 times. The averaged computing times are reported in Figure 4.

From Figure 4, we can see that the uniform subsampling method uses less
time than the MV and \(\operatorname { M V g }\) methods. This is
because the uniform method does not need to compute the optimal
subsampling probabilities. As expected, the computation time is very
sensitive to the number of updates for all methods since the labeling
cost takes most of the time even though we only assume that each subject
only requires 0.01 sec to label. These facts echo the conclusion in
Section 4.1 that the main cost is the data labeling for the measurement
constraints problem. Since annotating large datasets is a tedious and
time-intensive task, the MV subsampling method has its advantages in
dealing with massive data under measurement constraints.

\pandocbounded{\includegraphics[keepaspectratio]{images/57499fec46e0b3168f537ee94fc931762bc63c03ad36b3e26a754fabd462b573.jpg}}\\
Figure 4. The Computational time for different numbers of updates based
on the MV (triangle), MVg (square), and uniform subsampling (circle)
methods.

\pandocbounded{\includegraphics[keepaspectratio]{images/0411891ce894a824f770ac46a9bc12dbc0a89ebceafb7e6f3fcce7ba47f07fce.jpg}}\\
Figure 5. The log of MSEs and expected log-likelihood gain on the
testing set with different numbers of updates based on the MV
(triangle), MVg (square), and uniform subsampling (circle) methods.

\section{4.2. Forest Cover Type
Dataset}\label{forest-cover-type-dataset}

The Forest cover type dataset from the UCI Machine Learning repository
at https://archive.ics.uci.edu/dataset/31/covertype is being analyzed to
predict the forest cover type using geometric information. The cover
type is divided into seven categories in the research areas. Among the
seven categories, spruce or fir, lodgepole pine are the primary major
tree species with proportions of \(3 6 . 5 \%\) , \(4 8 . 8 \%\) ,
respectively. The primary goal is to distinguish forest cover type from
spruce or fir (type 1), lodgepole pine (type 2), and the rest. The model
uses elevation, aspect in degrees azimuth, slope in degrees, and
horizontal distance to the nearest surface water feature as covariates.
The data contains 581,012 instances, and the proportions of the seven
cover types are given. The data is divided into a training set of
\(8 5 \%\) (501,000 instances) and a testing set of \(1 5 \%\) (80,012
instances).

We again set the number of blocks to be 500, and use the first
\(n _ { 0 } ~ = ~ 1 0 0 0\) instances in the training set as historical
data. The target subsample size of each block is set to 30. The maximum
likelihood estimator calculated by the entire training set is treated as
``true'' parameters. The empirical MSE is reported in Figure 5(a).

From Figure 5(a), it is clear that our methods outperform the uniform
subsampling method. To measure the prediction performance, we report the
expected log-likelihood gain (EL) as adopted in Ando and Chau Li (2017).
To be precise, the EL is calculated by
\(\begin{array} { r } { \mathrm { E L } = \sum _ { i = 1 } ^ { n _ { \mathrm { t e s t } } } \sum _ { j = 1 } ^ { J } \mathbb { I } ( y _ { i } = j ) \log \pi _ { i j } ( \tilde { \boldsymbol { \beta } } _ { p , k } ^ { ( t ) } ) } \end{array}\)
with \(n _ { \mathrm { t e s t } }\) being the size of he test data.
Since it is the likelihood of the test data, a larger EL corresponds to
a better method. As expected, the performance of the MV method in both
estimation and prediction is better than the uniform subsampling method.

\section{4.3. Census Income Dataset}\label{census-income-dataset}

We also apply our method on a census income dataset, which is provided
in the UCI Machine Learning repository at https://
archive.ics.uci.edu/dataset/20/census+income.

In this dataset, the responses are labeled as 1 and 2, where 1 stands
for the person who makes less than or equal 50K a year and 2 for the
rest. The following five covariates are considered: age
\(( x _ { 1 } )\) , final weight \(( x _ { 2 } )\) , highest level of
education \(( x _ { 3 } )\) , capital loss \(( x _ { 4 } )\) , and hours
worked per week \(( x _ { 5 } )\) . Here, the final weight
\(( x _ { 2 } )\) represents the weights related to people who have
similar socio-economic characteristics, which is measured by the
Population Division at the Census Bureau. The variable \(x _ { 4 }\) is
the loss of income due to some bad investments. Readers may refer to
Kohavi (1996) for more details. A total of 48,842 observations are in
this dataset. The proportion of individuals whose income is less than or
equal to 50K a year and exceeds 50K is \(7 6 . 0 7 \%\) and
\(2 3 . 9 3 \%\) , respectively. In addition, this dataset is
partitioned into a training set with 32,561 instances and a testing set
with 16,281 instances by the UCI Machine Learning repository.

\pandocbounded{\includegraphics[keepaspectratio]{images/24f442b752a06fa5f6cba848c2a3e10730e96da7b6baa5be24ef1e2cf7c47ed6.jpg}}\\
Figure 6. The log of MSEs and Expected log-likelihood gain on the
testing set with the number of updates based on the MV (triangle), MVg
(square), and uniform subsampling (circle) methods.

We adopt the baseline-category logit model under the proportional odds
assumption here, which is also known as logistic regression. The
regression coefficients calculated using the full training set are
regarded as ``true'' parameters in this example. The total number of
blocks is set to 500, equivalently the size of each block is 64. In
addition, the first \(n _ { 0 } ~ = ~ 5 6 1\) instances are used as
historical data. The target subsample size is chosen as
\(\boldsymbol { r } ~ = ~ \boldsymbol { 3 }\) . The empirical MSE and
expected log-likelihood gain for the MV,
\(\begin{array} { r } { \operatorname { M V g } . } \end{array}\) and
uniform subsampling are shown in Figure 6.

As expected, the MV and \(\operatorname { M V g }\) methods perform
similarly and both of them dominate the uniform subsampling method. This
pattern is similar to that in the simulation studies. We also consider
the classification accuracy on the testing set with different numbers of
updates when the classification probability threshold is selected as
0.5. The results are presented in Figure
\(6 ( \boldsymbol { \mathbf { b } } )\) . It is clear to see that our
methods also have better performance than the uniform subsampling method
in prediction since our methods provide more accurate estimators. The
superiority of the MV method can also be found in the sense of
prediction, which agrees with Theorem 3.3.

\section{5. Conclusion}\label{conclusion}

In this article, we present an optimal online subsampling algorithm for
multinomial logistic models and simultaneously introduce sequential
updating estimators for the regression coefficients. Unlike most
existing works which focus on the case that the full data are given
before subsampling, our work gives a sequential subsampling solution to
accommodate the online big data analysis. The proposed method breaks the
storage barrier and the computational barrier under high-speed data
streams and is particularly useful for data under measurement
constraints. Note that the proposed methods need
\(\tilde { \boldsymbol { \beta } } ^ { ( t ) }\) together with a crude
estimate of the Hessian matrix based on a pilot sample. It is natural to
see the subsampling procedure can be performed at the sensor side with a
small data transaction cost on synchronizing
\(\tilde { \boldsymbol { \beta } } ^ { ( t ) }\) . Then perform the
analysis on the data center side. Since solar-powered Internet of Things
(IoT) devices are getting cheaper, our method can be a choice in
reducing the data transaction. The consistency and asymptotic normality
of the subsample-based estimator are derived. Based on the asymptotic
variance of the resultant estimator, the optimal subsampling
probabilities are obtained. Both theoretical and numerical results
demonstrate the great potential of the proposed methods in extracting
useful information from online data with categorical responses under
measurement constraints.

In this work, we focus on the scenario that the number of data blocks
grows to infinity at a rate faster than the size of each block, which is
common in high-velocity data streams. As for the scenarios that the data
arrival rate is slower, in the sense \(T\) is finite or much slower than
\(r ^ { 1 / 3 }\) , the distributed subsampling method proposed in Yu et
al.~(2022) is recommended to replace the proposed sampling probability
by the OSMUC subsampling probability. Due to data storage constraints,
the proposed method tries to avoid access to the historical data.
Another interesting direction is how to further use the labeled subdata
in the analysis processing when accessing the historically labeled
subdata is allowed.

\section{Supplementary Materials}\label{supplementary-materials}

All technical proofs and additional simulation results are included in
the supplementary material.

\section{Acknowledgments}\label{acknowledgments}

The authors sincerely thank the editor, associate editor, and two
referees for their valuable comments and insightful suggestions, which
lead to further improvement of this work.

\section{Disclosure Statement}\label{disclosure-statement}

The authors report there are no competing interests to declare.

\section{Funding}\label{funding}

Ai's work is supported by NSFC grants 12071014 and 12131001, National
Key R\&D Program of China 2020YFE0204200, and LMEQF. Yu's work is
supported by NSFC grant 12471244 and Beijing Municipal Natural Science
Foundation grant 1232019.

\section{ORCID}\label{orcid}

Mingyao Ai {\$\textbackslash textcircled\{1\}\$}
http://orcid.org/0000-0002-0421-0051

\section{References}\label{references}

Ai, M., Yu, J., Zhang, H., and Wang, H. (2021), ``Optimal Subsampling
Algorithms for Big Data Regressions,'' Statistica Sinica, 31, 749--772.
{[}3{]}\\
Ando, T., and Chau Li, K. (2017), ``A Weight-Relaxed Model Averaging
Approach for High-Dimensional Generalized Linear Models,'' The Annals of
Statistics, 45, 2654--2679. {[}9{]}\\
Cormen, T. H., Leiserson, C. E., Rivest, R. L., and Stein, C. (2009),
Introduction to Algorithms, Cambridge, MA: MIT Press. {[}5{]}\\
Dekel, O., Gilad-Bachrach, R., Shamir, O., and Xiao, L. (2012),
``Optimal Distributed Online Prediction Using Mini-Batches,'' Journal of
Machine Learning Research, 13, 165--202. {[}1{]}\\
Dowle, M., and Srinivasan, A. (2021), data.table: Extension of
`data.frame`, R package version 1.14.2. {[}8{]}\\
Han, L., Tan, K. M., Yang, T., and Zhang, T. (2020), ``Local Uncertainty
Sampling for Large-Scale Multiclass Logistic Regression,'' The Annals of
Statistics, 48, 1770--1788. {[}1,4{]}\\
Horn, R. A., and Johnson, C. R. (2012), Matrix Analysis (2nd ed.),
Cambridge: Cambridge University Press. {[}4{]}\\
Horvitz, D. G., and Thompson, D. J. (1952), ``A Generalization of
Sampling Without Replacement from a Finite Universe,'' Journal of the
American statistical Association, 47, 663--685. {[}3{]}\\
Kohavi, R. (1996), ``Scaling Up the Accuracy of Naive-Bayes Classifiers:
A Decision-Tree Hybrid,'' in Proceedings of the Second International
Conference on Knowledge Discovery and Data Mining, pp.~202--207.
{[}9{]}\\
Lee, A. H. (1987), ``Diagnostic Displays for Assessing Leverage and
Influence in Generalized Linear Models,'' Australian Journal of
Statistics, 29, 233--243. {[}5{]}\\
Li, F., Xie, R., Wang, Z., Guo, L., Ye, J., Ma, P., and Song, W. (2019),
``Online Distributed Iot Security Monitoring with Multidimensional
Streaming Big Data,'' IEEE Internet of Things Journal, 7, 4387--4394.
{[}1{]}\\
Ma, P., Chen, Y., Zhang, X., Xing, X., Ma, J., and Mahoney, M. W.
(2022), ``Asymptotic Analysis of Sampling Estimators for Randomized
Numerical Linear Algebra Algorithms,'' Journal of Machine Learning
Research, 23, 1--45. {[}1,5{]}\\
Ma, P., Mahoney, M. W, and Yu, B. (2015), ``A Statistical Perspective on
Algorithmic Leveraging,'' Journal of Machine Learning Research, 16,
861-- 911. {[}1,6{]}\\
McCullagh, P., and Nelder, J. A. (1989), Generalized Linear Models (2nd
ed.), Boca Raton, FL: Chapman \& Hall/CRC. {[}4{]}\\
Meng, C., Xie, R., Mandal, A., Zhang, X., Zhong, W., and Ma, P. (in
press), ``Lowcon: A Design-based Subsampling Approach in a Misspecified
Linear Model,'' Journal of Computational and Graphical Statistics.
{[}1{]}\\
Polyak, B. T., and Juditsky, A. B. (1992), ``Acceleration of Stochastic
Approximation by Averaging,'' SIAM Journal on Control and Optimization,
30, 838--855. {[}3{]}\\
R Core Team. (2021), R: A Language and Environment for Statistical
Computing, Vienna, Austria: R Foundation for Statistical Computing.
{[}7{]}\\
Robbins, H., and Monro, S. (1951), ``A Stochastic Approximation
Method,'' The Annals of Mathematical Statistics, 22, 400--407. {[}1{]}\\
Särndal, C.-E., Swensson, B., and Wretman, J. (2003), ``Model Assisted
Survey Sampling (2nd ed.), New York: Springer. {[}3{]}\\
Schifano, E. D., Wu, J., Wang, C., Yan, J., and Chen, M.-H. (2016),
``Online Updating of Statistical Inference in the Big Data Setting,''
Technometrics, 58, 393--403. {[}1,4{]}\\
Wang, H., Yang, M., and Stufken, J. (2019), ``Information-based Optimal
Subdata Selection for Big Data Linear Regression,'' Journal of the
American Statistical Association, 114, 393--405. {[}1{]}\\
Wang, H., Zhu, R., and Ma, P. (2018), ``Optimal Subsampling for Large
Sample Logistic Regression,'' Journal of the American Statistical
Association, 113, 829--844. {[}1,2,4{]}\\
Wang, L., Elmstedt, J., Wong, W. K., and Xu, H. (2021), ``Orthogonal
Subsampling for Big Data Linear Regression,'' The Annals of Applied
Statistics, 15, 1273--1290. {[}1{]}\\
Wu, X., Huo, Y., Ren, H., and Zou, C. (2024), ``Optimal Subsampling via
Predictive Inference,'' Journal of the American Statistical Association,
1-- 13. {[}1{]}\\
Xie, R., Bai, S., and Ma, P. (2023), ``Optimal Sampling Designs for
Multidimensional Streaming Time Series with Application to Power Grid
Sensor Data,'' The Annals of Applied Statistics, 17, 3195--3215.
{[}1{]}\\
Xie, R., Wang, Z., Bai, S., Ma, P., and Zhong, W. (2019), ``Online
Decentralized Leverage Score Sampling for Streaming Multidimensional
Time Series,'' in The 22nd International Conference on Artificial
Intelligence and Statistics, pp.~2301--2311, PMLR. {[}5{]}\\
Yao, Y., and Wang, H. (2019), ``Optimal Subsampling for Softmax
Regression,'' Statistical Papers, 60, 585--599. {[}2{]}\\
Yu, J., Wang, H., Ai, M., and Zhang, H. (2022), ``Optimal Distributed
Subsampling for Maximum Quasi-Likelihood Estimators with Massive Data,''
Journal of the American Statistical Association, 117, 265--276.
{[}1,4,10{]}\\
Zhang, A., Lipton, Z. C., Li, M., and Smola, A. J. (2021a), ``Dive into
Deep Learning,'' arXiv preprint arXiv:2106.11342. {[}1{]}\\
Zhang, H., and Wang, H. (2021), ``Distributed Subdata Selection for Big
Data via Sampling-based Approach,'' Computational Statistics and Data
Analysis, 153, 107072. {[}1{]}\\
Zhang, T., Ning, Y., and Ruppert, D. (2021b), ``Optimal Sampling for
Generalized Linear Models Under Measurement Constraints,'' Journal of
Computational and Graphical Statistics, 30, 106--114. {[}1,5{]}

\end{document}
