\begin{document}

\section{基于通用多路比较的光谱排序推断}	

\textbf{作者:}Jianqing Fan \ Zhipeng Lou \ Weichen Wang \ Mengxin Yu \\
\textbf{来源:}Operations Research,2025\\

\section*{摘要}\label{abstract}

本文研究了一种更通用、更现实的场景下,谱方法在估计和量化被比较实体未观测偏好分数的性能问题。具体而言,比较图由可能具有异质大小的超边组成,并且对于给定的超边,比较次数可以低至一次。这种设置在现实应用中普遍存在,避免了在常用的 Bradley-Terry-Luce (BTL) 或 Plackett-Luce (PL) 模型中需指定图的随机性或要求严格同质抽样假设的需要。此外,在 BTL 或 PL 模型适用的场景中,我们揭示了谱估计量与最大似然估计量 (MLE) 之间的关系。我们发现,一种两阶段的谱方法——即应用从等权朴素谱方法中估计出的最优权重——可以达到与 MLE 相同的渐近有效性。基于估计偏好分数的渐近分布,我们还引入了一个全面的框架来执行单样本和双样本排序推断,该框架适用于固定图和随机图设置。值得注意的是,这是首次提出有效的双样本秩检验方法。最后,我们通过全面的数值模拟验证了我们的发现,并随后应用我们开发的方法对统计学期刊和电影排名进行统计推断。


\subsection{引言}

排序聚合(Rank aggregation)在各种应用中至关重要,包括网络搜索(Dwork et al., 2001; Wang et al., 2016)、灵长类动物智力实验(Johnson et al., 2002)、品类优化(Aouad et al., 2018; Chen et al., 2020)、推荐系统(Baltrunas et al., 2010; Li et al., 2019)、体育排名(Massey, 1997; Turner and Firth, 2012)、教育(Avery et al., 2013; Caron et al., 2014)、投票(Plackett, 1975; Mattei and Walsh, 2013),以及近期流行的大型语言模型ChatGPT中使用的指令微调(Ouyang et al., 2022)。因此,它已成为心理学、计量经济学、教育学、运筹学、统计学、机器学习、人工智能等许多领域中的一个基本问题。

Luce (Luce, 1959) 提出了著名的Luce选择公理。令 $p(i\mid A)$ 表示在备选集合 $A$ 中选择项目 $i$ 而非其他所有项目的概率。根据该公理,当在任何包含 $i$ 和 $j$ 的备选集合 $A$ 中比较两个项目 $i$ 和 $j$ 时,选择 $i$ 而非 $j$ 的概率不受集合中其他备选项存在的干扰。具体而言,该公理假定:

$$
\frac{\mathbb{P}(i\mathrm{~is~preferred~in~}A)}{\mathbb{P}(j\mathrm{~is~preferred~in~}A)} = \frac{\mathbb{P}(i\mathrm{~is~preferred~in~}\{i,j\})}{\mathbb{P}(j\mathrm{~is~preferred~in~}\{i,j\})}.
$$

这一假设引出了一个独特的参数化选择模型:用于成对比较的 Bradley-Terry-Luce (BTL) 模型,以及用于 $M$ 项排序($M\geq 2$)的 Plackett-Luce (PL) 模型。

在本文中,我们考虑包含 $n$ 个项目的集合,其真实排序由某些未观测到的偏好分数 $\theta_{i}^{*}$(其中 $i = 1,\dots ,n$)决定。在此情境下,BTL模型假设一个个体或随机事件将项目 $i$ 排在项目 $j$ 之上的概率为 $\mathbb{P}(\mathrm{item}~i$ is preferred over item $j) =$ $e^{\theta_i^*} / (e^{\theta_i^*} + e^{\theta_j^*})$。 Plackett-Luce 模型是成对比较的扩展版本,允许进行更全面的 $M$ 项完整排序,最初由 Plackett (1975) 描述。该模型在给一个选定的大小为 $M< \infty$(在全部 $n$ 个项目之中)的项目子集排序时考虑了个体偏好,我们将这种排序表示为 $i_{1}\succ \dots \succ i_{M}$。可以把这个完整排序想象成 $M - 1$ 个独立事件:$i_{1}$ 优于集合 $\{i_{1},\ldots ,i_{M}\}$(即被选中为第一),然后是 $i_{2}$ 优于集合 $\{i_{2},\ldots ,i_{M}\}$(即剩余项目中选为第二),依此类推。PL模型计算一个完整排序 $i_{1}\succ \dots \succ i_{M}$ 的概率的公式为:

$$
\mathbb{P}(i_1\succ \dots \succ i_M) = \prod_{j = 1}^{M - 1}\left[e^{\theta_{ij}^*} / \sum_{k = j}^{M}e^{\theta_{ik}^*}\right].
$$

接下来,我们将简要介绍在参数化模型下对BTL和PL模型进行模型估计和不确定性量化方面取得进展的文献。

\subsubsection{相关文献}

一系列论文研究了基于BTL或PL模型的模型估计与推断。在Bradley-Terry-Luce模型方面,Hunter (2004)通过最小化-最大化算法巩固了其理论特性。此外,Negahban et al. (2012)开发了名为Rank Centrality(谱方法)的迭代排序聚合算法,该算法能以最优$\ell_2$-统计速率恢复BTL模型的潜在分数。随后,Chen and Suh (2015)采用两步法(谱方法后接MLE),在比较图基于Erdős-Rényi模型的情境下研究BTL模型——该模型假设每对项目有概率$p$被比较,且每对一旦连接将被比较$L$次。基于与Chen and Suh (2015)相似的设定,Chen et al. (2019)研究了恢复潜在分数的最优统计速率,证明当条件数为常数时,正则化最大似然估计(MLE)和谱方法都能最优恢复前$K$个项目,并推导了$\ell_2$-和$\ell_\infty$-收敛速率。此外,Chen et al. (2022)将研究扩展到部分恢复场景,并将分析改进至非正则化MLE。

超越简单两两比较,研究亦关注$M$项比较的排序问题($M \geq 2$)。Plackett-Luce模型及其变体是该领域的典型代表,相关文献丰富(Plackett, 1975; Guiver and Snelson, 2009; Cheng et al., 2010; Hajek et al., 2014; Maystre and Grossglauser, 2015; Szorenyi et al., 2015; Jang et al., 2018)。特别地,Jang et al. (2018)在均匀超图背景下研究PL模型:大小为$M$的元组以概率$p$被比较,且每个连接元组在超图中被比较$L$次。通过将$M$项比较数据分解为成对数据,他们利用谱方法获得潜在分数的$\ell_\infty$-统计速率,并给出PL模型中识别前$K$项目所需样本复杂度的下界。近期,Fan et al. (2022b)在相同模型设定下强化了Jang et al. (2018)的结论(聚焦于首选项目),应用MLE替代成对分解方法,达到了恢复前$K$项目所需样本复杂度的下界——这与Jang et al. (2018)需更稠密比较图或更多比较次数的要求形成对比。

上述文献主要关注项目分数估计的非渐近统计一致性,但排序模型的极限分布结果仍有待探索。在Erdős-Rényi比较图(连接概率$p$、每对比较次数$L$)下的BTL模型中,仅有少量关于估计排序分数渐近分布的发现:Simons and Yao (1999)建立了所有项目对完全比较(即$p=1$)时BTL模型MLE的渐近正态性;Han et al. (2020)将结论扩展至稠密非全连接图($p \gtrsim n^{-1/10}$);Liu et al. (2022)提出拉格朗日去偏方法处理稀疏比较图($p \asymp 1/n$但要求$L \gtrsim n^2$);Gao et al. (2021)采用"留二法"确定排序分数渐近分布,在稀疏图下($p \asymp 1/n$)实现最优样本复杂度且允许$L=\mathcal{O}(1)$;Fan et al. (2022a)通过创新证明将协变量信息融入BTL模型,当$p \asymp 1/n$且$L=\mathcal{O}(1)$时给出MLE的最优渐近方差;Fan et al. (2022b)进一步将渐近结果推广至$M\geq2$的PL模型(仍保持最优样本复杂度),并开发构建排序置信区间的统一框架(要求比较次数$L \gtrsim \mathrm{poly}(\log n)$);最近Han and Xu (2023)探索不同规模Erdos-Renyi图混合或超图随机块模型生成的比较数据,扩展了Fan et al. (2022b)的MLE渐近分布设定。

最后讨论本框架的重要应用领域:品类优化(assortment optimization)——收益管理中的关键问题(Talluri and Van Ryzin, 2004; Rusmevichientong and Topaloglu, 2012; Vulcano et al., 2012; Davis et al., 2014; Zhang et al., 2020; Chen et al., 2020, 2023; Shen et al., 2023)。具体场景中,产品(含未购选项)关联未知顾客偏好分数,可表征顾客对产品集的选择行为。基于估计的偏好分数和产品利润信息,已有研究提出各类高效算法以确定不同实际约束下的最优品类组合(Talluri and Van Ryzin, 2004; Rusmevichientong et al., 2010; Gallego and Topaloglu, 2014; Sumida et al., 2021),且偏好分数的统计推断能支持对最优品类统计特性的不确定性量化(Shen et al., 2023)。



\subsubsection{动机与贡献}

\paragraph{比较图}
在本节中,我们将讨论研究动机与问题设置,并从比较图性质、谱方法与极大似然估计(MLE)的联系、排序推断三个维度对比本文结果与现有文献。

现有参数化排序模型研究大多要求比较图源自特定随机图模型。例如在多项比较研究中(如Jang et al. (2016);Fan et al. (2022b)),通常假设比较数据生成于已知分布的均匀图。这一假设在某些场景下面临挑战——尽管存在均匀比较的实际应用,但假定所有比较均源自已知同质分布往往并不符合现实。事实上,更多案例呈现异质比较特性:部分项目比较频率更高,且比较图生成机制未知。下文通过示例1.1阐明研究动机。

**示例1.1.** 考虑顾客序列购买商品的场景。对第$l$位顾客,其浏览的商品集(选择集)为$A_l$,最终选购商品$c_l \in A_l$(首选商品)。数据集形式为$\{(A_l,c_l)\}_{l=1}^D$。若所有选择集以相同概率出现且商品数量固定(如成对比较),则称比较图是同质的;反之若呈现不同大小的选择集、且部分集合因顾客偏好特征频繁出现或被任意选择,此类异质比较图难以用给定随机图模型近似。此时比较数据可能不服从基于Erdos-Renyi均匀图的BTL或PL模型。

示例1.1的异质比较机制广泛存在于实际场景。例如在品类优化(assortment optimization)中,选择集$A_l$通常包含未购选项(no-purchase alternative)。图1展示含5件商品的示例,其中$A_l$大小分布于$\{2,3,4\}$。由于异质性,假设$A_l$具有相同大小或从显式随机图采样是不现实的。然而既有研究多聚焦特定随机图下的统计性质(尤以Erdős-Rényi均匀图为主),如Chen and Suh (2015); Chen et al. (2019); Jang et al. (2016); Gao et al. (2021); Liu et al. (2022); Fan et al. (2022b); Han and Xu (2023)。值得关注的是Shah et al. (2015)研究了固定比较图下成对比较的MLE在$\ell_2$估计最优性,Li et al. (2022)进一步探讨了$\ell_{\infty}$误差最优性,但通用固定比较图的推断结果仍知之甚少。

本文聚焦一般性固定比较图设置,规避复杂图生成过程建模。具体而言,我们通过谱方法研究异质选择集下偏好分数的统计估计与不确定性量化,同时分析谱方法在同质随机比较图中的理论性能并对比结果。因需确保各项目有充足排序信息,固定图设定需略强的条件假设。

一般化设定下,记第$l$次比较的选择偏好为$(c_l,A_l)$:其中$A_l$表示大小可变的异质选择集(可固定或随机),$c_l \in A_l$代表$A_l$中最受偏好项。即在第$l$次比较中,$c_l$优于$A_l$内所有元素。最广义的比较数据集记为$\mathcal{D} = \{l|(c_l,A_l)\}$,关联选择集集合为$\mathcal{G} = \{A_l|l\in \mathcal{D}\}$。该框架亦包含Plackett-Luce模型作为特例——若将PL模型视为$M-1$次选择(作用于$M-1$个选择集)。在温和条件下,我们成功获得谱方法在给定比较图下的最优统计速率,并给出渐近分布的显式表达式。这为异质比较图场景提供高效解决方案。此外,因图结构固定或条件给定,无需要求每次比较重复$L\geq 1$次,甚至可以容纳选择集仅被比较一次的情况(符合多数实际应用场景)。


\paragraph{谱估计与 MLE 之间的联系}

如第1.2.1节所述,本文的通用框架包含同质随机比较图作为特例。先前研究主要集中于评估同质比较场景下的最大似然估计(MLE)与谱方法(Chen and Suh, 2015; Chen et al., 2019, 2022; Fan et al., 2022a,b)。这两种方法在BTL或PL模型背景下研究排序问题时均具效力,因而引出一个核心问题:MLE与谱方法之间存何种关联?

少量研究对此提供了见解:Maystre and Grossglauser (2015) 在多维比较中发现谱估计量与MLE的关联——通过迭代更新特定加权函数构建谱估计量时,二者具有一致性。但此关联仅停留于一阶性质,因其仅关注收敛特性。此外,Gao et al. (2021) 在BTL模型成对比较中证明谱方法的渐近方差大于MLE,但此差异源于其采用了次优加权方案。

本文基于同质随机比较图(该设定在既有文献中被广泛研究)论证:通过使用特定最优估计的信息加权函数,谱估计量的渐近方差在PL模型多维比较下将与MLE完全匹配。因此,MLE可视为一种"两阶段"谱方法——第一阶段一致估计未知偏好分数,第二阶段通过优化加权使谱方法获得有效估计量。尤为值得注意的是,我们在最稀疏采样图(至对数项)下实现了最优样本复杂度。

\paragraph{排序推断:单样本 VS 双样本}

作为另一项贡献,我们还研究了若干排序相关推断问题。以下示例说明研究动机:

**示例1.2.** 首先考虑单样本推断问题:给定候选项目集$\{1,\dots,n\}$及其比较数据集,关注
- 为目标项$\{r_1,\dots,r_m\}$构建秩置信区间
- 检验给定项$m$是否属于前$K$名集合(包含最优$K$项)

其次考虑双样本推断问题:针对相同项目集$\{1,\dots,n\}$的两组数据集,关注
- 检验特定项$m$的秩是否在两组间保持一致(如跨群体/时间段)
- 检验前$K$名集合是否发生变化

排序在现实应用中无处不在(如大学排名、运动队评级、网页排序),但多数仅提供一阶信息而缺乏不确定性度量。例如在BTL模型中,当两项基础分数相当时,其排序高低仅具50\%概率。此时排序结果因基础分数不可区分而不可靠,凸显秩置信区间的必要性。

基于此,本研究提出综合框架,可在异质比较图下高效解决示例1.2所述问题。此外,本方法优化了既有研究的样本复杂度:
1. **单样本推断**:在同质随机比较图限定下,Liu et al. (2022)要求$L \gtrsim n^2$进行统计推断,Fan et al. (2022b)改进至$L \gtrsim \mathrm{poly}(\log n)$,而本框架允许$L = \mathcal{O}(1)$甚至$L = 1$
2. **双样本推断**:此类可广泛应用于政策评估、处理效应比较、变点检测等场景的方法此前尚未被研究。本文首次建立双样本排序推断的通用框架,同样提供最优样本复杂度

\paragraph{理论贡献}

我们建立的理论分析基于Gao et al. (2021)和Chen et al. (2019)的先前技术,但证明过程具有以下创新性:在这两篇论文及其他采用随机比较图的研究中,图的随机性与排序结果的随机性通常在分析中相互交织。我们将分离这两种随机性,揭示固定比较图中关键信息量的适定表征,研究这些量与谱排序性能的关联,并提供固定图下有效排序推断的充分条件。这一理论尝试在现有文献中前所未见。此外,所有分析均支持可变比较规模及各比较集的任意重复次数。这显著拓宽了方法的适用性——实践中大量排序问题包含非重复性、项目数各异的比较。我们也发展了含图随机性的理论:同质性采样各比较元组可放宽假设条件。借助更宽松的条件,我们清晰展示了改进性能保证的实现路径与机制(见定理4.4的假设与证明)。该部分凸显了固定图与随机图的差异,深化了对图随机性在谱排序中作用的理论认知。


\subsubsection{路线图和记号}

在第 2 节中,我们建立了模型并介绍了谱排序方法。第 3 节分别基于固定比较图和具有 PL 模型的随机图,专门探讨谱估计量的渐近分布。在同一节中,我们还介绍了针对单样本和双样本分析设计的构建秩置信区间和秩检验统计量的框架。第 4 节详细阐述了所提出的所有方法的理论保证。第 5 节和第 6 节包含了全面的数值研究以验证理论结果,并用两个真实数据示例说明了我们排序推断方法的实用性。最后,我们在第 7 节以一些讨论作结。所有证明均放在附录中。

在整篇工作中,我们使用 $[n]$ 表示索引集 $\{1,2,\dots ,n\}$。对于任意给定的向量 $\mathbf{x} \in \mathbb{R}^n$ 和 $q \geq 0$,我们使用 $\| \mathbf{x}\| _q = (\sum_{i = 1}^{n} |x_i|^q)^{1 / q}$ 表示向量 $\ell_q$ 范数。对于任意给定的矩阵 $\mathbf{X} \in \mathbb{R}^{d_1 \times d_2}$,我们使用 $\| \cdot \|$ 表示 $\mathbf{X}$ 的谱范数,若 $\mathbf{X}$ 或 $- \mathbf{X}$ 是半正定矩阵,则分别记为 $\mathbf{X} \succcurlyeq 0$ 或 $\mathbf{X} \preccurlyeq 0$。对于事件 $A$,$1(A)$ 表示指示函数,当 $A$ 成立时等于 1,否则为 0。对于两个正数序列 $\{a_n\}_{n \geq 1}$ 和 $\{b_n\}_{n \geq 1}$,若存在正常数 $C$ 使得 $a_{n} / b_{n} \leq C$,则记作 $a_n = \mathcal{O}(b_n)$ 或 $a_{n} \lesssim b_{n}$;若 $a_{n} / b_{n} \to 0$,则记作 $a_{n} = o(b_{n})$。此外,$\mathcal{O}_{p}(\cdot)$ 和 $o_{p}(\cdot)$ 分别与 $\mathcal{O}(\cdot)$ 和 $o(\cdot)$ 含义相似,但这些关系是渐近的且概率趋于 1。类似地,若存在常数 $c > 0$ 使得 $a_{n} / b_{n} \geq c$,则记作 $a_{n} = \Omega (b_{n})$ 或 $a_{n} \gtrsim b_{n}$。若 $a_{n} = \mathcal{O}(b_{n})$ 且 $a_{n} = \Omega (b_{n})$,则记作 $a_{n} = \Theta (b_{n})$ 或 $a_{n} \asymp b_{n}$。对于两个随机变量 $A_{n}, B_{n}$,若记作 $A_{n} \approx B_{n}$,则意味着 $A_{n} - B_{n} = o(1)$ 的概率趋于 1。给定 $n$ 个项目,我们使用 $\theta_{i}^{*}$ 表示第 $i$ 个项目的潜在偏好得分。定义 $r: [n] \to [n]$ 为对 $n$ 个项目的排序映射算子,它根据偏好得分将每个项目映射到其总体秩。我们将第 $i$ 个项目的秩记为 $r_{i}$ 或 $r(i)$。默认情况下,我们考虑按分值从大到小排序。


\subsection{多路比较模型与谱排序}

我们首先介绍一个通用的离散选择模型,它涵盖了经典的普拉基特-卢斯 (Plackett-Luce) 模型以及固定比较图场景。

\subsubsection{离散选择模型}

假设有 $n$ 个项目需要排序。根据卢斯选择公理 (Luce, 1959),一组给定 $n$ 个项目的偏好得分可以参数化为向量 $(\theta_{1}^{*}, \ldots , \theta_{n}^{*})^{\top}$,使得对于任意选择集 $A$ 和项目 $i \in A$,有 $\mathbb{P}(i \text{ 在 } A \text{ 中获胜}) = e^{\theta_{i}^{*}} / (\sum_{k \in A} e^{\theta_{k}^{*}})$。由于参数仅在平移变换下可识别,为便于识别,不失一般性,我们假设 $\sum_{i = 1}^{n} \theta_{i}^{*} = 0$。我们考虑通用比较模型,其中给定一组比较及其结果 $\{(c_{l}, A_{l})\}_{l \in \mathcal{D}}$。此处 $c_{l}$ 表示在选择集 $A_{l}$ 中被选中的项目,其概率为 $e^{\theta_{c_{l}}^{*}} / (\sum_{k \in A_{l}} e^{\theta_{k}^{*}})$。

**备注 2.1.** 此通用比较模型包含许多众所周知的特例。

-   对于布拉德利-特里-卢斯 (Bradley-Terry-Luce, BTL) 模型,每次只需将 $A_{l}$ 设定为被比较的对 (pair)。如果每一对项目被独立地比较 $L$ 次,我们只需将结果写为 $(c_{l}, A_{l})$ 并对 $l$ 重新索引。
-   对于普拉基特-卢斯 (Plackett-Luce, PL) 模型,我们已经获得了一个选择集 $B = \{i_{1}, \dots , i_{|B|}\}$ 的完整排序。观察到某一特定排序的概率为
$$
\begin{array}{r l} & {\mathbb{P}(i_{1} > i_{2} > \dots >i_{|B|}) = \mathbb{P}(i_{1}\text{ 在 }C_{1}\text{ 中获胜} \mid C_{1} = B)\cdot \mathbb{P}(i_{2}\text{ 在 }C_{2}\text{ 中获胜} \mid C_{2} = B\{-i_{1}\})\cdot \dots}\\ & {\qquad \cdot \mathbb{P}(i_{|B| - 1}\text{ 在 }C_{|B| - 1}\text{ 中获胜} \mid C_{|B| - 1} = B\{-i_{1},\dots , - i_{|B| - 2}\})}\\ & {\qquad = \frac{e^{\theta_{i_{1}}^{*}}}{\sum_{j = 1}^{|B|}e^{\theta_{i_{j}}^{*}}}\cdot \frac{e^{\theta_{i_{2}}^{*}}}{\sum_{j = 2}^{|B|}e^{\theta_{i_{j}}^{*}}}\cdot \dots \frac{e^{\theta_{i_{|B| - 1}}^{*}}}{\sum_{j = |B| - 1}^{|B|}e^{\theta_{i_{j}}^{*}}},} \end{array}
$$
其中 $C_{i}, i \geq 1$ 是第 $i$ 个比较集,$B\{- i_{1}, \dots , - i_{M}\}$ 表示移除集合 $\{i_{1}, \dots , i_{M}\}$ 后剩余的项目集合。这些比较结果也可以分解为若干次比较:
$$
\{(i_{1}, B), (i_{2}, B\{-i_{1}\}), \dots , (i_{|B| - 1}, B\{-i_{1}, \dots , -i_{|B| - 2}\})\} .
$$
-   在固定比较图情形下,$\{A_{l}, l\in \mathcal{D}\}$ 是给定的且因此没有随机性,所以假设比较结果 $c_{l}$ 是独立的。相反,在随机比较图(例如 PL 模型)情形下,$A_{l}$ 可能是相关的。例如,$(c_{l}=i_{1},A_{l}=B)$ 和 $(c_{l}=i_{2},A_{l}=B\{-i_{1}\})$ 是相关的,因为 $B\{-i_{1}\}$ 依赖于第一次比较的胜者 $i_{1}$。因此,为了研究其理论性质,我们必须针对不同情形明确阐述比较生成的随机过程假设。

回顾一下,通用比较数据表示为 $\{(c_{l},A_{l})\}_{l\in \mathcal{D}}$。相应的选择集集合为 $\mathcal{G} = \{A_{l}\mid l\in \mathcal{D}\}$。当我们只有成对比较数据时,$|A_{l}| = 2$ 且 $\mathcal{G}$ 表示所有已比较的边的集合。但在通用设定下,$A_{l}$ 可以有不同的大小,我们记 $M = \max_{l\in \mathcal{D}}|A_{l}|< \infty$ 为比较超图(hyper-graph)边的最大尺寸。此外,如果同一个比较超边 $A_{l}$ 有 $L$ 次独立的比较,我们会用不同的 $l$ 来标记这些比较。所以在 $\mathcal{G}$ 中,超边 $A_{l}$ 可能被比较了多次并有不同的结果 $c_{l}$。

在本文中,我们考虑对基于多元比较(multiway comparisons)的偏好数据使用谱方法。我们将首先关注固定比较图,然后再考虑常用的随机比较图结构。请注意,无论比较图 $\mathcal{G}$ 是固定的还是随机的,我们的谱排序方法都将以观测到的 $\mathcal{G}$ 为条件(在实践中是给定的)。生成 $\mathcal{G}$ 的底层模型可以非常通用:它可以是给定的,也可以是基于 Erdős-Rényi 随机图以相同概率 $p$ 随机生成的,或者更一般地由某些其他比较规则诱导生成(这些规则甚至可能导致一些 $A_{l}$ 相互依赖)。例如,如果我们将 PL 模型的每次比较视为 $M - 1$ 次成对比较(涉及第 1 名 vs 第 2 名,第 2 名 vs 第 3 名,...,第 $M - 1$ 名 vs 第 $M$ 名)。那么得到的比较数据,表示为 $(c_{l} = i_{k},A_{l} = \{i_{k},i_{k + 1}\})$($k = 1,\ldots ,M - 1$),这些数据是相关的(即使 $A_{l}$ 的定义也依赖于完整的比较结果)。


\subsubsection{谱排序}

在谱方法中,我们正式定义一个马尔可夫链,记为 $M = (S,P)$。其中 $S$ 表示与 $n$ 个待比较项目对应的 $n$ 个状态集合,这些状态可视作有向比较图的顶点。而 $P$ 是由下文定义的转移矩阵构成。该矩阵通过表示 $S$ 中任意两个特定状态是否能够通过非零转移概率相连,来控制状态间的转移。

定义两个比较索引集 $\mathcal{W}_{j},\mathcal{L}_{i}$,其中 $j$ 为胜者,$i$ 为败者:
$$
\mathcal{W}_{j} = \{l\in \mathcal{D}\mid j\in A_{l},c_{l} = j\} ,\qquad \mathcal{L}_{i} = \{l\in \mathcal{D}\mid i\in A_{l},c_{l}\neq i\} .
$$
因此当 $i\neq j$ 时,它们的交集给出了 $i,j$ 被比较且 $j$ 获胜的所有情形,即 $\mathcal{W}_{j}\cap \mathcal{L}_{i} = \{l\in \mathcal{D}\mid i,j\in A_{l},c_{l} = j\}$。定义转移矩阵 $P$,其转移概率为
$$
P_{ij} = \left\{ \begin{array}{ll}\frac{1}{d}\sum_{l\in \mathcal{W}_{j}\cap \mathcal{L}_{i}}\frac{1}{f(A_{l})}, & \text{若 } i\neq j, \\ 1 - \sum_{k:k\neq i}P_{ik}, & \text{若 } i = j. \end{array} \right.
$$
此处 $d$ 需选择足够大使对角元素非负,但不过大使转移概率充分分配。当比较图随机时,我们通过研究 $\sum_{k:k\neq i}P_{i k}$ 的集中不等式,选择 $d$ 以大概率保证对角元素非负(概率趋于 1)。其中 $f(A_{l}) > 0$ 是一个加权函数,用于编码第 $l$ 次比较中的总信息量。一个自然的选择是 $f(A_{l}) = |A_{l}|$,这会为比较项目数较少的超边赋予更高权重。我们将在后文讨论 $f(\cdot)$ 的最优选择。

当 $i\neq j$ 时,$P_{i j}$ 也可写为
$$
P_{i j} = \frac{1}{d}\sum_{l\in \mathcal{D}}\mathbf{1}(i,j\in A_{l})\mathbf{1}(c_{l} = j)\frac{1}{f(A_{l})}.
$$
以 $\mathcal{G}$ 为条件时,总体转移概率为
$$
P_{i j}^{*} = E[P_{i j}|\mathcal{G}] = \left\{ \begin{array}{l l}{\frac{1}{d}\sum_{l\in \mathcal{D}}\mathbf{1}(i,j\in A_{l})\frac{e^{\theta_{j}^{*}}}{\sum_{u\in A_{l}}e^{\theta_{u}^{*}}}\frac{1}{f(A_{l})},} & {\text{若 } i\neq j,}\\ {1 - \sum_{k:k\neq i}P_{i k}^{*},} & {\text{若 } i = j.} \end{array} \right.
$$
令
$$
\pi^{*} = (e^{\theta_{1}^{*}},\ldots ,e^{\theta_{n}^{*}}) / \sum_{k = 1}^{n}e^{\theta_{k}^{*}}.
$$
注意到分母中 $\sum_{u\in A_{l}}e^{\theta_{u}^{*}}$ 和 $f(A_{l})$ 均关于 $\theta_{i}^{*},\theta_{j}^{*}$ 对称(只要 $i,j$ 都属于 $A_{l}$)。因此有 $P_{i j}^{*}\pi_{i}^{*} = P_{i j}^{*}\pi_{j}^{*}$。这就是所谓的细致平衡条件,它导致 $\pi^{*}$ 成为上述总体转移矩阵对应马尔可夫链的平稳测度(对任意 $f(\cdot)$ 均成立)。即 $\pi^{*\top}P^{*} = \pi^{*\top}$,换言之 $\pi^{*}$ 是 $P^{*}$ 的(对应特征值 1 的)左特征向量。

最后,在可识别条件 $1^{\top}\theta^{*} = 0$ 下,我们可通过下式估计 $\theta_{i}^{*}$:
\[
\widehat{\theta}_{i}\coloneqq \log \widehat{\pi}_{i} - \frac{1}{n}\sum_{k = 1}^{n}\log \widehat{\pi}_{k}. \tag{2.1}
\]
值得强调的是,该谱估计量在实践中更易计算,仅需一步特征分解。事实上,我们只需计算最大特征值对应的特征向量,这甚至可通过幂法快速求解。相比之下,极大似然估计(MLE)在梯度下降算法的实现过程中,通常在数据存储和步长确定方面计算负担更重。



\subsection{排序推断方法}

In this section, we study the inference methodology for the spectral estimator for the underlying scores $\{\theta_{i}^{*}\}_{i\in [n]}$ of $n$ items. To be specific, we need to establish the statistical convergence rates and asymptotic normality for $\widetilde{\theta_{i}}$ .
在本节中,我们研究 $n$ 个项目潜在得分 $\{\theta_{i}^{*}\}_{i\in [n]}$ 的谱估计器的推断方法。具体而言,我们需要为 $\widetilde{\theta_{i}}$ 建立统计收敛速度和渐近正态性。

\subsubsection{Uncertainty Quantification for a Fixed Comparison Graph}
\subsubsection{固定比较图的不确定性量化}

For the estimation of $\pi^{*}$ , we use the following two approximations, which we will justify later to be accurate enough so as not to affect the asymptotic variance. Let us first focus on our intuition. Firstly, we have
对于估计 $\pi^{*}$,我们使用以下两个近似,稍后我们将证明这些近似足够精确,以至于不会影响渐近方差。让我们首先关注直观理解。首先,我们有
$$
\widehat{\pi}_{i} = \frac{\sum_{j:j\neq i}P_{ji}\widehat{\pi}_{j}}{\sum_{j:j\neq i}P_{ij}}\approx \frac{\sum_{j:j\neq i}P_{ji}\pi_{j}^{*}}{\sum_{j:j\neq i}P_{ij}} \eqqcolon \bar{\pi}_{i}.
$$
Equivalently,
等价地,
\[
\frac{\widehat{\pi}_{i} - \pi_{i}^{*}}{\pi_{i}^{*}}\approx \frac{\bar{\pi}_{i} - \pi_{i}^{*}}{\pi_{i}^{*}} = \frac{\sum_{j:j\neq i}(P_{ji}\pi_{j}^{*} - P_{ij}\pi_{i}^{*})}{\pi_{i}^{*}\sum_{j:j\neq i}P_{ij}}. \tag{3.1}
\]
Secondly, the denominator above can be approximated by its expected value so that (3.1) can further be approximated as
其次,上式中的分母可以用其期望值来近似,因此 (3.1) 可以进一步近似为
\[
J_{i}^{*}\coloneqq \frac{\sum_{j:j\neq i}(P_{ji}e^{\theta_{j}^{*}} - P_{ij}e^{\theta_{i}^{*}})}{\sum_{j:j\neq i}E[P_{ij}|\mathcal{G}]e^{\theta_{i}^{*}}}, \tag{3.2}
\]
by using $\pi_{i}^{*}\propto e^{\theta_{i}^{*}}$ . We will rigorously argue that the asymptotic distributions of $\frac{\widehat{\pi}_{i} - \pi_{i}^{*}}{\pi_{i}^{*}}$ and $J_{i}^{*}$ are identical. For now, let us look at the asymptotic distribution of $J_{i}^{*}$ . Obviously, it is mean zero due to the detailed balance: $E[P_{ji}|\mathcal{G}]\pi_{j}^{*} = E[P_{ij}|\mathcal{G}]\pi_{i}^{*}$ . The denominator of $J_{i}^{*}$ is a constant and can be explicitly written out as follows:
这里利用了 $\pi_{i}^{*}\propto e^{\theta_{i}^{*}}$ 的关系。我们将严格论证 $\frac{\widehat{\pi}_{i} - \pi_{i}^{*}}{\pi_{i}^{*}}$ 和 $J_{i}^{*}$ 具有相同的渐近分布。现在,让我们考虑 $J_{i}^{*}$ 的渐近分布。显然,由于细致平衡 $E[P_{ji}|\mathcal{G}]\pi_{j}^{*} = E[P_{ij}|\mathcal{G}]\pi_{i}^{*}$,其均值为零。$J_{i}^{*}$ 的分母是一个常数,可以显式地写出如下:
\[
\tau_{i}(\theta^{*})\coloneqq \sum_{j:j\neq i}E[P_{ij}|\mathcal{G}]e^{\theta_{i}^{*}} = \frac{1}{d}\sum_{l\in \mathcal{D}}1(i\in A_{l})\left(1 - \frac{e^{\theta_{i}^{*}}}{\sum_{u\in A_{l}}e^{\theta_{u}^{*}}}\right)\frac{e^{\theta_{i}^{*}}}{f(A_{l})}. \tag{3.3}
\]
where $\tau_{i}$ is short for $\tau_{i}(\theta^{*})$ . Since each $(c_{l},A_{l})$ is independent in the fixed graph setting (see Remark 2.1 for discussions), the variance of $J_{i}^{*}$ is
其中 $\tau_{i}$ 是 $\tau_{i}(\theta^{*})$ 的简写。由于在固定图设定下每个 $(c_{l},A_{l})$ 是独立的(讨论见备注 2.1),$J_{i}^{*}$ 的方差为
\[
\begin{array}{r l} & {\mathrm{~\gamma~}_{i}^{*}|\mathcal{G}) = \frac{1}{d^{2}\tau_{i}^{2}}\sum_{l\in \mathcal{D}}\frac{1(i\in A_{l})}{f^{2}(A_{l})}\cdot \mathrm{Var}\bigg(1(c_{l} = i)\sum_{u\in A_{l},u\neq i}e^{\theta_{u}^{*}} - e^{\theta_{i}^{*}}\sum_{u\in A_{l},u\neq i}1(c_{l} = u)\bigg)}\\ & {\quad = \bigg(\sum_{l\in \mathcal{D}}1(i\in A_{l})\frac{(\sum_{u\in A_{l}}e^{\theta_{u}^{*}} - e^{\theta_{i}^{*}})e^{\theta_{i}^{*}}}{f(A_{l})^{2}}\bigg)\bigg / \bigg[\sum_{l\in \mathcal{D}}1(i\in A_{l})\bigg(\frac{\sum_{u\in A_{l}}e^{\theta_{u}^{*}} - e^{\theta_{i}^{*}}}{\sum_{u\in A_{l}}e^{\theta_{u}^{*}}}\bigg)\frac{e^{\theta_{i}^{*}}}{f(A_{l})}\bigg]^{2}} \end{array} \tag{3.5}
\]
A few important comments are in order. Firstly, the function $f$ achieves the minimal variance in (3.5) when $\begin{array}{r}{f(A_{l})\propto \sum_{u\in A_{l}}e^{\theta_{u}^{*}}} \end{array}$ due to simply applying the Cauchy- Schwarz inequality. Actually, Maystre and Grossglauser (2015) showed that when $\begin{array}{r}f(A_l) = \sum_{u\in A_l}e^{\theta_u^*} \end{array}$ , spectral method estimator converges to the MLE up to the first order. Secondly, under the situation of pairwise comparison in the BTL model, each $(c_l,A_l)$ is independent and we assume in $\mathcal{D}$ each pair $(i,j)$ is either compared for $L$ times (denoted as $\widetilde{A}_{ij} = 1$ ) or never compared (denoted as $\widetilde{A}_{ij} = 0$ ). Further assuming $f(A_{l}) = |A_{l}| = 2$ , we have
在此需要做几点重要说明。首先,函数 $f$ 在 $\begin{array}{r}{f(A_{l})\propto \sum_{u\in A_{l}}e^{\theta_{u}^{*}}} \end{array}$ 时能在 (3.5) 式中实现最小方差,这只需应用 Cauchy-Schwarz 不等式即可得证。实际上,Maystre and Grossglauser (2015) 表明当 $\begin{array}{r}f(A_l) = \sum_{u\in A_l}e^{\theta_u^*} \end{array}$ 时,谱方法估计量在一阶收敛的意义上与 MLE 相同。其次,在 BTL 模型的成对比较情形下,每个 $(c_l,A_l)$ 是独立的,并且我们假设在 $\mathcal{D}$ 中每对 $(i,j)$ 要么比较了 $L$ 次(记为 $\widetilde{A}_{ij} = 1$),要么从未比较(记为 $\widetilde{A}_{ij} = 0$)。进一步假设 $f(A_{l}) = |A_{l}| = 2$ ,我们有
$$
\operatorname {Var}(J_i^* |\mathcal{G}) = \frac{1}{L}\bigg(\sum_{j:j\neq i}\widetilde{A}_{ij}e^{\theta_i^*}e^{\theta_j^*}\bigg)\bigg / \bigg[\sum_{j:j\neq i}\widetilde{A}_{ij}\frac{e^{\theta_i^*}e^{\theta_j^*}}{e^{\theta_i^*} + e^{\theta_j^*}}\bigg]^2.
$$
This exactly matches with Proposition 4.2 of Gao et al. (2021). In addition, if we choose $f(A_{l}) =$ $\textstyle \sum_{u\in A_l}e^{\theta_u^*}$ , we get the most efficient variance just like the MLE variance given in Proposition 4.1 of Gao et al. (2021), which is
这与 Gao 等人 (2021) 的命题 4.2 完全一致。此外,如果我们选择 $f(A_{l}) =$ $\textstyle \sum_{u\in A_l}e^{\theta_u^*}$,我们将得到最有效方差,正如 Gao 等人 (2021) 命题 4.1 中给出的 MLE 方差:
$$
\operatorname {Var}(J_i^* |\mathcal{G}) = \left(L\cdot \sum_{j:j\neq i}\widetilde{A}_{ij}\frac{e^{\theta_i^*}e^{\theta_j^*}}{(e^{\theta_i^*} + e^{\theta_j^*})^2}\right)^{-1}.
$$
With the above discussion and computation, after some additional derivations, we come to the conclusion that $\widetilde{\theta}_{i} - \theta_{i}^{*}$ has the same asymptotic distribution as $\frac{\widetilde{\pi}_{i} - \pi_{i}^{*}}{\pi_{i}^{*}}$ and $J_{i}^{*}$ . Therefore,
基于以上讨论和计算,经一些额外推导后,我们得出结论:$\widetilde{\theta}_{i} - \theta_{i}^{*}$ 与 $\frac{\widetilde{\pi}_{i} - \pi_{i}^{*}}{\pi_{i}^{*}}$ 和 $J_{i}^{*}$ 具有相同的渐近分布。因此,
$$
\operatorname {Var}(J_i^* |\mathcal{G})^{-1 / 2}(\widetilde{\theta}_i - \theta_i^*)\Rightarrow N(0,1),
$$
for all $i\in [n]$ . Based on this result, we can make inference for $\widetilde{\theta}_{i}$ and additionally the rank of item $i$ (see Section 3.3). The rigorous derivations for this conclusion will be provided in Section 4.
对所有 $i\in [n]$ 成立。基于此结果,我们可以对 $\widetilde{\theta}_{i}$ 以及项目 $i$ 的排名(见第 3.3 节)进行推断。关于此结论的严格推导将在第 4 节给出。


\subsubsection{PL 模型的不确定性量化}

在本节中,我们考虑随机比较图的情况,这可能导致依赖的 $(c_l,A_l)$,与前一节中固定图的情况不同。请注意,由于随机比较图的生成可以是任意的,我们无法处理每种情况。作为一个说明性例子,我们考虑来自Erdos-Renyi图的经典PL模型,原因有两个。首先,这是排名文献中最受欢迎的随机比较模型(Chen和Suh,2015;Chen等,2022;Han等,2020;Liu等,2022;Gao等,2021;Fan等,2022a,b)。其次,我们可以使用该模型来验证光谱方法的不确定性量化,并将其与MLE进行比较。事实证明,该模型足够好,可以为我们提供新的见解。为了进一步简化讨论和展示,我们将仅关注PL模型中的 $M = 3$。对于一般的 $M$,可以以相同的结论类似地推导结果。

使用PL模型,我们可以写出$J_{i}^{*}$的特定方差。考虑编码三项比较的最自然方式。假设一个人将$(i,j,k)$排名为$i\succ j\succ k$,其中$a\succ b$表示$a$优于$b$。受到似然函数的启发,该函数将从$\{i,j,k\}$中选择$i$作为最佳项的概率与从$\{j,k\}$中选择$j$的概率相乘,我们将这个完整的三项比较分解为两个相关的比较数据:$(i,\{i,j,k\})$和$(j,\{j,k\})$。我们称这种多层次分解为第一层比较所有三项时,$i$被优先选择,而在第二层比较剩余项时,$j$被优先选择。通过这样做,我们可以自然地链接和将结果与MLE估计量进行比较。Azari Soufiani等(2013)还提出了将$M$ - 方式比较分解为成对比较的其他方法,但不同的分解方法将导致不同的依赖结构,我们不打算在本工作中逐一分析。因此,在后续中,我们仅考虑多方式分解,基于似然函数的动机,并将对其他可能的分解方法的研究留到未来。我们使用$\widetilde{A}_{ijk} = 1$或$0$来表示$(i,j,k)$是否已比较$L$次或从未比较。

让我们处理多级比较(multi-level breaking)问题。现在的关键区别在于诱导出的比较图 $\mathcal{G}$ 不能再被视为固定图。相反,我们以 $\widetilde{\mathcal{G}} = \{\widetilde{A}_{ijk}\}$ 为条件。类似于公式 (3.2),我们有

\[
\frac{\bar{\pi}_{i} - \pi_{i}^{*}}{\pi_{i}^{*}} = \frac{\sum_{j:j\neq i}P_{ji}\pi_{j}^{*} - P_{ij}\pi_{i}^{*}}{\pi_{i}^{*}\sum_{j:j\neq i}P_{ij}}\approx \frac{\sum_{j:j\neq i}P_{ji}e^{\theta_{j}^{*}} - P_{ij}e^{\theta_{i}^{*}}}{\sum_{j:j\neq i}E[P_{ij}|\widetilde{\mathcal{G}}]e^{\theta_{i}^{*}}} \eqqcolon J_{i}^{*}. \tag{3.6}
\]

在随机比较图的情况下(即以 $\widetilde{\mathcal{G}}$ 为条件),我们得到

\[
P_{ij} = \frac{1}{d}\sum_{\ell = 1}^{L}\sum_{k:k\neq i,j}\widetilde{A}_{ijk}Z_{ijk}^{\ell}, \tag{3.7}
\]

其中 $Z_{ijk}^{\ell} = 1(y_{k\gamma >j > i}^{(\ell)} = 1) / f(\{i,j\}) + 1(y_{j\gamma >k}^{(\ell)} = 1) / f(\{i,j,k\}) + 1(y_{j\gamma k\gamma i}^{(\ell)} = 1) / f(\{i,j,k\})$。这里 $y_{i_1\gamma i_2\gamma i_3}^{(\ell)}$ 是一个二元变量,当事件 $i_1\searrow i_2\searrow i_3$ 在第 $\ell$ 次 $\{i_1,i_2,i_3\}$ 项的比较中成立时等于 $1$。本质上,我们需要将所有源自同一次比较的项整合为一个项 $Z_{ijk}^{\ell}$,以确保求和始终作用于独立项。

我们稍许滥用了 $J_{i}^{*}$ 的记号(虽然此处的期望是以 $\widetilde{\mathcal{G}}$ 为条件,而非固定图情形中使用的 $\mathcal{G}$)。注意

$$
E[Z_{ijk}^{\ell}|\widetilde{\mathcal{G}} ] = \frac{e^{\theta_{k}^{*}}e^{\theta_{j}^{*}}}{(e^{\theta_{i}^{*}} + e^{\theta_{j}^{*}} + e^{\theta_{k}^{*}})(e^{\theta_{i}^{*}} + e^{\theta_{j}^{*}})f(\{i,j\})} +\frac{e^{\theta_{j}^{*}}}{(e^{\theta_{i}^{*}} + e^{\theta_{j}^{*}} + e^{\theta_{k}^{*}})f(\{i,j,k\})}.
$$

利用 $\sum_{j\neq k}a_{ijk} = \sum_{j< k}(a_{ijk} + a_{ikj})$,$J_{i}^{*}$ 的分母可表示为

$$
\begin{array}{r l} & {\tau_{i}^{\diamond}(\theta^{*})\coloneqq \sum_{j:j\neq i}E[P_{i j}|\widetilde{\mathcal{G}} ]e^{\theta_{i}^{*}} = \frac{L}{d}\sum_{j< k:j,k\neq i}\widetilde{A}_{i j k}e^{\theta_{i}^{*}}\Big(\frac{e^{\theta_{j}^{*}}e^{\theta_{k}^{*}}}{(e^{\theta_{i}^{*}} + e^{\theta_{j}^{*}} + e^{\theta_{k}^{*}})(e^{\theta_{i}^{*}} + e^{\theta_{j}^{*}})f(\{i,j\})}}\\ & {\qquad +\frac{e^{\theta_{j}^{*}}e^{\theta_{k}^{*}}}{(e^{\theta_{i}^{*}} + e^{\theta_{j}^{*}} + e^{\theta_{k}^{*}})(e^{\theta_{i}^{*}} + e^{\theta_{k}^{*}})f(\{i,j,k\})} +\frac{e^{\theta_{j}^{*}} + e^{\theta_{k}^{*}}}{(e^{\theta_{i}^{*}} + e^{\theta_{j}^{*}} + e^{\theta_{k}^{*}})f(\{i,j,k\})}\Big).} \end{array}
$$

因此,$J_{i}^{*}$ 的表达式给定如下: 
\[
\begin{array}{r l} & {\mathcal{I}_{i}^{*} = \frac{1}{\tau_{i}^{\diamond}}\Big(\sum_{j:j\neq i}P_{j i}e^{\theta_{j}^{*}} - P_{i j}e^{\theta_{i}^{*}}\Big) = \frac{1}{d\tau_{i}^{\diamond}}\Big(\sum_{\ell = 1}^{L}\sum_{j:j\neq i}\sum_{k:k\neq i,j}\widetilde{A}_{i j k}(Z_{j i k}^{\ell}e^{\theta_{j}^{*}} - Z_{i j k}^{\ell}e^{\theta_{i}^{*}})\Big)}\\ & {\quad = \frac{1}{d\tau_{i}^{\diamond}}\sum_{\ell = 1}^{L}\sum_{j< k:j,k\neq i}\widetilde{A}_{i j k}(Z_{j i k}^{\ell}e^{\theta_{j}^{*}} + Z_{k i j}^{\ell}e^{\theta_{k}^{*}} - Z_{i j k}^{\ell}e^{\theta_{i}^{*}} - Z_{i k j}^{\ell}e^{\theta_{i}^{*}}) = \frac{1}{d}\sum_{\ell = 1}^{L}\sum_{j< k:j,k\neq i}J_{i j k\ell}(\theta^{*}),} \end{array} \tag{3.8}
\]
其中 $\tau_{i}^{\diamond}$ 是 $\tau_{i}^{\diamond}(\theta^{*})$ 的简写。由于每次三路比较独立,可证 $J_{i}^{*}$ 的方差为

$$
\widetilde{J}_{i}^{\diamond}(\widetilde{J}) = \frac{L}{d^{2}(\tau_{i}^{\diamond})^{2}}\sum_{j< k:j,k\neq i}\widetilde{A}_{ijk}e^{\theta_{i}^{*}}\Big(\frac{(e^{\theta_{j}^{*}} + e^{\theta_{k}^{*}})}{f^{2}(\{i,j,k\})} +\frac{e^{\theta_{j}^{*}}e^{\theta_{k}^{*}}}{e^{\theta_{i}^{*}} + e^{\theta_{j}^{*}} + e^{\theta_{k}^{*}}}\Big(\frac{1}{f^{2}(\{i,k\})} +\frac{1}{f^{2}(\{i,j,k\})}\Big)\Big)
$$

核心部分需计算 $EJ_{ijkl}(\theta^{*})^{2}$(源于独立性和零均值特性)。对给定三元组 $(i,j,k)$,存在6种可能的偏好结果(其概率由PL模型决定)。将平方随机结果对6种概率取平均即得 $EJ_{ijkl}(\theta^{*})^{2}$,从而导出上述表达式。我们省略这些计算的细节。

考虑所有 $\theta_{i}^{*}$ 相等的简单情况。此时若采用最优加权函数 $f(A_{l})\propto \sum_{u\in A_{l}}e^{\theta_{u}^{*}}$(即 $f(\{i,j,k\}) = 3$,$f(\{i,j\}) = f(\{i,k\}) = 2$),则有 $\mathrm{Var}(J_i^* |\widetilde{G}) = 18 / (7L)$。然而若朴素选择常数函数 $f$,则得 $\mathrm{Var}(J_i^* |\widetilde{G}) = 8 / (3L)$(这显然更大)。值得注意的是:当选择 $f(A_{l})\propto \sum_{u\in A_{l}}e^{\theta_{u}^{*}}$ 时,前述方差与Fan等人(2022b)中MLE的方差一致。

基于上述PL模型的 $\mathrm{Var}(J_i^* |\widetilde{G})$ 公式,我们可进一步得出结论:
$$
\mathrm{Var}(J_i^* |\widetilde{G})^{-1 / 2}(\widetilde{\theta}_i - \theta_i^*)\Rightarrow N(0,1),
$$
对所有 $i\in [n]$ 成立。严格证明将在第4节给出。

\subsubsection{排序推断:单样本置信区间}

在许多实际应用中,个体频繁接触涉及排名的数据与挑战。利用排名的常规方法通常围绕计算偏好得分,然后按排名顺序展示这些分数。这些方法仅提供排名的一阶信息,无法回答许多问题,例如:

如何以高置信度确定某物品的真实排名属于前 3 位(或一般地,$K, K \geq 1$)?以及,如何建立一个高置信度的候选集,确保真实的前 3 名候选不被遗漏?

如何分析给定产品阵列的排名偏好在两个不同群体(如男性和女性)中或同一群体在两个不同时间段的偏好是否一致?

总之,在涉及排名的现实应用中,特别是在比较数据来自一般比较图的情况下,需要有针对这些及其他深刻问题的工具和方法。

在本节中,我们首先提出一个为构建排名双边置信区间的综合框架。在尝试为排名建立同时置信区间时,一种直观的方法是从推导经验排名(记为 $\widetilde{r}_{m}, m\in \mathcal{M}$)的渐近分布入手,进而确定临界值。然而,众所周知,这项任务面临实质性挑战,因为 $\widetilde{r}_{m}$ 是整数且本质上依赖于所有估计得分,使得其渐近行为分析异常困难。

通过利用得分与其对应排名之间的内在联系,我们认识到构造排名的置信区间问题,可以有效地转化为构造总体得分间两两差异的同时置信区间问题。值得注意,这些经验得分差异的分布更易于描述。因此,我们聚焦于估计得分 $\widetilde{\theta}_{m} \in [n]$ 的统计特性,并提出通过估计的得分差异来构造排名的双边(同时)置信区间的方法。

**例 3.1.** 令 $\mathcal{M} = \{m\}$(其中 $1 \leq m \leq n$)代表待考虑的物品。我们关注为真实总体排名 $r_{m}$ 构建 $(1 - \alpha) \times 100\%$ 置信区间,其中 $\alpha \in (0,1)$ 表示预先设定的显著性水平。假设我们能够为两两差异 $\theta_{k}^{*} - \theta_{m}^{*}, k \neq m (k \in [n])$ 构造同时置信区间 $[\mathcal{C}_L(k,m), \mathcal{C}_U(k,m)], k \neq m, (k \in [n])$,使其满足以下性质:
\[
\mathbb{P}\Big(\mathcal{C}_L(k,m) \leq \theta_k^* - \theta_m^* \leq \mathcal{C}_U(k,m) \text{ 对所有 } k \neq m\Big) \geq 1 - \alpha . \tag{3.9}
\]
可以观察到,若 $\mathcal{C}_U(k,m) < 0$(相应地,$\mathcal{C}_L(k,m) > 0$),则意味着 $\theta_k^* < \theta_m^*$(相应地,$\theta_k^* > \theta_m^*$)。枚举得分高于物品 $m$ 的物品数量给出了排名 $r_{m}$ 的下界,反之亦然。换言之,我们从 (3.9) 推导出:
\[
\mathbb{P}\left(1 + \sum_{k \neq m} 1 \{C_L(k,m) > 0\} \leq r_m \leq n - \sum_{k \neq m} 1 \{C_U(k,m) < 0\}\right) \geq 1 - \alpha . \tag{3.10}
\]
这就产生了一个针对 $r_{m}$ 的 $(1 - \alpha) \times 100\%$ 置信区间,而我们的任务则简化为构造两两差异 (3.9) 的同时置信区间。

我们现在正式介绍同时为多个排名 $\{r_m\}_{m \in \mathcal{M}}$ 构造置信区间的过程。受例 3.1 启发,关键步骤是为两两得分差异 $\{\theta_k^* - \theta_m^*\}_{m \in \mathcal{M}, k \neq m}$ 构造满足 (3.9) 式的同时置信区间。为此,我们令
\[
T_{\mathcal{M}} = \max_{m \in \mathcal{M}} \max_{k \neq m} \left| \frac{\widetilde{\theta}_k - \widetilde{\theta}_m - (\theta_k^* - \theta_m^*)}{\widetilde{\sigma}_{km}} \right|, \tag{3.11}
\]
其中 $\{\widetilde{\sigma}_{km}\}_{k \neq m}$ 是由下文 (3.13) 式给出的正数标准化序列。对于任意 $\alpha \in (0,1)$,令 $Q_{1 - \alpha}$ 为满足 $\mathbb{P}(T_{\mathcal{M}} \leq Q_{1 - \alpha}) \geq 1 - \alpha$ 的临界值。那么,如例 3.1 所示,我们为 $\{r_m\}_{m \in \mathcal{M}}$ 提供的 $(1 - \alpha) \times 100\%$ 同时置信区间为 $\{[R_{mL}, R_{mU}] \}_{m \in \mathcal{M}}$,其中
$$
R_{mL} = 1 + \sum_{k \neq m} 1 \left(\widetilde{\theta}_k - \widetilde{\theta}_m > \widetilde{\sigma}_{km} \times Q_{1 - \alpha}\right), \quad R_{mU} = n - \sum_{k \neq m} 1 \left(\widetilde{\theta}_k - \widetilde{\theta}_m < -\widetilde{\sigma}_{km} \times Q_{1 - \alpha}\right).
$$

\subsubsection{多步 bootstrap 程序}

构建目标排名置信区间的关键步骤是选取临界值 $Q_{1 - \alpha}$。为计算该临界值,我们提出使用自助法(wild bootstrap procedure)。谱估计量(3.2)的不确定性量化表明,在 $i\in [n]$ 上一致成立 $\widetilde{\theta_{i}} - \theta_{i}^{*}\approx J_{i}(\theta^{*})$(详见第4节),这意味着渐近情况下有:

\[
T_{\mathcal{M}}\approx \max_{m\in \mathcal{M}}\max_{k\neq m}\left|\frac{J_{k}(\theta^{*}) - J_{m}(\theta^{*})}{\widetilde{\sigma}_{km}}\right|. \tag{3.12}
\]

本文聚焦固定图设定,随机图设定详见下文\textbf{注记3.2}。实践中,$J_{i}(\theta^{*})$ 的经验版本可通过代入谱估计量 $\widetilde{\theta}$ 获得,即由(3.4)式得:

$$
J_{i}(\widetilde{\theta}) = \frac{1}{d}\sum_{l\in \mathcal{D}}J_{il}(\widetilde{\theta}), i\in [n].
$$

令 $\sigma_{km}^{2} = \mathrm{Var}\{J_{k}(\theta^{*}) - J_{m}(\theta^{*})|\mathcal{G}\}$($k\neq m$)。则 $\sigma_{km}^{2}$ 的估计量定义为:

\[
\widetilde{\sigma}_{km}^{2} = \frac{e^{\widetilde{\theta}_{k}}}{d^{2}\tau_{k}^{2}(\widetilde{\theta})}\sum_{l\in \mathcal{D}}\frac{1(k\in A_{l})}{f^{2}(A_{l})}\left(\sum_{j\in A_{l}}e^{\widetilde{\theta}_{j}} - e^{\widetilde{\theta}_{k}}\right) + \frac{e^{\widetilde{\theta}_{m}}}{d^{2}\tau_{m}^{2}(\widetilde{\theta})}\sum_{l\in \mathcal{D}}\frac{1(m\in A_{l})}{f^{2}(A_{l})}\left(\sum_{j\in A_{l}}e^{\widetilde{\theta}_{j}} - e^{\widetilde{\theta}_{m}}\right) \tag{3.13}
\]

其中 $\tau_{k}(\widetilde{\theta})$ 和 $\tau_{m}(\widetilde{\theta})$ 同样代入 $\widetilde{\theta}$(见(3.5)式)。设 $\omega_{1},\ldots ,\omega_{|\mathcal{D}|}\in \mathbb{R}$ 为独立同分布的 $N(0,1)$ 随机变量,则高斯乘自助统计量定义为:

\[
G_{\mathcal{M}} = \max_{m\in \mathcal{M}}\max_{k\neq m}\left|\frac{1}{d\widetilde{\sigma}_{km}}\sum_{l\in \mathcal{D}}\{J_{kl}(\widetilde{\theta}) - J_{ml}(\widetilde{\theta})\} \omega_{l}\right|. \tag{3.14}
\]

令 $\mathbb{P}^{*}(\cdot) = \mathbb{P}(\cdot | [(\sigma_{i},A_{i})]_{l\in \mathcal{D}})$ 表示条件概率。则对 $\alpha \in (0,1)$,$Q_{1 - \alpha}$ 的估计量定义为 $G_{\mathcal{M}}$ 的 $(1 - \alpha)$ 分位条件量,即:
$$
\mathcal{Q}_{1 - \alpha} = \inf \{z:\mathbb{P}^{*}(G_{\mathcal{M}}\leq z)\geq 1 - \alpha \} ,
$$

可通过蒙特卡洛模拟计算。则同步置信区间 $\{[\mathcal{R}_{mL},\mathcal{R}_{mU}]\}_{m\in \mathcal{M}}$ 由下式给出:
\[
\mathcal{R}_{mL} = 1 + \sum_{k\neq m}1\left(\widetilde{\theta}_{k} - \widetilde{\theta}_{m} > \widetilde{\sigma}_{km}\times \mathcal{Q}_{1 - \alpha}\right),\qquad \mathcal{R}_{mU} = n - \sum_{k\neq m}1\left(\widetilde{\theta}_{k} - \widetilde{\theta}_{m}< -\widetilde{\sigma}_{km}\times \mathcal{Q}_{1 - \alpha}\right). \tag{3.15}
\]

\textbf{注记 3.1} (单样本单侧置信区间). 现详述构建总体排名的同步单侧区间。单侧区间构建流程与双侧置信区间类似。具体而言,令

\[
G_{\mathcal{M}}^{\circ} = \max_{m\in \mathcal{M}}\max_{k\neq m}\frac{1}{d\widetilde{\sigma}_{km}}\sum_{l\in \mathcal{D}}\{J_{kl}(\widetilde{\theta}) - J_{ml}(\widetilde{\theta})\} \omega_{l}, \tag{3.16}
\]

其中 $\omega_{1},\ldots ,\omega_{|\mathcal{D}|}$ 为前述独立同分布 $N(0,1)$ 随机变量。相应地,令 $\mathcal{Q}_{1 - \alpha}^{\circ}$ 为其 $(1 - \alpha)$ 分位数。则 $\{r_{m}\}_{m\in \mathcal{M}}$ 的 $(1 - \alpha)\times 100\%$ 同步置信下界为 $\{[\mathcal{R}_{mL}^{o},n]\}_{m\in \mathcal{M}}$,其中

\[
\mathcal{R}_{mL}^{\circ} = 1 + \sum_{k\neq m}1\left(\widetilde{\theta}_{k} - \widetilde{\theta}_{m} > \widetilde{\sigma}_{km}\times \mathcal{Q}_{1 - \alpha}^{\circ}\right). \tag{3.17}
\]

\textbf{注记 3.2} (随机比较图的PL模型排名推断). 第3.2节表明在 $i \in [n]$ 上一致成立 $\widetilde{\theta}_{i} - \theta_{i}^{*} \approx J_{i}(\theta^{*})$,其中根据(3.8)式:

$$
J_{i}(\theta^{*}) = \frac{1}{d} \sum_{\ell = 1}^{L} \sum_{j < s:j, s \neq i} J_{ijs\ell}(\theta^{*}).
$$

为实现PL模型排名推断,需改写此式为:令 $\mathcal{N} = \sum_{i< j< k}\widetilde{A}_{i j k}$ 表示随机图 $\widetilde{G}$ 上连通分量总数,记 $\{(i,j,k):i< j< k$ 且 $\widetilde{A}_{i j k} = 1\} \eqqcolon \{\widetilde{A}_{q}\}_{q = 1,\ldots ,\mathcal{N}}$。令 $y_{q}^{(\ell)}$ 表示 $A_{q}$ 的第 $\ell$ 次全排序比较结果,则可改写 $P_{ij}$ 为:

$$
P_{ij} = \frac{1}{d} \sum_{\ell = 1}^{L} \sum_{q = 1}^{N} \sum_{k \neq i, j} 1\{(i,j,k) = \widetilde{A}_{q}\} Z_{ijkq}^{(\ell)}, i \neq j,
$$

其中当 $i\neq j\neq k$ 且 $q\in \lfloor \mathcal{N}\rfloor$ 时,$Z_{i j k q}^{(\ell)} = \dfrac{1\{y_{q}^{(\ell)} = (k\cdot \gamma \cdot j\cdot \gamma \cdot i)\}}{f(\{i,j\})} + \dfrac{1\{y_{q}^{(\ell)} = (j\cdot \gamma \cdot i)\}}{f(\{i,j,k\})} + \dfrac{1\{y_{q}^{(\ell)} = (j\cdot k\cdot \gamma \cdot i)\}}{f(\{i,j,k\})}$。易证此 $P_{ij}$ 与(3.7)式等价。因此改写 $J_{i}(\theta^{*}) = d^{- 1}\sum_{\ell = 1}^{L}\sum_{q = 1}^{\mathcal{N}}J_{i q\ell}^{\diamond}(\theta^{*})$,其中

\[
J_{iq\ell}^{\diamond}(\theta^{*}) = \sum_{j < s:j, s \neq i} 1\{(i,j,s) = \widetilde{A}_{q}\} J_{iq\ell}^{\diamond}(\theta^{*}). \tag{3.18}
\]

依假设,在给定比较图 $\widetilde{G}$ 时,$\{J_{iq\ell}^{\diamond}(\theta^{*})\}_{\ell \in [L], q \in [\mathcal{N}]}$ 对每个 $i \in [n]$ 独立。设 $\{\omega_{q\ell}^{\diamond}\}_{q, \ell \in \mathbb{N}}$ 为独立同分布 $N(0,1)$ 随机变量,则仿照(3.14)式,相应自助检验统计量为:

$$
G_{\mathcal{M}}^{\diamond} = \max_{m \in \mathcal{M}} \max_{k \neq m} \left| \frac{1}{d \sigma_{km}^{\diamond}} \sum_{\ell = 1}^{L} \sum_{q = 1}^{\mathcal{N}} \{J_{kq\ell}^{\diamond}(\widetilde{\theta}) - J_{mq\ell}^{\diamond}(\widetilde{\theta})\} \omega_{q\ell} \right|,
$$

其中 $\{\widetilde{\sigma}_{km}^{\diamond}\}_{k \neq m}$ 为前述正标准化序列,其计算方式类似于(3.13)式中的 $J_{k}(\theta^{*})$ 与 $J_{m}(\theta^{*})$ 方差之和。由此可类似构建排名的同步置信区间。


\subsubsection{排序推断:双样本和单样本检验应用}

在本节中,我们将进一步说明如何将推断方法应用于单样本和双样本的几种显著性检验场景。

\textbf{示例 3.2}(前 $K$ 名位置检验)。设 $\mathcal{M} = \{m\}$($m \in [n]$),并令 $K \geq 1$ 为给定正整数。我们的目标是确定项目 $m$ 是否属于排名前 $K$ 的项目。因此,需检验假设:

\[
H_{0}:r_{m}\leq K\mathrm{~vs~}H_{1}:r_{m} > K. \tag{3.19}
\]

基于(3.17)式的单侧置信区间 $[\mathcal{R}_{mL}^{\circ}, n]$,对于任意 $\alpha \in (0,1)$,假设 (3.19) 的水平为 $\alpha$ 的检验可直接定义为 $\phi_{m,K} = 1\{\mathcal{R}_{mL}^{\circ} > K\}$。在定理 E.1 的条件下,有 $\mathbb{P}(\phi_{m,K} = 1|H_{0}) \leq \alpha + \mathrm{o}(1)$,即当零假设成立时,I 类错误率可有效控制在显著性水平 $\alpha$ 以下。

\textbf{示例 3.3}(前 $K$ 名确信筛选集)。另一应用是构建高概率包含前 $K$ 名项目的候选筛选集。这在高校招生或企业招聘决策中尤为重要。通常,高校或企业希望设计一种招生或招聘策略,确保能以高概率筛选出真实的前 $K$ 名候选人。

令 $\mathcal{K} = \{r^{- 1}(1),\ldots ,r^{- 1}(K)\}$ 表示排名算子 $r:[n]\to [n]$ 的前 $K$ 名项目。我们的目标是选择候选集合 $\widehat{\mathcal{I}}_K$,使其以给定概率包含前 $K$ 名候选人。数学上可表述为 $\mathbb{P}(\mathcal{K}\subseteq \widehat{\mathcal{I}}_K)\geq 1 - \alpha$($\alpha \in (0,1)$)。此处定义 $\mathcal{M} = [n]$,并令 $\{[\mathcal{R}_{mL}^{\circ},n],m\in [n]\}$ 表示 (3.17) 给出的 $(1 - \alpha)\times 100\%$ 同步左侧置信区间集。易证不等式 $\mathcal{R}_{mL}^{\circ} > K$ 可推断 $r_m > K$。因此,满足概率约束 $\mathbb{P}(\mathcal{K}\subseteq \widehat{\mathcal{I}}_K)\geq 1 - \alpha$ 的筛选集为:

$$
\widehat{\mathcal{I}}_K = \{m\in [n]:\mathcal{R}_{mL}^{\circ}\leq K\} .
$$

\textbf{示例 3.4}(双样本排名检验)。许多应用中需关注特定项目的排名在两个样本间是否发生变化。例如:
- 某项处理或政策变化前后排名分布是否不同。
- 不同群体(如男性 vs 女性)对同一产品集的排名偏好是否存在差异。
- 两个时期人们对同一事物的感知偏好是否改变。

假设观测到两个独立数据集 $\mathcal{D}_1$ 和 $\mathcal{D}_2$,其偏好得分分别为 $\theta_{[1]}^{*} = (\theta_{11}^{*},\ldots ,\theta_{1n}^{*})^{\top}$ 和 $\theta_{[2]}^{*} = (\theta_{21}^{*},\ldots ,\theta_{2n}^{*})^{\top}$。对应的真实排名为:

$$
r_{[1]} = (r_{11},\ldots ,r_{1n})^{\top} \text{ 和 } r_{[2]} = (r_{21},\ldots ,r_{2n})^{\top}.
$$

给定 $m\in [n]$,检验项目 $m$ 在两个样本中排名是否一致,即检验假设:

\[
H_{0}:r_{1m} = r_{2m} \text{ vs } H_{1}:r_{1m}\neq r_{2m}. \tag{3.20}
\]

为此,首先构建满足下式的同步置信区间 $[R_{1mL},R_{1mU}]$ 和 $[R_{2mL},R_{2mU}]$:

\[
\mathbb{P}(r_{1m}\in [R_{1mL},R_{1mU}]\text{ 且 } r_{2m}\in [R_{2mL},R_{2mU}])\geq 1 - \alpha . \tag{3.21}
\]

则 (3.20) 的水平为 $\alpha$ 的检验定义为:

$$
\phi_{m} = 1\{\|[R_{1mL},R_{1mU}]\cap [R_{2mL},R_{2mU}]\| = 0\} .
$$

易证 $\mathbb{P}(\phi_{m} = 1|H_{0})\geq 1 - \alpha$。

\textbf{示例 3.5}(双样本前 $K$ 名集合检验)。除单项目检验外,常需评估两个群体的前 $K$ 名集合是否相同(如两个时期、重大事件前后)。令 $\mathcal{S}_{1K} = \{r_{[1]}^{- 1}(1),\ldots ,r_{[1]}^{- 1}(K)\}$ 和 $\mathcal{S}_{2K} = \{r_{[2]}^{- 1}(1),\ldots ,r_{[2]}^{- 1}(K)\}$ 分别表示两组的前 $K$ 名项目集。检验假设:
\[
H_{0}:S_{1K} = S_{2K}\mathrm{~vs~}H_{1}:S_{1K}\neq S_{2K}. \tag{3.22}
\]

对于 $\alpha \in (0,1)$,先构建 $S_{1K}$ 和 $S_{2K}$ 的 $(1 - \alpha)\times 100\%$ 同步置信集 $\widehat{\mathcal{I}}_{1K}$ 和 $\widehat{\mathcal{I}}_{2K}$ 满足:

\[
\mathbb{P}\left(S_{1K}\subset \widehat{\mathcal{I}}_{1K}\mathrm{~且~}S_{2K}\subset \widehat{\mathcal{I}}_{2K}\right)\geq 1 - \alpha . \tag{3.23}
\]

则 (3.22) 的水平为 $\alpha$ 的检验定义为:

$$
\widetilde{\phi}_{K} = 1\{|\widehat{\mathcal{I}}_{1K}\cap \widehat{\mathcal{I}}_{2K}|< K\} .
$$

\textbf{注记 3.3}。多种方法可实现 (3.21) 和 (3.23) 的同步置信区间,包括 Bonferroni 校正法(为每样本构建 $(1 - \alpha /2)\times 100\%$ 置信区间)和高斯近似法(取各样本检验统计量的最大值)。为简化后续表述,本文对双样本直接采用 Bonferroni 校正。此外,示例 3.4 和 3.5 的框架可直接推广至评估三个或更多来源的项目排名或集合是否一致。

\subsection{理论验证}

在本小节中,我们将严格证明第3节中的结论,并明确阐述得出这些结论所需的假设。第一个假设旨在确保我们以一种有意义的方式按相同顺序比较 $\theta_{i}^{*}$。否则,我们总可以将项目归类到质量相近的类别中,然后分别处理每个子组,或者筛除一些极端项目。此外,正如我们已经讨论过的,我们需要 $\theta^{*}$ 的一个可识别性条件。

假设 4.1. 存在某个正常数 $\bar{\kappa} < \infty$,使得

$$
\max_{i\in [n]}\theta_{i}^{*} - \min_{i\in [n]}\theta_{i}^{*}\leq \bar{\kappa}.
$$

同时,为了可识别性,不妨设 $1^{\top}\theta^{*} = 0$。

在假设 4.1 中,我们假设 $\bar{\kappa}$ 是有限的,这表明我们只对偏好评分处于同一量级上的项目进行排序。如果 $\bar{\kappa}$ 是发散的,那么某些项目显然比其他项目更受欢迎或更不受欢迎。在这种情况下,实践中通常很容易将项目分成具有相似偏好评分的子组,然后我们可以在每个组内进行排序推断。尽管我们假设了有界的 $\bar{\kappa}$,但它充当了条件数的角色,其影响已在我们所有结果中为感兴趣的读者明确说明。然而,我们并不声称这种依赖性是最优的,因为我们的非平凡分析很容易遇到 $e^{\bar{\kappa}}$ 的幂次,比如在估计 $\pi_{i}^{*} / \pi_{j}^{*}$ 的比值时。


\subsubsection{固定比较的估计准确性和渐近正态性}

为推导谱估计量的渐近分布,我们需要严格证明近似式 (3.1) 和 (3.2)。首先处理利用 (3.2) 近似 (3.1),其中假设 $\mathcal{G}$ 中的所有比较关系固定。注意到

$$
P_{ij} - E[P_{ij}|\mathcal{G}] = \frac{1}{d}\sum_{l\in \mathcal{D}}1(i,j\in A_l)\left[1(c_l = j) - \frac{\pi_j^*}{\sum_{u\in A_l}\pi_u^*}\right]\frac{1}{f(A_l)}.
$$

令 $Z_{A_l}^j = 1(c_l = j) / f(A_l)$ ,当 $f$ 有界时该量上下有界。此外,每个 $Z_{A_l}^j$ 相互独立。因此 $P_{ij} - E[P_{ij}|\mathcal{G}] = d^{-1}\sum_{l\in \mathcal{D}}1(i,j\in A_l)[Z_{A_l}^j - E(Z_{A_l}^j)]$ 。由 Hoeffding 不等式,在 $\mathcal{G}$ 条件下,以 $1 - o(1)$ 的高概率成立:

$$
\max_{i\neq j}\left|P_{ij} - E[P_{ij}|\mathcal{G}]\right|\lesssim \frac{1}{d}\sqrt{(\log n)n^{\ddagger}}.
$$

其中 $\begin{array}{r}{n^{\ddagger} = \max_{i\neq j}\sum_{l\in \mathcal{D}}1(i,j\in A_{l})} \end{array}$ 为每对项目比较次数的最大值。类似地可得 $\sum_{j:j\neq i}P_{ij}$ 的集中界。由于 $Z_{A_{l}}^{j}$ 相互独立,对 $j$ 的额外求和导出下式。再次应用 Hoeffding 不等式,以趋近于 1 的高概率得:

$$
\max_{i}\left|\sum_{j:j\neq i}P_{ij} - \sum_{j:j\neq i}E[P_{ij}|\mathcal{G}]\right|\lesssim \frac{1}{d}\sqrt{(\log n)n^{\dagger}},
$$

其中 $\begin{array}{r}{n^{\dagger} = \max_{i}\sum_{l\in \mathcal{D}}1(i\in A_{l})} \end{array}$ 为每项项目比较次数的最大值。同时假设

$$
\sum_{j:j\neq i}E[P_{ij}|\mathcal{G}] = \tau_{i}e^{-\theta_{i}^{*}}\asymp \frac{1}{d} n^{\dagger},
$$

此处在 (3.3) 定义的 $\tau_{i}$ 是 $J_{i}^{*}$ 的分母。该假设合理,因 $\tau_{i}e^{- \theta_{i}^{*}}\lesssim \sum_{l\in \mathcal{D}}1(i\in A_{l}) / d$ ,且表明对每个 $i$ ,比较图不能过于不对称。注意不同项目对 $(i,j)$ 的 $\sum_{l\in \mathcal{D}}1(i,j\in A_{l})$ 仍可能与 $n^{\ddagger}$ 差异显著。若 $n^{\dagger}\gtrsim \log n$ ,期望项主导偏差,易证在 (3.2) 中将分母替换为其期望仅产生微小阶差,不影响渐近分布。

基于上述讨论,我们提出以下假设。

假设 4.2. 对固定比较图,假设图连通,且对所有 $i\in [n]$ 满足 $\tau_{i}e^{- \theta_{i}^{*}}\asymp n^{\dagger} / d$ ,同时 $e^{2\bar{\kappa}}\log n = o(n)$ 且 $e^{3\bar{\kappa}}n^{\ddagger}n^{1 / 2}(\log n)^{1 / 2} = o(n^{\dagger})$ 。

此假设对固定比较图合理。若每对 $(i,j)$ 至少比较一次,则 $\sum_{l\in \mathcal{D}}1(i,j\in A_{l})\geq 1$ 。若所有比较同阶,则 $\sum_{l\in \mathcal{D}}1(i\in A_{l}) = \sum_{j:j\neq i}\sum_{l\in \mathcal{D}}1(i,j\in A_{l})$ 确为 $n^{\ddagger}n$ 阶。假设 4.2 允许部分 $(i,j)$ 对无直接比较,故需利用 $i$ 和 $j$ 分别与其他项目比较的信息。此外,由于仅要求最大配对比较数 $n^{\ddagger}$ 满足假设 4.2,不要求任意 $i,j:i\neq j$ 的 $\sum_{l\in \mathcal{D}}1(i,j\in A_l)$ 同阶。但对固定图,因缺乏图的随机性,需足够稠密以保证排序信息充分。在 4.2 节的齐次随机比较图情形下,此条件可放宽至 $n^{\dagger}\gtrsim n^{\ddagger}\log n$ 。

需另一关于比较图结构的技术条件。定义 $\Omega = \{\Omega_{ij}\}_{i\leq n,j\leq n}$ ,其中 $i\neq j$ 时 $\Omega_{ij} = - P_{ji}\pi_j^*$ ,$\Omega_{ii} = \sum_{j:j\neq i}P_{ij}\pi_i^*$ 。注意到前文推导有 $E[\Omega_{ii}|\mathcal{G}]\asymp n^{\dagger}/(dn)$ 。我们期望厘清其特征值阶数。因 $\Omega$ 最小特征值为零(对应特征向量 $\mathbf{1}$),我们仅关注 $\mathbf{1}$ 的正交补空间。沿用 Gao 等 (2021) 记号,

$$
\lambda_{\min ,\bot}(A) = \min_{\| v\| = 1,v^{\top}\mathbf{1} = 0}v^{\top}A v.
$$

假设 4.3. 存在 $C_1,C_2 > 0$ 使得

\[
C_1e^{-\bar{\kappa}}\frac{n^\dagger}{dn}\leq \lambda_{\min ,\bot}(E[\Omega |\mathcal{G}])\leq \lambda_{\max}(E[\Omega |\mathcal{G}])\leq C_2e^{\bar{\kappa}}\frac{n^\dagger}{dn}, \tag{4.1}
\]

\[
\| \Omega -E[\Omega |\mathcal{G}]\| = o_{P}\left(\frac{n^{\dagger}}{dn}\right). \tag{4.2}
\]

当 $\bar{\kappa} = O(1)$ 时,假设 4.3 要求 $E[\Omega |\mathcal{G}]$ 的所有特征值(除最小者外)均为 $n^{\dagger}/(dn)$ 阶,且 $\Omega$ 与 $E[\Omega |\mathcal{G}]$ 同阶。该假设直观合理,因 $i\neq j$ 时 $E[\Omega_{ij}]\lesssim n^{\ddagger}/(dn)$ ,而 $E[\Omega_{ii}]\asymp n^{\dagger}/(dn)$ 。后文将严格证明该条件在 PL 模型下成立(定理 4.3)。

定理 4.1. 在假设 4.1-4.3 下,谱估计量 $\widetilde{\theta_{i}}$ 满足以下一致逼近:对所有 $i\in [n]$ 一致地有 $\widetilde{\theta}_{i} - \theta_{i}^{*} = J_{i}^{*} + \delta_{i}$ ,其中 $\| \delta \coloneqq (\delta_{1},\dots ,\delta_{n})\|_{\infty} = o(1 / \sqrt{n^{\dagger}})$ 的概率为 $1 - o(1)$ 。

为证定理 4.1,需验证 (3.1),详细证明见附录。由定理 4.1 结合 $J_{i}^{*}$ 的性质,易得下述定理,给出 $\widetilde{\theta}$ 的收敛速度及渐近正态性。

注 4.1. 定理 4.1 及后续结果均通过 Bernstein 与 Hoeffding 型不等式及 $n$ 个项目的并界证明。因此所有高概率项(主文中的 $o_{p}(\cdot)$ 和 $O_{p}(\cdot)$ 同理)均以 $1 - O(n^{-\zeta})$ 形式成立,其中 $\zeta \geq 2$ 为正整数(不同 $\zeta$ 选择仅影响集中不等式中的常数项)。

定理 4.2. 在假设 4.1-4.3 下,谱估计量 (2.1) 满足

\[
\| \widetilde{\theta} -\theta^{*}\|_{\infty}\asymp \| J^{*}\|_{\infty}\lesssim e^{\bar{\kappa}}\sqrt{\frac{\log n}{n^{\dagger}}}, \tag{4.3}
\]

概率为 $1 - o(1)$ ,其中 $J^{*} = (J_{1}^{*},\dots ,J_{n}^{*})$ 且 $J_{i}^{*}$ 由 (3.4) 定义。此外,

$$
\rho_{i}(\theta)(\widetilde{\theta}_{i} - \theta_{i}^{*})\Rightarrow N(0,1),
$$

对所有 $i\in [n]$ 成立,其中

$$
\begin{array}{r}{\mathbf{\Phi}_{i}(\theta) = \Big[\sum_{l\in \mathcal{D}}1(i\in A_{l})\Big(\frac{\sum_{u\in A_{l}}e^{\theta_{u}} - e^{\theta_{i}}}{\sum_{u\in A_{l}}e^{\theta_{u}}}\Big)\frac{e^{\theta_{i}}}{f(A_{l})}\Big] \Big/ \Big[\sum_{l\in \mathcal{D}}1(i\in A_{l})\Big(\frac{\sum_{u\in A_{l}}e^{\theta_{u}} - e^{\theta_{i}}}{f(A_{l})}\Big)\frac{e^{\theta_{i}}}{f(A_{l})}\Big]^{1 / 2}} \end{array}
$$

对 $\theta = \theta^{*}$ 及 $\theta =$ $\theta^{*}$ 的任意相合估计均适用。

注意定理 4.2 表明 $f(\cdot) > 0$ 的选择不影响收敛速度,但影响估计效率。如 3.1 节所述,在谱估计类中最小化渐近方差的加权最优选择为 $f(A_{l})\propto \sum_{u\in A_{l}}e^{\theta_{u}^{*}}$ 。但实际中 $\theta_{u}^{*}$ 未知,故可采用两阶段方法提升谱估计效率:第一阶段以 $f(A_{l}) = |A_{l}|$ 等加权获得初始相合估计 $\widehat{\theta_{u}^{(\mathrm{initial})}}$ ;第二阶段以 $\widehat{\theta_{u}^{(\mathrm{initial})}}$ 代入估计 $f(A_{l}) = \sum_{u\in A_{l}}e^{\theta_{u}^{*}}$ ,并使用此最优加权再次运行谱方法得最终渐近有效估计 $\widehat{\theta_{u}^{(\mathrm{final})}}$ 。由于第二阶段加权数据依赖性导致非独立同分布排序结果,此时一致逼近分析高度复杂,故不拟证明此两阶段估计的理论性质。但可通过划分数据规避困难:以微量样本 ($o(|\mathcal{D}|)$) 作第一步(获得具较差收敛速度的相合估计),剩余主要样本 ($|\mathcal{D}| - o(|\mathcal{D}|)$) 作第二步,以保持相同渐近行为。实证表明大样本下两阶段使用全体数据可获得良好性能,详参数值研究部分。


\subsubsection{PL 模型的估计准确性和渐近正态性}

在随机图情形下,需明确图生成过程以研究理论性质。我们考虑常用的 PL 模型:以概率 $p$ 抽取每组 $M$ 元比较,并对该组进行 $L$ 次比较。此外,为聚焦于透明直观的讨论,将仅处理 $M = 3$ 情形。所有讨论均可推广至一般 $M$,但推导与公式会更为繁琐。

Fan 等 (2022b) 通过 MLE 研究了含三元比较的 PL 模型,其中显式写出了似然函数。所提出的谱方法适用于任意固定图,包括由 PL 模型生成的图。本节拟比较谱方法与 MLE 的性能。为确保谱方法在 PL 模型中适用,需证明近似式 (3.1) 与 (3.6)。

首先处理 (3.6)。考虑以 $\widetilde{\mathcal{G}}$ 为条件,其中 $\widetilde{\mathcal{G}}$ 中所有比较独立;若 $\widetilde{A}_{ijk} = 1$,则 $\widetilde{A}_{ijk}$ 比较 $L$ 次。此时 $c_{l}$ 与 $A_{l}$ 由 $\widetilde{\mathcal{G}}$ 导出,可能存在依赖。此时可写为

$$
P_{ij} - E[P_{ij}|\widetilde{\mathcal{G}} ] = \frac{1}{d}\sum_{\ell = 1}^{L}\sum_{k:k\neq j,i}\widetilde{A}_{ijk}[Z_{ijk}^{l} - EZ_{ijk}^{l}],
$$

其中 $Z_{ijk}^{l} = 1(A_{l} = \{i,j\} ,c_{l} = j) / f(\{i,j\}) + 1(A_{l} = \{i,j,k\} ,c_{l} = j) / f(\{i,j,k\})$,该量在给定 $\widetilde{A}_{ijk}$ 时仍上下有界且独立。此处稍作符号复用,重新定义

$$
n^{\dagger} = L\max_{i\neq j}\sum_{k:k\neq j,i}\widetilde{A}_{ijk},\quad n^{\dagger} = L\max_{i}\sum_{j< k:j,k\neq i}\widetilde{A}_{ijk}.
$$

类似于第 4.1 节,在 $\widetilde{G}$ 条件下有

$$
\begin{array}{r l} & {\max_{i}\bigg|\sum_{j:j\neq i}P_{i j} - \sum_{j:j\neq i}E[P_{i j}|\widetilde{\mathcal{G}} ]\bigg| = \mathcal{O}_{P}(d^{-1}\sqrt{n^{\dagger}\log n}),}\\ & {\max_{i\neq j}\bigg|P_{i j} - E[P_{i j}|\widetilde{\mathcal{G}} ]\bigg| = \mathcal{O}_{P}(d^{-1}\sqrt{n^{\ddagger}\log n}),}\\ & {\sum_{j:j\neq i}E[P_{i j}|\widetilde{\mathcal{G}} ] = \tau_{i}^{\diamond}e^{-\theta_{i}^{*}}\asymp \frac{1}{d} n^{\dagger} \quad \text{(假设)}.} \end{array}
$$

将假设 4.2 调整为如下假设。注意对 $L$ 无假设限制,故 $L$ 可低至 1。

假设 4.4. 在含 $M$ 元完全比较的 PL 模型中,谱排序选择 $d\asymp n^{\dagger}$,且假设对所有 $i\in [n]$ 满足 $\tau_{i}^{\diamond}e^{- \theta_{i}^{*}}\asymp n^{\dagger} / d$,$e^{4\bar{\kappa}} = o(n)$ 且 $p\gtrsim e^{6\bar{\kappa}}\mathrm{poly}(\log n) / \binom{n-1}{M-1}$。

在假设 4.4 下,可证概率 $1 - o(1)$ 成立:

$$
n^{\dagger}\asymp \binom{n-1}{M-1}p L,\qquad\max \Big\{\binom{n-2}{M-2}p-\log n,0\Big\}L\lesssim n^{\ddagger}\lesssim\Big[\binom{n-2}{M-2}p+\log n\Big]L.
$$

注意在 $n^{\dagger}$ 中,由假设 4.4 知主导项为 $\binom{n-1}{M-1}p L$。但在 $n^{\ddagger}$ 中,存在附加项 $\log n$(源于 Bernstein 不等式的亚指数尾衰减),若 $p$ 极小则 $\log n$ 可能主导 $n^{\ddagger}$。当 $p$ 较大时(即 $\binom{n-2}{M-2}p\gtrsim\log n$),有 $n^{\ddagger}\asymp \binom{n-2}{M-2}pL \asymp n n^{\dagger}/M$,此时假设 4.2 成立。故此时比较图稠密,证明过程与定理 4.2 类似。当 $p$ 较小时(即 $\binom{n-2}{M-2}p\lesssim\log n$),若 $L$ 有界则 $n^{\ddagger}\lesssim L\log n$。此时将修改定理 4.2 的证明以适应随机图情形,以证得下述定理 4.4。此外,因 $\sum_{j:j\neq i}P_{i j} = \mathcal{O}_{P}(n^{\dagger}/d)$,在假设 4.4 中选择 $d\asymp n^{\dagger}$ 以使转移矩阵对角元素为常数阶是合理的。注意在固定图情形下,因比较图无随机性,无需对 $d$ 施加速率假设。

接下来验证在 PL 模型下,假设 4.3 高概率成立。

定理 4.3. 在 PL 模型及假设 4.4 下,当以 $\widetilde{G}$ 替代 $\mathcal{G}$ 为条件时,假设 4.3 以概率 $1 - o(1)$ 成立。

我们希望证明:在假设 4.4 下,谱估计量 $\widehat{\theta}_{i}$ 满足一致逼近:对所有 $i\in [n]$,$\widetilde{\theta}_{i} - \theta_{i}^{*}$ 与 $J_{i}^{*}$ 的差为 $o_{P}(1 / \sqrt{n^{\dagger}})$。关键步骤仍是在此较弱的假设 4.4 下验证 (3.1) 对随机比较图成立。

定理 4.4. 在 PL 模型及假设 4.1 与 4.4 下,谱估计量 $\widetilde{\theta}_{i}$ 满足一致逼近:对所有 $i\in [n]$ 一致地有 $\widetilde{\theta}_{i} - \theta_{i}^{*} = J_{i}^{*} + o_{P}\big(1 / \sqrt{n^{\dagger}}\big)$。因此,谱估计量 (2.1) 满足

\[
\| \widetilde{\theta} -\theta^{*}\|_{\infty}\lesssim e^{\bar{\kappa}}\sqrt{\frac{\log n}{\binom{n - 1}{M - 1}pL}}, \tag{4.5}
\]

其概率为 $1 - o(1)$。此外,

$$
\rho_{i}(\theta)(\widetilde{\theta}_{i} - \theta_{i}^{*})\Rightarrow N(0,1),
$$

对所有 $i\in [n]$ 成立,其中 $\rho_{i}(\theta) = \mathrm{Var}(J_{i}^{*}|\widetilde{G})^{-1/2}$,且在 $\mathrm{Var}(J_{i}^{*}|\widetilde{G})$ 公式中 $\theta$ 可取 $\theta^{*}$ 或其任意相合估计。

注 4.2. 采用最优加权 $f(A_{l}) = \sum_{u\in A_{l}}e^{\theta_{u}^{*}}$ 的两阶段估计量(可通过单独小数据集进行相合估计),其方差与 MLE 估计量相同,达到了所有估计量的 Cramer-Rao 下界(Fan 等, 2022a,b)。

推论 4.1. 在定理 4.4 条件下,若 $\theta_{(K)}^{*} - \theta_{(K + 1)}^{*}\geq \Delta$($\theta_{(i)}^{*}$ 表示真实排名为 $i$ 的项目得分),且当样本复杂度满足

$$
e^{2\bar{\kappa}}\Delta^{-2}\cdot \log n = \mathcal{O}\bigg(\binom{n-1}{M-1}pL\bigg),
$$

则有 $\{i\in [n],\widehat{r_{i}}\leq K\} = \{i\in [n],r_{i}^{*}\leq K\}$(所选 top-K 集合与真实 top-K 集合相同),其中 $\widehat{r_{i}}$, $r_{i}^{*}$ 分别表示 $\widehat{\theta}_{i}$ 在 $\{\widehat{\theta}_{i},i\in [n]\}$ 中的经验排名及第 $i$ 项的真实排名。

我们指出,当 $M = 2$ 且 $\bar{\kappa} = \mathcal{O}(1)$ 时,推论 4.1 的结论可约化为 Chen 等 (2019) 定理 1 的结论。


\subsubsection{Bootstrap 程序的有效性证明}

本节的主要目的是验证 3.4 节所提出的 bootstrap 方法的有效性。目标量 $T_{\mathcal{M}}$ 是随机向量的最大模: 

$$
\Delta_{\mathcal{M}}\coloneqq \left\{\frac{\widetilde{\theta}_{k} - \widetilde{\theta}_{m} - (\theta_{k}^{*} - \theta_{m}^{*})}{\widetilde{\sigma}_{km}}\right\}_{m\in \mathcal{M},k\neq m}.
$$

对于 $\Delta_{\mathcal{M}}$ 的每个边际分量,均可通过定理 4.2 类似地建立渐近正态性。但研究 $\| \Delta_{\mathcal{M}}\|_{\infty}$ 的渐近分布极具挑战性,因其维度 $(n - 1)|\mathcal{M}|$ 随项目数量增加而增长。特别地,传统的多元中心极限定理对 $\Delta_{\mathcal{M}}$ 可能渐近失效(Portnoy, 1986)。为处理高维性,我们将借助现代高斯逼近理论(Chernozhukov 等, 2017, 2019)推导 $\| \Delta_{\mathcal{M}}\|_{\infty}$ 的渐近分布(见定理 E.1)。此外,下述定理验证了我们乘数 bootstrap 方法的有效性。

定理 4.5. 假设 $e^{3\bar{\kappa}}(\log n)^2 = o(n)$ 且 $e^{5\bar{\kappa}}n^{\ddagger}n^{1 / 2}(\log n)^3 = o(n^{\dagger})$ 。则在定理 4.1 条件下,成立

$$
|\mathbb{P}(T_{\mathcal{M}} > \mathcal{Q}_{1 - \alpha}) - \alpha |\to 0.
$$

备注 4.3. 定理 4.5 表明,基于高斯乘数 bootstrap 估计的分位数 $\mathcal{Q}_{1 - \alpha}$ 确实将排名 $\{r_m\}_{m\in \mathcal{M}}$ 的联合置信区间 (3.15) 的显著性水平控制在预定水平 $\alpha$,即

$$
\mathbb{P}\Big(r_{m}\in [\mathcal{R}_{mU},\mathcal{R}_{mR}]\text{ 对所有 }m\in \mathcal{M}\Big)\geq 1 - \alpha +o(1).
$$

最近 Fan 等 (2022b) 在仅观测每次比较最优选择的 PL 模型中提出了类似方法构建排名的联合置信区间,但其要求每个关联项目的比较次数 $L$ 充分大($L\gtrsim \mathrm{poly}(\log n)$)。相较之下,我们 3.4 节的方法无需对每个 $A_{l}$ 的比较次数施加任何约束(甚至允许所有比较 $L=1$),因而适用性更广——因为实际问题中,大小为 $M$ 的项目集合常在不同时间比较且有时仅比较一次。



\end{document}
